\documentclass[a4paper,german,12pt,smallheadings]{scrartcl}
\usepackage[T1]{fontenc}
\usepackage[utf8]{inputenc}
\usepackage{babel}
\usepackage{tikz}
\usepackage{geometry}
\usepackage{amsmath}
\usepackage{amssymb}
\usepackage{float}
%\usepackage{wrapfig}
\usepackage[thinspace,thinqspace,squaren,textstyle]{SIunits}
\restylefloat{table}
\geometry{a4paper, top=15mm, left=20mm, right=40mm, bottom=20mm, headsep=10mm, footskip=12mm}
\linespread{1.5}
\setlength\parindent{0pt}
\begin{document}
\begin{center}
\bfseries % Fettdruck einschalten
\sffamily % Serifenlose Schrift
\vspace{-40pt}
Elektrodynamik und Optik, Sommersemester 2013, 3. Blatt \\
Markus Fenske, Tutor: Dr. Marko Wietstruk
\vspace{-10pt}
\end{center}
\section*{Aufgabe 1}

Später.

\section*{Aufgabe 2}
\subsection*{Teil a}

Für die Energie im Kondensator gilt:

\begin{align*}
  E = \frac{1}{2} C U^2
\end{align*}

Die Kapazität steigt linear mit steigender Dielektrizitätskonstante. Dementsprechend ist also

\begin{align*}
  C_{\text{neu}} = \epsilon_r C_\text{alt}
\end{align*}

Einsetzen und umstellen:

\begin{align*}
  &E = \frac{1}{2} \epsilon_r C_\text{alt} U^2 \\
  \Leftrightarrow \quad &\epsilon_r = \frac{2E}{C_\text{alt} U^2}
\end{align*}

Durch Einsetzen der gegebenen Werte erhält man:

\begin{align*}
  \epsilon_r = \frac{2 \cdot 10 \cdot 10^{-6}\;\joule}{10 \cdot 10^{-12}\; \farad \left(700\; \volt\right)^2} \approx 4{,}08\; \farad\per\meter
\end{align*}

Diese Dielekrizitätszahl sollte ein Material haben, damit der Kondensator die
angegebene Energiemenge speichern kann.

\end{document}
