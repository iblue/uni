\documentclass[a4paper,german,12pt,smallheadings]{scrartcl}
\usepackage[T1]{fontenc}
\usepackage[utf8]{inputenc}
\usepackage{babel}
\usepackage{tikz}
\usepackage{geometry}
\usepackage{amsmath}
\usepackage{amssymb}
\usepackage{float}
%\usepackage{wrapfig}
\usepackage[thinspace,thinqspace,squaren,textstyle]{SIunits}
\restylefloat{table}
\geometry{a4paper, top=15mm, left=20mm, right=40mm, bottom=20mm, headsep=10mm, footskip=12mm}
\linespread{1.5}
\setlength\parindent{0pt}
\begin{document}
\begin{center}
\bfseries % Fettdruck einschalten
\sffamily % Serifenlose Schrift
\vspace{-40pt}
Analytische Mechanik, Sommersemester 2013, 3. Blatt \\
Luis Herrmann und Markus Fenske, Tutor: Clemens Meyer zu Rheda
\vspace{-10pt}
\end{center}
\section*{Aufgabe 1}
\subsection*{Teil a}

In der Aufgabe ist festgelegt, dass die $x$-Position des Keils durch die Lage
seiner Spitze festgelegt werden soll. Da der Keil auf einer ebenen Unterlage
ist, ist er in $y$-Richtung nicht beweglich (Zwangbedingung: $y=0$), so dass
sich die Position durch $x$ allein festlegen lässt. Von der Skizze des Blattes
weichen wir allerdings ab und spiegeln das Problem an der $y$-Achse, so dass
der Winkel zur anderen Seite offen ist. Des Ergebnis bleibt das selbe.

Der Klotz kann sich nur auf der Oberfläche des Keils entlang bewegen. Aufgrund
der Spiegelung ergibt sich dann folgende Gleichung, die auch eine
Zwangsbedingung darstellt.

\begin{align*}
  \vec{r} =  r \begin{pmatrix} \cos \alpha \\ \sin \alpha \end{pmatrix}
\end{align*}

Die unabhängigen Koordindaten lauten also $x$ und $r$.

\subsection*{Teil b}

Sei $T_x$ die kinetische Energie des Keils und $T_r$ die des Klotzes. Bei $T_r$
ist zu beachten, dass sich die Geschwindigkeit des Klotzes gegenüber dem
Nullpunkt durch Vektoraddition der Geschwindigkeit des Keils gegenüber dem
Nullpunkt und des Klotzes gegenüber dem Keil ergibt. Beim Auflösen der
Quadrierten Klammer ergibt sich ein Skalarprodukt, dass wir mit $\vec{a} \cdot
\vec{b} = |a| |b| \cos \theta$ lösen. $\theta$ ist dabei der Winkel zwischen
den Vektoren, der genau mit unserem $\alpha$ übereinstimmt. Sieht man aber
auch, wenn man die Vektoren komponentenweise ausmultipliziert.

\begin{align*}
  T &= T_x + T_r \\
    &= \frac{m_1}{2}\dot{x}^2 + \frac{m_2}{2} \left(\dot{\vec{x}} + \dot{\vec{r}}\right)^2 \\
    &= \frac{m_1}{2}\dot{x}^2 + \frac{m_2}{2}(\dot{x}^2 + \dot{r}^2 + 2\dot{\vec{x}}\cdot\dot{\vec{r}}) \\
    &= \frac{m_1}{2}\dot{x}^2 + \frac{m_2}{2}(\dot{x}^2 + \dot{r}^2 + 2\dot{x}\dot{r} \cos \alpha) \\
    &= \frac{m_1}{2}\dot{x}^2 + \frac{m_2}{2}\dot{x}^2 + \frac{m_2}{2}\dot{r}^2 + m_2\dot{x}\dot{r} \cos \alpha
\end{align*}

Das Gravitationspotential ist $V(y) = mgy$. Sei $V_x$ die potentielle Energie
des Keils und $V_r$ die des Klotzes. $V_x = 0$, denn die $y$-Koordinate des
Klotzes verschwindet aufgrund der Zwangsbedingung.

\begin{align*}
  V &= V_x + V_r \\
    &= 0 + m_2gr \sin \alpha
\end{align*}

Für die Lagrange-Funktion $L = T - V$ ergibt sich dann

\begin{align*}
  L &= T - V \\
    &= \frac{m_1}{2}\dot{x}^2 + \frac{m_2}{2}\dot{x}^2 + \frac{m_2}{2}\dot{r}^2 + m_2\dot{x}\dot{r} \cos \alpha - m_2gr \sin \alpha \\
    &= \frac{m_1+m_2}{2}\dot{x}^2 + \frac{m_2}{2}\dot{r}^2 + m_2\dot{x}\dot{r} \cos \alpha - m_2gr \sin \alpha
\end{align*}

\subsection*{Teil c}

Die Bewegungsgleichungen ergeben sich durch Anwendung der Euler-Lagrange-Gleichung:

\begin{align*}
  \left(\frac{d}{dt}\frac{\partial}{\partial \dot{q_i}} - \frac{\partial}{\partial q_i}\right) L = 0
\end{align*}

Für die $x$-Koordinate ist:

\begin{align}
  \frac{d}{dt}\frac{\partial L}{\partial \dot{x}} = \frac{d}{dt} (m_1+m_2)\dot{x} + m_2\dot{r} = (m_1+m_2)\ddot{x} + m_2\ddot{r} \cos \alpha \\
  \frac{\partial L}{\partial x} = 0 \\
  \Rightarrow (m_1+m_2)\ddot{x} = -m_2 \ddot{r} \cos \alpha
  \label{x_component}
\end{align}

Für die $r$-Koordinate:

\begin{align}
  \frac{d}{dt}\frac{\partial L}{\partial \dot{r}} = \frac{d}{dt} m_2\dot{r} + m_2\dot{x} \cos \alpha = m_2\ddot{r} + m_2\ddot{x} \cos \alpha \\
  \frac{\partial L}{\partial r} = -m_2g \sin \alpha \\
  \Rightarrow m_2\ddot{r} + m_2\ddot{x} \cos \alpha = -m_2g \sin \alpha
  \label{r_component}
\end{align}

Um nun zu zeigen, dass beide Massen konstant beschleunigt werden, müssen wir
zeigen dass $\ddot{r} = \operatorname{const.}$ und $\ddot{x} =
\operatorname{const.}$.

Dazu stellen wir (\ref{r_component}) nach $\ddot{r}$ um:

\begin{align}
  &m_2\ddot{r} + m_2\ddot{x} \cos \alpha = -m_2g \sin \alpha \\
  \Leftrightarrow \qquad &\ddot{r} + \ddot{x} \cos \alpha = -g \sin \alpha \\
  \Leftrightarrow \qquad &\ddot{r} = -\ddot{x} \cos \alpha -g \sin \alpha \label{solved_for_r}
\end{align}

Und setzen das Ergebnis in (\ref{x_component}) ein:

\begin{align*}
  &(m_1+m_2)\ddot{x} = -m_2 \ddot{r} \cos \alpha \\
  \Leftrightarrow \qquad &(m_1+m_2)\ddot{x} = -m_2 (-\ddot{x} \cos \alpha -g \sin \alpha) \cos \alpha \\
  \Leftrightarrow \qquad &(m_1+m_2)\ddot{x} = m_2 \ddot{x} \cos \alpha + m_2 g \sin \alpha \cos \alpha \\
  \Leftrightarrow \qquad &(m_1+m_2-m_2 \cos \alpha)\ddot{x} = m_2 g \sin \alpha \cos \alpha \\
  \Leftrightarrow \qquad &\ddot{x} = \frac{m_2 g \sin \alpha \cos \alpha}{m_1+m_2-m_2 \cos \alpha}
\end{align*}

Diese Gleichung ist ziemlich häßlich, muss aber nicht weiter vereinfacht
werden. Da auf der rechten Seite nur noch Konstanten stehen, muss $\ddot{x} =
\operatorname{const.}$.

Wenn $\ddot{x}$ konstant ist, muss aber gemäß (\ref{solved_for_r}) auch
$\ddot{r}$ konstant sein.


\subsection*{Teil d}
Nicht bearbeitet.

\section*{Aufgabe 3}
Das angegebene Beispiel erschien uns logisch etwas inkonsistent, weswegen wir
uns an dem Vorgehen aus dem Buch ''Physik mit Bleistift'' orientieren möchten.

Dort ist die Schreibweise anders. Gegeben sei ein Funktional $I[f]$. Die
funktionale Ableitung ergibt sich dann nicht wie im Beispiel durch $\delta I[f]
= I[f_0 - \delta f] - I[f_0]$, wobei im Term $I[f_0 - \delta f]$ nur Terme bis
zur maximal ersten Ordnung erhalten bleiben, sondern durch

\begin{align*}
  \delta I = \left.I[f_0 - \eta]\right|_{\eta^1}
\end{align*}

Wobei $\eta^1$ bedeutet, dass ausschließlich Terme mit linearem $\eta$ erhalten
bleiben. Terme geringerer Ordnung ($\eta^0$) und höherer Ordnung ($\eta^2$,
$\eta^3$, \dots) entfallen.

Sofern die Terme dabei nicht frei stehen, führen wir eine Taylor-Entwicklung
(z.B. Aufgabenteil b) durch, ansonsten ist diese natürlich unnötig (z.B.
Aufgabenteil a).

\subsection*{Teil a}

\begin{align*}
  I[f] &= \int_a^b f(x)^2 dx \\
  \delta I &= \left.I[f_0 - \eta]\right|_{\eta^1} \\
  &=\left.\int_a^b dx (f+\eta)^2 \right|_{\eta^1} \\
  &=\left.\int_a^b dx f^2 + 2 \eta f + \eta^2 \right|_{\eta^1} \\
  &=\int_a^b dx \eta 2 f \\
  \Rightarrow& \frac{\delta I[f]}{\delta f} = 2f(x)
\end{align*}

\subsection*{Teil b}

\begin{align*}
  I[f] &= \int_a^b sin(f(x)) dx \\
  \delta I &= \left.I[f_0 - \eta]\right|_{\eta^1} \\
  &=\left.\int_a^b dx sin(f+\eta) \right|_{\eta^1}
\end{align*}

Hier entwickeln wir nun in eine Taylor-Reihe um freistehende $\eta^1$-Terme zu
erhalten. Wir benötigen nur den Term erster Ordnung, den Rest frisst unser
$\eta^1$-Operator sowieso, deswegen lassen wir die anderen auch gleich weg.

\begin{align*}
  &=\left.\int_a^b dx \dots + \frac{\left(\frac{\partial}{\partial \eta} \sin(f+\eta)\right)(\eta=0)}{1!} (\eta - 0)^1 + \dots\quad\right|_{\eta^1} \\
  &=\int_a^b dx \eta \cos(f) \\
  &\Rightarrow \frac{\delta I[f]}{\delta f} = \cos(f(x))
\end{align*}

\subsection*{Teil c}
\begin{align*}
  I[f] &= \int_a^b g(x)f'(x) dx \\
  \delta I &= \left.I[f_0 - \eta]\right|_{\eta^1} \\
  &=\left.\int_a^b dx g(x)(f' + \eta') \right|_{\eta^1}
\end{align*}

Taylor-Entwicklung ist hier nicht sinnvoll, man sieht direkt, welche Terme stehen bleiben:

\begin{align*}
  &=\int_a^b dx \eta' g(x)
\end{align*}

Nun integrieren wir partiell, um $\eta$-Terme zu erhalten:

\begin{align*}
  &=\left[\eta(x) g(x)\right]_{x=a}^{x=b} - \int_a^b dx \eta(x) g'(x)
\end{align*}

Die $\eta(x)$-Funktion verschwindet per definitionem in den Randbereichen, denn
es ist $\eta(a) = 0$ und $\eta(b) = 0$. Somit fällt der erste Term weg. Die
funktionale Ableitung lässt sich dann direkt ablesen:

\begin{align*}
  \frac{\delta I[f]}{\delta f} = g'(x)
\end{align*}

\subsection*{Teil d}
\begin{align*}
  I[f] &= \int_a^b f'(x)^2 dx \\
  \delta I &= \left.I[f_0 - \eta]\right|_{\eta^1} \\
  &=\left.\int_a^b dx (f' + \eta')^2 \right|_{\eta^1} \\
  &=\left.\int_a^b dx f'^2 + 2\eta'f' + \eta'^2 \right|_{\eta^1} \\
  &=\int_a^b dx 2\eta'f'
\end{align*}

Partielle Integration:

\begin{align*}
  &=\left[2\eta(x) f'(x)\right]_{x=a}^{x=b} - \int_a^b dx \eta(x) f''(x)
\end{align*}

Selbe Begründung wie oben: Der erste Term verschwindet, die funktionale Ableitung lässt sich ablesen:
\begin{align*}
  \frac{\delta I[f]}{\delta f} = f''(x)
\end{align*}

\subsection*{Teil e}
Da $\eta$ und $n$ schrecklich schwer außeinander zu halten sind,
definieren wir kurzerhand $a := n$.

\begin{align*}
  I[f] &= \int_a^b a(x,f(x)) \sqrt{1+f'(x)^2} dx \\
  \delta I &= \left.I[f_0 - \eta]\right|_{\eta^1} \\
  &=\left.\int_a^b dx \; a(x, f + \eta) \sqrt{1+(f'+\eta')^2} \right|_{\eta^1}
\end{align*}

Da hier nur Terme in mit $f$, $f'$ und $x$ vorkommen, bietet sich zur Lösung
die Euler-Lagrange-Gleichung an. Sei $L = a(x, f) \sqrt{1+f'^2}$

\begin{align*}
  \frac{d}{dx} \frac{\partial L}{\partial f'} &= \frac{d}{dx} \left(\underbrace{\frac{\partial}{\partial f'} (a(x, f))}_{=0} \sqrt{1+f'^2} + a(x,f) \frac{\partial}{\partial f'} \sqrt{1+f'^2} \right) \\
  &= \frac{d}{dx} \left(a(x,f) \frac{\partial}{\partial f'} \sqrt{1+f'^2} \right) \\
  &= \frac{d}{dx} \left(a(x,f) \frac{f'}{\sqrt{1+f'^2}} \right)
\end{align*}

\begin{align*}
  \frac{\partial L}{\partial f} &= \frac{\partial}{\partial f} \left(a(x, f)\right) \sqrt{1+f'^2} + a(x, f) \underbrace{\frac{\partial}{\partial f} \sqrt{1+f'^2}}_{=0} \\
  &= \frac{\partial}{\partial f} \left(a(x, f)\right) \sqrt{1+f'^2} \\
  &= f' \cdot a'(x, f) \sqrt{1+f'^2}
\end{align*}

Und damit ist:

\begin{align*}
  \frac{\delta I[f]}{\delta f} &= \frac{d}{dx} \frac{\partial L}{\partial f'} - \frac{\partial L}{\partial f} \\
  &= \frac{d}{dx} \left(a(x,f) \frac{f'}{\sqrt{1+f'^2}} \right) - f' \cdot a'(x, f) \sqrt{1+f'^2}
\end{align*}

\section*{Aufgabe 4}
\subsection*{Teil a}

Ausgehend von der Lösung in der letzten Ausgabe ergibt sich aus $a() = c$, dass
$a'() = 0$. Das Licht nimmt den kürzesten Weg, deswegen muss das Funktional
minimal sein und damit die Ableitung $=0$. Ob ein Minium oder ein Maximum
vorliegt, könnte man mithilfe der zweiten funktionalen Ableitung prüfen,
betrachten wir allerdings als sinnlos, da das Ergebnis mit der Erwartung
übereinstimmt.

\begin{align*}
  &\frac{d}{dx} \left(a(x,f) \frac{2f'}{\sqrt{1+f'^2}} \right) - f' \cdot a'(x, f) \sqrt{1+f'^2} \overset{!}{=} 0 \\
  \Leftrightarrow \quad &\frac{d}{dx} \left(a(x,f) \frac{2f'}{\sqrt{1+f'^2}} \right) = f' \cdot a'(x, f) \sqrt{1+f'^2}
\end{align*}

Einsetzen von $a = c$, $a' = 0$:

\begin{align*}
  \Leftrightarrow \quad &\frac{d}{dx} \left(c \frac{2f'}{\sqrt{1+f'^2}} \right) = 0
\end{align*}

Das braucht nicht explizit ausgerechnet zu werden. Damit der Term null wird,
muss über dem Bruchstrich eine Konstante stehen, also muss $f'(x) = A$. Daraus
folgt $f(x) = Ax+B$.

Diese Funktion beschreibt eine Gerade. Da Licht sich in Medien mit konstantem
Brechnungsindex immer auf einer Geraden ausbreitet, muss die Lösung stimmen.

\subsection*{Teil b}

\begin{align*}
  &\frac{d}{dx} \left(a(x,y) \frac{2f'}{\sqrt{1+f'^2}} \right) - y' \cdot a'(x, y) \sqrt{1+y'^2} \overset{!}{=} 0 \\
  \Leftrightarrow \quad &\frac{d}{dx} \left(a(x,y) \frac{2y'}{\sqrt{1+y'^2}} \right) = y' \cdot a'(x, y) \sqrt{1+y'^2}
\end{align*}

Einsetzen von $a = e^y$, $a' = y' e^y$:

\begin{align*}
  \Leftrightarrow \quad &\frac{d}{dx} \left(e^y \frac{2y'}{\sqrt{1+y'^2}} \right) = y'^2 e^y \sqrt{1+y'^2} \\
  \Leftrightarrow \quad &\frac{2e^y(y''+y'^4+y'^2)}{(y'^2+1)^\frac{3}{2}}= y'^2 e^y \sqrt{1+y'^2} \\
  \Leftrightarrow \quad &2(y''+y'^4+y'^2)= y'^2 (1+y'^2)^2\\
  \Leftrightarrow \quad &2(y''+y'^4+y'^2)= y'^2+y'^6\\
  \Leftrightarrow \quad &2y''+2y'^4+2y'^2= y'^2+y'^6\\
\end{align*}

Die Lösung dieser Differentialgleichung ergäbe den Lichtweg. Leider können wir
sie nicht lösen. Auch Wolfram Alpha nicht (Überschreitung der Rechenzeit).

\subsection*{Teil c}
Nicht bearbeitet.


\end{document}
