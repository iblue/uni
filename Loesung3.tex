\documentclass[a4paper,german,12pt,smallheadings]{scrartcl}
\usepackage[T1]{fontenc}
\usepackage[utf8]{inputenc}
\usepackage{babel}
\usepackage{tikz}
\usepackage{geometry}
\usepackage{amsmath}
\usepackage{amssymb}
\usepackage{float}
%\usepackage{wrapfig}
\usepackage[thinspace,thinqspace,squaren,textstyle]{SIunits}
\restylefloat{table}
\geometry{a4paper, top=15mm, left=20mm, right=40mm, bottom=20mm, headsep=10mm, footskip=12mm}
\linespread{1.5}
\setlength\parindent{0pt}
\begin{document}
\begin{center}
\bfseries % Fettdruck einschalten
\sffamily % Serifenlose Schrift
\vspace{-40pt}
Analytische Mechanik, Sommersemester 2013, 3. Blatt \\
Luis Herrmann und Markus Fenske, Tutor: Clemens Meyer zu Rheda
\vspace{-10pt}
\end{center}
\section*{Aufgabe 1}

\subsection*{Teil a}
Für den Klotz wählen wir die Koordinaten $(x_1, y_1)$, für den Keil die
Koordinaten $(x_2, y_2)$, jeweils vom Ursprung aus.

Der Keil bewegt sich nur auf seiner Unterlage. Der Klotz hingegen nur auf der
Schräge des Keils. Dies führt zu den folgenden Zwangsbedingungen:


\begin{align*}
  y_1 &= (x1-x2) \sin \alpha \\
  y_2 &= 0
\end{align*}

\subsection*{Teil b}

Sei $T_1$ die kinetische Energie des Klotzes und $T_2$ die des Keils. Die Masse
des Klotzes sei $m_1$, die Masse des Keils sei $m_2$.

\begin{align*}
  T_1 &= \frac{1}{2} m_1 \left(\dot{x_1}^2 + \dot{y_1}^2\right) \\
  T_2 &= \frac{1}{2} m_2 \dot{x_2}^2
\end{align*}

Durch Einsetzen der Zwangsbedingungen erhalten wir:

\begin{align*}
  T_1 &= \frac{1}{2}m_1\dot{x_1}^2 + \frac{1}{2}m_2 (\dot{x_1}^2 - 2\dot{x_1}\dot{x_2}+\dot{x_2}^2) \sin^2 \alpha \\
  T_2 &= \frac{1}{2} m_2 \dot{x_2}^2
\end{align*}

Für die kinetische Energie insgesamt dann also:

\begin{align*}
  T = \frac{1}{2}m_1\dot{x_1}^2 + \frac{1}{2}m_2 (\dot{x_1}^2 - 2\dot{x_1}\dot{x_2}+\dot{x_2}^2) \sin^2 \alpha + \frac{1}{2} m_2 \dot{x_2}^2
\end{align*}

Eine potentielle Energie erhält nur der Klotz, denn der Keil ändert seine Höhe nicht. Somit ist:

\begin{align*}
  V = V_2 = m_2gy_2 = m_2g (x_1 - x_2) \sin \alpha
\end{align*}

Damit ist die Lagrangefunktion insgesamt:

\begin{align*}
  L = T -V = \frac{1}{2}m_1\dot{x_1}^2 + \frac{1}{2} m_2 (\dot{x_1}^2 - 2\dot{x_1}\dot{x_2}+\dot{x_2}^2) \sin^2(\alpha) + \frac{1}{2} m_2 \dot{x_2}^2 - m_2g (x_1 - x_2) \sin \alpha
\end{align*}

\subsection*{Teil c}

Die Bewegungsgleichungen ergeben sich durch Anwendung der Euler-Lagrange-Gleichung:

\begin{align*}
  \left(\frac{d}{dt}\frac{\partial}{\partial \dot{q_i}} - \frac{\partial}{\partial q_i}\right) L = 0
\end{align*}

Für die $x_1$-Koordinate:

\begin{align*}
  \frac{\partial L}{\partial x_1} &= -mg \sin \alpha \\
  \frac{d}{dt}\frac{\partial L}{\partial \dot{x_1}} &= \frac{d}{dt} m_1 \dot{x_1} + m_2\dot{x_1}\sin^2 \alpha - m_2\dot{x_2} \sin^2 \alpha \\
                                                  &= m_1\ddot{x_1} + m_2 \ddot{x_1} \sin^2 \alpha - m_2 \ddot{x_2} \sin^2 \alpha \\
  \Rightarrow\quad &m_1\ddot{x_1} + m_2 \ddot{x_1} \sin^2 \alpha - m_2 \ddot{x_2} \sin^2 \alpha = -mg \sin \alpha
\end{align*}

Für $x_2$ ergibt sich:

\begin{align*}
  \frac{\partial L}{\partial x_2} &= mg \sin \alpha \\
  \frac{d}{dt}\frac{\partial L}{\partial \dot{x_2}} &= \frac{d}{dt} m_2\dot{x_2} \sin^2 \alpha - m_2 \dot{x_1} \sin^2 \alpha + m_2\dot{x_2} \\
  &= \ddot{x_2}(m_2+m_2 \sin^2 \alpha) - \ddot{x_1} m_2 \sin^2 \alpha \\
  \Rightarrow\quad &\ddot{x_2}(m_2+m_2 \sin^2 \alpha) - \ddot{x_1} m_2 \sin^2 \alpha = mg \sin \alpha
\end{align*}

Um zu zeigen, dass beide Massen konstant beschleunigt werden, addieren wir die beiden Bewegungsgleichungen:

\begin{align*}
&\ddot{x_1}(m_1+m_2 \sin^2 \alpha -m_2 \sin^2 \alpha)+\ddot{x_2}(m_2+m_2 \sin^2 \alpha - m_2 \sin^2 \alpha)=mg \sin \alpha - mg \sin \alpha \\
\Leftrightarrow\quad & \ddot{x_1} m_1 = -\ddot{x_2} m_2
\end{align*}

% iblue: Habe das hinzugefügt.
Setzt man dies in die zweite Bewegungsgleichung ein, erhält man:

\begin{align*}
  &\ddot{x_2}(m_2+m_2 \sin^2 \alpha) + \ddot{x_2} \frac{m_2^2}{m_1} \sin^2 \alpha = mg \sin \alpha \\
  \Leftrightarrow \quad &\ddot{x_2} (m_2+m_2 \sin^2 \alpha + \frac{m_2^2}{m_1} \sin^2 \alpha) = mg \sin \alpha \\
  \Leftrightarrow \quad &\ddot{x_2} = \frac{mg \sin \alpha}{m_2+m_2 \sin^2 \alpha + \frac{m_2^2}{m_1} \sin^2 \alpha}
\end{align*}

Da auf der rechten Seite nur Konstanten stehen, muss $\ddot{x_2}$ konstant
sein. Dies bedeutet, dass gemäß $\ddot{x_1} m_1 = -\ddot{x_2} m_2$ auch
$\ddot{x_1}$ konstant sein muss. Somit werden beide Massen konstant
beschleunigt.
% iblue: -----------


\subsection*{Teil d}
Nicht bearbeitet.

\section*{Aufgabe 2}
\subsection*{Teil a}
Wenn $x(0) = 0$ ist, ist klar, dass $A=0$ sein muss. Damit vereinfacht sich der
Bewegungsgleichungskandidat zu $x(t) = Bt+Ct^2$.

Ausrechnen der kinetischen Energie:
\begin{align*}
  T &= \frac{1}{2}m x'(t)^2 \\
    &= \frac{1}{2}m (B+2Ct)^2 \\
    &= \frac{m}{2} (B^2 + 4BCt + C^2t^2)
\end{align*}

Potenzielle Energie:
\begin{align*}
  V &= -Fx \\
    &= -F(Bt+Ct^2)
\end{align*}

Lagrange-Funktion:
\begin{align*}
  L &= T - V \\
    &= \frac{m}{2} (B^2 + 4BCt + C^2t^2) + F(Bt+Ct^2)
\end{align*}

\section*{Aufgabe 3}
Das angegebene Beispiel erschien uns logisch etwas inkonsistent, weswegen wir
uns an dem Vorgehen aus dem Buch ``Physik mit Bleistift'' orientieren möchten.

Dort ist die Schreibweise anders. Gegeben sei ein Funktional $I[f]$. Die
funktionale Ableitung ergibt sich dann nicht wie im Beispiel durch $\delta I[f]
= I[f_0 - \delta f] - I[f_0]$, wobei im Term $I[f_0 - \delta f]$ nur Terme bis
zur maximal ersten Ordnung erhalten bleiben, sondern durch

\begin{align*}
  \delta I = \left.I[f_0 - \eta]\right|_{\eta^1}
\end{align*}

Wobei $\eta^1$ bedeutet, dass ausschließlich Terme mit linearem $\eta$ erhalten
bleiben. Terme geringerer Ordnung ($\eta^0$) und höherer Ordnung ($\eta^2$,
$\eta^3$, \dots) entfallen.

Sofern die Terme dabei nicht frei stehen, führen wir eine Taylor-Entwicklung
(z.B. Aufgabenteil b) durch, ansonsten ist diese natürlich unnötig (z.B.
Aufgabenteil a).

\subsection*{Teil a}

\begin{align*}
  I[f] &= \int_a^b f(x)^2 dx \\
  \delta I &= \left.I[f_0 - \eta]\right|_{\eta^1} \\
  &=\left.\int_a^b dx (f+\eta)^2 \right|_{\eta^1} \\
  &=\left.\int_a^b dx f^2 + 2 \eta f + \eta^2 \right|_{\eta^1} \\
  &=\int_a^b dx \eta 2 f \\
  \Rightarrow& \frac{\delta I[f]}{\delta f} = 2f(x)
\end{align*}

\subsection*{Teil b}

\begin{align*}
  I[f] &= \int_a^b sin(f(x)) dx \\
  \delta I &= \left.I[f_0 - \eta]\right|_{\eta^1} \\
  &=\left.\int_a^b dx sin(f+\eta) \right|_{\eta^1}
\end{align*}

Hier entwickeln wir nun in eine Taylor-Reihe um freistehende $\eta^1$-Terme zu
erhalten. Wir benötigen nur den Term erster Ordnung, den Rest frisst unser
$\eta^1$-Operator sowieso, deswegen lassen wir die anderen auch gleich weg.

\begin{align*}
  &=\left.\int_a^b dx \dots + \frac{\left(\frac{\partial}{\partial \eta} \sin(f+\eta)\right)(\eta=0)}{1!} (\eta - 0)^1 + \dots\quad\right|_{\eta^1} \\
  &=\int_a^b dx \eta \cos(f) \\
  &\Rightarrow \frac{\delta I[f]}{\delta f} = \cos(f(x))
\end{align*}

\subsection*{Teil c}
\begin{align*}
  I[f] &= \int_a^b g(x)f'(x) dx \\
  \delta I &= \left.I[f_0 - \eta]\right|_{\eta^1} \\
  &=\left.\int_a^b dx g(x)(f' + \eta') \right|_{\eta^1}
\end{align*}

Taylor-Entwicklung ist hier nicht sinnvoll, man sieht direkt, welche Terme stehen bleiben:

\begin{align*}
  &=\int_a^b dx \eta' g(x)
\end{align*}

Nun integrieren wir partiell, um $\eta$-Terme zu erhalten:

\begin{align*}
  &=\left[\eta(x) g(x)\right]_{x=a}^{x=b} - \int_a^b dx \eta(x) g'(x)
\end{align*}

Die $\eta(x)$-Funktion verschwindet per definitionem in den Randbereichen, denn
es ist $\eta(a) = 0$ und $\eta(b) = 0$. Somit fällt der erste Term weg. Die
funktionale Ableitung lässt sich dann direkt ablesen:

\begin{align*}
  \frac{\delta I[f]}{\delta f} = g'(x)
\end{align*}

\subsection*{Teil d}
\begin{align*}
  I[f] &= \int_a^b f'(x)^2 dx \\
  \delta I &= \left.I[f_0 - \eta]\right|_{\eta^1} \\
  &=\left.\int_a^b dx (f' + \eta')^2 \right|_{\eta^1} \\
  &=\left.\int_a^b dx f'^2 + 2\eta'f' + \eta'^2 \right|_{\eta^1} \\
  &=\int_a^b dx 2\eta'f'
\end{align*}

Partielle Integration:

\begin{align*}
  &=\left[2\eta(x) f'(x)\right]_{x=a}^{x=b} - \int_a^b dx \eta(x) f''(x)
\end{align*}

Selbe Begründung wie oben: Der erste Term verschwindet, die funktionale Ableitung lässt sich ablesen:
\begin{align*}
  \frac{\delta I[f]}{\delta f} = f''(x)
\end{align*}

\subsection*{Teil e}
Da $\eta$ und $n$ schrecklich schwer außeinander zu halten sind,
definieren wir kurzerhand $a := n$.

\begin{align*}
  I[f] &= \int_a^b a(x,f(x)) \sqrt{1+f'(x)^2} dx \\
  \delta I &= \left.I[f_0 - \eta]\right|_{\eta^1} \\
  &=\left.\int_a^b dx \; a(x, f + \eta) \sqrt{1+(f'+\eta')^2} \right|_{\eta^1}
\end{align*}

Da hier nur Terme in mit $f$, $f'$ und $x$ vorkommen, bietet sich zur Lösung
die Euler-Lagrange-Gleichung an. Sei $L = a(x, f) \sqrt{1+f'^2}$

\begin{align*}
  \frac{d}{dx} \frac{\partial L}{\partial f'} &= \frac{d}{dx} \left(\underbrace{\frac{\partial}{\partial f'} (a(x, f))}_{=0} \sqrt{1+f'^2} + a(x,f) \frac{\partial}{\partial f'} \sqrt{1+f'^2} \right) \\
  &= \frac{d}{dx} \left(a(x,f) \frac{\partial}{\partial f'} \sqrt{1+f'^2} \right) \\
  &= \frac{d}{dx} \left(a(x,f) \frac{f'}{\sqrt{1+f'^2}} \right)
\end{align*}

\begin{align*}
  \frac{\partial L}{\partial f} &= \frac{\partial}{\partial f} \left(a(x, f)\right) \sqrt{1+f'^2} + a(x, f) \underbrace{\frac{\partial}{\partial f} \sqrt{1+f'^2}}_{=0} \\
  &= \frac{\partial a(x,f)}{\partial f} \sqrt{1+f'^2} \\
\end{align*}

Und damit ist:

\begin{align*}
  \frac{\delta I[f]}{\delta f} &= \frac{d}{dx} \frac{\partial L}{\partial f'} - \frac{\partial L}{\partial f} \\
  &= \frac{d}{dx} \left(a(x,f) \frac{f'}{\sqrt{1+f'^2}} \right) - \frac{\partial a(x,f)}{\partial f} \sqrt{1+f'^2}
\end{align*}

\section*{Aufgabe 4}
\subsection*{Teil a}

Ausgehend von der Lösung aus der vorherigen Aufgabe ergibt sich aus $a() = c$,
dass $\frac{\partial a}{\partial f} = 0$. Das Licht nimmt den kürzesten Weg,
deswegen muss das Funktional minimal sein und damit die Ableitung $=0$. Ob ein
Minium oder ein Maximum vorliegt, könnte man mithilfe der zweiten funktionalen
Ableitung prüfen, betrachten wir allerdings als sinnlos, da das Ergebnis mit
der Erwartung übereinstimmt.

\begin{align*}
  &\frac{d}{dx} \left(a(x,f) \frac{f'}{\sqrt{1+f'^2}} \right) - \frac{\partial a(x,f)}{\partial f} \sqrt{1+f'^2} \overset{!}{=} 0 \\
  \Leftrightarrow \quad &\frac{d}{dx} \left(a(x,f) \frac{f'}{\sqrt{1+f'^2}} \right) = \frac{\partial a(x,f)}{\partial f} \sqrt{1+f'^2}
\end{align*}

Einsetzen von $a = c$, $\frac{\partial a}{\partial f} = 0$:

\begin{align*}
  \Leftrightarrow \quad &\frac{d}{dx} \left(c \frac{f'}{\sqrt{1+f'^2}} \right) = 0 \\
  \Leftrightarrow \quad &c \frac{f''(x)}{\left(1 + f'(x)^2\right)^{\frac{3}{2}}} = 0
\end{align*}

Damit der Term null wird, muss über dem Bruchstrich eine $0$ stehen, unter dem
Bruchstrich hingegen darf keine stehen. Wir machen also den Ansatz $f(x) =
Ax+B$.

Diese Funktion beschreibt eine Gerade. Da Licht sich in Medien mit konstantem
Brechnungsindex immer auf einer Geraden ausbreitet, muss die Lösung stimmen.

\subsection*{Teil b}

Ausgehend von der Lösung aus Aufgabe 3e ergibt sich aus $a(x,y) = e^y$, dass
$\frac{\partial a}{\partial y} = e^y$. Eingesetzt führt dies zur
folgenden Differentialgleichung, die wir gleich null setzen, um das Minimum des
Funktionals zu finden.

\begin{align*}
  &\frac{d}{dx} \left(e^y \frac{f'}{\sqrt{1+f'^2}} \right) - e^y \sqrt{1+y'^2} \overset{!}{=} 0 \\
  \Leftrightarrow \quad &\frac{d}{dx} \left(e^y \frac{f'}{\sqrt{1+f'^2}} \right) = e^y \sqrt{1+y'^2} \\
  \Leftrightarrow \quad &\frac{e^y \left(y''(x) + y'(x)^4 + y'(x)^2\right)}{(y'(x)^2 + 1)^\frac{3}{2}} = e^y \sqrt{1+y'^2} \\
  \Leftrightarrow \quad &\frac{\left(y''(x) + y'(x)^4 + y'(x)^2\right)}{(y'(x)^2 + 1)^\frac{3}{2}} = \sqrt{1+y'^2} \\
  \Leftrightarrow \quad &y''(x) + y'(x)^4 + y'(x)^2 = \left(1+y'^2\right)^2 \\
  \Leftrightarrow \quad &y''(x) + y'(x)^4 + y'(x)^2 = 1+2y'^2+y'(x)^4 \\
  \Leftrightarrow \quad &y''(x) - y'(x)^2 = 1
\end{align*}

Diese Differentialgleichung ist zwar generell auch per Hand lösbar, jedoch
überlassen wir dies unter Anbetracht der bisherigen Schwierigkeit dieses
Aufgabenblattes doch lieber Wolfram Alpha und bitten um Verständnis.

Die Lösung lautet

\begin{align*}
  y(x) = A - \ln \cos\left(B + x\right)
\end{align*}

Diese Kurve wollen wir auch nicht weiter interpretieren.

\subsection*{Teil c}
Eine Fata Morgana ist eine Luftspiegelung. Verbunden damit ist für den
umherirrenden Wüstenforscher die Hoffnung, die rettende Oase zu erreichen,
bevor er verdurstet. Sollte der Leser Schwierigkeiten haben, sich in eine
solche Situation hineinzuversetzen, möge er sich vorstellen, er sei
Physikstudent im 2. Semester und müsste diesen Aufgabenzettel in einer Woche
bearbeiten.

\end{document}
