\documentclass[a4paper,german,12pt,smallheadings]{scrartcl}
\usepackage[T1]{fontenc}
\usepackage[utf8]{inputenc}
\usepackage{babel}
\usepackage{tikz}
\usepackage{geometry}
\usepackage{amsmath}
\usepackage{amssymb}
\usepackage{float}
%\usepackage{wrapfig}
\usepackage[thinspace,thinqspace,squaren,textstyle]{SIunits}
\restylefloat{table}
\geometry{a4paper, top=15mm, left=20mm, right=40mm, bottom=20mm, headsep=10mm, footskip=12mm}
\linespread{1.5}
\setlength\parindent{0pt}
\begin{document}
\begin{center}
\bfseries % Fettdruck einschalten
\sffamily % Serifenlose Schrift
\vspace{-40pt}
Analytische Mechanik, Sommersemester 2013, 3. Blatt \\
Luis Herrmann und Markus Fenske, Tutor: Clemens Meyer zu Rheda
\vspace{-10pt}
\end{center}
\section*{Aufgabe 1}
\subsection*{Teil a}

In der Aufgabe ist festgelegt, dass die $x$-Position des Keils durch die Lage
seiner Spitze festgelegt werden soll. Da der Keil auf einer ebenen Unterlage
ist, ist er in $y$-Richtung nicht beweglich (Zwangbedingung: $y=0$), so dass
sich die Position durch $x$ allein festlegen lässt. Von der Skizze des Blattes
weichen wir allerdings ab und spiegeln das Problem an der $y$-Achse, so dass
der Winkel zur anderen Seite offen ist. Des Ergebnis bleibt das selbe.

Der Klotz kann sich nur auf der Oberfläche des Keils entlang bewegen. Aufgrund
der Spiegelung ergibt sich dann folgende Gleichung, die auch eine
Zwangsbedingung darstellt.

\begin{align*}
  \vec{r} =  r \begin{pmatrix} \cos \alpha \\ \sin \alpha \end{pmatrix}
\end{align*}

Die unabhängigen Koordindaten lauten also $x$ und $r$.

\subsection*{Teil b}

Sei $T_x$ die kinetische Energie des Keils und $T_r$ die des Klotzes:

\begin{align*}
  T &= T_x + T_r \\
    &= \frac{m_1}{2}\dot{x}^2 + \frac{m_2}{2} \dot{r}^2
\end{align*}

Das Gravitationspotential ist $V(y) = mgy$. Sei $V_x$ die potentielle Energie
des Keils und $V_r$ die des Klotzes. $V_x = 0$, denn die $y$-Koordinate des
Klotzes verschwindet aufgrund der Zwangsbedingung.

\begin{align*}
  V &= V_x + V_r \\
    &= 0 + m_2gr \sin \alpha
\end{align*}

Für die Lagrange-Funktion $L = T - V$ ergibt sich dann

\begin{align*}
  L &= T - V \\
    &= \frac{m_1}{2}\dot{x}^2 + \frac{m_2}{2} \dot{r}^2 - m_2gr \sin \alpha
\end{align*}

\subsection*{Teil c}

Die Bewegungsgleichungen ergeben sich durch Anwendung der Euler-Lagrange-Gleichung:

\begin{align*}
  \left(\frac{d}{dt}\frac{\partial}{\partial \dot{q_i}} - \frac{\partial}{\partial q_i}\right) L = 0
\end{align*}

Für die $x$-Koordinate ist:

\begin{align*}
  \frac{d}{dt}\frac{\partial L}{\partial \dot{x}} = \frac{d}{dt} m_1\dot{x} = m_1\ddot{x} \\
  \frac{\partial L}{\partial x} = 0 \\
  \Rightarrow m_1\ddot{x} = 0
\end{align*}

\textbf{\dots und das ist komisch. Der Keil bewegt sich nicht?}

\end{document}
