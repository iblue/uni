\documentclass[a4paper,german,12pt,smallheadings]{scrartcl}
\usepackage[T1]{fontenc}
\usepackage[utf8]{inputenc}
\usepackage{babel}
\usepackage{tikz}
\usepackage{geometry}
\usepackage{amsmath}
\usepackage{amssymb}
\usepackage{float}
\usepackage{cancel}
%\usepackage{wrapfig}
\usepackage[thinspace,thinqspace,squaren,textstyle]{SIunits}
\restylefloat{table}
\geometry{a4paper, top=15mm, left=20mm, right=40mm, bottom=20mm, headsep=10mm, footskip=12mm}
\linespread{1.5}
\setlength\parindent{0pt}
\begin{document}
\begin{center}
\bfseries % Fettdruck einschalten
\sffamily % Serifenlose Schrift
\vspace{-40pt}
Elektrodynamik und Optik, Sommersemester 2013, 9. Blatt \\
Markus Fenske, Tutor: Dr. Marko Wietstruk
\vspace{-10pt}
\end{center}
\section*{Aufgabe 1: Wirbelstrom}
Wenn sich die Scheibe durch das Magnetfeld bewegt, werden in der Scheibe
Wirbelströme induziert. Diese erzeugen ein Magnetfeld, das widerrum gemäß
Lenzscher Regel dem ursprünglichen Magnetfeld entgegengerichtet sind. Die
Scheibe wird dadurch in ihrem Fall gebremst.

\section*{Aufgabe 2: Maxwellgleichungen}

Das Gaußsche Gesetz des elektrischen Feldes lautet:

\begin{align*}
  \oint \vec{E} \cdot d\vec{A} = \frac{1}{\epsilon_0} \int \rho\;dV \qquad\text{bzw.}\qquad \nabla \vec{E} = \frac{\rho}{\epsilon_0}
\end{align*}

Das Volumen $V$ habe die geschlossene Oberfläche $A$. Der elektrische Fluss
durch die Oberfläche ist proportional zu der im Volumen $V$ eingeschlossenen
elektrischen Ladung mit der Ladungsdichte $\rho$. Die elektrische Ladung ist
Quelle es Quelle des elektrischen Feldes.

Das Gaußsche Gesetz des Magnetfeldes lautet:

\begin{align*}
  \oint \vec{B} \cdot d\vec{A} = 0 \qquad\text{bzw.}\qquad \nabla \vec{B} = 0
\end{align*}

Der magnetische Fluss durch die Oberfläche $A$ verschwindet. Das Magnetfeld
wird als quellenfrei angenommen, es gäbe keine magnetischen Monopole.
Tatsächlich wurden magnetische Monopole bis jetzt nicht beobachtet.

Das Faradaysche Gesetz (Induktionsgesetz) lautet:

\begin{align*}
  \oint \vec{E} \cdot d\vec{s} = -\frac{d}{dt} \int \vec{B} \cdot d\vec{A} \qquad\text{bzw.}\qquad \nabla \times \vec{E} = -\frac{d}{dt} \vec{B}
\end{align*}

Ein sich zeitlich änderndes Magnetfeld induziert ein elektrisches Feld. Wenn
die Fläche $A$ von einem sich zeitlich ändernden Magnetfeld $\dot{\vec{B}}$
durchstoßen wird, erzeugt dies auf dem Rand dieser Fläche (charakterisiert
durch das tangentiale Linienelement $d\vec{s}$ ein umlaufendes elektrisches
Feld $\vec{E}$.

Das Ampere-Maxwell-Gesetz lautet:

\begin{align*}
  \oint \vec{B} \cdot d\vec{s} = \mu_0 I + \mu_i \epsilon_0 \frac{d\Phi_{E}}{dt} \qquad\text{bzw.}\qquad \nabla \times B = \mu_0 \vec{j} + \mu_0 \epsilon_0 \frac{d}{dt} \vec{E}
\end{align*}

Ein durch eine Fläche laufender Strom $I$ erzeugt, genau wie ein sich zeitlich
änderndes elektrisches Feld $\dot{\Phi_{E}}$ ein auf der Randkurve, die
charakterisiert ist durch das tangentiale Linienelement $d\vec{s}$, ein
Magnetfeld $\vec{B}$.

\subsection*{Teil b}

Im elektrostatischen Fall ist $\vec{E}$ ein konservatives Feld. Es kann dann
also als Gradient des Potentials $\Phi_{E}$ geschrieben werden:

\begin{align*}
  \vec{E}(\vec{r}) = -\nabla \Phi_{E}(\vec{r}) \\
  \Leftrightarrow\quad&\nabla \vec{E}(\vec{r}) = -\Delta \Phi_{E}(\vec{r}) \\
\end{align*}

Die erste Maxwell-Gleichung lautet:

\begin{align*}
  &\nabla \vec{E} = \frac{\rho}{\epsilon_0}
\end{align*}

Wenn man diese einsetzt erhält man die Poisson-Gleichung:

\begin{align*}
  \Delta \Phi_{E}(\vec{r}) = -\frac{\rho(\vec{r}}{\epsilon_0}
\end{align*}

Für den Fall, dass die Quellendichte $\rho = 0$ ist, erhält man die
Laplace-Gleichung:

\begin{align*}
  \Delta \Phi_{E}(\vec{r}) = 0
\end{align*}

\subsection*{Teil c}
Im statischen Fall, also $\frac{d}{dt} \vec{E} = \frac{d}{dt} \vec{B} = 0$,
erhält man dies aus den Maxwell-Gleichungen. Für das elektrische Feld:

\begin{align*}
  &\nabla \vec{E} = \frac{\rho}{\epsilon_0} \\
  &\nabla \times \vec{E} = 0
\end{align*}

Das elektrische Feld wird also zum elektrostatischen konservativen
Potentialfeld, das nicht an das Magnetfeld gekoppelt ist.

Für das Magnetische Feld gilt dann:
\begin{align*}
  &\nabla \vec{B} = 0 \\
  &\nabla \times \vec{B} = \mu_0 \vec{j}
\end{align*}

Das magnetische Feld hängt also nicht mehr mit dem elektrischen Feld zusammen.

\subsection*{Teil d}

Der maxwellsche Verschiebungsstrom löst das Paradoxon, dass durch die Anwendung
des Ampereschen Gesetzes an einem ladenden Kondensator auftritt.

Um zu beweisen, dass die Kontinuitätsgleichung gilt, benutzen wir das
Ampere-Maxwell-Gesetz:

\begin{align*}
  \nabla \times B = \mu_0 \vec{j} + \mu_0 \epsilon_0 \frac{d}{dt} \vec{E} \\
  \nabla(\nabla \times B) = \nabla \mu_0 \vec{j} + \nabla \mu_0 \epsilon_0 \frac{d}{dt} \vec{E} \\
\end{align*}

Für jeden beliebigen Vektor $\vec{a}$ verschwindet $\nabla(\nabla \times \vec{a})$, also:

\begin{align*}
  \nabla \mu_0 \vec{j} + \nabla \mu_0 \epsilon_0 \frac{d}{dt} \vec{E} = 0
\end{align*}

Aus dem ersten Maxwell-Gesetz erhalten wir $\rho = \epsilon_0 \nabla \vec{E}$.
Eingesetzt:

\begin{align*}
  \nabla \mu_0 \vec{j} + \mu_0 \frac{d}{dt} \rho = 0
\end{align*}

Wir kürzen $\mu_0$ und erhalten die Kontinuitätsgleichung:

\begin{align*}
  \frac{d \rho}{d t} + \nabla \vec{j} = 0
\end{align*}

\end{document}
