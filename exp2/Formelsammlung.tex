\documentclass[a4paper,german,12pt,smallheadings]{scrartcl}
\usepackage[T1]{fontenc}
\usepackage[utf8]{inputenc}
\usepackage{babel}
\usepackage{tikz}
\usepackage{geometry}
\usepackage{amsmath}
\usepackage{amssymb}
\usepackage{float}
\usepackage{cancel}
%\usepackage{wrapfig}
\usepackage[thinspace,thinqspace,squaren,textstyle]{SIunits}
\restylefloat{table}
\geometry{a4paper, top=15mm, left=20mm, right=40mm, bottom=20mm, headsep=10mm, footskip=12mm}
\linespread{1.5}
\setlength\parindent{0pt}
\begin{document}
\begin{center}
\bfseries % Fettdruck einschalten
\sffamily % Serifenlose Schrift
\vspace{-40pt}
Elektrodynamik und Optik, Sommersemester 2013, Formelsammlung \\
Markus Fenske
\vspace{-10pt}
\end{center}

\section{Einleitung}

Zur Klausur darf ein handgeschriebenes Blatt DIN-A4 (Vorder- und Rückseite)
mitgebracht werden. Dieses Blatt dient als Anhaltspunkt, was darauf sollte. Es
orientiert sich an den Vorlesungsunterlagen und den Übungsaufgaben und ist grob
chronologisch sortiert.

\end{document}
