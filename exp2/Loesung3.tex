\documentclass[a4paper,german,12pt,smallheadings]{scrartcl}
\usepackage[T1]{fontenc}
\usepackage[utf8]{inputenc}
\usepackage{babel}
\usepackage{tikz}
\usepackage{geometry}
\usepackage{amsmath}
\usepackage{amssymb}
\usepackage{float}
%\usepackage{wrapfig}
\usepackage[thinspace,thinqspace,squaren,textstyle]{SIunits}
\restylefloat{table}
\geometry{a4paper, top=15mm, left=20mm, right=40mm, bottom=20mm, headsep=10mm, footskip=12mm}
\linespread{1.5}
\setlength\parindent{0pt}
\begin{document}
\begin{center}
\bfseries % Fettdruck einschalten
\sffamily % Serifenlose Schrift
\vspace{-40pt}
Elektrodynamik und Optik, Sommersemester 2013, 3. Blatt \\
Markus Fenske, Tutor: Dr. Marko Wietstruk
\vspace{-10pt}
\end{center}
\section*{Aufgabe 1}

\subsection*{Teil a}
Die elektrische Verschiebungsdichte existiert überall und ist konstant. Damit
ist klar, dass es egal ist, ob die Platte mittig oder an der Seite eingeschoben
wird. Ich gehe von der Seite aus. Damit ergibt sich für die Teilspannungen in
Glas ($U_G$) und in Luft($U_L$):

\begin{align*}
  U &= U_G + U_L \\
    &= E_G \cdot d + E_L \cdot (a-d) \\
    &= \frac{D}{\epsilon_r \epsilon_0} d + \frac{D}{\epsilon_0} (a-d) \\
    &= \frac{E_G}{\epsilon_r} d + E_L \cdot (a-d)
\end{align*}

Die Teilspannung in Luft ist:
\begin{align*}
  U_L &= E_L \cdot (a-d) \\
      &= \frac{U (a-d)A}{(a-d) + \frac{d}{\epsilon_r}} \\
      &= \frac{2500 \volt \cdot 0{,}0055 \meter}{0{,}0055 \meter \cdot \frac{0{,}0025 \meter}{7{,} 5}}
      &\approx 2357 \volt
\end{align*}

Die Teilspannung in Glas muss also entsprechend sein:

\begin{align*}
  U_G = U_0 - U_L \approx 143 \volt
\end{align*}

\subsection*{Teil b}

Die Feldstärke ist wegen der Homogenität

\begin{align*}
  E_L = \frac{U}{(a-d) + \frac{d}{\epsilon_r}} \approx 429 \kilo\volt\per\meter \\
  E_G = \frac{E}{\epsilon_r} \approx 57 \kilo\volt\per\meter
\end{align*}

\subsection*{Teil c}
Der Kondensator entspricht einer Reihenschaltung, womit:

\begin{align*}
  C_{ges} &= \frac{1}{\frac{1}{C_L} + \frac{1}{C_G}} \\
  &= \frac{1}{\frac{d}{A\epsilon_0 \epsilon_r} + \frac{(d-a)}{A\epsilon_0}} \\
  &\approx 1{,}52\;\pico\farad
\end{align*}

\subsection*{Teil d}
Nicht bearbeitet.

\subsection*{Teil e}
\begin{align*}
  w = \frac{W}{V} = \frac{\frac{1}{2} C U^2}{Ad} \approx 5938\;\milli\joule\per\meter^3
\end{align*}

\subsection*{Teil f}
Nicht bearbeitet.


\section*{Aufgabe 2}
\subsection*{Teil a}

Für die Energie im Kondensator gilt:

\begin{align*}
  E = \frac{1}{2} C U^2
\end{align*}

Die Kapazität steigt linear mit steigender Dielektrizitätskonstante. Dementsprechend ist also

\begin{align*}
  C_{\text{neu}} = \epsilon_r C_\text{alt}
\end{align*}

Einsetzen und umstellen:

\begin{align*}
  &E = \frac{1}{2} \epsilon_r C_\text{alt} U^2 \\
  \Leftrightarrow \quad &\epsilon_r = \frac{2E}{C_\text{alt} U^2}
\end{align*}

Durch Einsetzen der gegebenen Werte erhält man:

\begin{align*}
  \epsilon_r = \frac{2 \cdot 10 \cdot 10^{-6}\;\joule}{10 \cdot 10^{-12}\; \farad \left(700\; \volt\right)^2} \approx \underline{4{,}08\; \farad\per\meter}
\end{align*}

Diese Dielekrizitätszahl sollte ein Material haben, damit der Kondensator die
angegebene Energiemenge speichern kann.

\subsection*{Teil b}

Für die Kapazität im Plattenkondensator gilt

\begin{align*}
  C = \epsilon_0 \epsilon_r \frac{A}{d}
\end{align*}

Für die Energie im Kondensator:

\begin{align*}
  E = \frac{1}{2} C U^2
\end{align*}


Setzte man die Kapazität ein, erhält man also:

\begin{align*}
  E = \frac{1}{2} \epsilon_0 \epsilon_r \frac{A}{d}  U^2
\end{align*}

Entfernt man die Platte, ist $\epsilon_r = 1$. Das bedeutet:

\begin{align*}
  \Delta E &= E_\text{vorher} - E_\text{nachher}  \\ 
           &= \frac{1}{2} \epsilon_0 \epsilon_r \frac{A}{d}  U^2 - \frac{1}{2} \epsilon_0 \frac{A}{d}  U^2 \\
           &= \frac{1}{2} \epsilon_0 (\epsilon_r - 1) \frac{A}{d}  U^2
\end{align*}

Es ist also Energie nötig, um die Platte zu entfernen. Diese muss mechanisch
aufgebracht werden, also ist $W = \Delta E$. Einsetzen und ausrechnen:

\begin{align*}
  W = \frac{1}{2} 8{,}85 \cdot 10^{-12} \;\farad\per\meter (5-1) \frac{1\;\meter^2}{5\cdot 10^{-3}\;\meter} \cdot 12^2 \volt^2 \approx \underline{4{,}25\;\joule}
\end{align*}

\section*{Aufgabe 3}

Das elektrische Feld einer Flächenladung ist homogen und bestimmt sich durch

\begin{align*}
  E = \frac{\sigma}{2 \epsilon_0} \\
\end{align*}

Die Flächenladungsdichte $\sigma$ ist dabei:

\begin{align*}
  \sigma = \frac{Q}{A}
\end{align*}

Die Fläche des Bandes, die in einer gegebenen Zeit $t$ an der Elektrode
vorbeiläuft ist ($b$ sei die Breite des Bandes):

\begin{align*}
  A = vt \cdot b
\end{align*}

Die zeitlich konstante Stromstärke ergibt sich durch:
\begin{align*}
  &I = \frac{Q}{t} \\
  \Leftrightarrow \quad &Q = It
\end{align*}

Einsetzen in die Flächenladungsdichte $\sigma$:

\begin{align*}
  \sigma = \frac{Q}{A} = \frac{It}{A}
\end{align*}

Einsetzen der Fläche $A$ in $\sigma$:

\begin{align*}
  \sigma = \frac{It}{vtb} = \frac{I}{vb} 
\end{align*}

Einsetzen von $\sigma$ in $E$:

\begin{align*}
  &E = \frac{\frac{I}{vb}}{2 \epsilon_0} \\
  &\quad = \frac{I}{2 \epsilon_0 vb} \\
  &\Leftrightarrow I = E 2\epsilon_0 vb
\end{align*}

Einsetzen und ausrechnen:
\begin{align*}
  I = E 2 \epsilon_0 v b = 3000\;\volt\per\meter \cdot 2 \cdot 8{,}85\cdot10^{-12}\;\farad\per\meter \cdot 20\;\meter\per\second \cdot 0{,}1\;\meter \approx \underline{1{,}06 \cdot 10^{-7}\;\ampere}
\end{align*}

\section*{Aufgabe 4}

Nicht bearbeitet.


\end{document}
