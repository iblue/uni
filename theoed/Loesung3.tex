\documentclass[a4paper,german,12pt,smallheadings]{scrartcl}
\usepackage[T1]{fontenc}
\usepackage[utf8]{inputenc}
\usepackage{babel}
\usepackage{geometry}
\usepackage[fleqn]{amsmath}
\usepackage{amssymb}
\usepackage{float}
\usepackage{enumerate}
\usepackage{commath} % http://tex.stackexchange.com/questions/14821/whats-the-proper-way-to-typeset-a-differential-operator
\usepackage{cancel}

% Number only referenced equations
\usepackage[fleqn]{mathtools}
\mathtoolsset{showonlyrefs}

%\usepackage{wrapfig}
\usepackage[thinspace,thinqspace,squaren,textstyle]{SIunits}

% New command for color underlining
\usepackage{xcolor}

\newsavebox\MBox
\newcommand\colul[2][red]{{\sbox\MBox{$#2$}%
  \rlap{\usebox\MBox}\color{#1}\rule[-1.2\dp\MBox]{\wd\MBox}{0.5pt}}}

\restylefloat{table}
\geometry{a4paper, top=15mm, left=10mm, right=20mm, bottom=20mm, headsep=10mm, footskip=12mm}
\linespread{1.5}
\setlength\parindent{0pt}
\DeclareMathOperator{\Tr}{Tr}
\DeclareMathOperator{\Var}{Var}
\begin{document}
\allowdisplaybreaks % Seitenumbrüche in Formeln erlauben
\begin{center}
\bfseries % Fettdruck einschalten
\sffamily % Serifenlose Schrift
\vspace{-40pt}
Theoretische Elektrodynamik, Sommersemester 2014, Aufgabenblatt 3

Markus Fenske, Mattis Riediger, Tutor: Clemens Meyer zu Rheda
\vspace{-10pt}
\end{center}

\section*{Aufgabe 1: Delta-Funktion I}

Die $\delta$-Funktion verschwindet an allen Stellen an denen das Argument $\neq
0$ ist. Seien $x_i$ die Nullstellen der Funktion $f(x)$, dann müssen wir also
nur alle $\epsilon$-Umgebungen um die Nullstellen betrachten.

\begin{equation}
  \int\limits_{-\infty}^\infty \dif x \; \delta(f(x)) g(x) = \sum_{i} \int\limits_{x_i - \epsilon}^{x_i + \epsilon} \dif x \; \delta(f(x)) g(x)
\end{equation}

Wir können den Integrationsbereich auch so verschieben, dass nur genau um die
Nullstellen integriert wird. Das geschieht durch Transformation $v = x - x_i$,
$\dif v = \dif x$.

\begin{equation}
  = \sum_{i} \int\limits_{ -\epsilon}^{+\epsilon} \dif v \; \delta(f(v + x_i) g(v + x_i)
\end{equation}

An diesen Stellen können wir $f(v + x_i)$ in eine Taylor-Reihe entwickeln: $f(v +
x_i) = f(x_i) + f'(x_i) v + \mathcal{O}(v^2)$, wobei $f(x_i)$ eine Nullstelle
ist, also $f(x_i + v) = f'(x_i) v + \mathcal{O}(v^2)$. Der Quadratische Term
fällt natürlich in beliebig kleinen $\epsilon$-Umgebungen nicht ins Gewicht,
kann also weggelassen werden.

\begin{equation}
  = \sum_{i} \int\limits_{- \epsilon}^{+ \epsilon} \dif v \; \delta(f'(x_i) v) g(v + x_i)
\end{equation}

Wir wissen außerdem, dass $\delta(ax) = \frac{1}{\envert{a}} \delta(x)$.

\begin{equation}
  = \sum_{i} \int\limits_{- \epsilon}^{+ \epsilon} \dif v \; \frac{1}{\envert{f'(x_i)}} \delta(v) g(v + x_i)
\end{equation}

Nun machen wir die Verschiebung rückgängig, transformieren also $x = v + x_i$:

\begin{equation}
  = \sum_{i} \int\limits_{x_i - \epsilon}^{x_i + \epsilon} \dif x \; \frac{1}{\envert{f'(x_i)}} \delta(x - x_i) g(x)
\end{equation}

Der Integrationsbereich ist natürlich beliebig, da alle Werte, außer die
Nullstellen, verschwinden (\textbf{Achtung}: Das funktioniert natürlich nicht,
wenn doppelte Nullstellen vorhanden sind).


\begin{equation}
  = \sum_{i} \int\limits_{- \infty}^{+ \infty} \dif x \; \frac{1}{\envert{f'(x_i)}} \delta(x - x_i) g(x)
\end{equation}

Die Summe von Integralen ist das Integral der Summe.

\begin{equation}
  = \int\limits_{- \infty}^{+ \infty} \dif x \; \sum_i \frac{1}{\envert{f'(x_i}} \delta(x - x_i) g(x)
\end{equation}

Das ist, was zu beweisen war.

\section*{Aufgabe 2: Delta-Funktion II}
\begin{enumerate}[a)]
  \item
    \begin{equation}
      \lim_{\epsilon \to 0} \int\limits_{-\infty}^{\infty} \dif x \; \delta_\epsilon(x - x_0) h(x)
      =
      \lim_{\epsilon \to 0} \int\limits_{-\infty}^{\infty} \dif x \; \frac{1}{\pi} \frac{\epsilon}{(x - x_0)^2 + \epsilon^2} h(x)
    \end{equation}

    Wir transformieren $u = x - x_0$, $\dif u = \dif x$:
    \begin{align}
      &= \lim_{\epsilon \to 0} \int\limits_{-\infty}^{\infty} \dif u \; \frac{1}{\pi} \frac{\epsilon}{u^2 + \epsilon^2} h(u + x_0) \\
      &= \lim_{\epsilon \to 0} \int\limits_{-\infty}^{\infty} \dif u \; \frac{1}{\pi} \frac{\epsilon}{\epsilon^2 \del{\frac{u^2}{\epsilon^2} + 1}} h(u + x_0) \\
    \end{align}

    Wir transformieren $v = \frac{u}{\epsilon}$, $\dif v = \frac{1}{\epsilon} \dif u \Leftrightarrow \dif u = \epsilon \dif v$
    \begin{equation}
      = \lim_{\epsilon \to 0} \int\limits_{-\infty}^{\infty} \dif v \; \frac{1}{\pi} \frac{1}{v^2 + 1} h(\epsilon v + x_0)
    \end{equation}

    $h(x)$ ist stetig und beschränkt (laut Aufgabenstellung). Daraus folgt,
    dass es für $h(\epsilon v + x_0)$ im gesamten Definitionsbereich eine
    integrierbare Majorante gibt (Beweis trivial). Nach dem Satz über die
    majorisierte Konvergenz dürfen deshalb Grenzprozess und Integralbildung
    vertauscht werden. (Lutz Heindorf, Skript Analysis I, S. 125, Abschnitt
    9.2.3).

    \begin{align}
      &= \int\limits_{-\infty}^{\infty} \dif v \; \frac{1}{\pi} \frac{1}{v^2 + 1} \lim_{\epsilon \to 0} h(\epsilon v + x_0) \\
      &= \int\limits_{-\infty}^{\infty} \dif v \; \frac{1}{\pi} \frac{1}{v^2 + 1} h(x_0) \\
      &= \frac{h(x_0)}{\pi} \int\limits_{-\infty}^{\infty} \dif v \; \frac{1}{v^2 + 1}
    \end{align}

    Der Wert des Integrals ist bekannt als $\pi$. Somit
    \begin{align}
      &= h(x_0)
    \end{align}

    Was zu beweisen war.

  \item
    Sei $f(x) = x^2 - 3x + 2$. Dann sind die Nullstellen $x_1 = 1$, $x_2 = 2$,
    die Ableitung ist $f'(x) = 2x - 3$ und die Werte $f'(x_1) = -1$, $f'(x_2) =
    1$. Zusammen mit dem in Aufgabe 1 bewiesenen Satz ergibt sich:

    \begin{equation}
      \int\limits_{-\infty}^{+\infty} \dif x \; x^3 \delta(x^2 - 3x + 1)
      =
      \int\limits_{-\infty}^{+\infty} \dif x \; x^3 \del{
        \frac{1}{\envert{-1}} \delta(x - 1) +
        \frac{1}{\envert{1}} \delta(x - 2)
      }
      = 1^3 + 2^3 = 9
    \end{equation}

  \item
    Partielle Integration liefert
    \begin{equation}
      \int\limits_{-\infty}^{\infty} \dif x \; \ln(x) \delta'(x-a) = \sbr[2]{\ln(x) \delta(x-a)}_{-\infty}^{\infty} - \int\limits_{-\infty}^{\infty} \dif x \; \frac{1}{x} \delta(x-a) = -\frac{1}{a}, \; a \neq 0
    \end{equation}
\end{enumerate}
\end{document}
