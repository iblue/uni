\documentclass[a4paper,german,12pt,smallheadings]{scrartcl}
\usepackage[T1]{fontenc}
\usepackage[utf8]{inputenc}
\usepackage{babel}
\usepackage{geometry}
\usepackage[fleqn]{mathtools} % also includes mathclap
\usepackage[fleqn]{amsmath}
\usepackage{amssymb}
\usepackage{float}
\usepackage{enumerate}
\usepackage{commath} % http://tex.stackexchange.com/questions/14821/whats-the-proper-way-to-typeset-a-differential-operator
\usepackage{cancel}

% Number only referenced equations
\mathtoolsset{showonlyrefs}

%\usepackage{wrapfig}
\usepackage[thinspace,thinqspace,squaren,textstyle]{SIunits}

% New command for color underlining
\usepackage{xcolor}

\newsavebox\MBox
\newcommand\colul[2][red]{{\sbox\MBox{$#2$}%
  \rlap{\usebox\MBox}\color{#1}\rule[-1.2\dp\MBox]{\wd\MBox}{0.5pt}}}

\restylefloat{table}
\geometry{a4paper, top=15mm, left=10mm, right=20mm, bottom=20mm, headsep=10mm, footskip=12mm}
\linespread{1.2}
\setlength\parindent{0pt}
\DeclareMathOperator{\Tr}{Tr}
\DeclareMathOperator{\Var}{Var}
\newcommand*\laplace{\mathop{}\!\mathbin\Delta}
\begin{document}
\allowdisplaybreaks % Seitenumbrüche in Formeln erlauben
\begin{center}
\bfseries % Fettdruck einschalten
\sffamily % Serifenlose Schrift
\vspace{-40pt}
Theoretische Elektrodynamik, Sommersemester 2014, Aufgabenblatt 9

Markus Fenske, Mattis Riediger, Tutor: Clemens Meyer zu Rheda
\vspace{-10pt}
\end{center}

\section*{Aufgabe 1: Rotierende Kugel}
\begin{enumerate}[a)]
\item
Wir nutzen Zylinderkoordinaten.

Die Ladung $q$ verteilt sich auf der Kugeloberfläche $A = 4 \pi R^2$, daraus
folgt die Ladungsdichte $\rho = \dfrac{q}{4 \pi R^2}$.

Sei o.B.d.A. die Drehachse $\vec{\omega} = \omega \hat{e}_z$, dann ist am Ort
$\vec{r} = r\hat{e}_r + z \hat{e}_z$ der Geschwindigkeitsvektor $\vec{\omega}
\times \vec{r} = \omega r \hat{e}_\phi$.

In Zylinderkoordinaten ist die Stromdichte dann also
\begin{equation}
  \vec{j} =  \frac{q}{4 \pi R^2} \delta\del{r^2 + z^2 - R^2} \omega r \hat{e}_\phi
\end{equation}
\end{enumerate}
\end{document}
