\documentclass[a4paper,german,12pt,smallheadings]{scrartcl}
\usepackage[T1]{fontenc}
\usepackage[utf8]{inputenc}
\usepackage{babel}
\usepackage{geometry}
\usepackage[fleqn]{mathtools} % also includes mathclap
\usepackage[fleqn]{amsmath}
\usepackage{amssymb}
\usepackage{float}
\usepackage{enumerate}
\usepackage{commath} % http://tex.stackexchange.com/questions/14821/whats-the-proper-way-to-typeset-a-differential-operator
\usepackage{cancel}

% Number only referenced equations
\mathtoolsset{showonlyrefs}

%\usepackage{wrapfig}
\usepackage[thinspace,thinqspace,squaren,textstyle]{SIunits}

% New command for color underlining
\usepackage{xcolor}

\newsavebox\MBox
\newcommand\colul[2][red]{{\sbox\MBox{$#2$}%
  \rlap{\usebox\MBox}\color{#1}\rule[-1.2\dp\MBox]{\wd\MBox}{0.5pt}}}

\restylefloat{table}
\geometry{a4paper, top=15mm, left=10mm, right=20mm, bottom=20mm, headsep=10mm, footskip=12mm}
\linespread{1.2}
\setlength\parindent{0pt}
\DeclareMathOperator{\Tr}{Tr}
\DeclareMathOperator{\Var}{Var}
\newcommand*\laplace{\mathop{}\!\mathbin\Delta}
\begin{document}
\allowdisplaybreaks % Seitenumbrüche in Formeln erlauben
\begin{center}
\bfseries % Fettdruck einschalten
\sffamily % Serifenlose Schrift
\vspace{-40pt}
Theoretische Elektrodynamik, Sommersemester 2014, Aufgabenblatt 9

Markus Fenske, Mattis Riediger, Tutor: Clemens Meyer zu Rheda
\vspace{-10pt}
\end{center}

\section*{Aufgabe 1: Rotierende Kugel}
\begin{enumerate}[a)]
\item
\begin{equation}
  \vec{j}(\vec{r}) =  \frac{q}{4 \pi R^2} \delta\del{\envert{\vec{r}} - R} \vec{\omega} \times \vec{r}
\end{equation}
\item
  \begin{align}
    \vec{A}(\vec{r})
    &= \frac{\mu_0}{4 \pi} \int \dif V' \; \frac{\vec{j}(\vec{r}')}{\envert{\vec{r}' - \vec{r}}} \\
    &= \frac{\mu_0}{4 \pi} \frac{q}{4 \pi R^2} \int \dif V' \;
    \frac{\vec{\omega} \times \vec{r}'}{\envert{\vec{r}' - \vec{r}}}
    \delta\del{r' - R} \\
    &= \frac{q \mu_0}{16 \pi^2 R^2} \vec{\omega} \times \int \dif V' \;
    \frac{\vec{r}'}{\envert{\vec{r}' - \vec{r}}}
    \delta\del{r' - R} \\
  \end{align}

  Wir werten das Integral in Kugelkoordinaten aus. Dazu setzen wir o.B.d.A.
  $\vec{r}$ entlang der $z$-Achse fest.
  Dann ist der Winkel zwischen $\vec{r}$ und $\vec{r}'$ genau der Winkel
  $\theta$ aus den Kugelkoordinaten.  Über den Kosinus-Satz ist
  $\envert{\vec{r}' - \vec{r}} = \sqrt{r'^2 + r^2 - 2 r' r \cos \theta}$.
  Außerdem ist der Integrationsvektor in Kugelkoordinaten $\vec{r}' = r' (\sin
  \theta' \cos \phi', \sin \theta' \sin \phi', \cos \phi')$.

  Die Delta-Funktion ersetzen wir direkt und erhalten die Komponenten des
  Integrals.

  Die $x$-Komponente verschwindet, wenn man zuerst das Integral nach $\phi'$
  ausführt:

  \begin{align}
    \int\limits_0^\pi \dif \theta'
    \int\limits_0^{2 \pi} \dif \phi' \;
    \frac{R \sin \theta' \cos \phi'}{\sqrt{R^2 + r^2 - 2Rr \cos \theta'}} R^2 \sin \theta' = 0
  \end{align}

  $y$-Komponente analog:

  \begin{align}
    \int\limits_0^\pi \dif \theta'
    \int\limits_0^{2 \pi} \dif \phi' \;
    \frac{R \sin \theta' \sin \phi'}{\sqrt{R^2 + r^2 - 2Rr \cos \theta'}} R^2 \sin \theta' = 0
  \end{align}

  $z$-Komponente (letztes Integral gelöst mit Mathematica mit der Annahme $r >
  R$):
  \begin{align}
    &\int\limits_0^\pi \dif \theta'
    \int\limits_0^{2 \pi} \dif \phi' \;
    \frac{R \cos \theta'}{\sqrt{R^2 + r^2 - 2Rr \cos \theta'}} R^2 \sin \theta'
    =
    &2 \pi R^3
    \int\limits_0^\pi \dif \theta'
    \frac{\cos \theta' \sin \theta'}{\sqrt{R^2 + r^2 - 2Rr \cos \theta'}}
    = 2 \pi R^3 \frac{2R}{3 r^2}
  \end{align}

  Zusammengfasst und eingesetzt:
  \begin{equation}
    \vec{A}(\vec{r})
    = \frac{q \mu_0}{16 \pi^2 R^2} \vec{\omega} \times \frac{4 \pi R^4}{3 r^2} \hat{e}_z
    = \frac{q \mu_0 R^2}{12 \pi r^2} \vec{\omega} \times \hat{e}_z
  \end{equation}

  Wegen $\vec{r} = r \hat{e}_z$ ergibt sich insgesamt:
  \begin{equation}
    \vec{A}(\vec{r}) = \frac{q \mu_0 R^2}{12 \pi r^3} \vec{\omega} \times \vec{r}
  \end{equation}


\end{enumerate}
\end{document}
