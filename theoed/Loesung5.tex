\documentclass[a4paper,german,12pt,smallheadings]{scrartcl}
\usepackage[T1]{fontenc}
\usepackage[utf8]{inputenc}
\usepackage{babel}
\usepackage{geometry}
\usepackage[fleqn]{mathtools} % also includes mathclap
\usepackage[fleqn]{amsmath}
\usepackage{amssymb}
\usepackage{float}
\usepackage{enumerate}
\usepackage{commath} % http://tex.stackexchange.com/questions/14821/whats-the-proper-way-to-typeset-a-differential-operator
\usepackage{cancel}

% Number only referenced equations
\mathtoolsset{showonlyrefs}

%\usepackage{wrapfig}
\usepackage[thinspace,thinqspace,squaren,textstyle]{SIunits}

% New command for color underlining
\usepackage{xcolor}

\newsavebox\MBox
\newcommand\colul[2][red]{{\sbox\MBox{$#2$}%
  \rlap{\usebox\MBox}\color{#1}\rule[-1.2\dp\MBox]{\wd\MBox}{0.5pt}}}

\restylefloat{table}
\geometry{a4paper, top=15mm, left=10mm, right=20mm, bottom=20mm, headsep=10mm, footskip=12mm}
\linespread{1.2}
\setlength\parindent{0pt}
\DeclareMathOperator{\Tr}{Tr}
\DeclareMathOperator{\Var}{Var}
\begin{document}
\allowdisplaybreaks % Seitenumbrüche in Formeln erlauben
\begin{center}
\bfseries % Fettdruck einschalten
\sffamily % Serifenlose Schrift
\vspace{-40pt}
Theoretische Elektrodynamik, Sommersemester 2014, Aufgabenblatt 5

Markus Fenske, Mattis Riediger, Tutor: Clemens Meyer zu Rheda
\vspace{-10pt}
\end{center}

\section*{Aufgabe 1: Dipolmomente}
\begin{enumerate}[a)]
  \item
    Das Dipolmoment ist
    \begin{equation}
      \vec{p} = \iiint \dif V \; \vec{r} \rho(\vec{r})
    \end{equation}

    Wenn wir unser Koordinatensystem um den Vektor $\vec{a}$ verschieben,
    erhalten wir
    \begin{align}
      \vec{p'} &= \iiint \dif V \; \del{\vec{r} - \vec{a}} \rho(\vec{r}) \\
               &= \iiint \dif V \; \vec{r} \rho(\vec{r}) - \vec{a} \iiint \dif V \; \rho(\vec{r}) \\
               &= \vec{p} - \vec{a} Q
    \end{align}

    Wenn die Gesamtladung $Q = 0$ ist, ist das Dipolmoment also translationsinvariant.
  \item
    Durch Übergang zu diskreten Ladungen $\rho(\vec{r}) = q_1\delta(\vec{r} -
    \vec{r}_1) + \dots + q_n\delta(\vec{r} - \vec{r}_n)$ erhalten wir
    \begin{equation}
      \vec{p} = \iiint \dif V \; \sum_i q_i \delta(\vec{r} - \vec{x}_i) = \sum_i q_i \vec{r}_i
    \end{equation}

    Da bei allen angegebenen Ladungen die Gesamtladung verschwindet, können wir
    den Ursprung hinlegen, wo wir wollen. Das Koordinatensystem sei jeweils so
    orientiert wie auf dem Aufgabenzettel (wenn man ihn richtig herum hält).
    \begin{enumerate}[(i)]
      \item
        Legt man $-q$ in den Ursprung, dann sieht man sofort $\vec{p} = q a \hat{e}_x$
      \item
        Legt man den Ursprung zwischen die beiden $-q$, gleicht sich deren
        Dipolmoment aus. Die Höhe des Gleichseitigen Dreiecks ist
        $\frac{\sqrt{3}}{2} a$, also ist das Dipolmoment $\vec{p} = q a \sqrt{3} \hat{e}_y$.
      \item
        Legt man den Ursprung zwischen die beiden $-q$, gleicht sich deren
        Dipolomoment aus, genau wie das der beiden weiter entfernten Ladungen,
        und das Dipolmoment zwischen den außen liegenden Ladungen, also ist
        $\vec{p} = 0$.
      \item
        Legt man den Ursprung in die Mitte des Quadrats, gleichen sich die
        Dipolmomente der jeweils Diagonal gegenüberliegenden Ladungen aus, also
        $\vec{p} = 0$.
    \end{enumerate}
\end{enumerate}
\section*{Aufgabe 2: Multipolmomente}
Das Monopolmoment verschwindet, weil die Gesamtladung verschwindet. Also kann
man den Koordinatenursprung frei wählen. Den setzen wir also auf die untere
Grundseite des Dreiecks, in die Mitte zwischen den beiden $q$. In diesem Punkt
stehen sich zwei gleichnamige Ladungen in gleichem Abstand gegenüber, also
verschwindet das Dipolmoment dieser Ladungen. Die Höhe des gleichseitigen
Dreiecks ist $\frac{\sqrt{3}}{2}a$, dort auf der $y$-Achse sitzt die Ladung
$q$. Der Mittelpunkt eines gleichseitigen Dreiecks teilt die Höhe im Verhältnis
$2:1$, liegt also bei $(0, \frac{\sqrt{3}}{6})$.
In der Mitte, also auf der Hälfte der Strecke sitzt die Ladung $-3q$. Somit ist
das Dipolmoment
\begin{equation}
  \vec{p} = q \frac{\sqrt{3}}{2} a \hat{e}_y - 3q \frac{\sqrt{3}}{6} a \hat{e}_y = 0
\end{equation}

Wir berechnen also das Multipolmoment. Dafür haben wir jeweils die folgenden
Ladungen und Ortsvektoren: % FIXME: Wenn Zeit ist: Skizze.

\begin{align}
  &q_1 = q, &&r_1 = a\begin{pmatrix} -\frac{1}{2} \\ 0 \\ 0 \end{pmatrix} \\
  &q_2 = q, &&r_2 = a\begin{pmatrix} \frac{1}{2} \\ 0 \\ 0 \end{pmatrix} \\
  &q_3 = q, &&r_3 = a\begin{pmatrix} 0 \\ \frac{\sqrt{3}}{2} \\ 0 \end{pmatrix} \\
  &q_4 = -3q, &&r_4 = a\begin{pmatrix} 0 \\ \frac{\sqrt{3}}{6} \\ 0 \end{pmatrix} \\
\end{align}

Das Multipolmoment berechnet sich durch

\begin{align}
  Q_{kl} &= \sum_i q_i \del{3 r_{ik} r_{il} - (r_i)^2 \delta_{kl}} \\
       &= \sum_i q_i 3 q_i \begin{pmatrix}
        r_{i1}^2     & r_{i1} r_{i2}  & r_{i1} r_{i3} \\
        r_{i2} r_{i1} & r_{i2}^2      & r_{i2} r_{i3} \\
        r_{i2} r_{i1} & r_{i3} r_{i2} & r_{i3}^2 \\
      \end{pmatrix} - \sum_i q_i \begin{pmatrix}
        r_i^2 &       &       \\
              & r_i^2 &       \\
              &       & r_i^2
      \end{pmatrix}
\end{align}

Die Quadrupolmomente für die einzelnen Ladungen sind somit

\begin{equation}
  Q_1 = Q_2 = 3q\begin{pmatrix}
    \dfrac{a^2}{4} & 0 & 0 \\
    0 & 0 & 0 \\
    0 & 0 & 0
  \end{pmatrix} - \begin{pmatrix}
   \frac{a^2}{4} &       &       \\
                 & \dfrac{a^2}{4} &       \\
                 &       & \dfrac{a^2}{4}
  \end{pmatrix}
  =
  qa^2\begin{pmatrix}
   \dfrac{2}{4} &       &       \\
                 & -\dfrac{1}{4} &       \\
                 &       & -\dfrac{1}{4}
  \end{pmatrix}
\end{equation}

\begin{equation}
  Q_3 = 3q\begin{pmatrix}
    0 & 0 & 0 \\
    0 & \frac{3a^2}{4} & 0 \\
    0 & 0 & 0
  \end{pmatrix} - \begin{pmatrix}
   \frac{3a^2}{4} &       &       \\
                 & \dfrac{3a^2}{4} &       \\
                 &       & \dfrac{3a^2}{4}
  \end{pmatrix}
  =
  qa^2\begin{pmatrix}
   \dfrac{6}{4} &       &       \\
                 & -\dfrac{3}{4} &       \\
                 &       & -\dfrac{3}{4}
  \end{pmatrix}
\end{equation}

Für $Q_4$ ergibt sich aufgrund des gedrittelten Abstandes, aber der höheren Ladung
\begin{equation}
  Q_4 = -3 \cdot \frac{1}{9} Q_3 = -\frac{1}{3} Q_3 = 
    qa^2\begin{pmatrix}
    -\dfrac{2}{4} &       &       \\
                    & \dfrac{1}{4} &       \\
                    &       & \dfrac{1}{4}
    \end{pmatrix}
\end{equation}

Das Gesamtquadrupolmoment ist die Summe der Quadrupolmomente, also
\begin{equation}
  Q = qa^2 \begin{pmatrix}
    2 & & \\
      & -1 & \\
      &    & -1
  \end{pmatrix}
\end{equation}

Das ist das kleinste nicht-verschwindende Multipolmoment.

\section*{Aufgabe 3: Zwei elektrische Dipole}
\begin{enumerate}[a)]
  \item
    Das elektrische Feld des Punktdipols $\vec{p_1}$ ist
    \begin{equation}
      \vec{E}_1(\vec{r}) = \frac{1}{4 \pi \epsilon_0} \del{
        3 \frac{\vec{p}_1 \cdot \vec{r}}{r^5} \vec{r} - \frac{1}{r^3} \vec{p}_1
      }
    \end{equation}

    Dann wirkt auf $\vec{p}_2$ das Drehmoment
    \begin{equation}
      \vec{M}_2 = \vec{p}_2 \times \vec{E}_1
    \end{equation}

    Der Zeichnung entnehmen wir
    \begin{equation}
      \vec{p}_1 = p_1 \begin{pmatrix}
        \cos \theta_1 \\
        0 \\
        \sin \theta_1
      \end{pmatrix}, \quad
      \vec{p}_2 = p_2 \begin{pmatrix}
        \cos \theta_2 \\
        0 \\
        \sin \theta_2
      \end{pmatrix}, \quad
      \vec{r} = \begin{pmatrix}
        d \\
        0 \\
        0
      \end{pmatrix}
    \end{equation}

    Eingesetzt in das Drehmoment erhalten wir
    \begin{align}
      \vec{M}_2 &= p_2 \begin{pmatrix}
        \cos \theta_2 \\
        0 \\
        \sin \theta_2
      \end{pmatrix}
      \times
      \del{
        \frac{1}{4 \pi \epsilon_0}
        \frac{3 \vec{p}_1 \cos \theta_1 d}{d^5} \begin{pmatrix} d \\ 0 \\ 0 \end{pmatrix} -
        \frac{p_1}{d^3} \begin{pmatrix} \cos \theta_1 \\ 0 \\ \sin \theta_1 \end{pmatrix}
      } \\
      &= \frac{3 p_1 p_2 \cos \theta_1}{4 \pi \epsilon_0 d^4}
        \begin{pmatrix} \cos \theta_2 \\ 0 \\ \sin \theta_2 \end{pmatrix} \times
        \begin{pmatrix} d \\ 0 \\ 0 \end{pmatrix}
        -
        \frac{p_1 p_2}{4 \pi \epsilon_0 d^3}
        \begin{pmatrix} \cos \theta_2 \\ 0 \\ \sin \theta_2 \end{pmatrix} \times
        \begin{pmatrix} \cos \theta_1 \\ 0 \\ \sin \theta_1 \end{pmatrix} \\
      &= \frac{3 p_1 p_2 \cos \theta_1}{4 \pi \epsilon_0 d^4}
         \begin{pmatrix} 0 \\ d \sin \theta_2 \\ 0 \end{pmatrix}
         - \frac{p_1 p_2}{4 \pi \epsilon_0 d^3}
           \begin{pmatrix}
             0 \\
             \sin \theta_2 \cos \theta_1 - \cos \theta_2 \sin \theta_1 \\
             0
           \end{pmatrix} \\
      \Rightarrow M_{2y} &= \frac{p_1p_2}{4 \pi \epsilon_0 d^3} \del{
        2 \cos \theta_1 \sin \theta_2 + \cos \theta_2 \sin \theta_1
      }, M_{2x} = M_{2z} = 0
    \end{align}

    Für den umgekehrten Fall wird $d \to -d$ und die jeweiligen Indices
    vertauschen sich, also ist
    \begin{equation}
      M_{1y} = -\frac{p_1p_2}{4 \pi \epsilon_0 d^3} \del{
        2 \cos \theta_2 \sin \theta_1 + \cos \theta_1 \sin \theta_2
      }, M_{1x} = M_{1z} = 0
    \end{equation}
  \item
    % FIXME: Bullshit. Die einzelnen Drehmomente müssen verschwinden (und damit
    % AUCH die Summe).

    In der Gleichgewichtslage verschwinden die einzelnen Drehmomente und somit
    auch deren Summe. Das nutzen wir, um zuerst die Optionen zu reduzieren.
    \begin{align}
      0 &= \vec{M}_1 + \vec{M}_2 \\
      0 &= \frac{p_1 p_2}{4 \pi \epsilon_0 d^3} \del{
          2 \cos \theta_1 \sin \theta_2 + \cos \theta_2 \sin \theta_1 -
          2 \cos \theta_2 \sin \theta_1 - \cos \theta_1 \sin \theta_2
        } \\
      0 &= \cos \theta_1 \sin \theta_2 - \cos \theta_2 \sin \theta_1 \\
      0 &= \sin(\theta_1 - \theta_2)
    \end{align}

    Abgesehen von der Möglichkeit $p_1 = p_2 = 0$, muss also entweder $\theta_1
    = \theta_2$ oder $\theta_1 = \theta_2 \pm \pi$.


  \item
    Die Wechselwikungsenergie von $p_1$ auf $p_2$ ist
    \begin{align*}
      W_1 &= - \vec{p}_2 \cdot \vec{E}_1 \\
        &= - p_2 \begin{pmatrix}
          \cos \theta_2 \\ 0 \\ \sin \theta_2
        \end{pmatrix}
        \frac{1}{4 \pi \epsilon_0} \del {
          3 \frac{p_1 \cos \theta_1 d}{d^5} \begin{pmatrix}
            d \\ 0 \\ 0
          \end{pmatrix}
          -
          \frac{p_1}{d^3} p_1 \begin{pmatrix}
            \cos \theta_1 \\ 0 \\ \sin \theta_1
          \end{pmatrix}
        } \\
        &= \frac{p_1 p_2}{4 \pi \epsilon_0 d^3 } \del{
          \sin \theta_2 \sin \theta_1 - 2 \cos \theta_2 \cos \theta_1
        }
    \end{align*}

    Für die Möglichkeit $p_1 = p_2$ ergibt sich eine indifferente
    Gleichgewichtslage (denn $W = 0$ konstant).

    % FIXME: Erst das oben fixen.
    Für $\theta_1 = \theta_2$ ist $W \propto 1 - 3 \cos^2(\theta)$, die
    Funktion hat ihr Minimum bei $\theta = 0$ und $\theta = \pi$. Es gibt also
    zwei Gleichgewichtslagen, wenn die beiden Dipole jeweils entlang der
    $x$-Achse in die gleiche Richtung zeigen. Die Maxima sind bei $\theta =
    \dfrac{\pi}{2}$ und $\theta = \dfrac{3 \pi}{2}$. Wenn die Dipole also
    jeweils entlang der $z$-Achse orientiert sind, ist kein Gleichgewicht
    vorhanden.



\end{enumerate}
\end{document}
