\documentclass[a4paper,german,12pt,smallheadings]{scrartcl}
\usepackage[T1]{fontenc}
\usepackage[utf8]{inputenc}
\usepackage{babel}
\usepackage{geometry}
\usepackage[fleqn]{mathtools} % also includes mathclap
\usepackage[fleqn]{amsmath}
\usepackage{amssymb}
\usepackage{float}
\usepackage{enumerate}
\usepackage{commath} % http://tex.stackexchange.com/questions/14821/whats-the-proper-way-to-typeset-a-differential-operator
\usepackage{cancel}

% Number only referenced equations
\mathtoolsset{showonlyrefs}

%\usepackage{wrapfig}
\usepackage[thinspace,thinqspace,squaren,textstyle]{SIunits}

% New command for color underlining
\usepackage{xcolor}

\newsavebox\MBox
\newcommand\colul[2][red]{{\sbox\MBox{$#2$}%
  \rlap{\usebox\MBox}\color{#1}\rule[-1.2\dp\MBox]{\wd\MBox}{0.5pt}}}

\restylefloat{table}
\geometry{a4paper, top=15mm, left=10mm, right=20mm, bottom=20mm, headsep=10mm, footskip=12mm}
\linespread{1.2}
\setlength\parindent{0pt}
\DeclareMathOperator{\Tr}{Tr}
\DeclareMathOperator{\Var}{Var}
\begin{document}
\allowdisplaybreaks % Seitenumbrüche in Formeln erlauben
\begin{center}
\bfseries % Fettdruck einschalten
\sffamily % Serifenlose Schrift
\vspace{-40pt}
Theoretische Elektrodynamik, Sommersemester 2014, Aufgabenblatt 6

Markus Fenske, Mattis Riediger, Tutor: Clemens Meyer zu Rheda
\vspace{-10pt}
\end{center}

\section*{Aufgabe 1: Spiegelladungen}
Die Ladung $q$ befindet sich an der Position $(a,b)$. Es liegt nahe, einfach
zwei Spiegelladungen $-\dfrac{q}{2}$ an den Punkten $(-a, b)$ und $(a, -b)$
anzubringen.

Es entstehen Superpositionen der drei Coulombpotentiale, nämlich
\begin{align*}
  \Phi_0(x,y) = \frac{q}{4 \pi \epsilon_0} \left(
    \frac{1}{\sqrt{(a-x)^2 + (b-y)^2}} -
    \frac{1}{2} \frac{1}{\sqrt{(a+x)^2 + (b-y)^2}} -
    \frac{1}{2} \frac{1}{\sqrt{(a-x)^2 + (b+y)^2}}
  \right)
\end{align*}

Wir sehen, dass das Potential an den Grenzflächen verschwindet:

Für $x = 0$:

\begin{align*}
  \Phi_0(x,y) = \frac{q}{4 \pi \epsilon_0} \left(
    \frac{1}{\sqrt{a^2 + (b-y)^2}} -
    \frac{1}{2} \frac{1}{\sqrt{a^2 + (b-y)^2}} -
    \frac{1}{2} \frac{1}{\sqrt{a^2 + (b-y)^2}}
  \right) = 0
\end{align*}

Für $y = 0$:

\begin{align*}
  \Phi_0(x,y) = \frac{q}{4 \pi \epsilon_0} \left(
    \frac{1}{\sqrt{(a-x)^2 + b^2}} -
    \frac{1}{2} \frac{1}{\sqrt{(a-x)^2 + b^2}} -
    \frac{1}{2} \frac{1}{\sqrt{(a-x)^2 + b^2}}
  \right) = 0
\end{align*}

Also ist der Ansatz korrekt.

Die Kraft ergibt sich als $F = q \vec{E}$, wobei $\Phi = - \nabla \vec{E}$. Da
$\Phi$ bei $x=a$ und $b=y$ eine Unendlichkeitsstelle hat, die sich durch die
Deltafunktion ausdrücken lässt, muss die Kraft dementsprechend verschwinden.
Was ja klar ist, denn über eine geerdete Platte kann keinerlei Energie zugeführt
werden.

\section*{Aufgabe 3: Randwertproblem in zwei Dimensionen}

Da im Volumen $V$ keine Ladungen sind, kann auch kein Feld existieren. Somit
muss $A=0$, dann ist $\Phi(x,y) = 0$


\end{document}
