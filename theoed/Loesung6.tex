\documentclass[a4paper,german,12pt,smallheadings]{scrartcl}
\usepackage[T1]{fontenc}
\usepackage[utf8]{inputenc}
\usepackage{babel}
\usepackage{geometry}
\usepackage[fleqn]{mathtools} % also includes mathclap
\usepackage[fleqn]{amsmath}
\usepackage{amssymb}
\usepackage{float}
\usepackage{enumerate}
\usepackage{commath} % http://tex.stackexchange.com/questions/14821/whats-the-proper-way-to-typeset-a-differential-operator
\usepackage{cancel}

% Number only referenced equations
\mathtoolsset{showonlyrefs}

%\usepackage{wrapfig}
\usepackage[thinspace,thinqspace,squaren,textstyle]{SIunits}

% New command for color underlining
\usepackage{xcolor}

\newsavebox\MBox
\newcommand\colul[2][red]{{\sbox\MBox{$#2$}%
  \rlap{\usebox\MBox}\color{#1}\rule[-1.2\dp\MBox]{\wd\MBox}{0.5pt}}}

\restylefloat{table}
\geometry{a4paper, top=15mm, left=10mm, right=20mm, bottom=20mm, headsep=10mm, footskip=12mm}
\linespread{1.2}
\setlength\parindent{0pt}
\DeclareMathOperator{\Tr}{Tr}
\DeclareMathOperator{\Var}{Var}
\newcommand*\laplace{\mathop{}\!\mathbin\Delta}
\begin{document}
\allowdisplaybreaks % Seitenumbrüche in Formeln erlauben
\begin{center}
\bfseries % Fettdruck einschalten
\sffamily % Serifenlose Schrift
\vspace{-40pt}
Theoretische Elektrodynamik, Sommersemester 2014, Aufgabenblatt 6

Markus Fenske, Mattis Riediger, Tutor: Clemens Meyer zu Rheda
\vspace{-10pt}
\end{center}

\section*{Aufgabe 1: Spiegelladungen}
Die Ladung $q$ befindet sich an der Position $(a,b)$. Es liegt nahe, einfach
zwei Spiegelladungen $-q$ an den Punkten $(-a, b)$ und $(a, -b)$ und eine
Spiegelladung $q$ an $(-a, -b)$ anzubringen und zu prüfen, ob das Potential bei
$x=0$ bzw. $y=0$ verschwindet.

Es entstehen Superpositionen von vier Coulombpotentialen, nämlich
\begin{align*}
  \Phi_0(x,y) = \frac{q}{4 \pi \epsilon_0} \left(
    \frac{1}{\sqrt{(a-x)^2 + (b-y)^2}} -
    \frac{1}{\sqrt{(a+x)^2 + (b-y)^2}} - \right. \\
    \left. \frac{1}{\sqrt{(a-x)^2 + (b+y)^2}} +
    \frac{1}{\sqrt{(a+x)^2 + (b+y)^2}}
  \right)
\end{align*}

Prüfen, ob der Ansatz stimmt:

Für $x=0$:
\begin{align*}
  \Phi_0(0,y) = \frac{q}{4 \pi \epsilon_0} \left(
    \frac{1}{\sqrt{a^2 + (b-y)^2}} -
    \frac{1}{\sqrt{a^2 + (b-y)^2}} - \right. \\
    \left. \frac{1}{\sqrt{a^2 + (b+y)^2}} +
    \frac{1}{\sqrt{a^2 + (b+y)^2}}
  \right) = 0
\end{align*}

Für $y=0$:
\begin{align*}
  \Phi_0(x,0) = \frac{q}{4 \pi \epsilon_0} \left(
    \frac{1}{\sqrt{(a-x)^2 + b^2}} -
    \frac{1}{\sqrt{(a+x)^2 + b^2}} - \right. \\
    \left. \frac{1}{\sqrt{(a-x)^2 + b^2}} +
    \frac{1}{\sqrt{(a+x)^2 + b^2}}
  \right) = 0
\end{align*}

Der Ansatz stimmt, das Potential ist dann (wenn man das Erdpotential auf 0
festlegt)
\begin{align*}
  \Phi(x,y) = \begin{cases}
    \Phi_0(x,y) & \text{ wenn } x,y > 0 \\
    0 & \text{ sonst}
  \end{cases}
\end{align*}

Die Kraft erhält man durch das Couloumbsche Kraftgesetz als
\begin{align*}
  \vec{F} = \frac{q^2}{4 \pi \epsilon_0} \del{
    -\frac{1}{\sqrt{4a^2}^3} \begin{pmatrix} 2a \\ 0 \end{pmatrix}
    -\frac{1}{\sqrt{4b^2}^3} \begin{pmatrix} 0  \\ 2b \end{pmatrix}
    +\frac{1}{\sqrt{4a^2+4b^2}^3} \begin{pmatrix} 2a \\ 2b \end{pmatrix}
  }
\end{align*}

\section*{Aufgabe 2: Ladung und Kugel}
\begin{enumerate}[a)]
  \item
Angenommen, die Ladung $q' = -q \frac{R}{a}$ bilde im Abstand $b =
\frac{R^2}{a}$ eine Line mit dem Mittelpunkt der Kugel und der Ladung $q$, dann
wäre das Potential

\begin{equation}
  \Phi_0(x,y,z) = \frac{1}{4 \pi \epsilon_0} \del{
    \frac{q}{\sqrt{(a-x)^2 + y^2 + z^2}} +
    \frac{q'}{\sqrt{(b-x)^2 + y^2 +z^2}}
  }
\end{equation}

Wir wollen das Potential auf der Kugeloberfläche berechnen. Das Problem ist
rotationssymmetrisch um die $x$-Achse, also setzen wir oBdA $z = 0$ (denn wenn
$z \neq 0$ ist, drehen wir das Ding so, dass der Punkt in der $x$-$y$-Ebene
liegt und dann ist $z=0$).

Für die Kreisoberfläche ist $x^2 + y^2 = R^2 \Leftrightarrow y^2 = R^2 - x^2$.

Das Potential im Abstand $R$ ist dann also:

\begin{align*}
  \Phi_0 4 \pi \epsilon_0 &= \frac{q}{\sqrt{(a-x)^2 + R^2 - x^2}} -
  \frac{qR}{a\sqrt{\del{\frac{R^2}{a} - x}^2 + R^2 - x^2}} \\
   &= \frac{q}{\sqrt{(a-x)^2 + R^2 - x^2}} -
   \frac{q}{\sqrt{\frac{a^2}{R^2} \del{\del{\frac{R^2}{a} - x}^2 + R^2 - x^2}}} \\
   &= \frac{q}{\sqrt{(a-x)^2 + R^2 - x^2}} -
   \frac{q}{\sqrt{\frac{a^2}{R^2} \del{\del{\frac{R^2}{a} - x}^2 + R^2  - x^2}}} \\
   &= \frac{q}{\sqrt{a^2 - 2ax + x^2 + R^2 - x^2}} -
   \frac{q}{\sqrt{\frac{a^2}{R^2} \del{\frac{R^4}{a^2} - 2 \frac{R^2x}{a} + x^2 + R^2 - x^2}}} \\
   &= \frac{q}{\sqrt{a^2 - 2ax + x^2 + R^2 - x^2}} -
   \frac{q}{\sqrt{\frac{a^2}{R^2} \del{\frac{R^4}{a^2} - 2 \frac{R^2x}{a} + x^2 + R^2 - x^2}}} \\
   &= \frac{q}{\sqrt{a^2 - 2ax + R^2}} -
   \frac{q}{\sqrt{\frac{a^2}{R^2} \del{\frac{R^4}{a^2} - 2 \frac{R^2x}{a} + R^2}}} \\
   &= \frac{q}{\sqrt{a^2 - 2ax + R^2}} -
   \frac{q}{\sqrt{R^2 - 2ax + a^2}} \\
   &= 0
\end{align*}

Das Potential ist also
\begin{equation}
  \Phi = \begin{cases}
    \Phi_0 & \text{ wenn } x^2 + y^2 + z^2 > R^2 \\
    0 & \text{ sonst }
  \end{cases}
\end{equation}

\item
  Analog zu oben sei $\vec{r}$ irgendein Punkt und oBdA $z=0$. Dann können wir
  das Problem auch in Polarkoordinaten betrachten. Mit der Transformation
  \begin{align*}
    x &= r \cos \phi \\
    y &= r \sin \phi \\
    z &= 0
  \end{align*}

  eingesetzt in das Potential $\Phi_0$ erhalten wir
  \begin{equation}
    \Phi_0(x,y) = \frac{1}{4 \pi \epsilon_0} \del{
      \frac{q}{\sqrt{a^2 - 2 a r \cos \phi + R^2}} +
      \frac{q'}{\sqrt{b^2 - 2 b r \cos \phi + R^2}}
    }
  \end{equation}

  Die Oberflächenladungsdichte ist proportional zur Radialkomponente des
  elektrischen Feldes, die wiederrum aus dem $\nabla \Phi$ kommt.
  \begin{align}
    \sigma &= - \epsilon_0 \partial_r \Phi \\
           &= - \frac{1}{4 \pi} \del {
             \frac{-2qa \cos \phi}{\sqrt{a^2 - 2ar \cos \phi + R^2}^3} +
             \frac{-2qb \cos \phi}{\sqrt{b^2 - 2br \cos \phi + R^2}^3}
           } \\
           &= \frac{\cos \phi}{2 \pi} \del {
             \frac{qa}{\sqrt{a^2 - 2ar \cos \phi + R^2}^3} +
             \frac{q'b}{\sqrt{b^2 - 2br \cos \phi + R^2}^3}
           }
  \end{align}
  % FIXME: b und q' einsetzen. Bin zu faul

\item
  Die Gesamtladung der Kugel muss verschwinden. Gleichzeitig bildet die
  Kugeloberfläche eine Äquipotentialfläche, denn sie leitet. Wir brauchen also
  noch eine Pseudoladung mit $q'' = -q' = \frac{qR}{a}$ genau im Mittelpunkt
  der Kugel.

  Wenn das Potential auf der Kugeloberfläche vorher $\Phi = 0$ ist, ist es
  jetzt aufgrund des Superpositionsprinzips
  \begin{equation}
    \Phi(R) = \frac{1}{4 \pi \epsilon 0} \frac{q}{a} \text{ (das R kürzt sich weg)}
  \end{equation}

\item Überlegungen wie oben. Damit die Gesamtladung $q$ wird, muss $q'' + q' =
  q$ sein. Also ist $q'' = q \del{1 - \dfrac{R}{a}}$.

  Dann haben wir 3 Ladungen in einer Linie (wobei sich $q'$ natürlich in
  Richtung $q$ bewegt, wenn $q$ näher kommt um die Oberfläche auf konstantem
  Potential zu halten).

  Wir haben also zwei Ladungen zu denen Kräfte berechnet werden müssen:
  \begin{itemize}
    \item $q'' \to q$: Ladung: $q \del{1-\frac{R}{a}}$ im Abstand $a$ von $q$.
    \item $q' \to q$: Ladung $-q\frac{R}{a}$ im Abstand $a - b$ (mit $b =
      \frac{R^2}{a}$) von $q$
  \end{itemize}

  Über das Coulomb'sche Kraftgesetz erhalten wir
  \begin{equation}
    F = \frac{1}{4 \pi \epsilon_0} \del{
      \frac{q q \del{1 - \frac{R}{a}}}{a^2} -
      \frac{q q \frac{R}{a}}{\del{a-\frac{R^2}{a}}^2}
    } = \frac{q^2}{4 \pi \epsilon_0} \del{
      \frac{1 - \frac{R}{a}}{a^2} -
      \frac{\frac{R}{a}}{\del{a-\frac{R^2}{a}}^2}
    }
  \end{equation}

  Wir drücken $a$ in Vielfachen von $R$ aus: $a = x R$. Wenn wir das auf einen
  Nenner bringen und entsprechend kürzen (Maxima macht das), erhalten wir einen
  Bruch in dessen Nenner ein Polynom 5. Grades steht. Das muss verschwinden, um
  die Gleichung zu lösen.

  \begin{equation}
    x^5 - 2x^4 - 2x^3 + 2x^2 + x - 1 \overset{!}{=} 0
  \end{equation}

  Wir suchen hier Nullstellen, denn wir wissen, dass die Kraft für $R \to
  \infty$ abstoßend sein muss.  Bei einem Nulldurchgang wechselt also das
  Vorzeichen der Kraft.
  Bei $x = 1$ wird $a = R$, geht also in die Kugel. Diese Stellen interessieren
  uns nicht, also muss außerdem $x > 1$ sein.

  Dieses Polynom ist \textit{nicht faktorisierbar} (und das hab ich echt
  mehrfach nachgerechnet) und hat seine einzige reelle Nullstelle bei $x
  \approx 2{,}44305$. Also wird die Ladung ab einem Abstand von $a \approx
  2{,}44305 R$ angezogen.
\end{enumerate}

\section*{Aufgabe 3: Randwertproblem in zwei Dimensionen}

Über den Separationsansatz $\Phi(x,y) = f(x) g(y)$ erhalten wir
\begin{align*}
  &\laplace \Phi = g \partial_x^2 f + f \partial_y^2 g = 0 \\
  \Leftrightarrow \quad
  & \frac{1}{f} \partial_x^2 f = -\frac{1}{g} \partial_y^2 g
\end{align*}

Links steht eine Funktion von $x$, rechts eine Funktion von $y$. Damit die
Gleichung stimmt, müssen also beide Seiten konstant sein. Sei diese Konstante
$k^2$ erhalten wir die separierten Differentialgleichungen mit den Lösungen
\begin{align*}
  \partial_x^2 f - k^2f = 0 &\Rightarrow f(x) = ae^{kx} + be^{-kx} \\
  \partial_y^2 g + k^2g = 0 &\Rightarrow g(x) = c \sin(ky) + d \cos(ky)
\end{align*}

Aus der Randbedingung $\phi(0,y) = 0$ sehen wir $a + b
= 0$. Aus $\phi(b,y) = A \sin(\pi y /a)$ sehen wir $c = A$, $d=0$, $k = \pi/a$.

Wir raten damit die Lösung als
\begin{equation}
  \Phi(x,y) = A \sinh(kx) \sin\del{ky} \text{ mit } k = \frac{\pi}{a}
\end{equation}

Die partiellen Ableitungen sind dann
\begin{align*}
  &\partial_x^2 \Phi = -Ak^2 \sinh(kx) \sin(ky) = -k^2 \Phi(x,y) \\
  &\partial_y^2 \Phi = Ak^2 \sinh(kx) \sin(ky) = k^2 \Phi(x,y) \\
  &\Rightarrow \laplace \Phi = \partial_x^2 \Phi + \partial_y^2 \Phi = 0
\end{align*}

Also ist dies die Lösung.

\end{document}
