\documentclass[a4paper,german,12pt,smallheadings]{scrartcl}
\usepackage[T1]{fontenc}
\usepackage[utf8]{inputenc}
\usepackage{babel}
\usepackage{geometry}
\usepackage[fleqn]{mathtools} % also includes mathclap
\usepackage[fleqn]{amsmath}
\usepackage{amssymb}
\usepackage{float}
\usepackage{enumerate}
\usepackage{commath} % http://tex.stackexchange.com/questions/14821/whats-the-proper-way-to-typeset-a-differential-operator
\usepackage{cancel}

% Number only referenced equations
\mathtoolsset{showonlyrefs}

%\usepackage{wrapfig}
\usepackage[thinspace,thinqspace,squaren,textstyle]{SIunits}

% New command for color underlining
\usepackage{xcolor}

\newsavebox\MBox
\newcommand\colul[2][red]{{\sbox\MBox{$#2$}%
  \rlap{\usebox\MBox}\color{#1}\rule[-1.2\dp\MBox]{\wd\MBox}{0.5pt}}}

\restylefloat{table}
\geometry{a4paper, top=15mm, left=10mm, right=20mm, bottom=20mm, headsep=10mm, footskip=12mm}
\linespread{1.2}
\setlength\parindent{0pt}
\DeclareMathOperator{\Tr}{Tr}
\DeclareMathOperator{\Var}{Var}
\begin{document}
\allowdisplaybreaks % Seitenumbrüche in Formeln erlauben
\begin{center}
\bfseries % Fettdruck einschalten
\sffamily % Serifenlose Schrift
\vspace{-40pt}
Theoretische Elektrodynamik, Sommersemester 2014, Aufgabenblatt 4

Markus Fenske, Mattis Riediger, Tutor: Clemens Meyer zu Rheda
\vspace{-10pt}
\end{center}

\section*{Aufgabe 1: Geladene Kugeln}
Aus $\nabla E = \dfrac{\rho}{\epsilon_0}$ folgt per Gaußschem Satz
\begin{equation}
  \oint\limits_{\partial V} \dif \vec{a} \cdot \vec{E} = \int\limits_{V} \dif V \; \frac{\rho}{\epsilon_0} = \frac{Q(V)}{\epsilon_0}
\end{equation}
dabei soll $Q(V)$ die im Volumen $V$ eingeschlossene Ladung sein.

Da $\rho(r)$ nur von $r$ abhängt, ist das gesamte Problem kugelsymmetrisch.
Das bedeutet, Kugelschalen müssen Äquipotentialflächen bilden, auf denen
$\vec{E}(r)$ senkrecht steht. Somit ist
\begin{align}
  &E(r) \underbrace{\oint\limits_{\partial V} \dif A}_{\mathclap{= \text{ Kugeloberfläche: } 4 \pi r^2}} = \frac{Q(V)}{\epsilon_0} \\
  \Leftrightarrow\quad
  & \vec{E}(r) = \frac{Q(r)}{4 \pi \epsilon_0 r^2} \hat{e_r}
\end{align}

Dies gilt für alle rotationssymmetrischen elektrostatischen Probleme.

\begin{enumerate}[a)]
  \item
    In diesem Fall ist
    \begin{equation}
      Q(r) = \begin{cases}
        0 & \text{ für } r < R \\
        Q & \text{ für } r \ge R
      \end{cases}
    \end{equation}

    also
    \begin{equation}
      \vec{E}(r) = \begin{cases}
        % FIXME: Evil typographic hack.
        \; \; \; \; 0 & \text{ für } r < R
        \vspace{0.2 cm} \\
        \dfrac{Q}{4 \pi \epsilon_0 r^2} \hat{e}_r & \text{ für } r \ge R
      \end{cases}
    \end{equation}

    % TODO: Skizze. "Skizzieren Sie das elektrische Feld" kann auch sein, den
    % Betrag skizzieren und die Richung angeben? :)
  \item
    Bestimmung der Konstanten $c_n$:

    \begin{align}
      &Q = \iiint\limits_{\mathclap{\text{Kugel mit Radius $R$}}} \dif V \; \rho(r)
      = \int\limits_0^R \dif r \int\limits_0^\pi \dif \theta \int\limits_0^{2 \pi} \dif \phi \; r^2 \sin \theta c_n r^n
      = 4 \pi c_n \int\limits_0^R \dif r \; r^{n+2}
      = 4 \pi c_n \frac{r^{n+3}}{n+3} \\
      \Leftrightarrow\quad
      &c_n = \frac{Q}{4 \pi} \frac{n+3}{R^{n+3}}
    \end{align}

    Wir leiten nun die Funktion $Q(r)$ her, die die in der Kugel mit Radius $r
    \le R$ eingeschlossene Ladung angibt. Für $r > R$ ist $Q(r) = Q$. Für $r \le R$:

    \begin{align}
      &Q(r) = 4 \pi \int\limits_0^r \dif r' \; r'^2 \rho(r') = Q \frac{n+3}{R^{n+3}} \int\limits_0^r \dif r' \; r'^{n+2} = Q \frac{r^{n+3}}{R^{n+3}}
    \end{align}

    Mit der zu Beginn der Aufgabe hergeleiteten Formel ergibt sich das elektrische Feld
    \begin{equation}
      \vec{E}(r) = \dfrac{Q}{4 \pi \epsilon_0 r^2} \hat{e}_r \begin{cases}
        \dfrac{r^{n+3}}{R^{n+3}} & \text{ für } r < R \\
        1 & \text{ für } r \ge R
      \end{cases}
    \end{equation}

    Also explizit ausgeschrieben für den Fall $n = -2$

    \begin{equation}
      \vec{E}(r) = \dfrac{Q}{4 \pi \epsilon_0} \hat{e}_r \begin{cases}
        \dfrac{1}{rR} & \text{ für } r < R \\
        \frac{1}{r^2} & \text{ für } r \ge R
      \end{cases}
    \end{equation}

    Und für $n = 2$

    \begin{equation}
      \vec{E}(r) = \dfrac{Q}{4 \pi \epsilon_0} \hat{e}_r \begin{cases}
        \dfrac{r^3}{R^5} & \text{ für } r < R \\
        \frac{1}{r^2} & \text{ für } r \ge R
      \end{cases}
    \end{equation}

    %FIXME: Skizze!
\end{enumerate}

\section*{Aufgabe 2: Wasserstoffatom}
\newcommand*\laplace{\mathop{}\!\mathbin\Delta}

Sei $\phi(r)$ das Potential vom Aufgabenblatt ohne die Konstante $\frac{q}{4 \pi \epsilon_0}$

Wir spalten zuerst die Unendlichkeitsstelle von $\phi(r)$ ab:
\begin{align}
  \phi(r) &= \frac{1}{r} e^{-\alpha r} \del{1 + \frac{\alpha r}{2}} \\
          &= \frac{e^{-\alpha r}}{r} + \frac{\alpha e^{-\alpha r}}{2} \\
          &= \frac{e^{-\alpha r} - 1 + 1}{r} + \frac{\alpha e^{-\alpha r}}{2} \\
          &= \underbrace{\frac{e^{-\alpha r} - 1}{r} + \frac{\alpha e^{-\alpha r}}{2}}_{=: \phi_1} + \underbrace{\frac{1}{r}}_{=: \phi_2} \\
\end{align}

Nun ist ersichtlich (könnte man z.B. durch Taylor-Entwicklung zeigen), dass
$\phi_2(r) \neq \infty$ existiert, womit $\phi_2(r)$ als stetige Funktion also
auch bei $r = 0$ differenzierbar ist.

Aus $\laplace \phi(r) = -\frac{\rho(r)}{\epsilon_0}$ gewinnen wir die
Dichtefunktion. Da $\phi$ nur von $r$ abhängt, berechnen wir also

\begin{align}
  \laplace \phi_2(r) &= \frac{1}{r^2} \partial_r r^2 \partial_r \del{\frac{\alpha e^{-\alpha r}}{r} + \frac{e^{-\alpha r} - 1}{r}} \\
                     &= \frac{1}{r^2} \partial_r r^2 \del{ \frac{-\alpha^2 e^{-\alpha r}}{2} - \frac{\alpha e^{-\alpha r}}{r} - \frac{e^{-\alpha r} - 1}{r^2}} \\
                     &= \frac{1}{r^2} \partial_r r^2 e^{-\alpha r} \del{ \frac{-\alpha^2}{2} - \frac{\alpha}{r} - \frac{1 - e^{\alpha r}}{r^2}} \\
                     &= \frac{1}{r^2} \partial_r e^{-\alpha r} \del{ \frac{-\alpha^2r^2}{2} - \alpha r - 1 - e^{\alpha r}} \\
                     &= \frac{1}{r^2} \del{
                        -\alpha e^{-\alpha r} \del{ \frac{-\alpha^2r^2}{2} - \alpha r - 1 - e^{\alpha r}} +
                        e^{-\alpha r} \del{ -\alpha^2 r - \alpha - \alpha e^{\alpha r}}
                      } \\
                      &= \frac{1}{r^2} e^{-\alpha r} \del{
                        \frac{\alpha^3r^2}{2} + \alpha^2 r + \alpha + \alpha e^{\alpha r} +
                         -\alpha^2 r - \alpha - \alpha e^{\alpha r}
                      } \\
                      &= \frac{1}{r^2} e^{-\alpha r} \frac{\alpha^3r^2}{2} \\
                      &= e^{-\alpha r} \frac{\alpha^3}{2}
\end{align}

Auf dem Aufgabenblatt gegeben ist
\begin{equation}
  \laplace \phi_1(r) = \laplace \frac{1}{r} = - 4 \pi \delta(r)
\end{equation}

So dass sich insgesamt für die Dichtefunktion ergibt
\begin{equation}
  \rho(r) = \frac{q}{4 \pi} \del{4 \pi \delta(r) - \frac{\alpha^3 e^{-\alpha r}}{2}}
\end{equation}

Physikalisch bildet hierbei der erste Term den positiv geladenen als
punktförmig angenommenen Kern ab. Der zweite Term entspricht dem Betragsquadret
der Wellenfunktion des Wasserstoffatoms im Grundzustand, gibt also die ``Dichte
der Elektronenwolke'' an.

Die Gesamtladung im gesamten Raum ist
\begin{align}
  Q = \iiint \dif V \rho(r) &= q \iiint \dif V \delta(r) - \frac{q}{4 \pi} \iiint \dif V \; \frac{\alpha^3 e^{-\alpha r}}{2} \\
                            &= q - \frac{q \alpha^3}{2} \int_0^\infty \dif r \; r^2 e^{-\alpha r}
\end{align}

Mit
\begin{align}
  \int_0^\infty \dif r \; r^2 e^{-\alpha r}
  &=
    \sbr[3]{r^2 \frac{e^{-\alpha r}}{-\alpha}}_0^\infty
    - \int \dif r \; 2r \frac{e^{-\alpha r}}{-\alpha} \\
    &= \int \dif r \; 2r \frac{e^{-\alpha r}}{\alpha} \\
    &=  \sbr[3]{2r \frac{e^{-\alpha r}}{-\alpha^2}}_0^\infty
    - \int \dif r \; 2 \frac{e^{-\alpha r}}{-\alpha^2} \\
    &= \sbr[3]{ 2 \frac{e^{-\alpha r}}{\alpha^3}}_0^\infty \\
    &= \frac{2}{\alpha^3}
\end{align}

ist
\begin{equation}
  Q = q - q = 0
\end{equation}

Also ist das Wasserstoffatom elektrisch neutral.

\end{document}
