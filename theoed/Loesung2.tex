\documentclass[a4paper,german,12pt,smallheadings]{scrartcl}
\usepackage[T1]{fontenc}
\usepackage[utf8]{inputenc}
\usepackage{babel}
\usepackage{geometry}
\usepackage{amsmath}
\usepackage{amssymb}
\usepackage{float}
\usepackage{enumerate}
\usepackage{commath} % http://tex.stackexchange.com/questions/14821/whats-the-proper-way-to-typeset-a-differential-operator
\usepackage{cancel}
%\usepackage{wrapfig}
\usepackage[thinspace,thinqspace,squaren,textstyle]{SIunits}

% New command for color underlining
\usepackage{xcolor}

\newsavebox\MBox
\newcommand\colul[2][red]{{\sbox\MBox{$#2$}%
  \rlap{\usebox\MBox}\color{#1}\rule[-1.2\dp\MBox]{\wd\MBox}{0.5pt}}}

\restylefloat{table}
\geometry{a4paper, top=15mm, left=10mm, right=20mm, bottom=20mm, headsep=10mm, footskip=12mm}
\linespread{1.5}
\setlength\parindent{0pt}
\DeclareMathOperator{\Tr}{Tr}
\DeclareMathOperator{\Var}{Var}
\begin{document}
\allowdisplaybreaks % Seitenumbrüche in Formeln erlauben
\begin{center}
\bfseries % Fettdruck einschalten
\sffamily % Serifenlose Schrift
\vspace{-40pt}
Theoretische Elektrodynamik, Sommersemester 2014, Aufgabenblatt 2

Markus Fenske, Tutor: Clemens Meyer zu Rheda
\vspace{-10pt}
\end{center}

\section*{Aufgabe 1: Vektoranalysis: Produktregeln}

Wir definieren $\partial_i f := \frac{\partial}{\partial x_i} f$.
Es seien $\hat{e}_i$ Einheitsvektoren.
Es gelte außerdem Einsteinsche Summenkonvention (über gleichnamige Indicies
wird summiert): $a_ib_i := \sum_{i=1}^3 a_ib_i$.
Damit ist $\nabla f = \hat{e}_i \partial_i f$ für $f$ skalar bzw. $\nabla
\vec{v} = \hat{e}_i \partial_i (\hat{e}_j \vec{v}_j) = \delta_{ij} \partial_i
v_j = \partial_i v_i$ für einen Vektor $\vec{v}$.

\begin{enumerate}[a)]
  \item $\nabla fg = \partial_i f g = f \partial_i g + g \partial_i f = f \nabla g + g \nabla f$
  \item $\dots$
  \item $\nabla f \vec{v}
    = \partial_i f v_i
    = f \partial_i v_i + v_i \partial_i f
    = f \partial_i v_i + \delta_{ij} v_i \partial_j f
    = f \partial_i v_i + \hat{e}_iv_i \hat{e}_j \partial_j f
    = f \nabla \vec{v} + \vec{v} \cdot \nabla f$
\end{enumerate}

\end{document}
