\documentclass[a4paper,german,12pt,smallheadings]{scrartcl}
\usepackage[T1]{fontenc}
\usepackage[utf8]{inputenc}
\usepackage{babel}
\usepackage{geometry}
\usepackage[fleqn]{amsmath}
\usepackage{amssymb}
\usepackage{float}
\usepackage{enumerate}
\usepackage{commath} % http://tex.stackexchange.com/questions/14821/whats-the-proper-way-to-typeset-a-differential-operator
\usepackage{cancel}

% Number only referenced equations
\usepackage[fleqn]{mathtools}
\mathtoolsset{showonlyrefs}

%\usepackage{wrapfig}
\usepackage[thinspace,thinqspace,squaren,textstyle]{SIunits}

% New command for color underlining
\usepackage{xcolor}

\newsavebox\MBox
\newcommand\colul[2][red]{{\sbox\MBox{$#2$}%
  \rlap{\usebox\MBox}\color{#1}\rule[-1.2\dp\MBox]{\wd\MBox}{0.5pt}}}

\restylefloat{table}
\geometry{a4paper, top=15mm, left=10mm, right=20mm, bottom=20mm, headsep=10mm, footskip=12mm}
\linespread{1.5}
\setlength\parindent{0pt}
\DeclareMathOperator{\Tr}{Tr}
\DeclareMathOperator{\Var}{Var}
\begin{document}
\allowdisplaybreaks % Seitenumbrüche in Formeln erlauben
\begin{center}
\bfseries % Fettdruck einschalten
\sffamily % Serifenlose Schrift
\vspace{-40pt}
Theoretische Elektrodynamik, Sommersemester 2014, Aufgabenblatt 2

Markus Fenske, Mattis Riediger, Tutor: Clemens Meyer zu Rheda
\vspace{-10pt}
\end{center}

Wir definieren für dieses und alle folgenden Aufgabenblätter $\partial_i f :=
\frac{\partial}{\partial x_i} f$.

Es gelte außerdem Einsteinsche Summenkonvention. Das heißt:

\begin{enumerate}
  \item
    Über gleichnamige Indicies in Produkten wird summiert: $a_ib_i := \sum_{i=1}^3 a_ib_i$.
  \item
    Einheitsvektoren werden weggelassen. $\vec{v} = \hat{e}_iv_i$ wird nur als
    $\vec{v} =: v_i$ geschrieben. Heißt effektiv: Freistehende Indices bilden einen Tensor
    (bzw. auf diesem Aufgabenblatt nur Tensoren 1. Stufe, also Vektoren).
\end{enumerate}

Auserdem wird der Epsilon-Tensor $\epsilon_{ijk}$ (und seine Eigenschaft, das
Vorzeichen zu ändern, wenn zwei seiner Indices vertauscht werden
$\epsilon_{ijk} = -\epsilon_{kji}$) und das Kronecker-Delta
$\delta_{ij}$ benutzt, sowie die Relation $\epsilon_{ijk} \epsilon_{ilm} =
\delta_{jl}\delta_{km} - \delta_{jm}\delta_{kl}$.

Es folgt für das Skalarprodukt $\vec{a} \cdot \vec{b} := a_ib_i$ und das
Kreuzprodukt $\vec{a} \times \vec{b} := \epsilon_{ijk} a_j b_k$.  Damit ist auch
klar, dass $\nabla f := \partial_i f$ für $f$ skalar bzw. $\nabla \vec{v} :=
\partial_i v_i$ für einen Vektor $\vec{v}$.

\section*{Aufgabe 1: Vektoranalysis: Produktregeln}

\begin{enumerate}[a)]
  \item
    \begin{equation}
      \nabla fg = \partial_i f g = f \partial_i g + g \partial_i f = f \nabla g + g \nabla f
    \end{equation}
  \item
    \begin{align}
      &\vec{v} \times (\nabla \times \vec{w}) + \vec{w} \times (\nabla \times \vec{v}) + (\vec{v} \cdot \nabla) \vec{w} + (\vec{w} \cdot \nabla) \vec{v} \\
      = &\epsilon_{ijk} v_j \epsilon_{klm} \partial_l w_m + \epsilon_{ijk} w_j \epsilon_{klm} \partial_l v_m + v_j \partial_j w_i + w_j \partial_j v_i\\
      = &\epsilon_{kij} \epsilon_{klm} v_j \partial_l w_m + \epsilon_{kij} \epsilon_{klm} w_j \partial_l v_m + v_j \partial_j w_i + w_j \partial_j v_i\\
      = &(\delta_{il} \delta_{jm} - \delta_{im} \delta_{jl}) v_j \partial_l w_m + (\delta_{il} \delta_{jm} - \delta_{im}\delta_{jl}) w_j \partial_l v_m + v_j \partial_j w_i + w_j \partial_j v_i\\
      = &v_j \partial_i w_j \cancel{- v_j \partial_j w_i} + w_j \partial_i v_j \cancel{- w_j \partial_j v_i} \cancel{+ v_j \partial_j w_i} \cancel{+ w_j \partial_j v_i}\\
      = &v_j \partial_i w_j + w_j \partial_i v_i\\
      = &\partial_i v_j w_j\\
      = &\nabla(\vec{v} \cdot \vec{w})
    \end{align}
  \item
    \begin{equation}
      \nabla f \vec{v}
      = \partial_i f v_i
      = f \partial_i v_i + v_i \partial_i f
      = f \nabla \vec{v} + \vec{v} \cdot \nabla f
    \end{equation}
  \item
    \begin{align}
      &\vec{w} \cdot (\nabla \times \vec{v}) - \vec{v} \cdot (\nabla \times \vec{w}) \\
      =  &\epsilon_{ijk} w_i \partial_j v_k - \epsilon_{ijk} v_i \partial_j w_k \\
      =  &\epsilon_{ijk} w_k \partial_i v_j + \epsilon_{ijk} v_j \partial_i w_k \\
      =  &\epsilon_{ijk} (w_k \partial_i v_j + v_j \partial_i w_k) \\
      =  &\epsilon_{ijk} (\partial_i v_j w_k) \\
      =  &\nabla (\vec{v} \times \vec{w})
    \end{align}
  \item
    \begin{align}
      &f(\nabla \times \vec{v}) - \vec{v} \times (\nabla f) \\
      = &f \epsilon_{ijk} \partial_j v_k - \epsilon_{ijk} v_j \partial_k f \\
      = &\epsilon_{ijk} f \partial_j v_k + \epsilon_{ijk} v_k \partial_j f \\
      = &\epsilon_{ijk} (f \partial_j v_k + v_k \partial_j f) \\
      = &\epsilon_{ijk} (\partial_j f v_k) \\
      = &\nabla \times (f \vec{v})
    \end{align}
  \item
    \begin{align}
      &\nabla \times (\vec{v} \times \vec{w}) \\
      = &\epsilon_{ijk} \partial_j \epsilon_{klm} v_l w_m \\
      = &\epsilon_{kij} \epsilon_{klm} \partial_j v_l w_m \\
      = &(\delta_{il} \delta_{jm} - \delta_{im} \delta_{jl}) \partial_j v_l w_m \\
      = &\partial_j v_i w_j - \partial_j v_j w_i \\
      = &w_j \partial_j v_i + v_i \partial_j w_j - \partial_j v_j w_i - v_j \partial_j w_i \\
      = &(w \cdot \nabla) v + (\nabla \cdot w) v - (\nabla \cdot v) w - (v \cdot \nabla) w
    \end{align}
\end{enumerate}

\section*{Aufgabe 2: Vektoranalysis: Zweite Ableitung}

\begin{align}
  &\nabla \times (\nabla \times \vec{v}) \\
  = &\epsilon_{ijk} \partial_j \epsilon_{klm} \partial_l v_m \\
  = &\epsilon_{ijk} \epsilon_{klm} \partial_j \partial_l v_m \\
  = &(\delta_{il} \delta_{jm} - \delta_{im} \delta_{jl}) \partial_j \partial_l v_m \\
  = &\partial_j \partial_i v_j - \partial_j \partial_j v_i \\
  = &\partial_i \partial_j v_j - \partial_i^2 v_j \\
  = &\nabla (\nabla \cdot \vec{v}) - \nabla^2 \vec{v}
\end{align}

\section*{Aufgabe 3: Krummlinige Koordinaten}
\begin{enumerate}[a)]
  \item
    Für Kugelkoordinaten gilt:

    \begin{align}
      &\operatorname{div} \vec{F} = \frac{1}{r^2} \partial_r r^2 F_r + \frac{1}{r \sin \theta} \partial_\theta F_\theta \sin \theta + \frac{1}{r \sin \theta} \partial_\phi F_\phi \\
      &\operatorname{grad} = \partial_r \hat{e}_r + \frac{1}{r} \partial_\theta \hat{e}_\theta + \frac{1}{r \sin \theta} \partial_\phi \hat{e}_\phi \\
      &\operatorname{rot} \vec{F} =
      \frac{1}{r \sin \theta} \del{\partial_\theta F_\theta \sin \theta - \partial_\phi F_\theta} \hat{e}_r +
      \del{\frac{1}{r \sin \theta} \partial_\phi F_r - \frac{1}{r} \partial_r r F_\phi} \hat{e}_\theta +
      \frac{1}{r} \del{\partial_r r F_\theta - \partial_\theta} \hat{e}_\phi
    \end{align}

    Somit

    \begin{align}
      &\operatorname{grad} \operatorname{div} \hat{e}_r
      = \operatorname{grad} \frac{1}{r^2} \partial_r r^2
      = \operatorname{grad} \frac{2}{r}
      = \hat{e}_r \partial_r \frac{2}{r}
      = -\frac{2}{r^2} \hat{e}_r \text{ mit } r \neq 0 \\
      &\operatorname{rot} \hat{e}_\theta
      = \hat{e}_\phi \frac{1}{r} \partial_r r
      = \frac{\hat{e}_\phi}{r}
    \end{align}
  \item
    Wir nehmen an, dass $\vec{\omega}$ irgendeine Drehachse mit
    Rotationsgeschwindigkeit $\omega$ ist und dass $\vec{r}$ irgendein Punkt im
    Abstand $r$ ist, der darum rotiert.

    Dann legen wir unser Koordinatensystem natürlich so, dass die $z$-Achse mit
    der Drehachse zusammenfällt und $\vec{r}$ im Abstand $r$ bei $\phi = 0$ und
    $z=0$ liegt (denn man kann den Winkel beliebig wählen und wenn die Höhe der
    Drehachse nicht passt, kann man sie verschieben ohne die Drehung zu
    ändern).

    Damit ist
    \begin{equation}
      \vec{A}
      = \vec{\omega} \times \vec{r}
      = \begin{pmatrix} 0 \\ 0 \\ \omega \end{pmatrix} \times  \begin{pmatrix} r \\ 0 \\ 0\end{pmatrix}
      = \begin{pmatrix} 0 \\ \omega r\\ 0 \end{pmatrix}
    \end{equation}

    Die Rotation in Zylinderkoordinaten ist
    \begin{equation}
      \operatorname{rot} \vec{F} =
      \del{\frac{1}{r} \partial_\phi F_z - \partial_z F_\phi} \hat{e}_r +
      \del{\partial_z F_z - \partial_r F_z} \hat{e}_\phi +
      \frac{1}{r} \del{\partial_r r F_\phi - \partial_\phi F_r} \hat{e}_z
    \end{equation}

    Wir haben nur die $\phi$-Komponente, also ist
    \begin{equation}
      \operatorname{rot} \vec{A} = \hat{e}_z \frac{1}{r} \partial_r r \omega r = 2 \omega \hat{e}_z \text{ mit } r \neq 0
    \end{equation}
\end{enumerate}

\end{document}
