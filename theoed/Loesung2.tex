\documentclass[a4paper,german,12pt,smallheadings]{scrartcl}
\usepackage[T1]{fontenc}
\usepackage[utf8]{inputenc}
\usepackage{babel}
\usepackage{geometry}
\usepackage[fleqn]{amsmath}
\usepackage{amssymb}
\usepackage{float}
\usepackage{enumerate}
\usepackage{commath} % http://tex.stackexchange.com/questions/14821/whats-the-proper-way-to-typeset-a-differential-operator
\usepackage{cancel}

% Number only referenced equations
\usepackage[fleqn]{mathtools}
\mathtoolsset{showonlyrefs}

%\usepackage{wrapfig}
\usepackage[thinspace,thinqspace,squaren,textstyle]{SIunits}

% New command for color underlining
\usepackage{xcolor}

\newsavebox\MBox
\newcommand\colul[2][red]{{\sbox\MBox{$#2$}%
  \rlap{\usebox\MBox}\color{#1}\rule[-1.2\dp\MBox]{\wd\MBox}{0.5pt}}}

\restylefloat{table}
\geometry{a4paper, top=15mm, left=10mm, right=20mm, bottom=20mm, headsep=10mm, footskip=12mm}
\linespread{1.5}
\setlength\parindent{0pt}
\DeclareMathOperator{\Tr}{Tr}
\DeclareMathOperator{\Var}{Var}
\begin{document}
\allowdisplaybreaks % Seitenumbrüche in Formeln erlauben
\begin{center}
\bfseries % Fettdruck einschalten
\sffamily % Serifenlose Schrift
\vspace{-40pt}
Theoretische Elektrodynamik, Sommersemester 2014, Aufgabenblatt 2

Markus Fenske, Tutor: Clemens Meyer zu Rheda
\vspace{-10pt}
\end{center}

\section*{Aufgabe 1: Vektoranalysis: Produktregeln}

Wir definieren $\partial_i f := \frac{\partial}{\partial x_i} f$.
Es seien $\hat{e}_i$ Einheitsvektoren.
Es gelte außerdem Einsteinsche Summenkonvention (über gleichnamige Indicies
wird summiert): $a_ib_i := \sum_{i=1}^3 a_ib_i$.
Damit ist $\nabla f = \hat{e}_i \partial_i f$ für $f$ skalar bzw. $\nabla
\vec{v} = \hat{e}_i \partial_i (\hat{e}_j \vec{v}_j) = \delta_{ij} \partial_i
v_j = \partial_i v_i$ für einen Vektor $\vec{v}$.

\begin{enumerate}[a)]
  \item
    \begin{equation}
      \nabla fg = \partial_i f g = f \partial_i g + g \partial_i f = f \nabla g + g \nabla f
    \end{equation}
  \item
    \begin{align}
      &\vec{v} \times (\nabla \times \vec{w}) + (\vec{v} \cdot \nabla) \vec{w} + \vec{w} \times (\nabla \times \vec{v}) + (\vec{w} \cdot \nabla) \vec{v} \\
      = &\epsilon_{ijk} v_j \epsilon_{klm} \partial_l w_m + v_j \partial_j w_i + \epsilon_{ijk} w_j \epsilon_{klm} \partial_l v_m + w_j \partial_j v_i\\
      = &\epsilon_{kij} \epsilon_{klm} v_j \partial_l w_m + v_j \partial_j w_i + \epsilon_{kij} \epsilon_{klm} w_j \partial_l v_m + w_j \partial_j v_i\\
      = &(\delta_{il} \delta_{jm} - \delta_{im} \delta_{jl}) v_j \partial_l w_m + v_j \partial_j w_i + (\delta_{il} \delta_{jm} - \delta_{im}\delta_{jl}) w_j \partial_l v_m + w_j \partial_j v_i\\
      = &v_j \partial_i w_j + w_j \partial_i v_i\\
      = &\partial_i v_j w_j\\
      = &\nabla(\vec{v} \cdot \vec{w})
    \end{align}
  \item
    \begin{equation}
      \nabla f \vec{v}
      = \partial_i f v_i
      = f \partial_i v_i + v_i \partial_i f
      = f \nabla \vec{v} + \vec{v} \cdot \nabla f
    \end{equation}
  \item
    \begin{align}
      &\vec{w} \cdot (\nabla \times \vec{v}) - \vec{v} \cdot (\nabla \times \vec{w}) \\
      =  &\epsilon_{ijk}w_i\partial_jv_k - \epsilon_{ijk} v_i \partial_j w_k \\
      = &-\epsilon_{ijk} w_j \partial_ iv_k + \epsilon_{ijk} v_j \partial_i w_k \\
      =  &\epsilon_{ijk} v_j \partial_i w_k - \epsilon_{ijk} w_j \partial_i v_k \\
      =  &\epsilon_{ijk} v_j \partial_i w_k + \epsilon_{ijk} w_k \partial_i v_j \\
      =  &\epsilon_{ijk} (v_j \partial_i w_k + w_k \partial_i v_j) \\
      =  &\epsilon_{ijk} (\partial_i v_j w_k) \\
      =  &\nabla (\vec{v} \times \vec{w})
    \end{align}
  \item
    \begin{align}
      &f(\nabla \times \vec{v}) - \vec{v} \times (\nabla f) \\
      = &f \epsilon_{ijk} \partial_j v_k - \epsilon_{ijk} v_j \partial_k f \\
      = &f \epsilon_{ijk} \partial_j v_k + \epsilon_{ijk} v_k \partial_j f \\
      = &\epsilon_{ijk} (f \partial_j v_k + v_k \partial_j f) \\
      = &\epsilon_{ijk} (\partial_j f v_k) \\
      = &\nabla \times (f \vec{v})
    \end{align}
  \item
    \begin{align}
      &\nabla \times (\vec{v} \times \vec{w}) \\
      = &\epsilon_{ijk} \partial_j \epsilon_{klm} v_l w_m \\
      = &\epsilon_{kij} \epsilon_{klm} \partial_j v_l w_m \\
      = &(\delta_{il} \delta_{jm} - \delta_{im} \delta_{jl}) \partial_j v_l w_m \\
      = &\partial_j v_i w_j - \partial_j v_j w_i \\
      = &w_j \partial_j v_i + v_i \partial_j w_j - \partial_j v_j w_i - v_j \partial_j w_i \\
      = &(w \cdot \nabla) v + (\nabla \cdot w) v - (\nabla \cdot v) w - (v \cdot \nabla) w
    \end{align}
\end{enumerate}

\section*{Aufgabe 2: Vektoranalysis: Zweite Ableitung}

\begin{align}
  &\nabla \times (\nabla \times \vec{v}) \\
  = &\epsilon_{lmn} \hat{e}_l \partial_m \hat{e}_n (\epsilon_{ijk} \hat{e}_i \partial_j v_k) \\
  = &\epsilon_{lmn} \hat{e}_l \partial_m \epsilon_{njk} \partial_j v_k \\
  = &\hat{e}_l \epsilon_{lmn} \epsilon_{njk} \partial_m \partial_j v_k \\
  = &\hat{e}_l (\delta_{lj}\delta_{mk} - \delta_{lk}\delta_{mj}) \partial_m \partial_j v_k \\
  = &\hat{e}_l \partial_l \partial_m v_m - \partial_m^2 v_l \\
  = &\nabla(\nabla \vec{v}) - \nabla^2 \vec{v}
\end{align}

\end{document}
