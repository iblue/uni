\documentclass[a4paper,german,12pt,smallheadings]{scrartcl}
\usepackage[T1]{fontenc}
\usepackage[utf8]{inputenc}
\usepackage{babel}
\usepackage{geometry}
\usepackage[fleqn]{mathtools} % also includes mathclap
\usepackage[fleqn]{amsmath}
\usepackage{amssymb}
\usepackage{float}
\usepackage{enumerate}
\usepackage{commath} % http://tex.stackexchange.com/questions/14821/whats-the-proper-way-to-typeset-a-differential-operator
\usepackage{cancel}

% Number only referenced equations
\mathtoolsset{showonlyrefs}

%\usepackage{wrapfig}
\usepackage[thinspace,thinqspace,squaren,textstyle]{SIunits}

% New command for color underlining
\usepackage{xcolor}

\newsavebox\MBox
\newcommand\colul[2][red]{{\sbox\MBox{$#2$}%
  \rlap{\usebox\MBox}\color{#1}\rule[-1.2\dp\MBox]{\wd\MBox}{0.5pt}}}

\restylefloat{table}
\geometry{a4paper, top=15mm, left=10mm, right=20mm, bottom=20mm, headsep=10mm, footskip=12mm}
\linespread{1.2}
\setlength\parindent{0pt}
\DeclareMathOperator{\Tr}{Tr}
\DeclareMathOperator{\Var}{Var}
\newcommand*\laplace{\mathop{}\!\mathbin\Delta}
\begin{document}
\allowdisplaybreaks % Seitenumbrüche in Formeln erlauben
\begin{center}
\bfseries % Fettdruck einschalten
\sffamily % Serifenlose Schrift
\vspace{-40pt}
Theoretische Elektrodynamik, Sommersemester 2014, Aufgabenblatt 6

Markus Fenske, Mattis Riediger, Tutor: Clemens Meyer zu Rheda
\vspace{-10pt}
\end{center}

\section*{Aufgabe 1: Kugelschale}
Die allgemeine zylindersymmetrische Lösung der Laplace-Gleichung ist
\begin{equation}
  \Phi(r, \theta) = \sum_{l=0}^\infty \del{A_lr^l + \frac{B_l}{r^{l+1}}} P_l(\cos \theta)
\end{equation}

Heuristisch schmeißen wir die Terme für $l>2$ weg (weil wir keine Kosinusterme
höherer Ordnung erwarten) und setzen auch direkt $A_0 = \phi_0, B_0 = 0$. Somit
erhalten wir
\begin{equation}
  \Phi(r, \theta) = \del{Ar + \frac{B}{r^2}} \cos \theta + \phi_0
\end{equation}

Wir fangen mit dem Außenbereich $r > R_2$ an. Das Potential darf nicht gegen
unendlich gehen, also gilt für den Außenbereich offentsichtlich $A=0$.

Aus der Oberflächenladungsdichte bauen wir uns über $\sigma = - \epsilon_0
\partial_r \phi$ das Oberflächenpotential

\begin{equation}
  \phi(r, \theta) \overset{!}{=} - \frac{r k \cos(\theta)}{\epsilon_0} + C \quad\text{ (bei $r=R_2$)}
\end{equation}

Somit
\begin{align*}
  -\frac{R_2 k \cos \theta}{\epsilon_0} + C = \frac{B}{R_2^2} \cos(\theta) + \phi_0
\end{align*}

Mit $C=\phi_0$ und Kürzen von $\cos \theta$ erhalten wir
\begin{align*}
  -\frac{R_2 k}{\epsilon_0} = \frac{B}{R_2^2} \Rightarrow B = -\frac{k}{\epsilon_0} R_2^3
\end{align*}

Somit ist für den Außenbereich

\begin{align*}
  \phi_\text{A}(r,\theta) &= -\frac{k}{\epsilon_0} \frac{R_2^3}{r^2} \cos(\theta) + \phi_0
\end{align*}

Für den Innenbereich muss das Potential stetig an das Potential des
Außenbereichs anschließen und gleichzeitig $\phi(R_1,\theta) = \phi_0$ erfüllt
werden. Daraus folgt das Gleichungssystem
\begin{align*}
  -\frac{R_2 k \cos(\theta)}{\epsilon_0} + \phi_0 &= \del{AR_2 + \frac{B}{R_2^2}} \cos(\theta) + \phi_0 \\
                                           \phi_0 &= \del{AR_1 + \frac{B}{R_1^2}} \cos(\theta) + \phi_0
\end{align*}

Aus der zweiten Gleichung folgt $AR_1 + \dfrac{B}{R_1^2} = 0$, also $B =
-AR_1^3$. Eingesetzt in die erste Gleichung und durch Kürzung folgt
\begin{equation}
  -\frac{R_2 k}{\epsilon_0} = A \del{R_2 - \frac{R_1^3}{R_2^2}} \Rightarrow
  A = -\frac{k}{\epsilon_0} \frac{1}{(1- \del{\frac{R_1}{R_2}}^3}
\end{equation}

Und somit für den Innenbereich
\begin{equation}
  \phi_I(r, \theta) = -\frac{k}{\epsilon_0} \frac{1}{1-\del{\frac{R_1}{R_2}}^3} \del{r - \frac{R_1^3}{r^2}} \cos(\theta) + \phi_0
\end{equation}
\end{document}
