\documentclass[a4paper,german,12pt,smallheadings]{scrartcl}
\usepackage[T1]{fontenc}
\usepackage[utf8]{inputenc}
\usepackage{babel}
\usepackage{geometry}
\usepackage[fleqn]{mathtools} % also includes mathclap
\usepackage[fleqn]{amsmath}
\usepackage{amssymb}
\usepackage{float}
\usepackage{enumerate}
\usepackage{commath} % http://tex.stackexchange.com/questions/14821/whats-the-proper-way-to-typeset-a-differential-operator
\usepackage{cancel}

% Number only referenced equations
\mathtoolsset{showonlyrefs}

%\usepackage{wrapfig}
\usepackage[thinspace,thinqspace,squaren,textstyle]{SIunits}

% New command for color underlining
\usepackage{xcolor}

\newsavebox\MBox
\newcommand\colul[2][red]{{\sbox\MBox{$#2$}%
  \rlap{\usebox\MBox}\color{#1}\rule[-1.2\dp\MBox]{\wd\MBox}{0.5pt}}}

\restylefloat{table}
\geometry{a4paper, top=15mm, left=10mm, right=20mm, bottom=20mm, headsep=10mm, footskip=12mm}
\linespread{1.2}
\setlength\parindent{0pt}
\DeclareMathOperator{\Tr}{Tr}
\DeclareMathOperator{\Var}{Var}
\newcommand*\laplace{\mathop{}\!\mathbin\Delta}
\begin{document}
\allowdisplaybreaks % Seitenumbrüche in Formeln erlauben
\begin{center}
\bfseries % Fettdruck einschalten
\sffamily % Serifenlose Schrift
\vspace{-40pt}
Theoretische Elektrodynamik, Sommersemester 2014, Aufgabenblatt 7

Markus Fenske, Mattis Riediger, Tutor: Clemens Meyer zu Rheda
\vspace{-10pt}
\end{center}

\section*{Aufgabe 1: Kugelschale}
Die allgemeine zylindersymmetrische Lösung der Laplace-Gleichung ist
\begin{equation}
  \Phi(r, \theta) = \sum_{l=0}^\infty \del{A_lr^l + \frac{B_l}{r^{l+1}}} P_l(\cos \theta)
\end{equation}

Heuristisch schmeißen wir die Terme für $l>2$ weg (weil wir keine Kosinusterme
höherer Ordnung erwarten) und setzen auch direkt $A_0 = \phi_0, B_0 = 0$. Somit
erhalten wir
\begin{equation}
  \Phi(r, \theta) = \del{Ar + \frac{B}{r^2}} \cos \theta + \phi_0
\end{equation}

Wir fangen mit dem Außenbereich $r > R_2$ an. Das Potential darf nicht gegen
unendlich gehen, also gilt für den Außenbereich offentsichtlich $A=0$.

Aus der Oberflächenladungsdichte bauen wir uns über $\sigma = - \epsilon_0
\partial_r \phi$ das Oberflächenpotential

\begin{equation}
  \phi(r, \theta) \overset{!}{=} - \frac{r k \cos(\theta)}{\epsilon_0} + C \quad\text{ (bei $r=R_2$)}
\end{equation}

Somit
\begin{align*}
  -\frac{R_2 k \cos \theta}{\epsilon_0} + C = \frac{B}{R_2^2} \cos(\theta) + \phi_0
\end{align*}

Mit $C=\phi_0$ und Kürzen von $\cos \theta$ erhalten wir
\begin{align*}
  -\frac{R_2 k}{\epsilon_0} = \frac{B}{R_2^2} \Rightarrow B = -\frac{k}{\epsilon_0} R_2^3
\end{align*}

Somit ist für den Außenbereich

\begin{align*}
  \phi_\text{A}(r,\theta) &= -\frac{k}{\epsilon_0} \frac{R_2^3}{r^2} \cos(\theta) + \phi_0
\end{align*}

Für den Innenbereich muss das Potential stetig an das Potential des
Außenbereichs anschließen und gleichzeitig $\phi(R_1,\theta) = \phi_0$ erfüllt
werden. Daraus folgt das Gleichungssystem
\begin{align*}
  -\frac{R_2 k \cos(\theta)}{\epsilon_0} + \phi_0 &= \del{AR_2 + \frac{B}{R_2^2}} \cos(\theta) + \phi_0 \\
                                           \phi_0 &= \del{AR_1 + \frac{B}{R_1^2}} \cos(\theta) + \phi_0
\end{align*}

Aus der zweiten Gleichung folgt $AR_1 + \dfrac{B}{R_1^2} = 0$, also $B =
-AR_1^3$. Eingesetzt in die erste Gleichung und durch Kürzung folgt
\begin{equation}
  -\frac{R_2 k}{\epsilon_0} = A \del{R_2 - \frac{R_1^3}{R_2^2}} \Rightarrow
  A = -\frac{k}{\epsilon_0} \frac{1}{(1- \del{\frac{R_1}{R_2}}^3}
\end{equation}

Und somit für den Innenbereich
\begin{equation}
  \phi_I(r, \theta) = -\frac{k}{\epsilon_0} \frac{1}{1-\del{\frac{R_1}{R_2}}^3} \del{r - \frac{R_1^3}{r^2}} \cos(\theta) + \phi_0
\end{equation}

\section*{Aufgabe 3: Leiterschleife}

Die Leiterschleife befinde sich in einem ruhenden Inertialsystem $S$. Der
Ursprung liege im Mittelpunkt des Halbkreises. Dort befinde sich eine
Probeladung $q$, die sich mit der Geschwindigkeit $v$ in $x$-Richtung bewege.
Wenn wir die Kraft auf diese Ladung kennen würden, könnten wir über die
Lorentzkraft $\vec{F} = q\vec{v} \times \vec{B}$ das Magnetfeld berechnen. Das
machen wir also.

Der Draht ist elektrisch neutral. Das bedeutet, die Linienladungsdichte
$\lambda$ verschwindet. Sei $\lambda_-$ die Linienladungsdichte, die durch die
Elektronen erzeugt wird und $\lambda_+$ die Linienladungsdichte, die durch die
Protonen erzeugt wird. Dann ist

\begin{equation}
  \lambda = \lambda_- + \lambda_+ = 0
\end{equation}

Damit die Elektronen einen Strom $I$ hervorrufen, müssen sie sich bewegen, denn
Strom ist Ladung pro Zeit. Wir sehen:

\begin{equation}
  I = \frac{dQ}{dt} = \frac{dQ}{dx} \frac{dx}{dt} = \lambda v
\end{equation}

Wir können die Geschwindigkeit $v$ also frei wählen, solange wir entsprechend
$\lambda = \frac{I}{v}$ setzen. Der Einfachheit halber bewegen sich also die
Elektronen mit der selben Geschwindigkeit $v$ wie die Probeladung.

Wir transformieren nun in ein mit $v$ bewegtes Inertialsystem $S'$.

Betrachten wir den oberen Draht in $S'$. Dort bewegen sich die Elektronen nun
nicht mehr. Auch die Ladung $q$ ist unbewegt, also wirkt auf sie kein
Magnetfeld und die Geschwindigkeit der Protonen ist uns deswegen egal. Die neue
Linienladungsdichte ist aufgrund der Längenkontraktion in $x$-Richtung.
\begin{equation}
  \lambda_+' = \frac{\lambda_+}{\sqrt{1 - \frac{v^2}{c^2}}}
\end{equation}

Die Ruhedichte der Elektronen liegt im System $S'$, also muss die
Elektronendichte genau anders herum transformiert werden (von bewegt zu
unbewegt):
\begin{equation}
  \lambda_-' = \lambda_- \sqrt{1 - \frac{v^2}{c^2}}
\end{equation}

Daraus ergibt sich, dass der Draht in diesem Inertialsystem geladen ist. Seine
Linienladungsdichte ist nämlich genau
\begin{align*}
  \lambda' &= \lambda_+' + \lambda_-' \\
       &= \frac{\lambda_+}{\sqrt{1 - \frac{v^2}{c^2}}} + \lambda_- \sqrt{1 - \frac{v^2}{c^2}} \\
       &= \lambda_{+} \frac{v^2}{c^2} \frac{1}{\sqrt{1-\frac{v^2}{c^2}}}
\end{align*}

Daraus folgt, dass auf die Ladung $q$ nun eine elektrische Kraft $\vec{F} = q
\vec{E}$ wirkt. Wir integrieren also über den mit $\lambda'$ homogen geladenen
den Draht entlang der $x$-Achse von $-\infty$ bis $0$.

\begin{align*}
  \vec{E}'_1 &= \frac{1}{4 \pi \epsilon_0} \int_{-\infty}^0 \dif x \;
  \frac{\lambda}{\sqrt{x^2 + R^2}^3} \begin{pmatrix} x \\ R \end{pmatrix} \\
  &= \frac{\lambda}{4 \pi \epsilon_0 R^3} \int_{-\infty}^0
  \frac{\dif x}{\sqrt{1 + \frac{x^2}{R^2}}^3} \begin{pmatrix} x \\ R \end{pmatrix}
\end{align*}

Mit $u = \frac{x}{R}$ ist $\dif x = R \dif u$:
\begin{align*}
  &= \frac{\lambda}{4 \pi \epsilon_0 R^3} \int_{-\infty}^0
  \frac{R \dif u}{\sqrt{1 + u^2}^3} \begin{pmatrix} u R \\ R \end{pmatrix} \\
  &= \frac{\lambda}{4 \pi \epsilon_0 R} \int_{-\infty}^0
  \frac{\dif u}{\sqrt{1 + u^2}^3} \begin{pmatrix} u \\ 1 \end{pmatrix} \\
  &= \frac{\lambda}{4 \pi \epsilon_0 R} \begin{pmatrix} -1 \\ 1 \end{pmatrix}
\end{align*}

% TODO: Hier Ladungsdichte einsetzen

Im unteren Draht geht der Strom in die andere Richtung. Ob sich dabei nun
negative Ladungen in die eine Richtung oder positive Ladungen in die andere
Richtung bewegen ist für das Ergebnis egal. Betrachtet man also einen Strom von
Protonen in die entgegengesetzte RIchtung ist der Draht aus Sicht von $S'$
genau anders herum geladen.

Es muss also das folgende Feld herauskommen:
\begin{align*}
  \vec{E}'_2 = \frac{\lambda}{4 \pi \epsilon_0 R} \begin{pmatrix} 1 \\ 1 \end{pmatrix}
\end{align*}

In Summe entsteht also ein Feld
\begin{equation}
  \vec{E}' = \frac{\lambda'}{2 \pi \epsilon_0 R} \hat{e}_y
\end{equation}

Betrachten wir nun das Feld des Halbkreises in $S'$. Der Halbkreis sei durch
den Winkel $\phi$ parametrisiert, der im Uhrzeigersinn von $0$ bis $\pi$ laufe.
Die $x$-Komponente des Stromes ist dort proportional zu $\cos(\phi)$, so dass
die Längenkontraktion dort die Ladungsdichte $\lambda' \cos \phi$ verursacht.

% FIXME: r = R
Wir integrieren also über den Ring
\begin{equation}
  \vec{E}' = \frac{1}{4 \pi \epsilon_0} \int_\text{Ring} \dif V \;
  \frac{\lambda' \cos \phi}{r^2} \begin{pmatrix} \sin \phi \\ \cos \phi \end{pmatrix}
\end{equation}

% FIXME: Sollte eigentlich das Doppelte sein.
Man sieht, dass die $x$-Komponente sowieso verschwindet (gerade mal ungerade
Funktion). Mit $\dif V = r \dif \phi$ und den entsprechenden Grenzen:
\begin{equation}
  \vec{E}' = \frac{1}{4 \pi \epsilon_0} \int_0^\phi \frac{\lambda' \cos^2 \phi}{r} \hat{e}_y
  = \frac{\lambda'}{8 \epsilon_0 R} \hat{e}_y
\end{equation}

% Ab hier doppeltes eingesetzt, damit passt.
Die Summe der 3 Felder ist dann
\begin{align*}
  \vec{E}' = \frac{\lambda'}{\epsilon_0 R} \hat{e}_y \del{\frac{1}{2 \pi} + \frac{1}{4}}
\end{align*}

Sei zur Kürzung:
\begin{equation}
  k := \frac{\dfrac{1}{2 \pi} + \frac{1}{4}}{R}
\end{equation}

Dann ist
\begin{equation}
  \vec{E} = \frac{k}{\epsilon_0} \lambda' \hat{e}_y
\end{equation}

Auf die Ladung wirkt dann die Kraft
\begin{align*}
  \vec{F}' = q \vec{E}' = \frac{qk}{\epsilon_0} \lambda' \hat{e}_y
  &= \frac{qk}{\epsilon_0} \lambda_{+} \frac{v^2}{c^2} \frac{1}{\sqrt{1-\frac{v^2}{c^2}}} \hat{e}_y
\end{align*}

Mit $\lambda_+ = -\lambda_-$ und $I = q \lambda$ sowie $\mu_0 =
\frac{1}{\epsilon_0 c^2}$ wird das zu

\begin{align*}
  \vec{F}' = \frac{qv kI \mu_0}{\sqrt{1 - \frac{v^2}{c^2}}} \hat{e}_y
\end{align*}

Transformiert in das ruhende System $S$:
\begin{equation}
  \vec{F} = qvkI \mu_0 \hat{e}_y
\end{equation}

Da $\vec{F} = q\vec{v} \times \vec{B}$ und $\vec{v} = v \hat{e}_x$, ist das
Magnetfeld

\begin{equation}
  \vec{B} = - \mu_0 I k \vec{e_z} = - \frac{\mu_0 I}{R} \del{\frac{1}{4} + \frac{1}{2\pi}}
\end{equation}


\end{document}
