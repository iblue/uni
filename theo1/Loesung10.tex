\documentclass[a4paper,german,12pt,smallheadings]{scrartcl}
\usepackage[T1]{fontenc}
\usepackage[utf8]{inputenc}
\usepackage{babel}
\usepackage{tikz}
\usepackage{geometry}
\usepackage{amsmath}
\usepackage{amssymb}
\usepackage{float}
\usepackage{wrapfig}
\usepackage[thinspace,thinqspace,squaren,textstyle]{SIunits}
\restylefloat{table}
\geometry{a4paper, top=15mm, left=20mm, right=40mm, bottom=20mm, headsep=10mm, footskip=12mm}
\linespread{1.5}
\setlength\parindent{0pt}
\begin{document}
\begin{center}
\bfseries % Fettdruck einschalten
\sffamily % Serifenlose Schrift
\vspace{-40pt}
Theoretische Physik I, Wintersemester 2012/2013, 10. Übungsblatt

Normen Peulecke und Markus Fenske, Tutorin: Alexandra Junck
\vspace{-10pt}
\end{center}


\section*{1. Fingerübungen}
\subsection*{a. Fläche der Ellipse}

In einem kartesischen Koordinatensystem erfüllt die Ellipse die Gleichung
$\frac{x^2}{a^2} + \frac{y^2}{b^2} = 1$. Diese Gleichung ist nach $y$
umstellbar. Dabei kann man die durch Wurzelziehen entstehnden
Doppeldeutigkeiten von vorn herein erledigen, indem man $y > 0$ und $x > 0$
setzt und sich auf die Berechnung einer Viertelellipse beschränkt.

\begin{align*}
  1 &= \frac{x^2}{a^2} + \frac{y^2}{b^2}\\
  1 - \frac{x^2}{a^2} &= \frac{y^2}{b^2}\\
  b^2 - \frac{x^2b^2}{a^2} &= y^2\\
  y &= \sqrt{b^2 - \frac{x^2b^2}{a^2}} \\
  y &= \frac{b}{a} \sqrt{a^2 - x^2}
\end{align*}

Damit gilt für die gesamte Fläche $A$ der Ellipse (Berechnung des Integrals siehe Formelsammlung):

\begin{align*}
  A &= 4 \cdot \int_0^a dx \frac{b}{a} \sqrt{a^2 - x^2} \\
  A &= \frac{4b}{a} \cdot \int_0^a dx \sqrt{a^2 - x^2} \\
  A &= \frac{4b}{a} \left[ \frac{x}{2} \sqrt{a^2 - x^2} + \frac{a^2}{2} \arcsin \frac{x}{a} \right]_0^a \\
  A &= \frac{4b}{a} \left( \left(\frac{a}{2} \sqrt{a^2 - a^2} + \frac{a^2}{2} \arcsin \frac{a}{a} \right) -
    \left( \frac{0}{2} \sqrt{a^2 - 0^2} + \frac{a^2}{2} \arcsin \frac{0}{a} \right) \right) \\
  A &= \frac{4b}{a} \left( \frac{a^2}{2} \arcsin 1 - \frac{a^2}{2} \arcsin 0 \right) \\
  A &= 2ba \left( \arcsin 1 - \arcsin 0 \right) \\
  A &= \pi ba
\end{align*}


\end{document}
