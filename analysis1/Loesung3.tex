\documentclass[a4paper,german,12pt,smallheadings]{scrartcl}
\usepackage[T1]{fontenc}
\usepackage[utf8]{inputenc}
\usepackage{babel}
\usepackage{tikz}
\usepackage{geometry}
\usepackage{amsmath}
\usepackage{amssymb}
\usepackage{float}
%\usepackage{wrapfig}
\usepackage{pdflscape}
\pagenumbering{gobble}
\usepackage[thinspace,thinqspace,squaren,textstyle]{SIunits}
\restylefloat{table}
\geometry{a4paper, top=15mm, left=20mm, right=40mm, bottom=20mm, headsep=10mm, footskip=12mm}
\linespread{1.5}
\setlength\parindent{0pt}
\begin{document}
\begin{center}
\bfseries % Fettdruck einschalten
\sffamily % Serifenlose Schrift
\vspace{-40pt}
Analysis I, Sommersemester 2013, 3. Übungsblatt \\
Luis Herrmann und Markus Fenske, Tutor: Adam Schienle
\vspace{-10pt}
\end{center}

\section*{Aufgabe 3.1}

Siehe letzte Seite.

\section*{Aufgabe 3.2}
\subsection*{Teil a}
Wir wissen, dass $x > 0$ und $a > 0$, deswegen:

\begin{align*}
  \leftrightarrow \frac{x_n^2 + a}{2x} &\ge \sqrt{a} \\
  \leftrightarrow x_n^2 + a &\ge \sqrt{a}2x \\
  \leftrightarrow x_n^2 -\sqrt{a}2x+ a &\ge 0 \\
  \leftrightarrow x_n^2 - 2\sqrt{a}x+ \sqrt{a}^2 &\ge 0 \\
  \leftrightarrow (x_n - \sqrt{a})^2 &\ge 0
\end{align*}

Diese Aussage ist wahr, denn $a,x_n > 0$.

\subsection*{Teil b}
Der Induktionsanfang von $x_0$ nach $x_1$ gilt nur für $a \ge 1$:

\begin{align*}
  x_0 &\ge \frac{x_1^2 + a}{2a} \\
  \leftrightarrow a &\ge \frac{a^2 + a}{2a} \\
  \leftrightarrow 2a^2 &\ge a^2 + a \\
  \leftrightarrow a^2 &\ge a \\
  \leftrightarrow a &\ge 1
\end{align*}

Somit ist das Gegenteil bewiesen.

Allerdings gilt der Induktionsanfang von $x_1$ nach $x_2$ immer:

\begin{align*}
  x_1 &\ge x_2 \\
  \leftrightarrow \frac{x_0^2 + a}{2x_0} &\ge \frac{x_1^2 + a}{2x_1} \\
  \leftrightarrow \frac{x_0^2 + a}{2x_0} &\ge \frac{\left(\frac{x_0^2 + a}{2x_0}\right)^2 + a}{2\left(\frac{x_0^2 + a}{2x_0}\right)} \\
  \leftrightarrow \frac{a^2 + a}{2a} &\ge \frac{\left(\frac{a^2 + a}{2a}\right)^2 + a}{2\left(\frac{a^2 + a}{2a}\right)} \\
  \leftrightarrow \frac{a + 1}{2} &\ge \frac{\left(\frac{a + 1}{2}\right)^2 + a}{2\left(\frac{a + 1}{2}\right)} \\
  \leftrightarrow a + 1 &\ge \frac{\left(\frac{a + 1}{2}\right)^2 + a}{\frac{a + 1}{2}} \\
  \leftrightarrow a + 1 &\ge \frac{\frac{(a+1)^2}{4} + a}{\frac{a + 1}{2}} \\
  \leftrightarrow a + 1 &\ge \frac{\frac{(a+1)^2}{2} + 2a}{a+1} \\
  \leftrightarrow 2(a + 1)^2 &\ge (a+1)^2 + 4a \\
  \leftrightarrow (a + 1)^2 &\ge 4a \\
  \leftrightarrow a^2+2a+1 &\ge 4a \\
  \leftrightarrow a^2-2a+1 &\ge 0 \\
  \leftrightarrow (a-1)^2 &\ge 0
\end{align*}

Damit können wir nun den Induktionsschritt $x_n$ nach $x_{n+1}$ machen:

\begin{align*}
  x_{n} \ge x_{n+1} \\
  \leftrightarrow x_n \ge \frac{x_n^2 + a}{2x_n} \\
  \leftrightarrow 2x_n^2 \ge x_n^2 + a \\
  \leftrightarrow x_n^2 \ge a \\
  \leftrightarrow x_n \ge \sqrt{a}
\end{align*}

Dass jedes $x_n \ge \sqrt{a}$ ist aus Teil a bekannt, somit ist die Aussage
wahr und die Induktion gültig. Die Folge fällt für $n \ge 1$ monoton.

\subsection*{Teil c}
Unter der Annahme, dass ein Grenzwert existiert ist:
\begin{align*}
\lim x_{n+1}=\lim \frac{x_n^2 + a}{2x_n}
\end{align*}
Wir haben gezeigt, dass ein Grenzwert existiert, da die Folge monoton fällt und nach unten beschränkt ist. Wie in der Vorlesung können wir schreiben:
\begin{align*}
x = \frac{x^2 + a}{2x}
\end{align*}
Sei x der Grenzwert der Folge.
\begin{align*}
\leftrightarrow 2x^2 = x^2 + a\\
\leftrightarrow x^2 = a\\
\leftrightarrow x = \sqrt{a}
\end{align*}

\section*{Aufgabe 3.3}
\subsection*{Teil a}

Um eine explizite Formel zu finden, verschaffen wir uns erstmal einen Überblick:

\begin{align*}
  &x_0 = s \\
  &x_1 = as + b \\
  &x_2 = a(as+b) + b &= a^2s+ab+b\\
  &x_3 = a(a(as+b)+b)+b &= a^3s + a^2b +ab + b
\end{align*}

Offensichtlich ist
\begin{align*}
  x_n &= a^ns + a^{n-1}b + a^{n-2}b + \dots + ab + b \\
  x_n &= a^ns + b(a^{n-1} + a^{n-2} + \dots + a + 1) \\
  x_n &= \left\{\begin{array}{ll} a^ns + b \frac{1-a^n}{1-a} & a \neq 1 \\
         s + bn & a = 1\end{array}\right.
\end{align*}

\subsection*{Teil b}
Die Wahl von b und s ist für das Konvergenzverhalten der Folge irrelevant, da:
\begin{align*}
\lim_{n \to \infty} a^ns + b\cdot\frac{1-a^n}{1-a} = s\cdot\lim_{n \to \infty} a^n + b\cdot \frac{1}{1-a} \cdot \lim_{n \to \infty} 1-a^n
\end{align*}
Es reicht also aus, das Konvergenzverhalten von $a^n$ zu betrachten.\\

Für $a>1$:
\begin{align*}
n>m=>a^n>a^m \;wenn\; a>1
\end{align*}
Daraus folgt:
\begin{align*}
a^n>a^m \; \forall n,m\ \;wenn\; n>m
\end{align*}
Die Folge ist also für das betrachtete $a$ monoton steigend. Falls Konvergenz vorhanden ist, muss die Folge auch nach oben beschränkt sein.\\
Tatsächlich finden wir keine obere Schranke, da $n \in \mathbb{N}$, wobei $\mathbb{N}$ nicht nach oben beschränkt ist. Wegen der Monotonie von $a^n$; $a>1$ folgt daraus, dass auch $a^n$ nicht beschränkt ist und somit ist die Folge auch nicht konvergent.\\

Für $a=1$:\\
Betrachten wir $x_n=s+bn$. Offensichtlich ist:
\begin{align*}
\lim_{n \to \infty}\,s+bn = s + b \cdot \lim_{n \to \infty} n = s +b \cdot \infty = \infty\\
\end{align*}

Für $0<a<1$:
\begin{align*}
a^n<a^m\;\forall n,m\ \;wenn\; n>m
\end{align*}
Die Folge ist also für das betrachtete $a$ monoton fallend. Falls Konvergenz vorhanden ist muss eine untere Schranke existieren.\\
Tatsächlich ist: 
\begin{align*}
a^n>0\;\forall a>0\
\end{align*}
Also ist $s=0$ untere Schranke der Folge für betrachtetes a und die Folge konvergent.\\

Für $a=0$:
\begin{align*}
0^n=n\;\forall n \
\end{align*}
...die Folge ist statisch und somit konvergent.\\

Für $-1<a<0$:\\
Jede Cauchy-Folge in $\mathbb{R}$ ist konvergent. Also reicht es zu zeigen, dass $(a^n)$ für betrachtetes n Cauchy-Folge ist.\\

Für $-1<a<0$ gilt:
\begin{align*}
|a^n|>|a^m|\;\forall n,m \;wenn\; n>m
\end{align*}
Sei $\epsilon>0$. Wir können davon ausgehen, dass ein N existiert, sodass:
\begin{align*}
2|a^N|<\epsilon
\end{align*}
Für dieses N wird auch gelten:
\begin{align*}
2a^N<\epsilon
\end{align*}
oBdA sei $N<m<=n$. Dann ist:
\begin{align*}
|a_m-a_n|=|a^m-a^n|\leq|a^m|+|a^n|\leq|a^N|+|a^N|=2|a^N|<\epsilon
\end{align*}
Damit wird die Voraussetzung für eine Cauchy-Folge erfüllt:
\begin{align*}
\forall \epsilon>0\ \exists N \in \mathbb{N} \: |a_m-a_n|<\epsilon \; \forall m,n \leq N \
\end{align*}

Für $a=-1$:
\begin{align*}
a^n=-1 \;oder\; a^n=1 \quad \forall n\
\end{align*}
...abhängig davon, ob n ungerade oder gerade ist. Offensichtlich divergiert die Folge, wie wir schon in der Vorlesung gesehen haben.\\

Für $a<-1$:
\begin{align*}
a^n<a^m\;\forall n,m\ \;wenn\; n>m
\end{align*}
Die Folge ist für betrachtetes $a$ also monoton fallend. Falls Konvergenz vorhanden, muss eine untere Schranke existieren.\\
Tatsächlich finden wir keine untere Schranke, da $n \in \mathbb{N}$ und $\mathbb{N}$ nicht nach oben beschränkt ist. Wegen der Monotonie von $(a^n)$; $a<-1$ folgt daraus, dass die Folge nicht nach unten beschränkt ist und somit auch nicht konvergent.\\

Zusammenfassend lässt sich also sagen: Für das Konvergenzverhalten sind $b$ und
$s$ unerheblich; damit die Folge konvergiert, muss $-1 < a < 1$ gelten.

\subsection*{Teil c}

Vorrausgesetzt, $a \in ]-1,1[$

\begin{align*}
  &\lim_{n \to \infty} a^ns + b \frac{1-a^n}{a-1} \\
  = & \lim_{n \to \infty} a^ns + b \frac{1-\lim_{n \to \infty} a^n}{a-1} \\
  = & 0 + b \frac{1-0}{a-1} \\
  = & \frac{b}{a-1}
\end{align*}

\section*{Aufgabe 3.4}

\begin{align*}
  \lim_{n \to \infty} \left(1+ \frac{1}{n}\right)^{-n} = \frac{1}{\lim_{n \to \infty} \left(1+ \frac{1}{n}\right)^{n}} = \frac{1}{e}
\end{align*}

\begin{align*}
  &\lim_{n \to \infty} \left(1-\frac{1}{n}\right)^n 
  = \lim_{n \to \infty} \left(1+\frac{1}{-n}\right)^n \\
  & =\left(\lim_{n \to \infty} \left(1+\frac{1}{-n}\right)^{-n}\right)^{-1}
  = \left(e\right)^{-1}=\frac{1}{e}
\end{align*}

\begin{align*}
  \lim_{n \to \infty} \left(1- \frac{1}{n}\right)^{-n} 
  = \frac{1}{\lim_{n \to \infty} \left(1- \frac{1}{n}\right)^{n}} 
  = \frac{1}{\frac{1}{e}} = e
\end{align*}

\begin{align*}
  & \lim_{n \to \infty} \left(1 + \frac{1}{n^2}\right)^{n}
   = \lim_{n \to \infty} \left(1 - \frac{1}{-n^2}\right)^{n} \\
  & =\lim_{n \to \infty} \left(\left(1 - \frac{1}{-n^2}\right)^{-n}\right)^{-1}
  = \left(\lim_{n \to \infty} \left(1 - \frac{1}{-n^2}\right)^{-n}\right)^{-1} \\
  & = \left(\lim_{n \to \infty} \left(1 - \frac{1}{n^2}\right)^{n}\right)^{-1} 
  = \left(\lim_{n \to \infty} \left(\left(1 - \frac{1}{n}\right)\left(1 + \frac{1}{n}\right)\right)^{n}\right)^{-1} \\
  & = \left(\lim_{n \to \infty} \left(1 - \frac{1}{n}\right)^{n} \lim_{n \to \infty} \left(1 + \frac{1}{n}\right)^{n}\right)^{-1} 
  = \left(\left(\frac{1}{e}\right)\left(e\right)\right)^{-1} = 1
\end{align*}


\begin{landscape}
\begin{align*}
  &\lim_{n \to \infty} \sqrt[3]{(n+a)(n+b)(n+c)} - n \\
  = &\lim_{n \to \infty} \frac{\sqrt[3]{(n+a)(n+b)(n+c)}^3 - n^3}{\left(\sqrt[3]{(n+a)(n+b)(n+c)}\right)^2 + n\left(\sqrt[3]{(n+a)(n+b)(n+c)}\right) + n^2} \\
  = &\lim_{n \to \infty} \frac{(n+a)(n+b)(n+c) - n^3}{\left(\sqrt[3]{(n+a)(n+b)(n+c)}\right)^2 + n\left(\sqrt[3]{(n+a)(n+b)(n+c)}\right) + n^2} \\
  = &\lim_{n \to \infty} \frac{n^3 + n^2(a+b+c) + n(ab+bc+ac) + abc - n^3}{\left(\sqrt[3]{n^3 + n^2(a+b+c) + n(ab+bc+ac) + abc }\right)^2 + n\left(\sqrt[3]{n^3 + n^2(a+b+c) + n(ab+bc+ac) + abc }\right) + n^2} \\
  = &\lim_{n \to \infty} \frac{n^2\left((a+b+c) + \frac{ab+bc+ac}{n} + \frac{abc}{n^2}\right)}{\sqrt[3]{n^3 \left(1 + \frac{a+b+c}{n} + \frac{ab+bc+ac}{n^2} + \frac{abc}{n^3}\right)}^2 + n\sqrt[3]{n^3 \left(1 + \frac{a+b+c}{n} + \frac{ab+bc+ac}{n^2} + \frac{abc}{n^3}\right)} + n^2} \\
  = &\lim_{n \to \infty} \frac{n^2\left((a+b+c) + \frac{ab+bc+ac}{n} + \frac{abc}{n^2}\right)}{n^2\left(\sqrt[3]{1 + \frac{a+b+c}{n} + \frac{ab+bc+ac}{n^2} + \frac{abc}{n^3}}^2 + \sqrt[3]{1 + \frac{a+b+c}{n} + \frac{ab+bc+ac}{n^2} + \frac{abc}{n^3}} + 1\right)} \\
  = &\lim_{n \to \infty} \frac{(a+b+c) + \frac{ab+bc+ac}{n} + \frac{abc}{n^2}}{\sqrt[3]{1 + \frac{a+b+c}{n} + \frac{ab+bc+ac}{n^2} + \frac{abc}{n^3}}^2 + \sqrt[3]{1 + \frac{a+b+c}{n} + \frac{ab+bc+ac}{n^2} + \frac{abc}{n^3}} + 1} \\
  = &\frac{\lim_{n \to \infty} (a+b+c) + \frac{ab+bc+ac}{n} + \frac{abc}{n^2}}{\lim_{n \to \infty} \sqrt[3]{1 + \frac{a+b+c}{n} + \frac{ab+bc+ac}{n^2} + \frac{abc}{n^3}}^2 + \sqrt[3]{1 + \frac{a+b+c}{n} + \frac{ab+bc+ac}{n^2} + \frac{abc}{n^3}} + 1} \\
  = &\frac{(a+b+c) + \lim_{n \to \infty} \left(\frac{ab+bc+ac}{n}\right) + \lim_{n \to \infty}\left(\frac{abc}{n^2}\right)}{\sqrt[3]{1 + \lim_{n \to \infty} \left(\frac{a+b+c}{n}\right) + \lim_{n \to \infty} \left(\frac{ab+bc+ac}{n^2}\right) + \lim_{n \to \infty}\left(\frac{abc}{n^3}\right)}^2 + \sqrt[3]{1 + \lim_{n \to \infty} \left(\frac{a+b+c}{n}\right) + \lim_{n \to \infty} \left(\frac{ab+bc+ac}{n^2}\right) + \lim_{n \to \infty} \left(\frac{abc}{n^3}\right)} + 1} \\
  = &\frac{(a+b+c) + 0 + 0}{\sqrt[3]{1 + 0 + 0 + 0}^2 + \sqrt[3]{1 + 0 + 0 + 0} + 1} \\
  = &\frac{a+b+c}{3}
\end{align*}
\end{landscape}
\end{document}
