\documentclass[a4paper,german,12pt,smallheadings]{scrartcl}
\usepackage[T1]{fontenc}
\usepackage[utf8]{inputenc}
\usepackage{babel}
\usepackage{tikz}
\usepackage{geometry}
\usepackage{amsmath}
\usepackage{amssymb}
\usepackage{float}
%\usepackage{wrapfig}
\usepackage{pdflscape}
\pagenumbering{gobble}
\usepackage[thinspace,thinqspace,squaren,textstyle]{SIunits}
\restylefloat{table}
\geometry{a4paper, top=15mm, left=20mm, right=40mm, bottom=20mm, headsep=10mm, footskip=12mm}
\linespread{1.5}
\setlength\parindent{0pt}
\begin{document}
\begin{center}
\bfseries % Fettdruck einschalten
\sffamily % Serifenlose Schrift
\vspace{-40pt}
Analysis I, Sommersemester 2013, 9. Übungsblatt \\
Luis Herrmann und Markus Fenske, Tutor: Adam Schienle
\vspace{-10pt}
\end{center}

\section*{Aufgabe 9.1}

Wir benutzen die Stetigkeit der Exponential- und der Logarithmusfunktion (denn
dann dürfen wir den Limes in die Funktion ziehen), außerdem, dass $\exp \log a
= a$ und die Regel von L'Hospital. Und natürlich, dass $\lim_{x \to \infty}
f(x) = \lim_{t \searrow 0} f\left(\frac{1}{t}\right)$.

\begin{align*}
  \lim_{x \to \infty} \left(2-a^\frac{1}{x}\right)^x &= \exp \lim_{x \to \infty} x \cdot \log \left(2-a^\frac{1}{x}\right) \\
  &= \exp \lim_{t \searrow 0} \frac{\log (2-a^t)}{t} \\
  &= \exp \lim_{t \searrow 0} \frac{a^t \cdot \log a}{2-a^t} \\
  &= \exp \frac{a^0 \cdot \log a}{2-a^0} \\
  &= \exp \frac{1 \cdot \log a}{2-1} \\
  &= \exp (-\log a) \\
  &= \frac{1}{a}
\end{align*}

\section*{Aufgabe 9.2}

Die Berechnung mit L'Hospital ist möglich, aber langweilig. Wir folgen dem
Hinweis der Aufgabenstellung und schreiben die Funktionen als Reihen um.

\begin{align*}
  \lim_{x \to 0} \frac{\cos x - e^{-\frac{x^2}{2}}}{x^4} &= \lim_{x \to 0} \frac{\left(\sum_{k=0}^\infty (-1)^k \frac{x^{2k}}{(2k)!}\right) - \left(\sum_{k=0}^{\infty} \frac{\left(-\frac{x^2}{2}\right)^k}{k!} \right)}{x^4} \\
   &= \lim_{x \to 0} \frac{\sum_{k=0}^\infty (-1)^k \frac{x^{2k}}{(2k)!} - \sum_{k=0}^{\infty} (-1)^k \frac{x^{2k}}{2^k \cdot k!}}{x^4} \\
   &= \lim_{x \to 0} \frac{\sum_{k=0}^\infty (-1)^k \frac{x^{2k}}{(2k)!} - (-1)^k \frac{x^{2k}}{2^k \cdot k!}}{x^4} \\
   &= \lim_{x \to 0} \frac{\sum_{k=0}^\infty (-1)^k x^{2k} \left(\frac{1}{(2k)!} - \frac{1}{2^k \cdot k!}\right)}{x^4} \\
   &= \lim_{x \to 0} \sum_{k=0}^\infty (-1)^k x^{2k-4} \left(\frac{1}{(2k)!} - \frac{1}{2^k \cdot k!}\right) \\
\end{align*}

Da $x \to 0$, ist klar, dass in der Reihe alle Glieder verschwinden, außer die
$x^0$-Summanden, diese werden zu $x^0 = 1$. Die $x^0$-Glieder erhalten wir bei
$2k-4 = 0$, beachten also nur $k=2$ und werfen den Rest weg.

\begin{align*}
   &= \left.(-1)^k x^{2k-4} \left(\frac{1}{(2k)!} - \frac{1}{2^k \cdot k!}\right)\right|_{k=2} \\
   &= \left.\frac{1}{(2k)!} - \frac{1}{2^k \cdot k!}\right|_{k=2} \\
   &= \frac{1}{4!} - \frac{1}{4\cdot 2!} \\
   &= \frac{1}{24} - \frac{1}{8} \\
   &= -\frac{1}{12}
\end{align*}

\section*{Aufgabe 9.3}
\subsection*{Teil a}

Wir zeigen die Konvergenz mittels Wurzelkriterium, dazu sei $a_n$ ein
Reihenglied.

\begin{align*}
  \lim_{n \to \infty} \sqrt[n]{|a_n|} &= \lim_{n \to \infty} \sqrt[n]{\left|\left(\frac{1+x}{1+n}\right)^n\right|} \\
  &= \lim_{n \to \infty} \left|\frac{1+x}{1+n}\right| \\
  &\le \lim_{n \to \infty} \left|\frac{1}{1+n}\right| + \left|\frac{x}{1+n}\right|\\
  &= \lim_{n \to \infty} \left|\frac{1}{1+n}\right| + \lim_{n \to \infty} \left|\frac{x}{1+n}\right|\\
  &= 0 + x \cdot \lim_{n \to \infty} \left|\frac{1}{1+n}\right|\\
  &= 0 < 1
\end{align*}

Also ist die Folge konvergent für alle $x$.

\subsection*{Teil b}
Wir bilden die Ableitung durch gliedweise Ableitung der Folge.

\begin{align*}
  \frac{d}{dx} f(x) &= \sum_{n=0}^\infty \frac{d}{dx} \left(\frac{1+x}{1+n}\right)^n \\
  &= \sum_{n=0}^\infty \frac{\frac{d}{dx} (1+x)^n}{(1+n)^n} \\
  &= \sum_{n=0}^\infty \frac{n (1+x)^{n-1}}{(1+n)^n}
\end{align*}

Wenn eine Folge konvergiert, konvergiert auch ihre Ableitung (wissen wir aus
dem Zusatztutorium).  Also konvergiert diese Folge.

\subsection*{Teil c}
Durch Koeffizientenvergleich mit der Taylor-Reihe erhalten wir:

\begin{align*}
  &\frac{f^{(6)}(x)}{6!}\cdot x^6 = \left(\frac{1+x}{1+6}\right)^6 \\
  \Leftrightarrow\quad &\frac{f^{(6)}(-1)}{6!}\cdot (-1)^6 = \left(\frac{1+(-1)}{1+6}\right)^6 \\
  \Leftrightarrow\quad &\frac{f^{(6)}(-1)}{6!}\cdot (-1)^6 = 0 \\
  \Leftrightarrow\quad &f^{(6)}(-1) = 0 \\
\end{align*}

\section*{Aufgabe 9.4}

\subsection*{Erster Beweis}
Für $x=0$ liegt Gleichheit vor. Das bedeutet, damit die angegebene Gleichung
gilt, muss für $x>0$ die Funktion $e^x$ stärker wachsen als $1+x$, für $x<0$
muss $1+x$ schneller fallen als $e^x$.

Oder anders:

\begin{align*}
  \forall x > 0: (e^x)' &> (1+x)' \\
  \forall x < 0: (e^x)' &< (1+x)'
\end{align*}

Das beweisen wir, indem wir die Ableitung bilden und erhalten:

\begin{align*}
  \forall x > 0: e^x &> 1 \\
  \forall x < 0: e^x &< 1
\end{align*}

Was offensichtlich wahr ist.

\end{document}
