\documentclass[a4paper,german,12pt,smallheadings]{scrartcl}
\usepackage[T1]{fontenc}
\usepackage[utf8]{inputenc}
\usepackage{babel}
\usepackage{tikz}
\usepackage{geometry}
\usepackage{amsmath}
\usepackage{amssymb}
\usepackage{float}
%\usepackage{wrapfig}
\usepackage{pdflscape}
\pagenumbering{gobble}
\usepackage[thinspace,thinqspace,squaren,textstyle]{SIunits}
\restylefloat{table}
\geometry{a4paper, top=15mm, left=20mm, right=40mm, bottom=20mm, headsep=10mm, footskip=12mm}
\linespread{1.5}
\setlength\parindent{0pt}
\begin{document}
\begin{center}
\bfseries % Fettdruck einschalten
\sffamily % Serifenlose Schrift
\vspace{-40pt}
Analysis I, Sommersemester 2013, 8. Übungsblatt \\
Luis Herrmann und Markus Fenske, Tutor: Adam Schienle
\vspace{-10pt}
\end{center}

\section*{Aufgabe 8.1}

\subsection*{Teil a}

Das machen wir induktiv. Induktionsvorraussetzung: Sei $h_0(x) = 1$,
dann ist $a_0(x) = h_0(x) \cdot e^{-x^2}$. Induktionsschritt: Sei
$a_n(x) = h_n(x) \cdot e^{-x^2}$, wobei $h_n(x)$ ein Polynom ist. Dann ist:

\begin{align*}
  a_{n+1}(x) &= \frac{d}{dx} h_n(x) \cdot e^{-x^2} \\
             &= h_n'(x) \cdot e^{-x^2} + h_n(x) \cdot -2xe^{-x^2} \\
             &= (h_n'(x) - 2x h_n(x)) \cdot e^{-x^2} \\
             &= h_{n+1}(x) \cdot e^{-x^2}
\end{align*}

Da $h_0$ ein Polynom ist, muss gemäß Rekursionsformel $h_{n+1}(x) = (h_n'(x) -
2x h_n(x))$ auch $h_n(x)$ ein Polynom sein, denn die Ableitung eines Polynoms
gibt wieder ein Polynom. Außerdem ist klar, dass die Multiplikation zweier
Polynome ($2x$ ist ein Polynom) wieder ein Polynom ergibt, die Subtraktion
ebenfalls.

Für den Grad der Polynome gilt gemäß Rekursionsformel

\begin{align*}
  \deg h_{n+1} &= \deg (h_n'(x) - 2x h_n(x)) \\
               &= \max (\deg h_n'(x), \deg 2x h_n(x)) \\
               &= \max (\deg(h_n) - 1, \deg(h_n) + 1) \\
               &= \deg(h_n) + 1
\end{align*}

Da $h_0(x) = 1$ ein Polynom vom Grad 0 ist, muss gemäß Induktion $h_n(x)$ vom Grad
$n$ sein.

\subsection*{Teil b}

\begin{align*}
  \lim_{x \to \pm \infty} a_n(x) &= \lim_{x \to \pm \infty} h_n(x) \cdot e^{-x^2} \\
  &= \lim_{x \to \pm \infty} \frac{h_n(x)}{e^{x^2}} \\
  &= 0
\end{align*}

Erklärung dazu: Wir haben in Aufgabe 5.4 bewiesen, dass die Exponentialfunktion
$e^x$ schneller wächst als jedes Polynom. Da $e^{x^2} > e^x$ für fast alle $x$
(außer vielleicht irgendwo in der Gegend von $|x| < 1$, was für die Betrachtung
im unendlichen irrelevant ist), wächst auch $e^{x^2}$ schneller als jedes
Polynom. Da $h_n(x)$ ein Polynom ist und $\lim_{x \to \pm \infty} e^{x^2} =
\infty$ muss der Grenzwert 0 sein.

\subsection*{Teil c}

Wir beweisen dies per Induktion.

Für die Anwendung des Satzes von Rolle auf $a_0(x)$ prüfen wir zuerst die wichtigen Vorraussetzungen:

\begin{itemize}
  \item Es müssen $a$ und $b$ existieren, für die $a_0(x_1) = a_0(x_2)$
  \item $a_0(x)$ muss stetig in $[a,b]$ sein
  \item $a_0(x)$ muss differenzierbar in $]a,b[$ sein
\end{itemize}

Die letzte Vorraussetzung erfüllt. Alle $a_n(x)$ sind überall differenzierbar.
Da sie überall differenzierbar sind, müssen sie auch stetig sein.

Die Funktion $a_0(x)$ ist symmetrisch, denn

\begin{align*}
  &a_0(x) = a_0(-x) \\
  \Leftrightarrow\quad &e^{-x^2} = e^{-(-x)^2} \\
  \Leftrightarrow\quad &e^{-x^2} = e^{-x^2}
\end{align*}

Das bedeutet, dass die erste Vorraussetzung erfüllt ist. Deswegen hat $a_1(x)$ eine Nullstelle.

Angenommen $a_n(x)$ habe $n$ Nullstellen, dann bedeutet das, dass zwischen
diesen Nullstellen nach dem Satz von Rolle insgesamt $n-1$ Nullstellen liegen.

Seien die Nullstellen von $a_n(x)$ jeweils bei $x_0, x_1, \dots, x_n$. Wenn wir
nun die äußersten Nullstellen $a_n(x_0) = 0$ und $a_n(x_n) = 0$ betrachten,
dann ist klar, dass $a_n'(x_0) \neq 0$ und $a_n'(x_n) \neq 0$.

Beweis:

Sei $a_n(x_i) = 0$, dann muss $h_n(x_i) = 0$ sein, denn $e^{-x^2} \neq 0
\forall x$. Dann ist $h_{n+1}(x_i) = h_n'(x_i) + 2x h_n(x_i) = h_n'(x_i)$. Wäre $h_n'(x_i) = 0$, dann wäre $h_n(x_i + \epsilon) = 0$ f


dann finden wir ein beliebig kleines $\epsilon_0 > 0$ so, dass $a_n(x_0 -
\epsilon_0) \neq 0$ und $a_n(x_0 + \epsilon_0)$, denn $a_n'(x_0) \neq 0$
(Beweis?). Das bedeutet, dass $a_n(x)$ die Nullstelle entweder von unten oder
von oben schneidet. Da aber $\lim_{x \to \infty} a_n(x) = 0$, kann die Funktion
nicht ungegrenzt wachsen, sie muss irgendwann gegen 0 gehen. Um das zu tun,
muss die Steigung der Funktion umdrehen. Das heisst, es existiert ein $a > 0$,
bei dem $a_n'(x_0 - a) = 0$.

Der obige Absatz gilt für $x_n$ analog.


\begin{align*}
  
\end{align*}

Für die inneren $n-2$ Nullstellen: Satz von Rolle. Die beiden äußeren
Nullstellen existieren, weil $a_n$ gegen 0 geht, deswegen muss $a_n$ ab einem
gewissen Punkt streng monoton fallen, die Steigung dreht sich also um, heisst
da muss $a_n'(x) = 0$ sein, das sind die beiden verbliebenen Nullstellen.

Zuerst beweisen wir, für die ersten drei Polynome, dass dort Nullstellenen in
der zu beweisenden Anzahl auftreten.

Zuerst prüf

Die Funktion $a_0(x) = e^{-x^2}$ hat offensichtlich keine Nullstellen, denn
$e^{-x^2} = 0$ hat keine Lösungen.

Die Funktion $a_1(x) = -2x e^{-x^2}$ kann nur dann null werden, wenn $-2x = 0$,
denn $e^{-x^2}$ hat keine Nullstellen (wie eben gezeigt).

Die Funktion $a_2(x) = (4x^2 - 2) e^{-x^2}$ hat zwei Nullstellen, denn $4x^2 -
2 = 0$ hat offensichtlich die Lösungen $x = \pm \sqrt{2}$.

Damit haben wir die Vorraussetzung für die Anwendung des Satzes von Rolle geschaffen, nämlich zwei gleiche Funktionswerte


\end{document}
