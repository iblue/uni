\documentclass[a4paper,german,12pt,smallheadings]{scrartcl}
\usepackage[T1]{fontenc}
\usepackage[utf8]{inputenc}
\usepackage{babel}
\usepackage{tikz}
\usepackage{geometry}
\usepackage{amsmath}
\usepackage{amssymb}
\usepackage{float}
%\usepackage{wrapfig}
\usepackage{pdflscape}
\pagenumbering{gobble}
\usepackage[thinspace,thinqspace,squaren,textstyle]{SIunits}
\restylefloat{table}
\geometry{a4paper, top=15mm, left=20mm, right=40mm, bottom=20mm, headsep=10mm, footskip=12mm}
\linespread{1.5}
\setlength\parindent{0pt}
\begin{document}
\begin{center}
\bfseries % Fettdruck einschalten
\sffamily % Serifenlose Schrift
\vspace{-40pt}
Analysis I, Sommersemester 2013, 8. Übungsblatt \\
Luis Herrmann und Markus Fenske, Tutor: Adam Schienle
\vspace{-10pt}
\end{center}

\section*{Aufgabe 8.1}

\subsection*{Teil a}

Das machen wir induktiv. Induktionsvorraussetzung: Sei $h_0(x) = 1$,
dann ist $a_0(x) = h_0(x) \cdot e^{-x^2}$. Induktionsschritt: Sei
$a_n(x) = h_n(x) \cdot e^{-x^2}$, wobei $h_n(x)$ ein Polynom ist. Dann ist:

\begin{align*}
  a_{n+1}(x) &= \frac{d}{dx} h_n(x) \cdot e^{-x^2} \\
             &= h_n'(x) \cdot e^{-x^2} + h_n(x) \cdot -2xe^{-x^2} \\
             &= (h_n'(x) - 2x h_n(x)) \cdot e^{-x^2} \\
             &= h_{n+1}(x) \cdot e^{-x^2}
\end{align*}

Da $h_0$ ein Polynom ist, muss gemäß Rekursionsformel $h_{n+1}(x) = (h_n'(x) -
2x h_n(x))$ auch $h_n(x)$ ein Polynom sein, denn die Ableitung eines Polynoms
gibt wieder ein Polynom. Außerdem ist klar, dass die Multiplikation zweier
Polynome ($2x$ ist ein Polynom) wieder ein Polynom ergibt, die Subtraktion
ebenfalls.

Für den Grad der Polynome gilt gemäß Rekursionsformel

\begin{align*}
  \deg h_{n+1} &= \deg (h_n'(x) - 2x h_n(x)) \\
               &= \max (\deg h_n'(x), \deg 2x h_n(x)) \\
               &= \max (\deg(h_n) - 1, \deg(h_n) + 1) \\
               &= \deg(h_n) + 1
\end{align*}

Da $h_0(x) = 1$ ein Polynom vom Grad 0 ist, muss gemäß Induktion $h_n(x)$ vom Grad
$n$ sein.

\subsection*{Teil b}

\begin{align*}
  \lim_{x \to \pm \infty} a_n(x) &= \lim_{x \to \pm \infty} h_n(x) \cdot e^{-x^2} \\
  &= \lim_{x \to \pm \infty} \frac{h_n(x)}{e^{x^2}} \\
  &= 0
\end{align*}

Erklärung dazu: Wir haben in Aufgabe 5.4 bewiesen, dass die Exponentialfunktion
$e^x$ schneller wächst als jedes Polynom. Da $e^{x^2} > e^x$ für fast alle $x$
(außer vielleicht irgendwo in der Gegend von $|x| < 1$, was für die Betrachtung
im unendlichen irrelevant ist), wächst auch $e^{x^2}$ schneller als jedes
Polynom. Da $h_n(x)$ ein Polynom ist und $\lim_{x \to \pm \infty} e^{x^2} =
\infty$ muss der Grenzwert 0 sein.

\end{document}
