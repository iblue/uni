\documentclass[a4paper,german,12pt,smallheadings]{scrartcl}
\usepackage[T1]{fontenc}
\usepackage[utf8]{inputenc}
\usepackage{babel}
\usepackage{tikz}
\usepackage{geometry}
\usepackage{amsmath}
\usepackage{amssymb}
\usepackage{float}
%\usepackage{wrapfig}
\usepackage{pdflscape}
\pagenumbering{gobble}
\usepackage[thinspace,thinqspace,squaren,textstyle]{SIunits}
\restylefloat{table}
\geometry{a4paper, top=15mm, left=20mm, right=40mm, bottom=20mm, headsep=10mm, footskip=12mm}
\linespread{1.5}
\setlength\parindent{0pt}
\begin{document}
\begin{center}
\bfseries % Fettdruck einschalten
\sffamily % Serifenlose Schrift
\vspace{-40pt}
Analysis I, Sommersemester 2013, 8. Übungsblatt \\
Luis Herrmann und Markus Fenske, Tutor: Adam Schienle
\vspace{-10pt}
\end{center}

\section*{Aufgabe 8.1}

\subsection*{Teil a}

Das machen wir induktiv. Induktionsvorraussetzung: Sei $h_0(x) = 1$,
dann ist $a_0(x) = h_0(x) \cdot e^{-x^2}$. Induktionsschritt: Sei
$a_n(x) = h_n(x) \cdot e^{-x^2}$, wobei $h_n(x)$ ein Polynom ist. Dann ist:

\begin{align*}
  a_{n+1}(x) &= \frac{d}{dx} h_n(x) \cdot e^{-x^2} \\
             &= h_n'(x) \cdot e^{-x^2} + h_n(x) \cdot -2xe^{-x^2} \\
             &= (h_n'(x) - 2x h_n(x)) \cdot e^{-x^2} \\
             &= h_{n+1}(x) \cdot e^{-x^2}
\end{align*}

Da $h_0$ ein Polynom ist, muss gemäß Rekursionsformel $h_{n+1}(x) = (h_n'(x) -
2x h_n(x))$ auch $h_n(x)$ ein Polynom sein, denn die Ableitung eines Polynoms
gibt wieder ein Polynom. Außerdem ist klar, dass die Multiplikation zweier
Polynome ($2x$ ist ein Polynom) wieder ein Polynom ergibt, die Subtraktion
ebenfalls.

Für den Grad der Polynome gilt gemäß Rekursionsformel

\begin{align*}
  \deg h_{n+1} &= \deg (h_n'(x) - 2x h_n(x)) \\
               &= \max (\deg h_n'(x), \deg 2x h_n(x)) \\
               &= \max (\deg(h_n) - 1, \deg(h_n) + 1) \\
               &= \deg(h_n) + 1
\end{align*}

Da $h_0(x) = 1$ ein Polynom vom Grad 0 ist, muss gemäß Induktion $h_n(x)$ vom Grad
$n$ sein.

\subsection*{Teil b}

\begin{align*}
  \lim_{x \to \pm \infty} a_n(x) &= \lim_{x \to \pm \infty} h_n(x) \cdot e^{-x^2} \\
  &= \lim_{x \to \pm \infty} \frac{h_n(x)}{e^{x^2}} \\
  &= 0
\end{align*}

Erklärung dazu: Wir haben in Aufgabe 5.4 bewiesen, dass die Exponentialfunktion
$e^x$ schneller wächst als jedes Polynom. Da $e^{x^2} > e^x$ für fast alle $x$
(außer vielleicht irgendwo in der Gegend von $|x| < 1$, was für die Betrachtung
im unendlichen irrelevant ist), wächst auch $e^{x^2}$ schneller als jedes
Polynom. Da $h_n(x)$ ein Polynom ist und $\lim_{x \to \pm \infty} e^{x^2} =
\infty$ muss der Grenzwert 0 sein.

\subsection*{Teil c}

Wir beweisen zuerst, das der Satz von Rolle nicht nur für für echte Nullstellen
gilt, sondern auch für ``Nullstellen im Unendlichen ($f(\pm\infty) = 0$)'' (alles
in Anführungszeichen weil mathematisch betrachtet hochgradig obszön - aber so
ist nachvollziehbar, was wir meinen).

Behauptung: Sei $f(x)$ eine im Intervall $[x_0, \infty[$ differenzierbare, im
Intervall $]x_0, \infty[$ stetige Funktion mit einer Nullstelle $f(x_0) = 0$
und einem Grenzwert $\lim_{x \to \infty} f(x) = 0$. Dann existiert mindestens
ein $c \in (x_0, \infty)$ für das gilt, dass $f'(c) = 0$.

Beweis: Sei $g(x) = f\left(x_0 - 1 + \frac{1}{x}\right)$ für $x \in ]0, 1]$ und $g(x) =
0$. Dann ist $g(x)$ stetig in $[0,1]$ und differenzierbar in $]0, 1[$. Außerdem
ist $g(0) = g(1) = 0$. Nach dem Satz von Rolle muss mindestens ein $d \in ]0,1[$
existieren für das gilt, dass $g'(d) = 0$. Daraus folgt 

\begin{align*}
  &0 = g'(d) \\
  \Leftrightarrow\quad&0 = -\frac{1}{d^2} f'\left(x_0 - 1 + \frac{1}{x}\right)(d)
\end{align*}

Da $d \neq 0$ kann dies nur erfüllt sein, wenn die Ableitung null wird.

\begin{align*}
  \Leftrightarrow\quad f'\left(x_0 - 1 + \frac{1}{d}\right) = 0
\end{align*}

Also hat $f'(x)$ mindestens eine Nullstelle im Intervall $[x_0, \infty[$. Damit
ist der Beweis erbracht. Für $]-\infty, x_0]$ analog.

Nun beweisen wir per Induktion, dass $a_n(x)$ genau $n$ reelle Nullstellen.

Wir beginnen mit der Feststellung, dass an der Nullstelle $a_n(x_i) = 0$ nur
erfüllt werden kann, wenn $h_n(x_i) = 0$, denn $e^{-x^2} \neq 0 \forall x$.

Induktionsvoraussetzung: Für die Anwendung des Satzes von Rolle und des
``modifizierten Satzes von Rolle'' (siehe oben) auf $a_0(x)$ prüfen wir zuerst
die wichtigen Vorraussetzungen:

\begin{itemize}
  \item Es müssen $a$ und $b$ existieren, für die $a_0(x_1) = a_0(x_2)$
  \item $a_0(x)$ muss stetig in $[a,b]$ sein
  \item $a_0(x)$ muss differenzierbar in $]a,b[$ sein
\end{itemize}

Die letzte Vorraussetzung ist erfüllt. Alle $a_n(x)$ sind überall
differenzierbar. Da nur Polynome und die $e$-Funktion enthalten sind, beide die
stetig sind, sind auch alle $a_n(x)$ stetig.

Die Funktion $a_0(x)$ ist symmetrisch, denn

\begin{align*}
  &a_0(x) = a_0(-x) \\
  \Leftrightarrow\quad &e^{-x^2} = e^{-(-x)^2} \\
  \Leftrightarrow\quad &e^{-x^2} = e^{-x^2}
\end{align*}

Das bedeutet, dass die erste Vorraussetzung erfüllt ist. Deswegen hat $a_1(x)$
eine Nullstelle, und zwar nur eine, denn $a_1(x)$ kann nur null werden, wenn
$h_1(x)$ null wird. Dies ist ein Polynom $1$ Grades, dass maximal einen
Nullstelle haben kann.

Induktionsschritt: Angenommen $a_n(x)$ habe $n$ Nullstellen, dann bedeutet das,
dass zwischen diesen Nullstellen nach dem Satz von Rolle insgesamt mindestens $n-1$
Nullstellen liegen.

Seien die Nullstellen von $a_n(x)$ jeweils bei $x_0, x_1, \dots, x_n$. Dann
gilt $a_n(x_0) = \lim_{x \to -\infty} a_n(x) = 0$ und $a_n(x_n) = \lim_{x \to
\infty} = 0$. Nach obigem Beweis des Satzes von Rolle für Grenzwerte muss $a_{n+1}(x)$
in den Intervallen $]-\infty,x_0]$ und $[x_n,\infty[$ jeweils mindestens eine
Nullstelle haben.

Somit hat $a_{n+1}(x)$ mindestens $n+1$ Nullstellen. Da die Nullstellen alle
aus dem Polynom $h_{n+1}(x)$ kommen und jedes Polynom vom Grad $n+1$ höchstens
$n+1$ Nullstellen haben kann, hat das Polynom $a_{n+1}(x)$ genau $n+1$
Nullstellen. Das war zu beweisen.

\section*{Aufgabe 8.2}
Sei $f(x) = \arctan \frac{1+x}{1-x}$ und $g(x) = \arctan x$. Die Ableitung von
$g(x)$ lässt sich ablesen:

\begin{align*}
  g'(x) = \frac{1}{x^2+1}
\end{align*}

Die Ableitung von $f(x)$ kann man ausrechnen (sofern $x \neq 1$, denn dort ist
eine Definitionslücke). Wir wenden die Kettenregel an:

\begin{align*}
  f'(x) &= \frac{1}{\left(\frac{1+x}{1-x}\right)^2+1} \cdot \frac{d}{dx} \frac{1+x}{1-x} \\
        &= \frac{1}{\left(\frac{1+x}{1-x}\right)^2+1} \cdot \frac{d}{dx} \frac{1}{1-x} + \frac{x}{1-x} \\
        &= \frac{1}{\left(\frac{1+x}{1-x}\right)^2+1} \cdot \frac{1}{(x-1)^2} + \frac{1}{(1-x)^2} \\
        &= \frac{1}{\frac{1+2x+x^2+(1-2x+x^2)}{1-2x+x^2}} \cdot \frac{2}{x^2-2x+1} \\
        &= \frac{1-2x+x^2}{2+2x^2} \cdot \frac{2}{x^2-2x+1} \\
        &= \frac{1}{(1+x^2)} \\
\end{align*}

Man sieht, dass $f'(x) - g'(x) = 0$. Die Steigung ist also jeweils von
$]-\infty, 1[$ bis $]1, +\infty[$ konstant. In $x = 1$ ist sie undefiniert. Da
die Steigung konstant ist, sind dort alle Werte identisch. Wir berechnen also
den Wert für $x=0$:

\begin{align*}
  \arctan \frac{1}{1} - \arctan 0 = \frac{\pi}{4} - 0 = \frac{\pi}{4}
\end{align*}

Dieser Wert gilt für alle $x \in ]-\infty, 1[$, also insbesondere auch für $x
\leq 0$. Für $x > 1$ dann ein anderer Wert. Von der Berechnung sehen wir ab, da
diese in der Aufgabenstellung nicht gefordert ist.

\section*{Aufgabe 8.3}

Wir machen einen Fallunterscheidung in $x = 0$ und $x > 0$.

Fall 1: $x=0$

Dann ist
\begin{align*}
  &\frac{1}{1+x} - \frac{1}{1+ax} \leq \frac{\sqrt{a} - 1}{\sqrt{a} + 1} \\
  \Leftrightarrow\quad& 0 \leq \frac{\sqrt{a} - 1}{\sqrt{a} + 1} \\
  \Rightarrow\quad&     0 \leq \sqrt{a} - 1
\end{align*}

Da $a > 1$ gegeben ist, ist dies immer wahr.

Fall 2: $x > 0$

Wir suchen den Extremwert von $f(x) = \frac{1}{1+x} - \frac{1}{1+ax}$, indem
wir die Ableitung null setzen $f'(x) \overset{!}{=} 0$. Anschließend müssen wir
nachsehen, ob die gefunden Lösung $x_0$ ein Maximum ist, indem wir $f''(x_0) <
0$ zeigen.

\begin{align*}
  &f'(x) \overset{!}{=} 0 \\
  \Leftrightarrow\quad& \frac{a}{(ax+1)^2} - \frac{1}{(x+1)^2} = 0 \\
  \Leftrightarrow\quad& \frac{a}{(ax+1)^2} = \frac{1}{(x+1)^2} \\
  \Leftrightarrow\quad& a(x+1)^2 = (ax+1)^2 \\
  \Leftrightarrow\quad& a(x^2+2x+1) = a^2x^2+2ax+1 \\
  \Leftrightarrow\quad& ax^2+2ax+a = a^2x^2+2ax+1 \\
  \Leftrightarrow\quad& ax^2+a = a^2x^2+1 \\
  \Leftrightarrow\quad& ax^2-a^2x^2 = 1-a \\
  \Leftrightarrow\quad& x^2(a-a^2) = 1-a \\
  \Leftrightarrow\quad& x^2 = \frac{1-a}{a(1-a)} \\
  \Leftrightarrow\quad& x^2 = \frac{1}{a} \\
  \Leftrightarrow\quad& x = \pm \frac{1}{\sqrt{a}} \\
\end{align*}

Da $x > 0$, fällt die negative Lösung weg. Die Extremstelle liegt also bei $x_0
= \frac{1}{\sqrt{a}}$.

Die zweite Ableitung ist:

\begin{align*}
  &f''(x) = \frac{2}{(x+1)^3} - \frac{2a^2}{(ax+1)^3}
\end{align*}

Zu zeigen:

\begin{align*}
  &f''(x_0) < 0\\
  \Leftrightarrow\quad&\frac{2}{(\frac{1}{\sqrt{a}}+1)^3} - \frac{2a^2}{(\frac{a}{\sqrt{a}}+1)^3} < 0 \\
  \Leftrightarrow\quad&\frac{2}{(\frac{1}{\sqrt{a}}+1)^3} < \frac{2a^2}{(\sqrt{a}+1)^3} \\
  \Leftrightarrow\quad&\frac{2}{(\frac{1}{\sqrt{a}}(1+\sqrt{a}))^3} < \frac{2a^2}{(\sqrt{a}+1)^3} \\
  \Leftrightarrow\quad&\frac{2}{\sqrt{a}^3 (1+\sqrt{a})^3} < \frac{2a^2}{(\sqrt{a}+1)^3} \\
  \Leftrightarrow\quad&\frac{2}{\sqrt{a}^3} < 2a^2 \\
  \Leftrightarrow\quad&\frac{1}{\sqrt{a}^3} < a^2 \\
  \Leftrightarrow\quad& 1 < a^3 \\
  \Leftrightarrow\quad& a^3 > 1\\
  \Leftrightarrow\quad& a > 1
\end{align*}

Die Stelle ist also nur ein Maximum, wenn $a > 1$. Dies ist gegeben.

Jetzt berechnen wir noch den Maximalwert $f(x_0)$:

\begin{align*}
  f(x_0) &= \frac{1}{1+\frac{1}{\sqrt{a}}} - \frac{1}{1+\frac{a}{\sqrt{a}}} \\
         &= \frac{\sqrt{a}}{\sqrt{a}+1} - \frac{1}{1+\sqrt{a}} \\
         &= \frac{\sqrt{a} - 1}{\sqrt{a}+1}
\end{align*}

Für alle $x \in ]0, +\infty[$ ist $f(x)$ also kleiner oder gleich diesem Wert.
Für $x=0$ haben wir den Zusammenhang gezeigt. Damit gilt insgesamt:

\begin{align*}
  \frac{1}{1+x} - \frac{1}{1+ax} \leq \frac{\sqrt{a} - 1}{\sqrt{a}+1} \qquad \forall\; x \geq 0, a > 1
\end{align*}

Genau das war zu beweisen.

\section*{Aufgabe 8.4}

\subsection*{Teil a}
Sei

\begin{align*}
  f_1(x) = \frac{t}{1+t}, \quad f_2(x) = \ln(1+t), \quad f_3(x) = t
\end{align*}

Wir zeigen, dass die Funktionen alle vom selben Punkt bei $t=0$ starten, aber
jeweils langsamer steigen.

\begin{align*}
  f_1(0) &= \frac{0}{1} = 0 \\
  f_2(0) &= \ln(1) = 0 \\
  f_3(0) &= 0
\end{align*}

Somit $f_1(0) = f_2(0) = f_3(0)$.

Unter der Vorraussetzung $t>0$:

\begin{align*}
&f_1'(x) < f_2(x) < f_3'(x) \\
\Leftrightarrow\quad&\frac{1}{(1+t)^2} < \frac{1}{1+t} < 1 \\
\Leftrightarrow\quad&\frac{1}{(1+t)} < 1 < 1+t
\end{align*}

Die zweite Hälfte ist wahr. Die erste Hälfte ist wahr, weil die zweite Hälfte
wahr ist.

Wenn also alle Funktionen vom selben Punkt starten, aber jeweils langsamer
steigen, muss die gegebene Ungleichung gelten.

\subsection*{Teil b}

Sei $f(x) = \left(1+\frac{1}{x}\right)^x$. Zu beweisen: $f'(x) > 0 \forall x > 0$.

Wir berechnen die Ableitung:

\begin{align*}
  f'(x) &= \frac{d}{dx} \left(1+\frac{1}{x}\right)^x \\
        &= \frac{d}{dx} e^{x \ln\left(1+\frac{1}{x}\right)} \\
        &= \left(1+\frac{1}{x}\right)^x \frac{d}{dx} x \ln\left(1+\frac{1}{x}\right)  \\
        &= \left(1+\frac{1}{x}\right)^x \left( \ln\left(1+\frac{1}{x}\right) + x \frac{d}{dx} \ln\left(1+\frac{1}{x}\right) \right) \\
        &= \left(1+\frac{1}{x}\right)^x \left( \ln\left(1+\frac{1}{x}\right) + x \frac{1}{1+\frac{1}{x}} \frac{d}{dx} \left(1+\frac{1}{x}\right) \right) \\
        &= \left(1+\frac{1}{x}\right)^x \left( \ln\left(1+\frac{1}{x}\right) + x \frac{1}{1+\frac{1}{x}} \frac{-1}{x^2} \right) \\
        &= \left(1+\frac{1}{x}\right)^x \left( \ln\left(1+\frac{1}{x}\right) - \frac{1}{1+\frac{1}{x}} \frac{1}{x} \right) \\
        &= \left(1+\frac{1}{x}\right)^x \left( \ln\left(1+\frac{1}{x}\right) - \frac{1}{x+1} \right) \\
\end{align*}

Zu zeigen: Wenn $x > 0$, dann gilt:

\begin{align*}
  &\underbrace{\left(1+\frac{1}{x}\right)^x}_{> 1} \left( \ln\left(1+\frac{1}{x}\right) - \frac{1}{x+1} \right) > 0 \\
  \Leftrightarrow\quad& \ln\left(1+\frac{1}{x}\right) - \frac{1}{x+1} > 0 \\
  \Leftrightarrow\quad& \ln\left(1+\frac{1}{x}\right) > \frac{1}{x+1} \\
  \Leftrightarrow\quad& \underbrace{(x+1)}_{>1} \ln\left(1+\frac{1}{x}\right) > 1 \\
  \Leftrightarrow\quad& \ln\underbrace{\left(1+\frac{1}{x}\right)}_{>1} > 1
\end{align*}

Der natürliche Logarithmus von einer Zahl die größer als 1 ist, ist immer
größer als 1. Also stimmt die Aussage. Womit bewiesen ist, dass
$\left(1+\frac{1}{x}\right)$ im Intervall $]0,+\infty[$ streng monoton wächst.

\end{document}
