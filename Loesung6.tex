\documentclass[a4paper,german,12pt,smallheadings]{scrartcl}
\usepackage[T1]{fontenc}
\usepackage[utf8]{inputenc}
\usepackage{babel}
\usepackage{tikz}
\usepackage{geometry}
\usepackage{amsmath}
\usepackage{amssymb}
\usepackage{float}
%\usepackage{wrapfig}
\usepackage{pdflscape}
\pagenumbering{gobble}
\usepackage[thinspace,thinqspace,squaren,textstyle]{SIunits}
\restylefloat{table}
\geometry{a4paper, top=15mm, left=20mm, right=40mm, bottom=20mm, headsep=10mm, footskip=12mm}
\linespread{1.5}
\setlength\parindent{0pt}
\begin{document}
\begin{center}
\bfseries % Fettdruck einschalten
\sffamily % Serifenlose Schrift
\vspace{-40pt}
Analysis I, Sommersemester 2013, 6. Übungsblatt \\
Luis Herrmann und Markus Fenske, Tutor: Adam Schienle
\vspace{-10pt}
\end{center}

\section*{Aufgabe 6.3}

Wir berechnen

\begin{align*}
  \qquad&\lim_{x \to 0} \frac{\sqrt[n]{1+x} - 1}{x}
\end{align*}

indem wir mit $u^n = 1+x$ substituieren. Für $x \to 0$ geht dann $u^n \to 1$, also $u \to 1$:

\begin{align*}
  = &\lim_{u \to 1} \frac{\sqrt[n]{u^n} - 1}{u^n -1} \\
  = &\lim_{u \to 1} \frac{u - 1}{u^n -1}
\end{align*}

Wir wissen, dass $\frac{u^n-1}{u-1} = \sum_0^{n-1} u^n$ und setzen das ein.

\begin{align*}
  = &\lim_{u \to 1} \frac{1}{\sum_0^{n-1} u^n} \\
  = &\lim_{u \to 1} \frac{1}{u^0 + u^1 + \dots + u^n} \\
  = &\frac{1}{\underbrace{1 + 1 + \dots + 1}_{n\;\text{Einsen}}} \\
  = &\frac{1}{n}
\end{align*}

\end{document}
