\documentclass[a4paper,german,12pt,smallheadings]{scrartcl}
\usepackage[T1]{fontenc}
\usepackage[utf8]{inputenc}
\usepackage{babel}
\usepackage{geometry}
\usepackage{pdfpages}
\usepackage{tikz}
\usetikzlibrary{calc,intersections,fadings}
\usepackage{wrapfig}
\usepackage[fleqn]{amsmath}
\usepackage{amssymb}
\usepackage{float}
\usepackage{enumerate}
\usepackage{listings} % Source code
\usepackage{lscape} % landscape
\usepackage{commath} % http://tex.stackexchange.com/questions/14821/whats-the-proper-way-to-typeset-a-differential-operator
\usepackage{cancel}
\usepackage[fleqn]{mathtools}
% Number only referenced equations
%\mathtoolsset{showonlyrefs}

%\usepackage{wrapfig}
\usepackage{siunitx}
\sisetup{separate-uncertainty=true,locale=DE}

% http://tex.stackexchange.com/questions/38818/best-way-to-denote-an-angle-in-tikz
\newcommand\markangle[6][red]{% [color] {X} {origin} {Y} {mark} {radius}
  % filled circle: red by default
  \begin{scope}
    \path[clip] (#2) -- (#3) -- (#4);
    \fill[color=#1,fill opacity=0.5,draw=#1,name path=circle]
    (#3) circle (#6mm);
  \end{scope}
  % middle calculation
  \path[name path=line one] (#3) -- (#2);
  \path[name path=line two] (#3) -- (#4);
  \path[%
  name intersections={of=line one and circle, by={inter one}},
  name intersections={of=line two and circle, by={inter two}}
  ] (inter one) -- (inter two) coordinate[pos=.5] (middle);
  % bissectrice definition
  \path[%
  name path=bissectrice
  ] (#3) -- (barycentric cs:#3=-1,middle=1.2);
  % put mark
  \path[
  name intersections={of=bissectrice and circle, by={middleArc}}
  ] (#3) -- (middleArc) node[pos=1.3] {#5};
  }

% New command for color underlining
\usepackage{xcolor}
\newcommand\invisiblesection[1]{%
    \refstepcounter{section}%
      \addcontentsline{toc}{section}{\protect\numberline{\thesection}#1}%
        \sectionmark{#1}}
\newsavebox\MBox
\newcommand\colul[2][red]{{\sbox\MBox{$#2$}%
  \rlap{\usebox\MBox}\color{#1}\rule[-1.2\dp\MBox]{\wd\MBox}{0.5pt}}}

\restylefloat{table}
\geometry{a4paper, top=15mm, left=20mm, right=10mm, bottom=20mm, headsep=10mm, footskip=12mm}
\linespread{1.5}
\setlength\parindent{0pt}
\DeclareMathOperator{\Tr}{Tr}
\DeclareMathOperator{\Var}{Var}
\begin{document}

\begin{titlepage}
\newcommand{\HRule}{\rule{\linewidth}{0.5mm}}

\begin{center}
  \textsc{\Large Physikalisches Grundpraktkum 1}
  \HRule\\[0.4 cm]
  {\huge \bfseries Gleichmäßig beschleunigte Drehbewegungen}
  \HRule\\[0.4 cm]

  \begin{minipage}{0.65\textwidth}
  \begin{flushleft}
    Markus Fenske \texttt{<iblue@zedat.fu-berlin.de>} \\
    Paul Rahmann \texttt{<paulrahmann@zedat.fu-berlin.de>}
  \end{flushleft}
  \end{minipage}
  \hfill
  \begin{minipage}{0.30\textwidth}
  \begin{flushright}
    Tutor: Christian Hindermann \\
    Versuchstag: 6. Juni 2014
  \end{flushright}
  \end{minipage}

  \vspace{1cm}

  \tableofcontents


  %{\large \today}
  \vfill
\end{center}
\newpage

\end{titlepage}

\allowdisplaybreaks % Seitenumbrüche in Formeln erlauben
\begin{center}
\bfseries % Fettdruck einschalten
\sffamily % Serifenlose Schrift
\vspace{-40pt}
Physikalisches Grundpraktikum 2, Wintersemester 2014/2015

Markus Fenske \texttt{<iblue@zedat.fu-berlin.de>}

Alexandra Krause \texttt{<alexandra.krause2@gmail.com>}

Fabry-Perot-Etalon
\vspace{-10pt}
\end{center}

\section{Physikalische Grundlagen}

Ein Fabry-Perot-Etalon besteht aus einem optisch durchlässigen Medium, das von
zwei parallelen teilverspiegelten Grenzflächen eingeschlossen wird. Wenn ein
Strahl unter dem Winkel $\alpha$ auf das Etalon trifft, wird er zwischen den
Spiegeln hin und her reflektiert, wobei ein Anteil das Etalon verlässt.

% FIXME: Abstand von caption zu groß
% FIXME: Grafik verständlicher machen (Lichtstrahl sollte Pfeile haben)
% FIXME: Dranschreiben, was was ist ("Licht", "Spiegel", ...)
\begin{figure}[h]
  \centering
  \begin{tikzpicture}
    \pgfmathsetmacro{\angle}{25}
    \pgfmathsetmacro{\distance}{4}
    \pgfmathsetmacro{\width}{17}
    \pgfmathsetmacro{\rayAt}{1}
    \pgfmathsetmacro{\reflectionWidth}{\distance*tan(\angle)}

    \pgfmathsetmacro{\perpX}{2*\reflectionWidth*sin(\angle)*sin(\angle)}
    \pgfmathsetmacro{\perpY}{2*\reflectionWidth*cos(\angle)*sin(\angle)}

    % Koordinaten
    \coordinate                        (RayStart)        at
      (-\width/2+\rayAt-    \reflectionWidth/2, -\distance   );
    \coordinate                        (RayIn)           at
      (-\width/2+\rayAt,                        -\distance/2 );
    \coordinate[label=above left:$A$]  (RayBounceTop1)   at
      (-\width/2+\rayAt+     \reflectionWidth,   \distance/2 );
    \coordinate[label=below:$C$]       (RayBounceBottom) at
      (-\width/2+\rayAt+ 2  *\reflectionWidth,  -\distance/2 );
    \coordinate[label=above right:$D$] (RayBounceTop2)   at
      (-\width/2+\rayAt+ 3  *\reflectionWidth,   \distance/2 );
    \coordinate                        (RayEnd)          at
      (-\width/2+\rayAt+ 3.5*\reflectionWidth,   0           );
    \coordinate                        (RayTrans1)          at
      (-\width/2+\rayAt +1.5*\reflectionWidth,   \distance   );
    \coordinate                        (RayTrans2)          at
      (-\width/2+\rayAt +3.5*\reflectionWidth,   \distance   );
    \coordinate[label=above left:$B$]  (RayTrans1Perp)      at
      (-\width/2+\rayAt+     \reflectionWidth+\perpX,   \distance/2+\perpY);

    \coordinate (RayInPerp) at (-\width/2+\rayAt, -\distance);
    \coordinate (RayBounceTop1Perp) at (-\width/2+\rayAt+\reflectionWidth, 0);
    \coordinate (RayBounceBottomPerp) at (-\width/2+\rayAt+ 2  *\reflectionWidth, 0);
    \coordinate[label=above:$E$] (RayBounceBottomTop) at (-\width/2+\rayAt+ 2  *\reflectionWidth, \distance/2);

    % Oberer und unterer Spiegel
    \draw (-\width/2,\distance/2)  -- (\width/2,\distance/2);
    \draw (-\width/2,-\distance/2) -- (\width/2,-\distance/2);

    % Strahl(-\width/2+\rayAt +1.5*\reflectionWidth,   \distance   );
    % FIXME: Fadein, Fadeout
    \draw[color=orange] (RayStart) -- (RayIn);
    \draw[color=orange] (RayIn) --(RayBounceTop1) -- (RayBounceBottom) -- (RayBounceTop2);
    \draw[color=orange] (RayBounceTop2) -- (RayEnd);

    % Outputstrahlen
    % FIXME: Fadeout
    \draw[color=yellow, ->] (RayBounceTop1) -- (RayTrans1);
    \draw[color=yellow, ->] (RayBounceTop2) -- (RayTrans2);

    % Gangunterschied
    \draw[dashed, color=black!50] (RayBounceTop2) -- (RayTrans1Perp);

    % Linie Einfallswinkel und Einfallswinkel
    \draw[color=black!50] (RayInPerp) -- (RayIn);
    \markangle[orange]{RayStart}{RayIn}{RayInPerp}{$\alpha$}{10};

    % Einfallswinkel bei A
    \draw[color=black!50] (RayBounceTop1) -- (RayBounceTop1Perp);
    \markangle[orange]{RayIn}{RayBounceTop1}{RayBounceTop1Perp}{$\alpha$}{10};

    % Winkel untere Reflexion (C)
    \draw[color=black!50] (RayBounceBottom) -- (RayBounceBottomPerp);
    \draw[dashed, color=black!50] (RayBounceBottom) -- (RayBounceBottomTop);
    \markangle[orange]{RayBounceTop1}{RayBounceBottom}{RayBounceBottomTop}{$\alpha$}{10};

    % Winkel bei D
    \markangle[orange]{RayBounceTop1}{RayBounceTop2}{RayTrans1Perp}{$\alpha$}{10};


    % Breite des Etalons
    \draw[<->] (\width/2-0.5, \distance/2) -- node[left] {$d$} (\width/2-0.5, -\distance/2);

  \end{tikzpicture}
  \caption{Strahlengang im Etalon}
  \label{fig:rays}
\end{figure}

Für die Interferenz relevant ist der Gangunterschied zwischen den Strahlen
(hier exemplarisch für die ersten beiden Strahlen dargestellt). Dieser ist
genau die Differenz der zurückgelegten Strecke der beiden Strahlen, also

\begin{equation}
  \delta = \overline{AC} + \overline{CD} - \overline{AB}
\end{equation}

Aus der Grafik erhält man geometrisch

\begin{equation}
  \cos \alpha = \frac{d}{\overline{AC}} \quad \Leftrightarrow \quad \overline{AC} = \frac{d}{\cos \alpha}
\end{equation}

Aufgrund des Reflexionsgesetzes (Einfallswinkel gleich Ausfallswinkel) gilt
außerdem

\begin{equation}
  \overline{AC} = \overline{CD}
\end{equation}

Wegen der Symmetrie der Dreiecke gilt
\begin{equation}
  \overline{AD} = 2 \overline{AE}
\end{equation}

Mit
\begin{equation}
  \tan \alpha = \frac{\overline{AE}}{d}
\end{equation}

erhalten wir
\begin{equation}
  \overline{AD} = 2 d \tan \alpha
\end{equation}

Mit
\begin{equation}
  \sin \alpha = \frac{\overline{AB}}{\overline{AD}}
\end{equation}

erhalten wir
\begin{equation}
  \overline{AB} = \overline{AD} \sin \alpha = 2d \tan \alpha \cos \alpha
\end{equation}

Somit
\begin{equation}
  \delta = \frac{2d}{\cos \alpha} - 2d \tan \alpha \sin \alpha
\end{equation}

Es gilt außerdem
\begin{align*}
  \frac{1}{\cos} - \tan \sin &= \frac{1}{\cos} - \frac{\sin^2}{\cos} \\
                             &= \frac{1}{\cos} \underbrace{\del{1-\sin^2}}_{\mathrlap{\sin^2 + \cos^2 = 1 \; \Leftrightarrow \; \cos^2 = 1 - \sin^2}} \\
                             &= \cos
\end{align*}

Somit ist der Gangunterschied

\begin{equation}
  \delta = 2d \cos \alpha
\end{equation}

Für konstruktive Interferenz muss dies ein Vielfaches der Wellenlänge betragen,
also

\begin{equation}
  2d \cos \alpha = n \lambda \quad \text{mit} \quad n \in \mathbb{N}
\end{equation}













\end{document}
