\documentclass[a4paper,german,12pt,smallheadings]{scrartcl}
\usepackage[T1]{fontenc}
\usepackage[utf8]{inputenc}
\usepackage{babel}
\usepackage{geometry}
\usepackage{pdfpages}
\usepackage{tikz}
\usepackage{wrapfig}
\usepackage[fleqn]{amsmath}
\usepackage{amssymb}
\usepackage{float}
\usepackage{enumerate}
\usepackage{listings} % Source code
\usepackage{lscape} % landscape
\usepackage{commath} % http://tex.stackexchange.com/questions/14821/whats-the-proper-way-to-typeset-a-differential-operator
\usepackage{cancel}
\usepackage[fleqn]{mathtools}
% Number only referenced equations
%\mathtoolsset{showonlyrefs}

%\usepackage{wrapfig}
\usepackage{siunitx}
\sisetup{separate-uncertainty=true,locale=DE}

% New command for color underlining
\usepackage{xcolor}
\newcommand\invisiblesection[1]{%
    \refstepcounter{section}%
      \addcontentsline{toc}{section}{\protect\numberline{\thesection}#1}%
        \sectionmark{#1}}
\newsavebox\MBox
\newcommand\colul[2][red]{{\sbox\MBox{$#2$}%
  \rlap{\usebox\MBox}\color{#1}\rule[-1.2\dp\MBox]{\wd\MBox}{0.5pt}}}

\restylefloat{table}
\geometry{a4paper, top=15mm, left=20mm, right=10mm, bottom=20mm, headsep=10mm, footskip=12mm}
\linespread{1.5}
\setlength\parindent{0pt}
\DeclareMathOperator{\Tr}{Tr}
\DeclareMathOperator{\Var}{Var}
\begin{document}

\begin{titlepage}
\newcommand{\HRule}{\rule{\linewidth}{0.5mm}}

\begin{center}
  \textsc{\Large Physikalisches Grundpraktkum 1}
  \HRule\\[0.4 cm]
  {\huge \bfseries Gleichmäßig beschleunigte Drehbewegungen}
  \HRule\\[0.4 cm]

  \begin{minipage}{0.65\textwidth}
  \begin{flushleft}
    Markus Fenske \texttt{<iblue@zedat.fu-berlin.de>} \\
    Paul Rahmann \texttt{<paulrahmann@zedat.fu-berlin.de>}
  \end{flushleft}
  \end{minipage}
  \hfill
  \begin{minipage}{0.30\textwidth}
  \begin{flushright}
    Tutor: Christian Hindermann \\
    Versuchstag: 6. Juni 2014
  \end{flushright}
  \end{minipage}

  \vspace{1cm}

  \tableofcontents


  %{\large \today}
  \vfill
\end{center}
\newpage

\end{titlepage}

\allowdisplaybreaks % Seitenumbrüche in Formeln erlauben
\begin{center}
\bfseries % Fettdruck einschalten
\sffamily % Serifenlose Schrift
\vspace{-40pt}
Physikalisches Grundpraktikum 2, Wintersemester 2014/2015

Markus Fenske \texttt{<iblue@zedat.fu-berlin.de>}

Alexandra Krause \texttt{<alexandra.krause2@gmail.com>}

Mikroskop, Tutor: Kevin Madsen
\vspace{-10pt}
\end{center}

\section{Physikalische Grundlagen}

% Einführung: Parallelstrahl, Mittelpunktstrahl, Brennpunktstrahl

Herleitung der Linsengleichung $\frac{1}{f} = \frac{1}{g} + \frac{1}{b}$.

% Aufgabe 1

% FIXME: Abbildung
Die Dreiecke bestehend aus den Strecken $G$, $g$ und dem halben
Mittelpunktstrahl und aus $B$, $b$ und dem halben Mittelpunktstrahl sind
ähnlich, denn sie stimmen in allen Winkeln überein (rechter Winkel zwischen $G$
und $g$ sowie $B$ und $b$ und Wechselwinkel am Mittelpunktstrahl). Somit müssen
die Streckenverhältnisse gleich sein. Wir definieren darüber den
Abbildungsmaßstab $\beta$.

\begin{equation}
  \beta = \frac{B}{G} = \frac{b}{g}
  \label{eq:beta}
\end{equation}

Auf der Seite des Bildes ergibt wird ein Dreieck gebildet aus der der Strecke
$f$, der Gegenstandshöhe $G$ in der Linsenebene und dem halben
Brennpunktstrahl. Das Dreieck aus $B$, $b-f$ und der anderen Hälfte des
Brennpunktstrahls ist dazu aus den selben Gründen wie oben ähnlich. Somit gilt

\begin{equation}
  \frac{B}{G} = \frac{b-f}{f}
\end{equation}

Zusammengesetzt ergibt sich
\begin{equation}
  \frac{b}{g} = \frac{b-f}{f}
  \label{eq:inter}
\end{equation}

Durch Division durch $b$ und Addition von $\frac{1}{b}$ erhält man die
Abbildungsgleichung

\begin{equation}
  \frac{1}{f} = \frac{1}{b} + \frac{1}{g}
\end{equation}

% Aufgabe 2

Für einen festen Abstand $e$ zwischen Gegenstand und Bild gilt $e = g+b$.

% http://www.wolframalpha.com/input/?i=solve+1%2Ff+%3D+1%2Fg+%2B+1%2F%28e-g%29+for+g
Durch Einsetzen in die Linsengleichung ergeben sich also nur zwei Lösungen für
ein scharfes Bild. % FIXME: Herleitung (Umstellen, PQ-Formel)

\begin{equation}
  g = \pm \frac{1}{2} \del{\sqrt{e^2 - 4ef} + e}
\end{equation}

Dabei handelt es sich einmal um eine Vergrößerung ($\beta > 1$) und einmal um
eine Verkleinerung ($\beta < 1$). % FIXME: Warum? beta = (e-g)/g für beide
% Lösungen ausrechnen!

% Aufgabe 3
% FIXME: Einführung dicke Linsen
Die geometrischen Überlegungen gelten genauso für dicke Linsen, denn entfernt
man die Hauptebenen, ergeben sich die selben geometrischen Zusammenhänge.
Problematisch ist nur, dass sich aufgrund des unbekannten Hauptebenenabstands
die Gegenstands- und die Bildweite nicht mehr bestimmen lassen. Der Brennwert
muss sich also aus einem anderen Zusammenhang ergeben. Messbar sind $g-b$ und
$\beta$, so dass daraus der Brennwert berechnet werden muss.

Durch Einsetzen von \ref{eq:beta} in \ref{eq:inter} erhalten wir

\begin{equation}
  \beta = \frac{b-f}{f}
\end{equation}

Umgestellt nach $f$:
\begin{equation}
  f = \frac{b}{\beta + 1}
\end{equation}

Durch Erweitern mit $\beta - 1$:
\begin{equation}
  f = \frac{b(\beta - 1)}{1-\beta^2}
\end{equation}

Durch Erweitern mit $\frac{1}{\beta} = \frac{g}{b}$:

\begin{equation}
  f = \frac{g - b}{\frac{1}{\beta} - \beta}
\end{equation}

Somit ist die Brennweite nur durch $g-b$ und $\beta$ ausgedrückt.

Aus $e = g + i + b$ erhalten wir

\begin{align*}
  i &= e - (g+b) \\
    &= e - (g-b) \frac{g+b}{g-b} \\
    &= e - (g-b) \frac{(g+b)\beta}{(g-b)\beta} \\
    &= e - (g-b) \frac{b + \beta b}{b - \beta b} \\
    &= e - (g-b) \frac{\beta + 1}{\beta - 1}
\end{align*}

% Aufgabe 4: Zeug malen (Aus Skript übernehmen?)

% Aufgabe 5:
% FIXME: Einleitung Mikroskop
Analog zu oben kann man sich den Vergrößerungsfaktor des Objektivs geometrisch
überlegen.

% FIXME: Hier Grafik mit Hervorhebung der beiden Dreiecke.
Auf der Bildseite des Objektivs bildet die Strecke $f_1$ zusammen mit $G$ und
der Hälfte des Mittelpunktstrahls ein Dreieck das ähnlich ist zu dem Dreieck
aus $t$, $B$ und der anderen Hälfte des Mittelpunktstrahls. Somit ist

\begin{equation}
  \beta_\text{ob} = \frac{B}{G} = \frac{t}{f_1}
\end{equation}

Durch die Definition $\Gamma_\text{ob} := \beta_\text{ob}$ des
Vergrößerungsfaktors ist also

\begin{equation}
  \Gamma_\text{ob} = \frac{t}{f_1}
\end{equation}

% FIXME: Herleitung Okular. Akkomodation??? Unklar. Zusammenstümpern!


\end{document}
