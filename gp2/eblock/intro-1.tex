\documentclass[a4paper,german,12pt,smallheadings]{scrartcl}
\usepackage[T1]{fontenc}
\usepackage[utf8]{inputenc}
\usepackage{babel}
\usepackage{geometry}
\usepackage[fleqn]{amsmath}
\usepackage{amssymb}
\usepackage{float}
\usepackage{enumerate}
\usepackage{commath} % http://tex.stackexchange.com/questions/14821/whats-the-proper-way-to-typeset-a-differential-operator
\usepackage{cancel}

\usepackage[fleqn]{mathtools}
% Number only referenced equations
%\mathtoolsset{showonlyrefs}

%\usepackage{wrapfig}
\usepackage[thinspace,thinqspace,squaren,textstyle]{SIunits}
\usepackage{tikz}
\usepackage[europeanresistors]{circuitikz}

% New command for color underlining
\usepackage{xcolor}

\newsavebox\MBox
\newcommand\colul[2][red]{{\sbox\MBox{$#2$}%
  \rlap{\usebox\MBox}\color{#1}\rule[-1.2\dp\MBox]{\wd\MBox}{0.5pt}}}

\restylefloat{table}
\geometry{a4paper, top=15mm, left=20mm, right=10mm, bottom=20mm, headsep=10mm, footskip=12mm}
\linespread{1.5}
\setlength\parindent{0pt}
\DeclareMathOperator{\Tr}{Tr}
\DeclareMathOperator{\Var}{Var}
\begin{document}
\allowdisplaybreaks % Seitenumbrüche in Formeln erlauben
\begin{center}
\bfseries % Fettdruck einschalten
\sffamily % Serifenlose Schrift
\vspace{-40pt}
Physikalisches Grundpraktikum 2, Wintersemester 2014/2015

Markus Fenske \texttt{<iblue@zedat.fu-berlin.de>}

Ohmscher Widerstand, Tutor: Andreas Maier
\vspace{-10pt}
\end{center}
\section{Einführung}
Ziel des Versuches ist die Untersuchung von (ohmschen und nicht-ohmschen)
Widerständen und daraus aufgebauten Schaltungen. Dabei behandeln wir die
Widerstandskennlinie, Strom- und Spannungsteiler (belastet und unbelastet),
Innenwiderstände von Spannungsquellen und Messgeräten und darauf aufbauend die
strom- und spannungsrichtige Messung.

\section{Theoretische Grundlagen}
% TODO: Kirchhoffsche Regeln
% TODO: Herleitung Gesamtwiderstand von Reihen- und Parallelschaltung
% TODO: Kennlinie eines Widerstands (Ohm: linear, ansonsten nicht-linear)
% TODO: Spannungsteiler, Stromteiler
% TODO: Innenwiederstände von Strom- und Spannungsquellen sowie Messgeräten
% TODO: Ersatzschaltbilder
% TODO: Analog von Messgeräten (mit Ersatzschaltbildern)
% TODO: Strom- und spannungsrichtige Messung
% TODO: Berechnung Spannungsfehler (siehe Kursmaterial)

\end{document}
