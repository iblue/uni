\documentclass[a4paper,german,12pt,smallheadings]{scrartcl}
\usepackage[T1]{fontenc}
\usepackage[utf8]{inputenc}
\usepackage{babel}
\usepackage{geometry}
\usepackage[fleqn]{amsmath}
\usepackage{amssymb}
\usepackage{float}
\usepackage{enumerate}
\usepackage{commath} % http://tex.stackexchange.com/questions/14821/whats-the-proper-way-to-typeset-a-differential-operator
\usepackage{cancel}

\usepackage[fleqn]{mathtools}
% Number only referenced equations
%\mathtoolsset{showonlyrefs}

%\usepackage{wrapfig}
\usepackage[thinspace,thinqspace,squaren,textstyle]{SIunits}
\usepackage{tikz}
\usepackage[europeanresistors]{circuitikz}

% New command for color underlining
\usepackage{xcolor}

\newsavebox\MBox
\newcommand\colul[2][red]{{\sbox\MBox{$#2$}%
  \rlap{\usebox\MBox}\color{#1}\rule[-1.2\dp\MBox]{\wd\MBox}{0.5pt}}}

\restylefloat{table}
\geometry{a4paper, top=15mm, left=20mm, right=10mm, bottom=20mm, headsep=10mm, footskip=12mm}
\linespread{1.5}
\setlength\parindent{0pt}
\DeclareMathOperator{\Tr}{Tr}
\DeclareMathOperator{\Var}{Var}
\begin{document}
\allowdisplaybreaks % Seitenumbrüche in Formeln erlauben
\begin{center}
\bfseries % Fettdruck einschalten
\sffamily % Serifenlose Schrift
\vspace{-40pt}
Physikalisches Grundpraktikum 2, Wintersemester 2014/2015

Markus Fenske \texttt{<iblue@zedat.fu-berlin.de>}

Ohmscher Widerstand, Tutor: Andreas Maier
\vspace{-10pt}
\end{center}
\section{Einführung}
Ziel des Versuches ist die Untersuchung von (ohmschen und nicht-ohmschen)
Widerständen und daraus aufgebauten Schaltungen. Dabei behandeln wir die
Widerstandskennlinie, Strom- und Spannungsteiler (belastet und unbelastet),
Innenwiderstände von Messgeräten und darauf aufbauend die strom- und
spannungsrichtige Messung.

% TODO: Was machen wir hier eigentlich?

\section{Theoretische Grundlagen}

\subsection{Kirchhoffsche Regeln}
In der elektrischen Schaltungstechnik verwendet man die Kirchhoffschen
Regeln, um den Zusammenhang zwischen mehreren elektrischen Strömen bzw.
mehreren elektrischen Spannungen zu beschreiben. Sie bestehen aus zwei
grundlegenden Sätzen, die im Folgenden beschrieben werden sollen.

\subsubsection{Knotenregel}

\begin{figure}[H]
  \begin{center}
    \begin{circuitikz}
      \draw (0,0) to (4,0);
      \draw (0,0) to (-4,0);
      \draw (0,0) to (0,2);
      \draw (0,0) to (0,-2);
      \draw node[circ] at (0,0) {};
      \draw[<-] (2.5,0.4) -- ++(1.4,0)    node [midway,fill=white] {$I_1$};
      \draw[<-] (-2.5,0.4) -- ++(-1.5,0)  node [midway,fill=white] {$I_3$};
      \draw[<-] (0.35,0.6) -- ++(0,1.4)   node [midway,fill=white] {$I_2$};
      \draw[<-] (0.35,-0.6) -- ++(0,-1.4) node [midway,fill=white] {$I_4$};
    \end{circuitikz}
    \caption{Knotenregel}
  \end{center}
\end{figure}


Betrachten wir einen beliebigen Knoten innerhalb einer elektrischen
Schaltung, also einen Punkt in dem mehrere Leitungen elektrisch verbunden
sind, so ist klar, dass in diesem Punkt keine Ladung gespeichert werden
kann. Aufgrund der Ladungserhaltung muss also zufließende Ladung wieder
abfließen. Den Fluss von Ladungen definiert man als elektrischen Strom

\begin{equation}
  I := \frac{\dif Q}{\dif t}
\end{equation}

Soll der Zufluss von Ladungen also dem Abfluss von Ladungen entsprechend,
muss die Summe aller Ströme $I_1, I_2, \dots, I_n$ in den Knoten hinein und
aus dem Knoten herraus verschwinden.

\begin{equation}
  \sum_{n=1}^k I_n = 0
\end{equation}

Diese Erkenntnis bezeichnet man als Knotensatz oder auch 1. Kirchhoffsches
Gesetz.

Es gilt natürlich nur für Knoten, die elektrisch neutral bleiben. Wird zum
Beispiel ein Kondensator benutzt, so werden die Ladungen auf der
Kondensatorplatte gespeichert. Betrachtet man also nur eine Kondensatorplatte,
so muss zusätzlich der Verschiebungsstrom berücksichtigt werden. Da wir in
diesem Versuch nicht mit Kondensatoren oder Spulen arbeiten, sondern nur
statische Fälle betrachten, kann dies unberücksichtigt bleiben.

\subsubsection{Maschenregel}
\begin{figure}[H]
  \begin{center}
    \begin{circuitikz}
      \draw (0,0) to[R=$R_1$, v=$U_1$] (6,0)
                  to[R=$R_2$, v=$U_2$] (6,2)
                  to[R=$R_3$, v=$U_3$] (0,2)
                  to[R=$R_4$, v=$U_4$] (0,0);
    \end{circuitikz}
    \caption{Maschenregel}
  \end{center}
\end{figure}

Die 3. Maxwellsche Gleichung stellt den Zusammengang her zwischen der Änderung
des magnetischen Flusses durch eine Fläche $A$ und der elektrischen Zirkulation
auf dem Rand $\partial A$ dieser Fläche. Sie lautet in Integralschreibweise

\begin{equation}
  \oint\limits_{\partial A} \vec{E} \cdot \dif \vec{s} = - \iint\limits_{A} \frac{\partial \vec{B}}{\partial t} \cdot \vec{A}
\end{equation}


Bei Betrachtungen elektrischer Schaltkreise mit zeitlich konstanten Strömen
existieren keine zeitlich variablen Magnetfelder, also folgt:

\begin{equation}
  \oint\limits_{\partial A} \vec{E} \cdot \dif \vec{s} = 0
\end{equation}


\begin{figure}[H]
  \begin{center}
    \begin{tikzpicture}
      \draw plot[smooth, tension=.7] coordinates {(-3.5,0.5) (-3,2.5) (-1,3.5) (1.5,3) (4,3.5) (5,2.5) (5,0.5) (2.5,-2) (0,-0.5) (-3,-2) (-3.5,0.5)};
      \draw (-3.5,0.5) node[circle,fill,inner sep=2pt,label=left:$p_1$] {};
      \draw (1.5,3)    node[circle,fill,inner sep=2pt,label=above:$p_2$] {};
      \draw (4,3.5)    node[circle,fill,inner sep=2pt,label=above:$p_3$] {};
      \draw (5,0.5)    node[circle,fill,inner sep=2pt,label=right:$p_4$] {};
      \draw (0,-0.5)   node[circle,fill,inner sep=2pt,label=above:$p_5$] {};
      \draw (1,1)      node[label=$A$] {};
      \draw (-3,2.5)   node[label=above:$\partial A$] {};
    \end{tikzpicture}
    \caption{Aufteilung des Ringintegrals in Teilstücke}
  \end{center}
\end{figure}

Das Ringintegral lässt sich aufteilen in $n$ Pfade zwischen jeweils zwei
Punkten. Dabei seien $p_i$ Punkte auf der Randkurve $\partial A$ und $[p_{i},
p_{i+1}]$ ein Pfad zwischen zwei Punkten, dann gilt

\begin{equation}
  \partial A = [p_1, p_2] \cup [p_2, p_3] \cup \dots \cup [p_{n-1}, p_n] \cup [p_n, p_1]
\end{equation}

Damit wird das Integral zu

\begin{equation}
  \int\limits_{p_1}^{p_2} \vec{E} \cdot \dif \vec{s} +
  \int\limits_{p_2}^{p_3} \vec{E} \cdot \dif \vec{s} +
  \dots +
  \int\limits_{p_{n-1}}^{p_n} \vec{E} \cdot \dif \vec{s} +
  \int\limits_{p_{n}}^{p_1} \vec{E} \cdot \dif \vec{s}
  = 0
\end{equation}

Die elektrische Spannung zwischen zwei Punkten $A$ und $B$ ist definiert durch
\begin{equation}
  U_{AB} = \int\limits_A^B \vec{E} \cdot \dif \vec{s}
\end{equation}

Somit

\begin{equation}
  \underbrace{U_{p_1p_2}}_{=: U_1} +
  \underbrace{U_{p_2p_3}}_{=: U_2} +
  \dots +
  \underbrace{U_{p_{n-1}p_n}}_{=: U_{n-1}} +
  \underbrace{U_{p_np_1}}_{=: U_n} = 0
\end{equation}

Das bedeutet die Summe der Spannungen innerhalb einer Masche verschwindet.

\begin{equation}
  \sum_{n=1}^k U_n = 0
\end{equation}

Diese Regel ist bekannt als 2. Kirchhoffsches Gesetz oder als Maschenregel.

Sie gilt nur für zeitlich konstante Magnetfelder. Werden Spulen oder
Kondensatoren eingesetzt, kann nur der statische Fall betrachtet werden. Es
gibt jedoch Korrekturen für Wechselströme (komplexe Wechselstromrechnung).
Diese sind jedoch für diesen Versuch nicht relevant und werden daher nicht
betrachtet.

\subsection{Ohmsche Widerstände}

Der elektrische Widerstand $R$ ist in der Elektrotechnik ein Maß dafür, welche
Spannung $U$ notwendig ist, um einen bestimmten Strom $I$ durch einen Leiter
fließen zu lassen. Er ist definiert durch das \textit{Ohmsche Gesetz}

\begin{equation}
  R = \frac{U}{I}
\end{equation}

Wenn $R$ unabhängig von Strom und Spannung ist (also $R = \text{const.}$),
spricht man von einem \textit{ohmschen Widerstand}.

Im Folgenden wollen wir einfache Schaltungen aus mehreren ohmschen Widerständen
betrachten, um deren effektiven Gesamtwiderstand zu berechnen.

\subsubsection{Reihenschaltung}

% Schaltbild Reihenschaltung R_1 ... R_n mit Spannung U_0
\begin{figure}[H]
  \begin{center}
    \begin{circuitikz}
      \draw (0,0)
      to[V,v=$U_0$] (0,2)
      to[R=$R_1$, v=$U_1$] (2,2)
      to[R=$R_2$, v=$U_2$] (4,2);
      \draw [dashed] (4,2) -- (6,2);
      \draw (6,2)
      to[R=$R_n$, v=$U_n$] (8,2)
      to[short] (8,0)
      to[short] (0,0);
    \end{circuitikz}
    \caption{Reihenschaltung von Widerständen}
  \end{center}
\end{figure}

Wenn $n$ Widerstände $R_1, \dots, R_n$ in Reihe geschaltet sind (siehe
Abbildung), fällt über jedem eine bestimmte Spannung $U_i$ ab.

Gemäß der Maschenregel folgt sofort, dass die Summe der abfallenden Spannungen
gleich der Summe der von der Spannungsquelle erzeugten Spannung sein muss.

\begin{equation}
  \sum_{i=1}^n U_i = U_0
\end{equation}

Wenn wir das Ohmsche Gesetz für jeden Widerstand einsetzen, erhalten wir sofort

\begin{equation}
  \sum_{i=1}^n I_i R_i = U_0
\end{equation}

Aus der Knotenregel wird klar, dass alle Ströme durch die Widerstände gleich
sein müssen, also $I_i =: I_0$. Wir können diesen Term vor die Summe ziehen.

\begin{equation}
  I_0 \sum_{i=1}^n R_i = U_0
\end{equation}

Durch Umstellen ergibt sich

\begin{equation}
  \frac{U_0}{I_0} = \sum_{i=1}^n R_i
\end{equation}

Die linke Seite hat die Dimension eines Widerstandes (ohmsches Gesetz), also
schreiben wir dafür $R_\text{ges} = \frac{U_0}{I_0}$ und erhalten


\begin{equation}
  R_\text{ges} = \sum_{i=1}^n R_i
\end{equation}

In einer Reihenschaltung ergibt sich also der effektive Gesamtwiderstand durch
Addition der einzelnen Widerstände.

\subsubsection{Parallelschaltung}

\begin{figure}[H]
  \begin{center}
    \begin{circuitikz}
      \draw (8,0)
      to[V,v=$U_0$,i=$I_0$] (0,0);
      \draw (0,0)
      to[short] (0,2)
      to[R=$R_1$,i=$I_1$,v=$U_1$] (8,2)
      to[short, i=$I_\text{ges}$] (8,0);
      \draw (0,2)
      to[short] (0,4)
      to[R=$R_2$,i=$I_2$,v=$U_2$] (8,4)
      to[short] (8,2);
      \draw[dashed] (0,4) -- (0,6);
      \draw[dashed] (8,4) -- (8,6);
      \draw (0,6)
      to[R=$R_n$,i=$I_n$,v=$U_n$] (8,6);
    \end{circuitikz}
    \caption{Parallelschaltung}
  \end{center}
\end{figure}

Wenn $n$ Widerstände $R_1, R_2, \dots, R_n$ parallel geschaltet werden (siehe
Abbildung), teilen sich die Ströme aufgrund der Knotenregel auf.
Fließt durch die Spannungsquelle der Strom $I_0$, gilt folglich

\begin{equation}
  I_0 = \sum_{i=1}^n I_i
\end{equation}

Durch Einsetzen des umgestellten ohmschen Gesetzes $I = \frac{U}{R}$ erhält man

\begin{equation}
  I_0 = \sum_{i=1}^n \frac{U_i}{R_i}
\end{equation}

In dieser Schaltung kann man jeden Kreis von Stromquelle und Widerstand mit der
Maschenregel betrachten, also sind alle Spannungen an den Widerständen gleich.

\begin{equation}
  U_i = U_0 \quad\text{für alle}\quad i = 1, 2, \dots, n
\end{equation}

Somit erhält man
\begin{equation}
  I_0 = U_0 \sum_{i=1}^n \frac{1}{R_i}
\end{equation}

Durch Umstellen
\begin{equation}
  \frac{I_0}{U_0} = \sum_{i=1}^n \frac{1}{R_i}
\end{equation}

Anhand des Ohmschen Gesetzes sieht man, dass die linke Seite dem inversen
Widerstand entspricht. Der Gesamtwiderstand ist also

\begin{equation}
  \frac{1}{R_\text{ges}} = \sum_{i=1}^n \frac{1}{R_i}
\end{equation}

\subsection{Nicht-Ohmsche Widerstände}
\begin{figure}[H]
  \begin{center}
    \begin{tikzpicture}[domain=0:8]
        \draw[very thin,color=gray] (-0.1,-0.1) grid (7.9,3.9);
        \draw[->] (-0.2,0) -- (8.2,0) node[right] {$U$};
        \draw[->] (0,-0.2) -- (0,4.2) node[above] {$I$};
        \draw plot[id=x] function{0.9*x/2}
            node[right] {$R_1$};
        \draw plot[id=exp] function{0.055*exp(x/2)}
            node[right] {$R_2$};
        \draw plot[id=log] function{4*(1-exp(-(x/2)))}
            node[above] {$R_3$};
    \end{tikzpicture}
    \caption{Kennlinien von Widerständen}
  \end{center}
\end{figure}

Wenn $R = \text{const.}$ spricht man von einem Ohmschen Widerstand. Allerdings
muss dies nicht zwangsläufig der Fall sein. Widerstände können eine
Abhängigkeit von Strom und Spannung zeigen, so dass die Kennlinie nichtlinear
wird. In der obigen Abbildung ist nur $R_1$ ein Ohmscher Widerstand. Die
anderen beiden zeigen ein nichtlineares Verhalten und sind somit keine Ohmschen
Widerstände.

Jeder in der Technik benutze Widerstand zeigt ein gewisses nichtlineares
Verhalten, z.B. durch Erwärmung. Der Ohmsche Widerstand ist ein Idealbild, das
in erster Näherung gilt.


\subsection{Messgeräte}

Um Ströme und Spannungen zu Messen, benutzt man Messgeräte. Das Messgerät für
die Strommessung nennt man \textit{Amperemeter}, das Gerät für die
Spannungsmessung \textit{Voltmeter}.

\subsubsection{Amperemeter}

% Ersatzschaltbild Amperemeter
\begin{figure}[H]
  \begin{center}
    \begin{circuitikz}
      \draw (0,0)
      to[R=$R_I$] (2,0)
      to[ammeter] (4,0);
    \end{circuitikz}
    \caption{Ersatzschaltbild für das Amperemeter}
  \end{center}
\end{figure}

Um den Strom an einer bestimmten Stelle zu messen, wird ein Amperemeter in
Reihe geschaltet. Es zeigt den Stromfluss durch sich selbst an. Um den
Schaltkreis durch die Messung nicht zu beeinträchtigen hat das ideale
Amperemeter dabei den Widerstand $R=0$. Dies ist technisch nicht möglich, so
dass auch ein gutes Amperemeter über einen Innenwiderstand verfügt.

Im Schaltbild stellt man das reale Amperemeterdurch ein ideales Amperemeter und
einen Innenwiderstand $R_I$ dar (sog. Ersatzschaltbild, siehe Abbildung).

\subsubsection{Voltmeter}
\begin{figure}[H]
  \begin{center}
    \begin{circuitikz}
      \draw (0,4)
      to[short] (0,2)
      to[voltmeter=$U$] (4,2)
      to[short] (4,4);
      \draw (0,2)
      to[short] (0,0)
      to[R=$R_{I}$] (4,0)
      to[short] (4,2);
    \end{circuitikz}
    \caption{Ersatzschaltbild für das Voltmeter}
  \end{center}
\end{figure}

Um die Spannung an einer Stelle zu messen, wird ein Voltmeter parallel
geschaltet. Es zeigt die Potentialdifferenz an zwei Stellen an. Um die Messung
nicht zu beinträchtigen hat das ideale Amperemeter dabei einen unendlich großen
Innenwiderstand $R_I \to \infty$. Technisch ist dies leider nicht möglich, so
dass auch das Voltmeter durch ein Ersatzschaltbild dargestellt wird. Es
besteht aus einem idealen Voltmeter und einem Innenwiderstand $R_I$ (siehe
oben).

\subsection{Strom und Spannungsquelle}

In der Elektrotechnik unterscheidet man zwischen Strom- und Spannungsquellen.
Die ideale Spannungsquelle erhält unabhängig vom Stromverbrauch eine feste
Spannung aufrecht, während die ideale Stromquelle unabhängig von der Spannung
einen konstanten Strom liefert.

Da wir im Versuch nur mit Spannungsquellen arbeiten, ignorieren wir die
Stromquellen im Folgenden.

Damit die ideale Spannungsquelle auch bei hohen Strömen die Spannung aufrecht
erhalten kann, verfügt sie über einen verschwindend geringen Innenwiderstand.
In der Praxis ist dieser jedoch vorhanden, was bei hohen Strömen zu einem
Spannungseinbruch führt. Im Fall stabilisierter Netzgeräte bei geringen Strömen
ist dieser Effekt jedoch völlig vernachlässigbar, so dass wir nicht weiter
darauf eingehen wollen, da er experimentell nicht relevant ist.

\subsection{Strom- und Spannungsrichtige Messung}

Wird eine Spannung und gleichzeitig ein Strom gemessen, führen die nichtidealen
Messgeräte zu einer Verfälschung der Messung. Nur eine der beiden Größen können
korrekt gemessen werden. Dies führt zu zwei möglichen Schaltungsarten.

\subsubsection{Stromrichtige Messung}
\begin{figure}[H]
  \begin{center}
    \begin{circuitikz}
      \draw (0,0)
      to[V,v=$U_0$] (0,2)
      to[R=$R_{I,A}$] (2,2)
      to[ammeter=$I$, l=$I$] (4,2)
      to[R=$R_V$] (6,2)
      to[short] (6,0)
      to[short] (0,0);
      \draw (0,2)
      to[short] (0,4)
      to[voltmeter, l=$U$] (6,4)
      to[short] (6,2);
      \draw (0,4)
      to[short] (0,6)
      to[R=$R_{I,V}$] (6,6)
      to[short] (6,4);
    \end{circuitikz}
    \caption{Stromrichtige Messung an einem ohmschen Verbraucher $R_V$}
  \end{center}
\end{figure}

In der Abbildung soll der Strom und die Spannung des Verbrauchers $R_V$
gemessen werden. Bei der stromrichtigen Messung wird dazu das Voltmeter
parallel zu Amperemeter und Verbraucher geschaltet. Das Amperemeter führt durch
seinen Innenwiderstand zu einem Spannungsabfall, der vom Voltmeter mitgemessen
wird. Der Strom durch den Verbraucher wird nicht beeinflusst. Man nennt dies
daher die Stromrichtige Messung oder auch Spannungsfehlerschaltung.

% FIXME: Stimmt das so?
Gemäß der Maschenregel gilt für die Spannung am Voltmeter

\begin{equation}
  U_V = U_0
\end{equation}

Gemäß der Knotenregel fließen durch den Innenwiderstand $R_{I,A}$ des Amperemeters, durch das Amperemeter $I$ und
den Verbraucher $R_V$ der selbe Strom den wir $I_A$ nennen wollen. Gemäß
Maschenregel und Ohmschen Gesetz gilt dann

\begin{equation}
  U_0 = I_A R_{I,A} + I_A R_{V}
\end{equation}

Der gemessene Strom ist dann also

\begin{equation}
  I_A = \frac{U_0}{R_{I,A} + R_V}
\end{equation}

% TODO: Berechnung Spannungsfehler (siehe Kursmaterial)
% FIXME: Der Fehler berechnet sich wie?

\subsubsection{Spannungsrichtige Messung}
\begin{figure}[H]
  \begin{center}
    \begin{circuitikz}
      \draw (0,0)
      to[V,v=$U_0$] (0,2)
      to[ammeter=$I$] (2,2)
      to[R=$R_{I,A}$] (4,2)
      to[R=$R_V$] (4,0)
      to[short] (0,0);
      \draw (4,2)
      to[short] (6,2)
      to[voltmeter=$U$] (6,0)
      to[short] (4,0);
      \draw (6,2)
      to[short] (8,2)
      to[R=$R_{I,V}$] (8,0)
      to[short] (6,0);
    \end{circuitikz}
    \caption{Spannungsrichtige Messung an einem Verbraucher $R_V$}
  \end{center}
\end{figure}

In der obigen Abbildung soll ebenfalls wieder der Strom und die Spannung am
Verbraucher $R_V$ gemessen werden. Diesmal wird eine Spannungsrichtige Messung
benutzt. Dabei wird das Voltmeter direkt parallel zum Verbraucher geschaltet,
so dass die tatsächliche Spannung durch den Verbraucher ermittelt wird. Das
Amperemeter misst jedoch nicht nur den Strom durch den Verbraucher sondern auch
den Strom durch das Voltmeter.

Da der Innenwiderstand moderner Digitalvoltmeter in der Größenordnung von $10
\;\text{M}\Omega$ liegt, führt die Spannungsrichtige Messung bei Verbrauchern
mit kleinerem Widerstand zu keiner signifikanten Verfälschung, so dass diese
Art der Messung zu bevorzugen ist.

Der Strom durch das Amperemeter entspricht dem Gesamtstrom durch die Schaltung,
denn das Amperemeter ist in Reihe mit der Spannungsquelle geschaltet. Es ergibt
sich nach den obigen Regeln für Parallel- und Reihenschaltung ein
Gesamtwiderstand der Schaltung von

\begin{equation}
  R = R_{I_A} + \tfrac{1}{\tfrac{1}{R_V} + \tfrac{1}{R_{I,V}}}
\end{equation}

Nach Ohmschem Gesetz ist dann der gemessene Strom

\begin{equation}
  I_A = \tfrac{U_0}{R_{I_A} + \tfrac{1}{\tfrac{1}{R_V} + \tfrac{1}{R_{I,V}}}}
\end{equation}

Nach Maschenregel ist die gemessene Spannung die Spannung der Spannungsquelle
minus der Spannung die am ersten Widerstand abfällt

\begin{equation}
  U_V = U_0 - I_A R_{I,A}
\end{equation}

\subsubsection{Beispielrechnung}

% FIXME
Wir berechnen für die Messungen an Verbrauchern von $R_1 = 10 \, \Omega$ und
$R_2 = 50\,\text{k}\Omega$ die jeweiligen relativen Fehler. Wir nehmen einen
Innenwiderstand des Amperemeters von $R_{I,A} = 0{,}1\,\Omega$ und $R_{I,U} =
50\;\text{k}\Omega$ an.

Im Fall der Stromrichtgen Messung ergibt sich der relative Spannungsfehler aus
dem Quotienten der angezeigten und der wahren Spannung.
\begin{equation}
  \frac{U_A}{U_W} = ?
\end{equation}



% FIXME
\subsection{Potentiometer}
Ein Potentiometer ist ein einstellbarer Widerstand. Technisch betrachtet
besteht es aus einem Widerstandsdraht bei dem ein in seiner Position
veränderlicher Schleifkontakt eine Abzapfung in der Mitte bereit stellt. Des
Schaltzeichen lehnt sich an diesen Aufbau an.

\begin{figure}[H]
  \begin{center}
    \begin{circuitikz}
      \draw (0,0) to[pR] (2,0);
    \end{circuitikz}
    \caption{Schaltzeichen für das Potentiometer}
  \end{center}
\end{figure}

Wird nur eine Seite und die Anzapfung angeschlossen, benutzt man aus
stilistischen Gründen oft das Schaltzeichen für den variablen Widerstand.

\begin{figure}[H]
  \begin{center}
    \begin{circuitikz}
      \draw (0,0) to[vR] (2,0);
    \end{circuitikz}
    \caption{Schaltzeichen für den variablen Widerstand}
  \end{center}
\end{figure}


% FIXME: Von oben hier her schieben und Schaltung erklären
\subsection{Spannungsteiler}

Mithilfe des Potentiometers kann ein Spannungsteiler aufgebaut werden.
Technische Anwendungen kann man sich viele ausdenken, beispielsweise zum Dimmen
von Lampen\footnote{Tatsächlich wird das nicht mehr gemacht. Man benutzt eine
Phasenanschnittsteuerung mit Thyristorstellern, was übrigens auch das lästige
Summen älterer Dimmer erklärt.} oder zum Einstellen einer Lautstärke\footnote{Wobei
dann allerdings die Spannung benutzt wird, um den Basisstrom des
Verstärkertransistors zu regulieren, nicht die Ausgangsspannung direkt.}.


\begin{figure}[H]
  \begin{center}
    \begin{circuitikz}
      \draw (0,0)
      to[V,v=$U_0$] (0,4)
      to[short] (2,4)
      to[pR, n=POT] (2,0)
      to[short] (0,0);
      \draw (POT.wiper) -| (4,2);
      \draw (4,2)
      to[voltmeter, l=$U_1$] (4,4)
      to[short] (2,4);
      \draw (2,0)
      to[short] (4,0)
      to[voltmeter, l=$U_2$] (4,2);
    \end{circuitikz}
    \caption{Spannungsteiler}
  \end{center}
\end{figure}

Die obige Abbildung zeigt den Aufbau. Für die Spannungen gilt nach Maschenregel
\begin{equation}
  U_0 = U_1 + U_2
\end{equation}

Wenn das Potentiometer einen Widerstand $R$ hat und ein Skalenwert $s =
0\dots1$ eingestellt ist, ergeben sich die Teilwiderstände $sR$ und $(1-s)R$.
Damit gilt für die Teilspannungen

\begin{equation}
  U_1 = s IR \qquad U_2 = (1-s) IR
\end{equation}

Und für das Verhältnis
\begin{equation}
  \frac{U_1}{U_2} = \frac{s}{1-s}
\end{equation}

\subsection{Stromteiler}
\begin{figure}[H]
  \begin{center}
    \begin{circuitikz}
      \draw (0,0)
      to[V,v=$U_0$] (0,4)
      to[short] (3,4);
      \draw (0,0)
      to[short] (3,0)
      to[short] (3,1);
      \draw (3,1)
      to[short] (2,1)
      to[ammeter] (2,3)
      to[pR, n=POT] (4,3)
      to[ammeter]  (4,1)
      to[short]  (3,1);
      \draw (POT.wiper) -| (3,4);

      %to[pR, n=POT] (2,0)
      %to[short] (0,0);
      %\draw (POT.wiper) -| (4,2);

    \end{circuitikz}
    \caption{Stromteiler}
  \end{center}
\end{figure}

Analog zum Spannungsteiler bei dem die ``internen Widerstände'' des
Potentiometers in Reihe geschaltet werden, kann man diese auch parallel
schalten um einen Stromteiler aufzubauen. Da wir dies im Experiment nicht
nutzen, gehen wir nicht weiter darauf ein.



\subsection{Wheatstonesche Brückenschaltung}
\begin{figure}[H]
  \begin{center}
    \begin{circuitikz}[x=2.5cm,y=2.5cm]
    \draw
      (0,0) to[V, v=$U_0$] (0,2) -- (2,2)
      to[R=$R_1$,*-*] (1,1)
      to[R=$R_3$, *-*] (2,0) -- (0,0);

    \draw
      (2,2) to[R=$R_2$, *-*] (3,1)
      to[R=$R_4$, *-*] (2,0);

    \draw
      (1,1) to[R=$R_5$, *-*] (3,1);

    \begin{scope}[>=latex,thick,text=black]
    \draw[->,rounded corners=7pt]
       (0.6,1.9) -- (1.6,1.9) --
       (0.7,1) node[anchor=east]{$I_1$} --
       (1.6,0.1) -- (0.5,0.1);

    \draw[->]
      (1.7,1.3)  arc(220:-50:0.4 and 0.15)
      node[pos=0.5,above] {$I_2$};

    \draw[<-]
      (1.7,0.8) arc(-220:50:0.4 and 0.15)
      node[midway,above] {$I_3$};
    \end{scope}
    \end{circuitikz}
    \caption{Wheatstonesche Brückenschaltung}
  \end{center}
\end{figure}

% TODO: Wheatstonesche Brückenschaltung

\section{Aufgaben}

\begin{enumerate}
  \item Aufnahme der Spannungskurve an einem belasteten und unbelasteten Spannungsteiler
  \item Aufnahme der Widerstandskennlinie eines ohmschen Widerstandes, einer Glühbirne und eines Graphitstabs
  \item Messung eines Widerstands mithilfe der Wheatstoneschen Brückenschaltung
\end{enumerate}

\end{document}
