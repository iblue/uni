\documentclass[a4paper,german,12pt,smallheadings]{scrartcl}
\usepackage[T1]{fontenc}
\usepackage[utf8]{inputenc}
\usepackage{babel}
\usepackage{geometry}
\usepackage[fleqn]{amsmath}
\usepackage{amssymb}
\usepackage{float}
\usepackage{enumerate}
\usepackage{commath} % http://tex.stackexchange.com/questions/14821/whats-the-proper-way-to-typeset-a-differential-operator
\usepackage{cancel}

\usepackage[fleqn]{mathtools}
% Number only referenced equations
%\mathtoolsset{showonlyrefs}

%\usepackage{wrapfig}
\usepackage[thinspace,thinqspace,squaren,textstyle]{SIunits}
\usepackage{tikz}
\usepackage[europeanresistors]{circuitikz}

% New command for color underlining
\usepackage{xcolor}

\newsavebox\MBox
\newcommand\colul[2][red]{{\sbox\MBox{$#2$}%
  \rlap{\usebox\MBox}\color{#1}\rule[-1.2\dp\MBox]{\wd\MBox}{0.5pt}}}

\restylefloat{table}
\geometry{a4paper, top=15mm, left=20mm, right=10mm, bottom=20mm, headsep=10mm, footskip=12mm}
\linespread{1.5}
\setlength\parindent{0pt}
\DeclareMathOperator{\Tr}{Tr}
\DeclareMathOperator{\Var}{Var}
\begin{document}
\allowdisplaybreaks % Seitenumbrüche in Formeln erlauben
\begin{center}
\bfseries % Fettdruck einschalten
\sffamily % Serifenlose Schrift
\vspace{-40pt}
Physikalisches Grundpraktikum 2, Wintersemester 2014/2015

Markus Fenske \texttt{<iblue@zedat.fu-berlin.de>}

Wechselstromwiderstände, Tutor: Andreas Maier
\vspace{-10pt}
\end{center}
\section{Einführung}
Ziel des Versuches ist die Untersuchung von Spulen, Kondensatoren und Ohmschen
Widerständen in Wechselstromkreisen. Wir betrachten den Auf- und Entladevorgang
eines Kondensators und aus Spule und Kondensator aufgebaute Hoch-, Tief- und
Bandpässe.

\section{Theoretische Grundlagen}

% Weiter nach unten schieben.
\subsection{Kirchhoffsche Regeln}

\textbf{Die zweite Kirchhoffsche Regel gilt hier nicht.}

Eine weiterverbreitete Fehlannahme, die auch in vielen Lehrbüchern reproduziert
wird, ist die Behauptung, die Kirchhoffschen Regeln hätten in Stromkreisen mit
Spulen Gültigkeit. Da wir es jedoch in diesen Stromkreisen mit zeitlich
variablen Magnetischen Feldern zu tun haben, gilt nach 3. Maxwellschem Gesetz

\begin{equation}
  \oint\limits_{\partial A} \vec{E} \cdot \dif \vec{s} = - \iint\limits_{A} \frac{\partial \vec{B}}{\partial t} \cdot \vec{A}
\end{equation}

Die Herleitung der Maschenregel beruht darauf, dass der rechte Term
verschwindet. Das tut er aber nicht mehr. Die Maschenregel muss daher angepasst
werden. Dies soll anhand eines Beispiels geschehen und dann verallgemeinert
werden. Wir betrachten wieder das Ringintegral an einem konkreten Stromkreis


\begin{figure}[H]
  \begin{center}
    \begin{circuitikz}
      \draw (0,0)
      to[battery1] (0,2)
      to[R=$R$] (2,2)
      to[L=$L$] (4,2)
      to[short] (4,0)
      to[short] (0,0);
    \end{circuitikz}
    \caption{Beispielschaltung}
  \end{center}
\end{figure}

Wir werten nun das Ringintegral schrittweise im Uhrzeigersinn aus.

Die Spannungsquelle liefert das bereits bekannte Potential $U_0$. Dies fließt
gegen den Uhrzeigersinn (von Plus nach Minus), ist daher negativ zu
berücksichtigen.

Über dem Widerstand fällt nach Ohmschem Gesetz eine Potentialdifferenz
proportional zum durchfließenden Strom $I$ ab.

\begin{equation}
  U_R = IR
\end{equation}

Die Spule bestehe aus einem Draht mit vernachlässigbarem Widerstand. In der
Spule existiert also kein elektrisches Feld. Das Integral verschwindet dort.

\begin{equation}
  U_L = 0
\end{equation}

Die linke Seite ist somit
\begin{equation}
  -U_0 + IR
\end{equation}

Werten wir nun den rechten Teil der 3. Maxwell-Gleichung aus. Bekannt ist,
dass sich das Flächenintegral schreiben lässt als Zeitableitung des
magnetischen Flusses durch die Fläche (also durch den gesammten Schaltkreis):

\begin{equation}
  \iint\limits_{A} \frac{\partial \vec{B}}{\partial t} \cdot \dif \vec{A} = \frac{\dif \Phi_B}{\dif t}
\end{equation}

Die Fläche hat nun eine Form, die sich grafisch nur schwer darstellen lässt.
Sie ist begrenzt durch den gesamten Draht, folgt also insbesondere auch der
Spule in einer Form, die am ehesten an eine Wendeltreppe erinnert.

Zur Vereinfachung teilen wir die Flächen auf in den Teil innerhalb und
außerhalb der Spule. Nehmen wir an, die Spule liege in einer Ebene mit dem
Schaltkreis, so laufen die magnetischen Feldlinien gerade parallel zur Fläche
außerhalb der Spule, so dass das Integral an dieser Stelle verschwindet.

Wenn die Windungen der Spule unendlich nah zusammenrücken, lässt sich die
``Wendeltreppe'' betrachten als mehrere übereinanderliegende Kreisflächen
senkrecht zur Spulenachse. Der magnetische Fluss an dieser Stelle ist gerade
proportional zum Strom $I$. Die Proportionalitätskonstante $L$ nennen wir
Induktivität.

\begin{equation}
  \Phi_B = L I
\end{equation}

Somit ist

\begin{equation}
  \frac{\dif \Phi_B}{\dif t} = L \frac{\dif I}{\dif t}
\end{equation}

Zusammensetzen der einzelnen Teile liefert

\begin{equation}
  U_0 - IR = -L \frac{\dif I}{\dif t}
\end{equation}

Dies widerspricht der Maschenregel, denn in dieser würde der rechte Teil
verschwinden.

Führt man diese Betrachtung für mehrere Spulen durch, sieht man schnell ein,
dass für jede Spule ein Zusatzterm der Form $U = -L \frac{\dif I}{\dif t}$
eingefügt werden muss. Dies führt bei vielen Autoren zu der Idee, dass die
Kirchhoffsche Regel hier doch gilt. Die Ergebnisse sind dann korrekt, weil sie
die Fehlannahmen ``\textit{Die Kirchhoffsche Regel gilt für Spulen}'' und
``\textit{In der Spule existiert eine Spannungsquelle}'' gegeneinander
aufheben. Es führt aber schnell zu Widersprüchen, sobald man einzelne
Spulenwindungen betrachtet.\footnote{Für mehr Details lohnt sich das Ansehen
des Videos ``Kirchhoff's rule is for the birds'' von Prof. Walter Lewin
<http://youtu.be/gJSEgANEkOo>}


\subsection{Kondensator}

% TODO: Schematische Zeichnung Kondensator

Ein Kondensator ist ein elektronisches Bauteil das elektrische Ladung speichern
kann. In der einfachsten Bauform besteht es aus zwei Platten, die durch einen
gewissen Abstand getrennt sind, so dass sich beim Anlegen einer Spannung ein
elektrisches Feld zwischen den Platten ausbildet.

Durch das Einfügen eines Dielektrikums und beispielsweise das Ausbilden der
Kondensatorplatten als Folien und anschließendes Aufrollen führen zum kleineren
Bauformen. Eine Vielzahl an Materialien und Dielektrika ergeben Kondensatoren
ganz unterschiedlicher Eigenschaften, auf deren technische Details nicht weiter
eingegangen werden soll.

Die Ladung $Q$ die sich auf den Platten aufbaut ist proportional zur angelegten
Spannung $U$. Die für jeden Kondensator unterschiedliche
Proportionalitätskonstante $C$ bezeichnet man als Kapazität. Je größer die
Kapazität ist, desto mehr Ladung kann der Kondensator bei einer bestimmten
Spannung speichern.

% FIXME: Schaltzeichen und Erklärung (stellt 2 Platten dar, ...)

\begin{equation}
  Q = C U
  \label{eq:capa}
\end{equation}

% FIXME: Hierzwischen Kirchhoffsche Regeln
\subsection{Aufladekurve des Kondensators}

% FIXME: Zeichung Aufladekurve, Entladekurve

% FIXME: Schaltung

Zuerst einmal müssen wir untersuchen, ob die Kirchhoffschen Regeln noch gelten,
da wir diese nur für statische Fälle verifiziert haben. Da sich hier Spannungen
und Ströme ändern, müssen wir sie erneut begründen.

Die Knotenregel gilt hier, sofern wir nicht einzelne Platten des Kondensators
getrennt betrachten, denn sonst gilt die Ladungserhaltung im Knoten nicht mehr
(denn es wird Ladung im Kondensator gespeichert). Also betrachten wir nur den
Kondensator als ganzes.

Wenn wir annehmen, dass in erster Näherung keine zeitlich variablen
Magnetfelder \textit{durch} die Oberfläche des von der betrachteten Masche
umspannten Stromkreises fließen, gilt auch die Maschenregel. Dies ist hier
allerdings nicht der Fall. Der sich zeitlich ändernde Strom erzeugt ein
zeitlich variables Magnetfeld. Nehmen wir allerdings kleine Ströme
und eine geringe Induktivität (siehe hinten) des Stromkreises an, gilt die
Maschenregel in guter Näherung.

Angenommen der Kondensator mit der Kapazität $C$ sei in Reihe mit einem
ohmschen Widerstand $R$ an eine Spannungsquelle $U$ angeschlossen.







\end{document}
