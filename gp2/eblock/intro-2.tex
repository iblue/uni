\section{Wechselstromwiderstände}
Ziel des Versuches ist die Untersuchung von Spulen, Kondensatoren und Ohmschen
Widerständen in Wechselstromkreisen. Wir betrachten den Auf- und Entladevorgang
eines Kondensators und aus Spule und Kondensator aufgebaute Hoch-, Tief- und
Bandpässe.

\subsection{Spule und Induktivität}

Eine Spule besteht aus Draht, der auf einen Kern aufgewickelt wird. Der Kern
kann dabei aus magnetischen Materialien bestehen (muss aber nicht). Fließt ein
Strom durch den Draht, entsteht ein Magnetfeld. Das sich ändernde Magnetfeld
induziert widerrum eine Spannung in der Spule. Diese Eigenschaft nennt man
Selbstinduktivität. Die Proportionalitätskonstante $L$ gibt die Größe der
Indukvität an.

\begin{equation}
  U = L \frac{\dif I}{\dif t}
\end{equation}

\subsection{Kirchhoffsche Regeln}

\textbf{Die zweite Kirchhoffsche Regel gilt hier nicht.}

Eine weiterverbreitete Fehlannahme, die auch in vielen Lehrbüchern reproduziert
wird, ist die Behauptung, die Kirchhoffschen Regeln hätten in Stromkreisen mit
Spulen Gültigkeit. Da wir es jedoch in diesen Stromkreisen mit zeitlich
variablen Magnetischen Feldern zu tun haben, gilt nach 3. Maxwellschem Gesetz

\begin{equation}
  \oint\limits_{\partial A} \vec{E} \cdot \dif \vec{s} = - \iint\limits_{A} \frac{\partial \vec{B}}{\partial t} \cdot \vec{A}
\end{equation}

Die Herleitung der Maschenregel beruht darauf, dass der rechte Term
verschwindet. Das tut er aber nicht mehr. Die Maschenregel muss daher angepasst
werden. Dies soll anhand eines Beispiels geschehen und dann verallgemeinert
werden. Wir betrachten wieder das Ringintegral an einem konkreten Stromkreis


\begin{figure}[H]
  \begin{center}
    \begin{circuitikz}
      \draw (0,0)
      to[battery1] (0,2)
      to[R=$R$] (2,2)
      to[L=$L$] (4,2)
      to[short] (4,0)
      to[short] (0,0);
    \end{circuitikz}
    \caption{Beispielschaltung}
  \end{center}
\end{figure}

Wir werten nun das Ringintegral schrittweise im Uhrzeigersinn aus.

Die Spannungsquelle liefert das bereits bekannte Potential $U_0$. Dies fließt
gegen den Uhrzeigersinn (von Plus nach Minus), ist daher negativ zu
berücksichtigen.

Über dem Widerstand fällt nach Ohmschem Gesetz eine Potentialdifferenz
proportional zum durchfließenden Strom $I$ ab.

\begin{equation}
  U_R = IR
\end{equation}

Die Spule bestehe aus einem Draht mit vernachlässigbarem Widerstand. In der
Spule existiert also kein elektrisches Feld. Das Integral verschwindet dort.

\begin{equation}
  U_L = 0
\end{equation}

Die linke Seite ist somit
\begin{equation}
  -U_0 + IR
\end{equation}

Werten wir nun den rechten Teil der 3. Maxwell-Gleichung aus. Bekannt ist,
dass sich das Flächenintegral schreiben lässt als Zeitableitung des
magnetischen Flusses durch die Fläche (also durch den gesammten Schaltkreis):

\begin{equation}
  \iint\limits_{A} \frac{\partial \vec{B}}{\partial t} \cdot \dif \vec{A} = \frac{\dif \Phi_B}{\dif t}
\end{equation}

Die Fläche hat nun eine Form, die sich grafisch nur schwer darstellen lässt.
Sie ist begrenzt durch den gesamten Draht, folgt also insbesondere auch der
Spule in einer Form, die am ehesten an eine Wendeltreppe erinnert.

Zur Vereinfachung teilen wir die Flächen auf in den Teil innerhalb und
außerhalb der Spule. Nehmen wir an, die Spule liege in einer Ebene mit dem
Schaltkreis, so laufen die magnetischen Feldlinien gerade parallel zur Fläche
außerhalb der Spule, so dass das Integral an dieser Stelle verschwindet.

Wenn die Windungen der Spule unendlich nah zusammenrücken, lässt sich die
``Wendeltreppe'' betrachten als mehrere übereinanderliegende Kreisflächen
senkrecht zur Spulenachse. Der magnetische Fluss an dieser Stelle ist gerade
proportional zum Strom $I$. Die Proportionalitätskonstante $L$ nennen wir
Induktivität.

\begin{equation}
  \Phi_B = L I
\end{equation}

Somit ist

\begin{equation}
  \frac{\dif \Phi_B}{\dif t} = L \frac{\dif I}{\dif t}
\end{equation}

Zusammensetzen der einzelnen Teile liefert

\begin{equation}
  U_0 - IR = -L \frac{\dif I}{\dif t}
  \label{eq:ind}
\end{equation}

Dies widerspricht der Maschenregel, denn in dieser würde der rechte Teil
verschwinden.

Führt man diese Betrachtung für mehrere Spulen durch, sieht man schnell ein,
dass für jede Spule ein Zusatzterm der Form $U = -L \frac{\dif I}{\dif t}$
eingefügt werden muss. Dies führt bei vielen Autoren zu der Idee, dass die
Kirchhoffsche Regel hier doch gilt. Die Ergebnisse sind dann korrekt, weil sie
die Fehlannahmen ``\textit{Die Kirchhoffsche Regel gilt für Spulen}'' und
``\textit{In der Spule existiert eine Spannungsquelle}'' gegeneinander
aufheben. Es führt aber schnell zu Widersprüchen, sobald man einzelne
Spulenwindungen betrachtet.\footnote{Für mehr Details lohnt sich das Ansehen
des Videos ``Kirchhoff's rule is for the birds'' von Prof. Walter Lewin
<http://youtu.be/gJSEgANEkOo>}


\subsection{Kondensator}

% TODO: Schematische Zeichnung Kondensator

Ein Kondensator ist ein elektronisches Bauteil das elektrische Ladung speichern
kann. In der einfachsten Bauform besteht es aus zwei Platten, die durch einen
gewissen Abstand getrennt sind, so dass sich beim Anlegen einer Spannung ein
elektrisches Feld zwischen den Platten ausbildet.

Durch das Einfügen eines Dielektrikums und beispielsweise das Ausbilden der
Kondensatorplatten als Folien und anschließendes Aufrollen führen zum kleineren
Bauformen. Eine Vielzahl an Materialien und Dielektrika ergeben Kondensatoren
ganz unterschiedlicher Eigenschaften, auf deren technische Details nicht weiter
eingegangen werden soll.

Die Ladung $Q$ die sich auf den Platten aufbaut ist proportional zur angelegten
Spannung $U$. Die für jeden Kondensator unterschiedliche
Proportionalitätskonstante $C$ bezeichnet man als Kapazität. Je größer die
Kapazität ist, desto mehr Ladung kann der Kondensator bei einer bestimmten
Spannung speichern.

\begin{equation}
  Q = C U
  \label{eq:capa}
\end{equation}

\begin{figure}[H]
  \begin{center}
    \begin{circuitikz}
      \draw (0,0) to[C] (2,0);
    \end{circuitikz}
    \caption{Schaltzeichen des Kondensators}
  \end{center}
\end{figure}

Das Schaltzeichen des Kondensators ist oben abgedruckt. Es zeigt stilisiert die
beiden Platten des Kondensators.

% FIXME: Hierzwischen Kirchhoffsche Regeln
\subsection{Auf- und Entladekurve des Kondensators}

Zuerst einmal müssen wir untersuchen, ob die Kirchhoffschen Regeln noch gelten,
da wir diese nur für statische Fälle verifiziert haben. Da sich hier Spannungen
und Ströme ändern, müssen wir sie erneut begründen.

Die Knotenregel gilt hier, sofern wir nicht einzelne Platten des Kondensators
getrennt betrachten, denn sonst gilt die Ladungserhaltung im Knoten nicht mehr
(denn es wird Ladung im Kondensator gespeichert). Also betrachten wir nur den
Kondensator als ganzes.

Wenn wir annehmen, dass in erster Näherung keine zeitlich variablen
Magnetfelder \textit{durch} die Oberfläche des von der betrachteten Masche
umspannten Stromkreises fließen, gilt auch die Maschenregel. Dies ist hier
allerdings nicht der Fall. Der sich zeitlich ändernde Strom erzeugt ein
zeitlich variables Magnetfeld. Nehmen wir allerdings kleine Ströme
und eine geringe Induktivität (siehe hinten) des Stromkreises an, gilt die
Maschenregel in guter Näherung.

Angenommen der Kondensator mit der Kapazität $C$ sei in Reihe mit einem
ohmschen Widerstand $R$ an eine Spannungsquelle $U$ angeschlossen.

\begin{figure}[H]
  \begin{center}
    \begin{circuitikz}
      \draw (0,0)
      to[battery1=$U_0$] (0,2)
      to[R=$R$] (2,2)
      to[C=$C$] (4,2)
      to[short] (4,0)
      to[short] (0,0);
    \end{circuitikz}
    \caption{Beispielschaltung}
  \end{center}
\end{figure}

Dann gilt hier gemäß der Maschenregel (mit obiger Begründung)

\begin{equation}
  U_0 = U_R + U_C
\end{equation}

Die Spannung am Kondensator ergibt sich aus der Kapazitätsgleichung
(\ref{eq:capa}) als $U_C = \frac{Q}{C}$. Die Spannung am Widerstand aus dem
Ohmschen Gesetz als $U_R = IR$. Damit folgt


\begin{equation}
  U_0 = R I  + \frac{Q}{C}
\end{equation}

Da der Strom die zeitliche Änderung der Ladung ist ($I = \frac{\dif Q}{\dif
t}$) ergibt sich damit für die Ladung am Kondensator inhomogene
Differentialgleichung 1. Ordnung in der Zeit.

\begin{equation}
  U_0 = R \frac{\dif Q}{\dif t} + \frac{Q}{C}
\end{equation}

Diese Lösen wir durch Separation der Variablen. Wer gehen von einer
zeitunabhängigen Spannung $U_0$ aus und erhalten

\begin{equation}
  \int \dif t \frac{1}{R} = \int \dif Q \frac{1}{U_0 - \frac{Q}{C}}
\end{equation}

Mit der Substitution $f = U_0 - \frac{Q}{C}$ und damit $\dif Q = -C \dif f$ im
linken Integral:

\begin{equation}
  \int \dif t \frac{1}{R} = -C \int \dif f \frac{1}{f}
\end{equation}

Integration und gleichzeitiges additives Zusammenfassen der
Integrationskonstanten zu $k$:

\begin{align*}
  &\frac{t}{R} = -C \ln(f) + k \\
  \Leftrightarrow\quad&
  \frac{t}{R} = -C \ln\del{U_0 - \frac{Q}{C}} + k \\
  \Leftrightarrow\quad&
  -\frac{t}{RC} = \ln\del{U_0 - \frac{Q}{C}} - \frac{k}{C} \\
  \Leftrightarrow\quad&
  \frac{k}{C}-\frac{t}{RC} = \ln\del{U_0 - \frac{Q}{C}} \\
  \Leftrightarrow\quad&
  \underbrace{\exp\del{\frac{k}{C}}}_{=: k'} \exp \del{-\frac{t}{RC}} = U_0 - \frac{Q}{C} \\
  \Leftrightarrow\quad&
  \frac{Q}{C} = U_0 - k' \exp \del{-\frac{t}{RC}}
\end{align*}

Da $\frac{Q}{C}$ die Spannung am Kondensator ist, erhalten wir

\begin{equation}
  U_C(t) = U_0 - k' \exp \del{-\frac{t}{RC}}
\end{equation}

Um die Integrationskonstante zu bestimmen, beginnen wir mit einem vollständig
entladenen Kondensator ($U_C(0) = 0$). Durch Einsetzen erhalten wir direkt

\begin{equation}
  0 = U_0 - k' \quad \Rightarrow \quad k' = U_0
\end{equation}

Dann ist die Gleichung für die \textit{Aufladekurve} in ihrer bekannten Form

\begin{equation}
  U_C(t) = U_0 \del{1 - e^{-\frac{t}{RC}}}
\end{equation}

Für die Entladekurve gehen wir einfach von $U_0 = 0$ aus, während am
Kondensator eine Anfangsspannung $U_C(0) = U_C > 0$ anliegt. Dann erhalten wir
\begin{equation}
  U_C = 0 - k' \quad \Rightarrow k' = U_C
\end{equation}

Und somit die \textit{Entladekurve}

\begin{equation}
  U_C(t) = U_C \; e^{-\frac{t}{RC}}
\end{equation}

\begin{figure}[H]
  \begin{center}
    \begin{tikzpicture}[domain=0:8]
        \draw[very thin,color=gray] (-0.1,-0.1) grid (7.9,3.9);
        \draw[->] (-0.2,0) -- (8.2,0) node[right] {$t$};
        \draw[->] (0,-0.2) -- (0,4.2) node[above] {$U/U_0$};
        \draw plot[id=auf] function{4*(1-exp(-x))}
            node[below] {Aufladung};
        \draw plot[id=ent] function{4*exp(-x)}
            node[above] {Entladung};
    \end{tikzpicture}
    \caption{Auf- und Entladekurve des Kondensators}
  \end{center}
\end{figure}

Man kann die Zeitkonstante $\tau = \frac{1}{RC}$ definieren. Sie gibt an, wie
schnell eine bestimmte Aufladespannung erreicht wird, bzw. wie schnell die
Entladung erfolgt. Je größer Kapazität und Widerstand sind, desto langsamer
erfolgen Auf- und Entladung.

\subsection{Wechselstromkreise}

Widerständen, Spulen und Kondensatoren im Schaltkreis lassen sich Terme
zuordnen, die das Verhältnis zwischen abfallender Spannung und fließendem Strom
angeben. Für den Widerstand gilt das Ohmsche Gesetz ($R = \frac{U}{I}$).  Die
Gleichung für den Kondensator gilt über die Zeitableitung von (\ref{eq:capa})
und mit $I := \frac{\dif Q}{\dif t}$. Die Gleichung für die Spule erhält man
durch die Identifizierung der rechten Seite von (\ref{eq:ind}) mit einer
Spannung. Die Terme sind negativ, weil wir jeweils Spannungsabfälle an den
Bauteilen betrachten, wie weiter unten ausgeführt wird.

\begin{align}
  \text{Widerstand:}  \qquad & U_R = -R I_R \\
  \text{Kondensator:} \qquad & I_C = -C \frac{\dif U_C}{\dif t} \\
  \text{Spule:}       \qquad & U_L = -L \frac{\dif I_L}{\dif t}
\end{align}

Im Folgenden wollen wir nun cosinusförmige Wechselspannungen betrachten. Dabei
wird die Spannung zeitabhängig in der Form
\begin{equation}
  U(t) = U_0 \cos \omega t
\end{equation}

Aufgrund der Homogenität der Zeit betrachten wir keine Phasenverschiebung der
Spannung.

Man sieht jetzt, dass sich durch obige Gleichungen je nach Wahl des Bauteiles
Phasenverschiebungen zwischen Strom und Spannung ergeben. Am Widerstand tritt
keine Phasenverschiebung auf. Beim Kondensator erhält man durch Bildung der
Zeitableitung

\begin{equation}
  I_C = U_0 C \omega \sin(\omega t) = U_0 C \omega \cos \del{\omega t + \frac{\pi}{2}}
\end{equation}

Der Stromfluss am Kondensator ist also um eine Phase $\phi = 90^\circ$
verschoben. Die Größe des Stroms hängt dabei nicht nur von der Kapazität $C$
und der Spannung $U_0$ ab, sondern auch von der (Kreis-)Frequenz $\omega$ des
Wechselstroms.

Bei der Spule muss integriert werden:

\begin{align}
  U_0 \int \dif t \; \cos(\omega t) &= -L I_L \\
  U_0 \frac{1}{\omega} \sin(\omega t) &= -L I_L \\
  I_L &= -\frac{U_0}{\omega L} \sin(\omega t) \\
  I_L &= \frac{U_0}{\omega L} \cos \del{\omega t - \frac{\pi}{2}}
\end{align}

Hier hängt der Stromfluss vom \textit{Inversen} der Frequenz $\omega$ ab, auf
die Induktivität geht invers ein. Die Phasenverschiebung beträgt $\phi =
-90^\circ$.

\subsection{Impedanz}

Man kann nun wieder den Widerstand als Verhältnis von Spannung und Strom
betrachten, allerdings hat dieser nun zwei Komponenten, nämlich die Größe und
die Phasenverschiebung. Um beide zu erfassen, ordnet man dem Widerstand eine
komplexe Zahl zu und nennt ihn nun \textit{Impedanz}.

\begin{equation}
  Z := \frac{U}{I}
\end{equation}

Aus den obigen Herleitungen ergeben sich die folgenden Impedanzen:

\begin{align}
  \text{Widerstand:}  \qquad & Z_R = R \\
  \text{Kondensator:} \qquad & Z_C = -\frac{i}{\omega C} \\
  \text{Spule:}       \qquad & Z_L = i \omega L
\end{align}

Grenzwertbetrachtungen des Betrages $\envert{Z}$ mit $\omega \to 0$ liefern das
Verhalten im Falle von Gleichstrom, nämlich einen unendlich großen Widerstand
des Kondensators bzw.  eine leitende Spule.

Analog zu Widerstandsnetzwerken kann nun mit den Impedanzen gerechnet werden,
um Gesamtipedanzen zu erhalten. In der Reihenschaltung werden die Impedanzen
addiert, in der Parallelschaltung ergibt sich das Inverse der Gesamtimpedanz
als das Inverse ihrer Summen:

\begin{align}
  \text{Reihenschaltung:}   \qquad & Z = \sum_n Z_n\\
  \text{Parallelschaltung:} \qquad & \frac{1}{Z} = \sum_n \frac{1}{Z_n}
\end{align}

Die Phasenverschiebung ergibt sich durch Umrechnen der komplexen Zahl in ihre
Polardarstellung.

\subsection{Filter}
Die Frequenzabhängigkeit der Impedanz kann man sich zunutze machen, um
bestimmte Frequenzen auszufiltern. Der einfachste Anwendungsfall ist die
Glättung der Eingangsspannung eines elektronischen Geräts. Durch die
Parallelschaltung eines Kondensators zur Eingangsspannung werden hohe
Frequenzen kurzgeschlossen, durch die Reihenschaltung einer Spule in die
Eingangsspannung Wechselströme verringert (sog. Drossel).

Analog kann man Kondensatoren in Reihe schalten, um niedrige Frequenzen zu
blockieren, bzw. Spulen parallel, um niedrige Frequenzen kurzzuschließen.

In Kombination mit Widerständen lassen sich so verschiedene Filter aufbauen,
die wir im folgenden vorstellen und berechnen wollen.

\subsubsection{Hochpass}
\begin{figure}[H]
  \begin{center}
    \begin{circuitikz}
      \draw (0,0)
      to[vsourcesin=$U_0$] (0,4)
      to[short] (2,4)
      to[C=$C$] (2,2)
      to[R=$R\qquad\; U_R$] (2,0)
      to[short] (0,0);
      \draw (2,2) to[short] (4,2);
      \draw node[ocirc] (A) at (4,2) {};
      \draw (2,0) to[short] (4,0);
      \draw node[ocirc] (B) at (4,0) {};
    \end{circuitikz}
    \caption{Schaltung des Hochpasses}
  \end{center}
\end{figure}

Der Widerstand $R$ simuliert einen Verbraucher. In Reihe dazu ist ein
Kondensator $C$ angeschlossen. Der Kreis wird von einem Frequenzgenerator mit
einstellbarer Frequenz mit Spannung versorgt.

Die Gesamtimpedanz der Reihenschaltung aus Kondensator und Widerstand ist

\begin{equation}
  Z = R - \frac{i}{\omega C}
\end{equation}

Um die Ausgangsspannung abhängig von der Frequenz zu erhalten, benutzen wir die
Spannungsteilerformel. Da uns die Phasenverschiebung vorerst nicht
interessiert, nutzen wir die Beträge.

\begin{equation}
  \frac{U_R}{U_0} = \frac{R}{|Z|} = \frac{R}{\sqrt{R^2 + \frac{1}{\omega^2 C^2}}}
\end{equation}

Zu erwarten ist also folgender Kurvenverlauf


\begin{figure}[H]
  \begin{center}
    \begin{tikzpicture}[domain=0:8]
        \draw[very thin,color=gray] (-0.1,-0.1) grid (7.9,3.9);
        \draw[->] (-0.2,0) -- (8.2,0) node[right] {$\omega$};
        \draw[->] (0,-0.2) -- (0,4.2) node[above] {$U_R/U_0$};
        \draw plot[id=hochpass,samples=120] function{3/sqrt(0.5+1/x)}
            node[below] {Hochpass};
    \end{tikzpicture}
    \caption{Frequenzgang des Hochpasses}
  \end{center}
\end{figure}

\subsubsection{Tiefpass}
\begin{figure}[H]
  \begin{center}
    \begin{circuitikz}
      \draw (0,0)
      to[vsourcesin=$U_0$] (0,4)
      to[short] (2,4)
      to[L=$L$] (2,2)
      to[R=$R\qquad\; U_R$] (2,0)
      to[short] (0,0);
      \draw (2,2) to[short] (4,2);
      \draw node[ocirc] (A) at (4,2) {};
      \draw (2,0) to[short] (4,0);
      \draw node[ocirc] (B) at (4,0) {};
    \end{circuitikz}
    \caption{Schaltung des Tiefpasses}
  \end{center}
\end{figure}

Durch Ersetzen des Kondensators durch eine Spule lässt sich ein Tiefpass
aufbauen. Die Rechnung ist völlig analog und liefert


\begin{equation}
  \frac{U_R}{U_0} = \frac{R}{|Z|} = \frac{R}{\sqrt{R^2 + \omega^2 L^2}}
\end{equation}

\begin{figure}[H]
  \begin{center}
    \begin{tikzpicture}[domain=0:8]
        \draw[very thin,color=gray] (-0.1,-0.1) grid (7.9,3.9);
        \draw[->] (-0.2,0) -- (8.2,0) node[right] {$\omega$};
        \draw[->] (0,-0.2) -- (0,4.2) node[above] {$U_R/U_0$};
        \draw plot[id=tiefpass,samples=120] function{3/sqrt(0.5+6*x)}
            node[above] {Tiefpass};
    \end{tikzpicture}
    \caption{Frequenzgang des Tiefpasses}
  \end{center}
\end{figure}

\subsubsection{Bandpass}

\begin{figure}[H]
  \begin{center}
    \begin{circuitikz}
      \draw (0,0)
      to[vsourcesin=$U_0$] (0,6)
      to[short] (2,6)
      to[L=$L$] (2,4)
      to[C=$C$] (2,2)
      to[R=$R\qquad\; U_R$] (2,0)
      to[short] (0,0);
      \draw (2,2) to[short] (4,2);
      \draw node[ocirc] (A) at (4,2) {};
      \draw (2,0) to[short] (4,0);
      \draw node[ocirc] (B) at (4,0) {};
    \end{circuitikz}
    \caption{Schaltung des Bandpasses}
  \end{center}
\end{figure}

Setzt man nun einen Kondensator und eine Spule ein, erhält man (Rechnung
ebenfalls analog) einen Bandpass.


\begin{equation}
  \frac{U_R}{U_0} = \frac{R}{|Z|} = \frac{R}{\sqrt{R^2 + \del{\omega L - \frac{1}{\omega C}}^2}}
\end{equation}

\begin{figure}[H]
  \begin{center}
    \begin{tikzpicture}[domain=0:8]
        \draw[very thin,color=gray] (-0.1,-0.1) grid (7.9,3.9);
        \draw[->] (-0.2,0) -- (8.2,0) node[right] {$\omega$};
        \draw[->] (0,-0.2) -- (0,4.2) node[above] {$U_R/U_0$};
        \draw plot[id=bandpass,samples=120] function{3/sqrt(0.5+(x-1/x)**2)}
            node[above] {Bandpass};
    \end{tikzpicture}
    \caption{Frequenzgang des Bandpasses}
  \end{center}
\end{figure}

Man sieht hier, dass der Bandpass eine Maximalfrequenz aufweist, die besonders
gut durchgelassen wird. Diese nennt man die Resonanzfrequenz. Man könnte sie
durch Ableiten und Nullsetzen der Gleichung erhalten. Da die Auswertung des
Bandpasses sowieso nicht Teil der Aufgabenstellung ist, belassen wir es dabei.

\subsection{Aufgaben}

\begin{enumerate}
  \item Messung der Auf- und Entladekurve des Kondensators
  \item Messung des Frequenzganges eines Hoch-, Tief- und Bandpasses.
\end{enumerate}
