\documentclass[a4paper,german,12pt,smallheadings]{scrartcl}
\usepackage[T1]{fontenc}
\usepackage[utf8]{inputenc}
\usepackage{babel}
\usepackage{geometry}
\usepackage[fleqn]{amsmath}
\usepackage{amssymb}
\usepackage{float}
\usepackage{enumerate}
\usepackage{commath} % http://tex.stackexchange.com/questions/14821/whats-the-proper-way-to-typeset-a-differential-operator
\usepackage{cancel}
\usepackage{pdfpages}

\usepackage[fleqn]{mathtools}
% Number only referenced equations
%\mathtoolsset{showonlyrefs}

%\usepackage{wrapfig}
%\usepackage[thinspace,thinqspace,squaren,textstyle]{SIunits}
\usepackage{tikz}
\usepackage[europeanresistors]{circuitikz}

\usepackage{siunitx}
\sisetup{separate-uncertainty=true,locale=DE}

% New command for color underlining
\usepackage{xcolor}

\newsavebox\MBox
\newcommand\colul[2][red]{{\sbox\MBox{$#2$}%
  \rlap{\usebox\MBox}\color{#1}\rule[-1.2\dp\MBox]{\wd\MBox}{0.5pt}}}

\restylefloat{table}
\geometry{a4paper, top=15mm, left=20mm, right=10mm, bottom=20mm, headsep=10mm, footskip=12mm}
\linespread{1.5}
\setlength\parindent{0pt}
\DeclareMathOperator{\Tr}{Tr}
\DeclareMathOperator{\Var}{Var}
\begin{document}
\allowdisplaybreaks % Seitenumbrüche in Formeln erlauben
\begin{center}
\bfseries % Fettdruck einschalten
\sffamily % Serifenlose Schrift
\vspace{-40pt}
Physikalisches Grundpraktikum 2, Wintersemester 2014/2015

Markus Fenske \texttt{<iblue@zedat.fu-berlin.de>}

E-Block, Tutor: Andreas Maier
\vspace{-10pt}
\end{center}
\section{Auswertung}

\subsection{Aufgaben}
\begin{enumerate}
  \item \textbf{Lade-/Entladekurve Kondensator}
    \begin{enumerate}
      \item  Bestimmung der Zeitkonstanten und Kapazität des Kondensators und
        Vergleich mit den berechneten Werten
      \item Diskussion des Einflusses der Frequenz der Rechteckspannung auf die
        Lade-/Entladekurven
    \end{enumerate}
  \item \textbf{Diode}
    \begin{enumerate}
      \item Bestimmung der Schwellenspannung sowie des (differentiellen)
        Wiederstandes im Durchlassbereich
      \item Zusatzaufgabe: Bestimmung des Idealitätsfaktors $n$ bei einer
        Temperatur von $20\,^\circ C$.
    \end{enumerate}
  \item \textbf{Transistor}
    \begin{enumerate}
      \item Bestimmung des differentiellen Ausgangswiderstandes im
        Durchlassbereich für Basisströme von 30, 60, 90 und 120 $\mu$A
      \item Bestimmen des Gleichstromverstärkungsfaktors
      \item Bestimmung des Arbeits- und Vorwiderstands aus dem aufgenommenen
        Kannlinienfeld für einer Verstärkerschaltung bei $U_\text{CE} = 6
        \text{V}$.
    \end{enumerate}
\end{enumerate}

\subsection{Lade-/Entladekurve des Kondensators}

Umstellen und Logarithmieren der Auf- und Entladekurve führt zu den
linearisierten Messgleichungen

\begin{equation}
  \ln \frac{U_C}{U_0} = -\frac{t}{RC} \qquad \text{(Entladekurve)}
\end{equation}

und

\begin{equation}
  \ln \del{1 - \frac{U_C}{U_0}} = -\frac{t}{RC} \qquad \text{(Aufladekurve)}
\end{equation}

eine entsprechende logarithmische Auftragung und linearer Fit (siehe Plot)
liefert direkt die Zeitkonstanten (Aufladung und Entladung)

\begin{equation}
  \tau = \num{1.65+-0.11} \; \text{ms}
\end{equation}

und

\begin{equation}
  \tau = \num{1.722+-0.065} \; \text{ms}
\end{equation}

Die beiden Werte sind verträglich. Das Endergebnis der Zeitkonstante ergibt
sich als Mittelung $(a+b)/2$ und dem Fehler (per Gauß) $\Delta = \sqrt{\Delta
a^2 + \Delta b^2}/2$.

\begin{equation}
  \tau = \num{1.69\pm0.07} \; \text{ms}
\end{equation}

Der theoretische Wert der Zeitkonstante ist $\tau =
\num{18.0+-0.9}\,\text{k}\Omega \cdot 0{,}1\,\mathrm{\mu} \text{F} =
\num{1.8+-0.9} \, \text{ms}$. Der Widerstand wurde nicht gemessen, eine
Minimalabschätzung des Fehlers ergibt sich aus ergibt sich aus der bei
Kohleschichtwiderständen üblichen Fertigungstoleranz von 5 \%. Der Fehler des
Kondensators ist unbekannt, wird daher aufgrund der Minimalabschätzung
vernachlässigt.

Der experimentell ermittelte Wert der Zeitkonstante ist mit dem theoretischen
Wert identisch.

Zur Bestimmung der Frequenz des Kondensators wird einfach durch den Widerstand
geteilt. Es gilt

\begin{equation}
  C = \frac{\tau}{R}
\end{equation}

mit dem Fehler
\begin{equation}
  \Delta C = \sqrt{
    \del{\frac{\partial C}{\partial R} \Delta R}^2 +
    \del{\frac{\partial C}{\partial \tau} \Delta \tau}^2
  }
  = \sqrt{
    \frac{\tau^2 \Delta R^2}{R^4} + \frac{\Delta \tau^2}{R^2}
  }
\end{equation}

Somit ist die Kapazität (als Endergebnis):
\begin{equation}
  C = \num{0.094+-0.007} \;\mu\text{F}
\end{equation}

Der Wert ist mit dem angegebenen Wert von $C = 0{,}01 \; \mu\text{F}$
identisch. Er zeigt außerdem, dass die Toleranzabschätzung von 5 \% korrekt
war.

Die Frequenz der Rechteckspannung hat einen Einfluss auf die
Lade/Entladekurve. Bei der verwendeten Frequenz von 50 Hz scheinen noch keine
Auswirkungen aufzutreten, wie die Übereinstimmung der experimentellen Resultate
mit den theoretischen Werten zeigen, jedoch kommen folgende Möglichkeiten in
Frage.

Abschwächung der End/Anfangsspannung: Ist die Periode geringer, wird der
Kondensator nicht lange genug geladen um sich der Endspannung anzunähern. Man
kann dann Spannung beim Phasenwechsel nicht mehr als näherungsweise $U_0$
betrachten. Dies würde zu einem systematischen Fehler führen, sofern man nicht
Endspannung und $U_0$ am Oszilloskop vergleicht. Da beiden Spannungen
gleichzeitig betrachtet werden müssen, nimmt dann jedoch die Spannung am
Kondensator einen geringeren Platz auf dem Oszilloskop ein, was die
Ablesegenauigkeit verringert.

Hochfrequenzeffekte: Bei höheren Frequenzen treten möglicherweise
Hochfrequenzeffekte auf. Zu berüchsichtigen wäre die Induktivität der
Verbindungskabel, die Repolarisationszeit des Dielektrikums des Kondensators
und eventuelle Phasenverschiebungen (lässt sich theoretisch eventuell durch
Entwicklung des Rechtecksignals in eine Fourierreihe und dann Behandlung der
einzelnen Frequenzen im Rahmen der komplexen Wechselstromrechnung), sowie
Reflektionen an Kabeln etc. Da diese Themen den Umfang des Grundpraktikums
übersteigen, werde ich darauf nicht weiter eingehen.

\subsection{Diode}

Die Schwellenspannung ergibt sich durch Verlängerung des scheinbar geradlinigen
Teils der Kennlinie. Da die Kennlinie der Shockley-Gleichung entspricht,
existiert die Schwellspannung mathematisch überhaupt nicht. Die Steigung geht
gegen 1 (ersichtlich durch Grenzwertbildung der Ableitung der
Shockley-Gleichung). Da die Gleichung selber $U$-Werte von unendlich erreicht,
ergibt die senkrechte Projektion auf die x-Achse auch einen unendlichen Wert.
Der differentielle Widerstand wird ebenfalls unendlich. Um dennoch eine
Schwellenspannung und einen entsprechenden Fehler zu berechnen, führe ich eine
lineare Regression auf den letzten 5 Messwerten durch (siehe Plot). Dazu nutze
ich die Werte der spannungrichtigen Messung.

Durch Ablesen der Projektion auf die x-Achse ergibt sich eine Schwellenspannung
von

\begin{equation}
  U_S = \num{0.72+-0.03} \text{V}
\end{equation}

Eine andere Definition der Schwellenspannung ist die Spannung, bei der der
Strom 1 mA erreicht. Auch dies lässt sich ablesen und führt auf eine
Schwellenspannung von
\begin{equation}
  U_S = \num{0.60\pm0.01} \text{V}
\end{equation}

Die beiden Werte sind nicht verträglich. Dies liegt daran, dass die
Schwellenspannung messabhängig definiert ist.

Die Berechnung eines differentiellen Widerstands ist ebenso messabhängig. Er
ergibt sich aus der Steigung der Regressionsgerade. Diese ist nach Rundung

\begin{equation}
  1/R_D = \num{990\pm590} / k \Omega
\end{equation}

Den Widerstand erhält man durch Bildung des Inversen. Der Fehler ist dann
$\Delta R_D = \frac{\Delta(1/R_D)}{R_D^2}$.

Das Endergebnis lautet
\begin{equation}
  R_D = \num{1.0+-0.6} \,\Omega
\end{equation}

Die geringe Genauigkeit liegt in der messabhängigen Definition der
Schwellenspannung begründet.

Zusatzaufgabe: Nicht bearbeitet.




\end{document}
