\documentclass[a4paper,german,12pt,smallheadings]{scrartcl}
\usepackage[T1]{fontenc}
\usepackage[utf8]{inputenc}
\usepackage{babel}
\usepackage{geometry}
\usepackage{tikz}
\usepackage{wrapfig}
\usepackage[fleqn]{amsmath}
\usepackage{amssymb}
\usepackage{float}
\usepackage{enumerate}
\usepackage{listings} % Source code
\usepackage{lscape} % landscape
\usepackage{commath} % http://tex.stackexchange.com/questions/14821/whats-the-proper-way-to-typeset-a-differential-operator
\usepackage{cancel}
\usepackage[fleqn]{mathtools}
% Number only referenced equations
%\mathtoolsset{showonlyrefs}

%\usepackage{wrapfig}
\usepackage{siunitx}
\sisetup{separate-uncertainty=true,locale=DE}

% New command for color underlining
\usepackage{xcolor}
\newcommand\invisiblesection[1]{%
    \refstepcounter{section}%
      \addcontentsline{toc}{section}{\protect\numberline{\thesection}#1}%
        \sectionmark{#1}}
\newsavebox\MBox
\newcommand\colul[2][red]{{\sbox\MBox{$#2$}%
  \rlap{\usebox\MBox}\color{#1}\rule[-1.2\dp\MBox]{\wd\MBox}{0.5pt}}}

\restylefloat{table}
\geometry{a4paper, top=15mm, left=20mm, right=10mm, bottom=20mm, headsep=10mm, footskip=12mm}
\linespread{1.5}
\setlength\parindent{0pt}
\DeclareMathOperator{\Tr}{Tr}
\DeclareMathOperator{\Var}{Var}
\begin{document}

\begin{titlepage}
\newcommand{\HRule}{\rule{\linewidth}{0.5mm}}

\begin{center}
  \textsc{\Large Physikalisches Grundpraktkum 1}
  \HRule\\[0.4 cm]
  {\huge \bfseries Gleichmäßig beschleunigte Drehbewegungen}
  \HRule\\[0.4 cm]

  \begin{minipage}{0.65\textwidth}
  \begin{flushleft}
    Markus Fenske \texttt{<iblue@zedat.fu-berlin.de>} \\
    Paul Rahmann \texttt{<paulrahmann@zedat.fu-berlin.de>}
  \end{flushleft}
  \end{minipage}
  \hfill
  \begin{minipage}{0.30\textwidth}
  \begin{flushright}
    Tutor: Christian Hindermann \\
    Versuchstag: 6. Juni 2014
  \end{flushright}
  \end{minipage}

  \vspace{1cm}

  \tableofcontents


  %{\large \today}
  \vfill
\end{center}
\newpage

\end{titlepage}

\allowdisplaybreaks % Seitenumbrüche in Formeln erlauben
\begin{center}
\bfseries % Fettdruck einschalten
\sffamily % Serifenlose Schrift
\vspace{-40pt}
Physikalisches Grundpraktikum 2, Wintersemester 2014/2015

Markus Fenske \texttt{<iblue@zedat.fu-berlin.de>}

Alexandra Krause \texttt{<alexandra.krause2@gmail.com>}

Optische Spektroskopie, Tutor: Madsen
\vspace{-10pt}
\end{center}

\section{Physikalische Grundlagen}

Untersucht werden soll die spektrale Zerlegung von Licht an einem Prisma und
einem Beugungsgitter.

\subsection{Brechung am Prisma}

Wechselt ein Lichtstrahl das Medium, gilt

\begin{equation}
  n_1 \sin(\delta_1) = n_2 \sin(\delta_2)
\end{equation}

dabei sind $n_1$, $n_2$ die Brechungsindizes der jeweiligen Medien, $\delta_1$
der Einfallswinkel und $\delta_2$ der Winkel des gebrochenen Strahls. Somit
tritt eine Ablenkung am Wechsel des Mediums auf.

Der Brechungsindex eines optischen Mediums ist dabei nicht nur vom Medium,
sondern auch von der Wellenlänge des Lichts abhängig, somit

\begin{equation}
  n_1(\lambda) \sin(\delta_1) = n_2(\lambda) \sin(\delta_2)
\end{equation}

\subsection{Prisma}

Ein Prisma ist ein durchsichtiges optisches Bauteil in Form eines Prismas, das
zur Spektralzerlegung und Ablenkung von Licht benutzt werden kann.

% TODO: Theoretische Herleitung erforderlich?
Für den Sonderfall, dass ein Lichtstrahl das Prisma parallel zu dessen Basis
durchquert gilt, wenn $\epsilon$ der Winkel an der Spitze des Prismas ist und
$\alpha$ der Einfallswinkel zum Lot und $\beta$ der Austrittswinkel zum Lot,
dann ergibt sich der gesamte Brechwinkel $\gamma$ durch

\begin{equation}
  \gamma = 2 \del{\alpha - \beta}
\end{equation}

gegeben.

Die beiden Austrittswinkel $\beta$ spannen zusammen mit dem dritten Winkel
$\delta$ ein Dreieck auf. Wegen Innenwinkelsatz gilt $\delta = 180^\circ -
\epsilon$. Aufgrund der Winkelsumme ich Dreieck ist somit

\begin{align*}
  2 \beta + \delta &= 180^\circ \\
  2 \beta + 180^\circ - \epsilon = 180^\circ \\
  \beta = \frac{\epsilon}{2}
\end{align*}

Durch Einsetzen der beiden Gleichungen ineinander erhält man

\begin{equation}
  \alpha = \frac{\gamma + \epsilon}{2}
\end{equation}

\subsection{Auflösungsvermögen des Prismas}

Um getrennte Spektrallinien zu erhalten, wird das Licht durch einen Einzelspalt
geschickt. Dabei treten Beugungserscheinungen auf. Linien können erst dann als
getrennt betrachtet werden, wenn der Abstand derart ist, dass das
Beugungsmaximum der einen Linie mit dem ersten Beugungsminimum der anderen
Linie zusammenfällt (Rayleigh-Kriterium).

Um dies quantitativ zu Bestimmen, entwickeln wir den Brechnungsindex in eine
Taylorreihe:

\begin{equation}
  n(\lambda + \Delta \lambda) \approx n + \frac{\dif n}{\dif \lambda} \Delta \lambda
\end{equation}

Läuft nun gemischtes Licht der Wellenlängen $\lambda$ und $\lambda + \Delta
\lambda$ parallel zur Basis auf einer Länge $t$ durch das Prisma, entsteht ein
Gangunterschied

\begin{align}
  n(\lambda_1) t - n(\lambda_1 + \Delta \lambda) t &= \lambda \\
  \del{ n + \frac{\dif n}{\dif \lambda} \Delta \lambda}t - nt &= \lambda
\end{align}

\begin{equation}
  \lambda = t \Delta \lambda \frac{\dif n}{\dif \lambda}
\end{equation}

Das Auflösungsvermögen eines Prismas bestimmt sich also durch die
materialabhängige Dispersion, die Wellenlänge des Lichts und insbesondere durch
die Basislänge $t$.

\subsection{Beugungsgitter}

Ein optisches Gitter lässt sich modellhaft betrachten als eine Blende mit einer
periodischen Folge von lichtdurchlässigen Spalten. Aufgrund des
Wellencharakters des Lichts entstehen beim Durchgang kohärenten Lichtes durch
das Gitter dabei Interferrenzeffekte. Zu beobachten sind scharfe Beugungsmaxima
in Abständen die von der jeweiligen Wellenlänge des Lichts abhängig sind. Somit
lässt sich auch ein Beugungsgitter zur Spektralzerlegung benutzen.

Die Hauptmaxima lassen sich dabei aus der Bedingung herleiten, dass für
konstruktive Interferrenz der Gangunterschied benachbarter Spalte ein
ganzzahliges Vielfaches der Wellenlänge betragen muss. Für eine Gitterkonstante
$d$ und die Wellenlänge $\lambda$ gilt für den Winkel $\alpha$:

\begin{equation}
  d \sin \alpha = z \lambda \quad \text{mit} \quad z \in \mathbf{N}
\end{equation}

$z$ wird dabei als Ordnung der Beugungsmaxima bezeichnet.

\subsection{Auflösungsvermögen des Beugungsgitters}

Neben den Hauptmaxima existieren noch Nebenmaxima, deren Intensität jedoch
schnell abnimmt, so dass wir nur das erste Nebenmaximum betrachten. Die Lage
des ersten Nebenminimums ist gegeben durch

\begin{equation}
  d \sin \alpha_\text{min} = \del{1 + \frac{1}{N}} \lambda
\end{equation}

wobei $N$ die Gesamtzahl der beitragenden Gitterspalte ist.

Gemäß Rayleigh-Kriterium (siehe Prisma) ergibt sich dann die auflösbare
Wellenlängendifferenz

\begin{equation}
  \Delta \lambda = \frac{\lambda}{z N}
\end{equation}

Das Auflösungsvermögen steigt also mit wachsender Spaltzahl und steigender Ordnung.

\subsection{Wasserstoffspektrum}

Aus dem Bohrschen Atommodell lässt sich das Wasserstoffspektrum ableiten als

\begin{equation}
  \frac{1}{\lambda} = R \del{ \frac{1}{m^2} - \frac{1}{n^2}}
  \quad \text{mit} \quad m,n \in \mathbf{N}
\end{equation}

Dabei ist
\begin{equation}
  R = \frac{2 \pi^2 m_e e^4}{h^3 c}
\end{equation}
die Rydberg-Konstante.

Für $m = 2$ erhält man die Balmer-Serie.


\end{document}
