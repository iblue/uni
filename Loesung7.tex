\documentclass[a4paper,german,12pt,smallheadings]{scrartcl}
\usepackage[T1]{fontenc}
\usepackage[utf8]{inputenc}
\usepackage{babel}
\usepackage{tikz}
\usepackage{geometry}
\usepackage{amsmath}
\usepackage{amssymb}
\usepackage{float}
%\usepackage{wrapfig}
\usepackage{pdflscape}
\pagenumbering{gobble}
\usepackage[thinspace,thinqspace,squaren,textstyle]{SIunits}
\restylefloat{table}
\geometry{a4paper, top=15mm, left=20mm, right=40mm, bottom=20mm, headsep=10mm, footskip=12mm}
\linespread{1.5}
\setlength\parindent{0pt}
\begin{document}
\begin{center}
\bfseries % Fettdruck einschalten
\sffamily % Serifenlose Schrift
\vspace{-40pt}
Analysis I, Sommersemester 2013, 7. Übungsblatt \\
Luis Herrmann und Markus Fenske, Tutor: Adam Schienle
\vspace{-10pt}
\end{center}

\section{Aufgabe 7.2}

Der Algorithmus funktioniert wie folgt: Wir wählen ein Intervall $[a,b]$, von
dem wir wissen, dass sich in diesem eine Nullstelle der Funktion $f(x)$
befindet. Wir berechnen den Funktionswert in der Mitte $m = \frac{a+b}{2}$. Hat
$f(m)$ das selbe Vorzeichen wie $f(a)$, befindet sich die Nullstelle im
Intervall $[m, b]$, ansonsten befindet sie sich im Intervall $[a, m]$. Setzen
wir dieses Intervall wieder als unser Ursprungsintervall, erhalten wir immer
genauere Näherungen $m$ für die Nullstelle.

Wenn man dies mit dem ``Taschenrechner'' ausrechnet,
erhält man folgende Werte bei 5 Durchläufen:

% Generiert mit folgendem Code:
% #!/usr/bin/env ruby
% def f(x) x*x - 2; end
% def sgn(x) x > 0 ? 1 : -1; end
% a,b = [1,2].map(&:to_f)
% class Float; def to_tex; to_s.gsub('.','{,}'); end; end
% class Fixnum; def to_tex; to_s; end; end
% 6.times.map{|x|
%   mid = (a+b)/2.0; f_a, f_mid, f_b = [a, mid, b].map{|x| f(x)}
%   puts [x, a, b, mid, f_a, f_b, f_mid].map(&:to_tex).join(" & ") + " \\"
%   sgn(f_a) != sgn(f_mid) ? b = mid : a = mid
% }
\vspace{20pt}
\begin{tabular}{lllllll}
\hline
$n$ & $a_n$   & $b_n$ & $m_n$ & $f(a_n)$ & $f(b_n)$ & $f(m_n)$ \\ \hline
0 & 1{,}0 & 2{,}0 & 1{,}5 & -1{,}0 & 2{,}0 & 0{,}25 \\
1 & 1{,}0 & 1{,}5 & 1{,}25 & -1{,}0 & 0{,}25 & -0{,}4375 \\
2 & 1{,}25 & 1{,}5 & 1{,}375 & -0{,}4375 & 0{,}25 & -0{,}109375 \\
3 & 1{,}375 & 1{,}5 & 1{,}4375 & -0{,}109375 & 0{,}25 & 0{,}06640625 \\
4 & 1{,}375 & 1{,}4375 & 1{,}40625 & -0{,}109375 & 0{,}06640625 & -0{,}0224609375 \\
5 & 1{,}40625 & 1{,}4375 & 1{,}421875 & -0{,}0224609375 & 0{,}06640625 & 0{,}021728515625 \\
\hline
\end{tabular}
\vspace{20pt}

Benutzt man $a_5$ als letzte Näherung erhält man für die Genauigkeit

\begin{align*}
  |\sqrt{2} - a_5| \approx 0{,}007963562373 \le \frac{1}{64} = 0{,}015625
\end{align*}

Die Näherung liegt also innerhalb der geforderten Toleranz.

\end{document}
