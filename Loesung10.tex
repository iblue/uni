\documentclass[a4paper,german,12pt,smallheadings]{scrartcl}
\usepackage[T1]{fontenc}
\usepackage[utf8]{inputenc}
\usepackage{babel}
\usepackage{tikz}
\usepackage{geometry}
\usepackage{amsmath}
\usepackage{amssymb}
\usepackage{float}
\usepackage{wrapfig}
\usepackage[thinspace,thinqspace,squaren,textstyle]{SIunits}
\restylefloat{table}
\geometry{a4paper, top=15mm, left=20mm, right=40mm, bottom=20mm, headsep=10mm, footskip=12mm}
\linespread{1.5}
\setlength\parindent{0pt}
\begin{document}
\begin{center}
\bfseries % Fettdruck einschalten
\sffamily % Serifenlose Schrift
\vspace{-40pt}
Experiementalphysik I, Wintersemester 2012/2013, 10. Übungsblatt

Normen Peulecke und Markus Fenske, Tutor: Alex Krüger
%Markus Fenske, Tutor: Alex Krüger
\vspace{-10pt}
\end{center}


\section*{32. Druck}

\begin{equation}
  P = \frac{F}{A} = \frac{F}{\pi r^2} = \frac{50 \newton}{\pi (50 \cdot 10^{-6} \meter)^2} \approx 6{,}3 \cdot 10^9 \newton\per\meter\squared = 6{,}3 \giga\pascal 
\end{equation}

Dieser Druck entspricht in etwa dem 62800-fachen des Atmossphärendrucks.

\section*{33. Luftdruck}

Wenn wir davon ausgehen, dass der Luftdruck planetenweit konstant ist, wirkt
auf die gesamte Erdoberfläche ($A = 4 \pi r^2$) die Kraft

\begin{equation}
  F = AP = 4 \pi r^2 P
\end{equation}

Da diese Kraft nur aus der Gravitation kommen kann, erhalten wir mit $F = mg$
\begin{equation}
  m = \frac{4 \pi r^2 P}{g}
\end{equation}

Dies ergibt eine Masse der Atmossphäre von $5{,}2 \cdot 10^{18} \kilo\gram$.

\section*{34. Barometrische Höhenformel}

Wir betrachten das Volumenelement $dV = dxdydz$ als Würfel. Sofern alle
Strömungen ausgeglichen sind und man von einer flachen Erdoberfläche ausgeht,
muss man nur den Drücke auf der Ober- und auf der Unterseite des Würfels
betrachten.

An der Oberseite des Würfels wirkt ein Druck $p_{u}$. An der Unterseite
zusätzlich zu $p_{u}$ noch der Druck, der durch das Gas innerhalb des
Volumenelements $dV$ erzeugt wird.

\begin{align*}
  p_d &= p_u + \frac{F}{A}
\end{align*}

Mit $F = m \cdot g$ und $m = \rho V$ erhält man:

\begin{align*}
  p_d &= p_u + \frac{\rho V g}{A}
\end{align*}

Volumen und Fläche kürzen sich:

\begin{align*}
  p_d &= p_u + \frac{\rho dxdydz g}{dxdy} \\
  p_d &= p_u + \rho dz g \\
\end{align*}

Bildet man nun die Druckdifferenz $dp = p_u - p_d$, erhält man:

\begin{align*}
  dp &= p_u - p_d \\
  dp &= p_u - (p_u + \rho dz g) \\
  dp &= - \rho dz g \\
  \frac{dp}{dz} &= - \rho g
\end{align*}

Die Dichte $\rho$ kann man aus dem idealen Gasgesetz einsetzen, um eine Differentialgleichung zu erhalten.

\begin{align*}
  pV &= nRT \\
  pV &= \frac{mRT}{M} \\
  p &= \frac{mRT}{VM} \\
  \frac{pM}{RT} &= \frac{m}{V} \\
  \rho &= \frac{pM}{RT}
\end{align*}

Eingesetzt ergibt sich:

\begin{equation}
  \frac{dp}{dz} = -p \frac{Mg}{RT}
\end{equation}

Wenn man von einer homogenen isothermen Atmpossphäre ($M$ und $T$ konstant) in
einem konstanten Schwerefeld ($g$ konstant) ausgeht, lässt sich durch Trennung
der Variablen lösen:

\begin{align*}
  \frac{dp}{dz} &= -p \frac{Mg}{RT} \\
  dp &= -p \frac{Mg}{RT} dz \\
  dp \frac{1}{p} &= -\frac{Mg}{RT} dz \\
  \int_{p_0}^{p_1} dp \frac{1}{p} &= \int_{h_0}^{h_1} -\frac{Mg}{RT} dz \\
  \frac{\ln p_1}{\ln p_0} &= -\frac{Mg}{RT} (h_1 - h_0) \\
  \ln p_1 &= - \frac{Mg}{RT} (h_1 - h_0) \ln p_0 \\
   p_1 &= e^{- \frac{Mg}{RT} (h_1 - h_0)} p_0 \\
\end{align*}

Berechnung der Bergwanderung mit den gegebenen Werten (SI-Einheiten):

\begin{align*}
  p &= 101000 e^{-\frac{0{,}028\cdot 9{,}8}{8{,}314 \cdot 288} (2000 - 400)} \\
  p &\approx 84079
\end{align*}

In einer Höhe von $2000 \meter$ herrscht demnach ein Luftdruck von etwa $841 \hecto\pascal$.

\section*{35. Kühlschranktür}

Aus $PV = nRT$ folgt für konstante $n, R, V$, dass $\frac{P}{T}$ konstant sein muss. Somit gilt

\begin{align*}
  \frac{P_2}{T_2} &= \frac{P_1}{T_1} \\
  P_2 &= P_1 \frac{T_2}{T_1}
\end{align*}

Mit $P = \frac{F}{A}$ gilt für die Kraft, mit der die Kühlschranktür zugehalten wird und dem Gegendruck der Atomossphäre.

\begin{align*}
  \frac{F}{A} &= P_1 \frac{T_2}{T_1} - P_1 \\
  F_P &= A P_1 \left(\frac{T_2}{T_1} - 1\right)
\end{align*}

Die Kraft wirkt gleichmäßig auf die ganze Tür, kann also auch in die Mitte gesetzt werden. Dann gilt für das Drehmoment (wenn sich das Scharnier im Nullpunkt befindet).

\begin{align*}
  M_P = F_P \cdot 0{,}25 \meter
\end{align*}

Dieses muss durch ein Drehmoment mindestens kompensiert werden, dass am Griff
angreift, der sich (sofern der Kühlschrank sinnvoll konstruiert ist), auf der
gegenüberliegenden Seite des Scharniers befindet, also im Abstand $0{,}45
\meter$.

Es ergibt sich (mit einem angenommenen Atmossphärendruck von $10^5 \pascal$):

\begin{align*}
  M_z &= F_z \cdot 0{,}45 \meter \\
  F_P \cdot 0{,}25 \meter &= F_z \cdot 0{,}45 \meter \\
  F_z &= F_P \cdot \frac{0{,}25 \meter}{0{,}45 \meter} \\
  F_z &= A P_1 \left(\frac{T_2}{T_1} - 1\right) \frac{0{,}25 \meter}{0{,}45 \meter} \\
  F_z &= 0{,}5 \meter\squared 10^5 \pascal \left(\frac{255 \kelvin}{298 \kelvin} - 1\right) \frac{0{,}25 \meter}{0{,}45 \meter} \\
  F_z &\approx -4{,}0 \kilo\newton
\end{align*}

Die Tür wird sich also kaum noch öffnen lassen, weswegen normale Kühlschränke
ein Loch zum Druckausgleich haben.

\end{document}
