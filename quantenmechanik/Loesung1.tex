\documentclass[a4paper,german,12pt,smallheadings]{scrartcl}
\usepackage[T1]{fontenc}
\usepackage[utf8]{inputenc}
\usepackage{babel}
\usepackage{tikz}
\usepackage{geometry}
\usepackage{amsmath}
\usepackage{amssymb}
\usepackage{float}
\usepackage{enumerate}
%\usepackage{wrapfig}
\usepackage[thinspace,thinqspace,squaren,textstyle]{SIunits}
\restylefloat{table}
\geometry{a4paper, top=15mm, left=20mm, right=40mm, bottom=20mm, headsep=10mm, footskip=12mm}
\linespread{1.5}
\setlength\parindent{0pt}
\begin{document}
\begin{center}
\bfseries % Fettdruck einschalten
\sffamily % Serifenlose Schrift
\vspace{-40pt}
Quantum Mechanics, winter term 2013/2014, exercise sheet 1

Markus Fenske, Luis Herrmann, Tutor: Earl Campbell
\vspace{-10pt}
\end{center}

\section*{Problem 1: Raditation and photons}

\begin{enumerate}[a)]

\item
  Using $E = hf = \frac{hc}{\lambda}$ I get the following values ($1$ eV $\approx 1.6\cdot10^{-19}\;\joule$).
  \vspace{7mm}

  \begin{tabular}{r | r | r | c}
    $\lambda$ & $E\;\text{[eV]}$ & $E\;\text{[J]}$ & spectrum \\
    \hline
    140.2 nm  & 8.8 eV & $1.4 \cdot 10^{-18}$ J & UV \\
    313.2 nm  & 4.0 eV & $6.4 \cdot 10^{-19}$ J & UV \\
    546.0 nm  & 2.3 eV & $3.7 \cdot 10^{-19}$ J & V \\
    1207.2 nm  & 1.0 eV & $1.6 \cdot 10^{-19}$ J & IR \\
  \end{tabular}

\item
  Using the given surface power density $S = 10^{-10}\;\watt\per\meter^2$
  and the surface $A = 5\cdot10^{-5}\;\meter^2$ I calulate the required power
  as $P = AS = 5 \cdot 10^{-15}\;\joule\per\second$. With the given photon
  energy $E=3.7 \cdot 10^{-19} J$ and the time interval $\Delta t =
  1\;\second$, this leads to the required number of photons.

  \begin{equation*}
    N = \frac{P}{E} \cdot \Delta t \approx 13500
  \end{equation*}

\end{enumerate}

\section*{Problem 2: Planck's law}
\begin{enumerate}[a)]
  \item
    Planck's law is
    \begin{equation*}
      \omega(\nu) = \frac{8 \pi h \nu^3}{c^3} \frac{1}{e^\frac{h \nu}{kT} - 1}
    \end{equation*}

    Substitute with $x = \frac{h \nu}{kT}$
    \begin{equation*}
      \omega(x) = \frac{8 \pi h}{c^3} \left(\frac{kT}{h}\right)^3 \frac{x^3}{e^x - 1}
    \end{equation*}

    Differentiate.
    \begin{equation*}
      \frac{d \omega}{dx} = \frac{8 \pi h}{c^3} \left(\frac{kT}{h}\right)^3 \frac{3x^2(e^x-1) - x^3e^x}{(e^x - 1)^2}
    \end{equation*}

    For $d \omega / dx \overset{!}{=} 0$, it must be that
    \begin{equation*}
      3x^2(e^x-1) - x^3e^x = 0
    \end{equation*}

    Because $x \neq 0$ and $e^x \neq 0$
    \begin{align*}
      &\quad 3(e^x-1) - xe^x = 0 \\
      \Leftrightarrow&\quad 3(e^x-1) = xe^x \\
      \Leftrightarrow&\quad 3(1 - e^{-x}) = x
    \end{align*}

    The solution is given as $x \approx 2.8215$, which leads to Wien's
    displacement law in the frequency dependent formulation.
    \begin{align*}
      \nu = 2.8215 \cdot \frac{kT}{h}
    \end{align*}

  \item
    By using above equation, we obtain

    \begin{equation*}
      \nu \approx 341 \; \tera\hertz
    \end{equation*}

  \item
    Given the wire is a long cylinder with diameter $d$ and length $l$, then
    the emitted total power (ignoring the cylinder top and bottom) is

    \begin{equation*}
      P_\gamma = \sigma A T^4 = \sigma \pi d l T^4
    \end{equation*}

    The required electrical power is
    \begin{equation*}
      P_\text{el} = UI = I^2R
    \end{equation*}

    The electrical resitance of the wire (cylinder) is
    \begin{equation*}
      R = \rho \frac{l}{A} = \rho \frac{4l}{\pi d^2}
    \end{equation*}

    So the electrical power becomes
    \begin{equation*}
      P_\text{el} = I^2 \rho \frac{4l}{\pi d^2}
    \end{equation*}

    The supplied eletrical power must equal the emitted raditation power ($P_\text{el} = P_\gamma$)
    \begin{align*}
      &\quad I^2 \rho \frac{4l}{\pi d^2} = \sigma \pi d l T^4 \\
      %\Leftrightarrow&\quad I^2 = \frac{\sigma \pi d l T^4 \pi d^2}{\rho 4l} \\
      \Leftrightarrow&\quad I = \sqrt{\frac{\sigma \pi^2 d^3 T^4}{4 \rho}}
    \end{align*}

    Using the given values this results in a required current of
    \begin{equation*}
      I \approx 1.48 \; \ampere
    \end{equation*}

    %Specifing a specfify resistance for a hot wire is nonsense, because hot
    %wires do not obey Ohm's law. So if any of the above calculation may be
    %wrong, it is not our fault. It would be wrong anyway. (Hint: We like bonus
    %points).
\end{enumerate}

\section*{Problem 3: Solar radiation}
\begin{enumerate}[a)]
  \item
    The radiation power of the earth can be calculated by Stefan–Boltzmann law as

    \begin{equation*}
      P_{\text{earth}} = \sigma A_{\text{Earth}} T^4 = \sigma 4 \pi r_{\text{Earth}}^2
    \end{equation*}

    Because this radiation power originates from the sun, the total solar
    radiation power can be calculated as

    \begin{equation*}
      P_{\text{solar}} = \frac{A_{\text{sphere}}}{A_{\text{earth shadow}}} \cdot P_{\text{earth}}
    \end{equation*}

    where $A_{\text{sphere}}$ is an imaginary sphere behind the earths orbit
    with its center in the center of the sun and $A_{\text{earth shadow}}$ is
    the shadow the earth creates on this sphere. Because the earth is small
    against the sun and at large distances the sun rays are nearly parallel,
    this can be approximated as

    \begin{equation*}
      \frac{A_{\text{sphere}}}{A_{\text{earth shadow}}} \approx \frac{4 \pi d^2}{\pi r_{\text{earth}}^2}
    \end{equation*}

    where $d$ is the distance between sun and earth.

    Putting together these equations leads to

    \begin{equation*}
      P_\text{solar} = \frac{4 \pi d^2}{\pi r_\text{Earth}^2} \sigma 4 \pi r_\text{Earth}^2 T^4 = 16\pi d^2 \sigma T^4
    \end{equation*}

    The total power generated by the sun can be estimated as

    \begin{equation*}
      P_\text{solar} = 16 \pi \left(1.5 \cdot 10^{11} \; \meter\right)^2 \cdot 5.67 \cdot 10^{-8}\;\watt\meter^{-2}\kelvin^{-4} \cdot (290 \; \kelvin)^4 \approx 4.5 \cdot 10^{26}\;\watt
    \end{equation*}
  \item
    Solving above equation for $T$ leads to

    \begin{equation*}
      T = \frac{\sqrt[4]{P_\text{solar}}}{2\sqrt{d} \sqrt[4]{\pi \sigma}} = 93.8\;\kelvin
    \end{equation*}

    where $d \approx 1.429 \cdot 10^{12} \;\meter$

  \item
    Using Stefan-Boltzmann law as above, the total power irridiated by the sun is

    \begin{equation*}
      P_\text{solar} = \sigma 4 \pi r_\text{sun}^2 T_\text{sun}^4
    \end{equation*}

    The energy is
    \begin{equation*}
      E = P \Delta t
    \end{equation*}
    where $\Delta t = 1\;\text{year}$

    From $E = mc^2$ the mass is
    \begin{equation*}
      m = \frac{E}{c^2} = \frac{P \Delta t}{c^2} = \frac{\sigma 4 \pi r_\text{sun}^2 T_\text{sun}^4 \Delta t}{c^2}
    \end{equation*}

    The percentage of mass lost per year is therefore
    \begin{equation*}
      p = \frac{m}{M_\text{sun}} = \frac{P \Delta t}{c^2} = \frac{\sigma 4 \pi r_\text{sun}^2 T_\text{sun}^4 \Delta t}{c^2 M_\text{sun}} \approx 2.0 \cdot 10^{-13}
    \end{equation*}

    which is $2.0 \cdot 10^{-11}\;\%$ or $0.2\;\text{ppt}$ (parts per trillion).


\end{enumerate}
\end{document}
