\documentclass[a4paper,german,12pt,smallheadings]{scrartcl}
\usepackage[T1]{fontenc}
\usepackage[utf8]{inputenc}
\usepackage{babel}
\usepackage{geometry}
\usepackage{amsmath}
\usepackage{amssymb}
\usepackage{float}
\usepackage{enumerate}
\usepackage{braket} % Teh quantum stuff
\usepackage{commath} % http://tex.stackexchange.com/questions/14821/whats-the-proper-way-to-typeset-a-differential-operator
\usepackage{cancel}
%\usepackage{wrapfig}
\usepackage[thinspace,thinqspace,squaren,textstyle]{SIunits}

% New command for color underlining
\usepackage{xcolor}

\newsavebox\MBox
\newcommand\colul[2][red]{{\sbox\MBox{$#2$}%
  \rlap{\usebox\MBox}\color{#1}\rule[-1.2\dp\MBox]{\wd\MBox}{0.5pt}}}

\restylefloat{table}
\geometry{a4paper, top=15mm, left=20mm, right=40mm, bottom=20mm, headsep=10mm, footskip=12mm}
\linespread{1.5}
\setlength\parindent{0pt}
\DeclareMathOperator{\Tr}{Tr}
\DeclareMathOperator{\Var}{Var}
\begin{document}
\allowdisplaybreaks % Seitenumbrüche in Formeln erlauben
\begin{center}
\bfseries % Fettdruck einschalten
\sffamily % Serifenlose Schrift
\vspace{-40pt}
Quantum Mechanics, winter term 2013/2014, exercise sheet 7+1

Markus Fenske, Luis Herrmann, Tutor: Adam Nagy
\vspace{-10pt}
\end{center}

\section*{Exercise 1: Disturbed harmonic oscillator}
\begin{enumerate}[a)]
  \item
We can write
\begin{equation*}
  H = \hbar \omega \del{a^\dagger a + \frac{1}{2}} + \lambda x
\end{equation*}

Using
\begin{align*}
  &a + a^\dagger = 2 \frac{\omega m}{\sqrt{2 \omega m \hbar}} x \\
  \Leftrightarrow\quad& x = \frac{\sqrt{2 \omega m \hbar}}{2 \omega m} \del{a + a^\dagger}
\end{align*}

we get
\begin{equation*}
  H = \hbar \omega \del{a^\dagger a + \frac{1}{2}} + \lambda \frac{\sqrt{2 \omega m \hbar}}{2 \omega m} \del{a + a^\dagger}
\end{equation*}

\item
Inserting $\tilde x = x + x_0 \Leftrightarrow x = \tilde x - x_0$ into the
given Hamiltonian leads:

\begin{align*}
  H &= \frac{1}{2m} p^2 + \frac{m \omega^2}{2} \del{\tilde x - x_0}^2 + \lambda \del{\tilde x - x_0} \\
    &= \frac{1}{2m} p^2 + \underbrace{\frac{m \omega^2}{2} \tilde x^2}_{= U_0(\tilde x} + \underbrace{\del{\lambda - m\omega^2 x_0}}_{= \tilde \lambda} \tilde x + \underbrace{\frac{m \omega^2}{2} x_0^2 - \lambda x_0}_{=u_0} \\
    &= \frac{1}{2m} p^2 + U_0(\tilde x) + \tilde \lambda \tilde x + u_0
\end{align*}

In order to find the $b$ and $b^\dagger$ operators, we write the Hamiltonian as
\begin{align*}
  H &= \hbar \omega \del{a^\dagger a + \frac{1}{2} + \lambda \frac{\sqrt{2 \omega m \hbar}}{2 \omega m \hbar \omega} \del{a + a^\dagger}} \\
    &\overset{?}{=} k_1 \del{b^\dagger b + k_2}
\end{align*}

We see that $k_1 = \hbar \omega$. We then define $b = a + c$ and $b^\dagger = a^\dagger + c^\dagger$. Which leads to

\begin{align*}
  &a^\dagger a + \frac{1}{2} + \lambda \frac{\sqrt{2 \omega m \hbar}}{2 \omega m \hbar \omega} \del{a + a^\dagger} = b^\dagger b + k_2 \\
  \Leftrightarrow\quad
  &a^\dagger a + \frac{1}{2} + \lambda \frac{\sqrt{2 \omega m \hbar}}{2 \omega m \hbar \omega} \del{a + a^\dagger} = \del{a^\dagger + c^\dagger}\del{a + c} + k_2 \\
  \Leftrightarrow\quad
  &\cancel{a^\dagger a} + \frac{1}{2} + \lambda \frac{\sqrt{2 \omega m \hbar}}{2 \omega m \hbar \omega} \del{a + a^\dagger} = \cancel{a^\dagger a} + a^\dagger c + c^\dagger a + c^\dagger c + k_2
  \intertext{Setting $c = c^\dagger = \lambda \frac{\sqrt{2 \omega m \hbar}}{2 \omega m \hbar \omega}$:}
  \Leftrightarrow\quad
  &\frac{1}{2} + \cancel{\lambda \frac{\sqrt{2 \omega m \hbar}}{2 \omega m \hbar \omega} a} + \cancel{\frac{\sqrt{2 \omega m \hbar}}{2 \omega m \hbar \omega} a^\dagger} = \cancel{\frac{\sqrt{2 \omega m \hbar}}{2 \omega m \hbar \omega} a^\dagger}  + \cancel{\frac{\sqrt{2 \omega m \hbar}}{2 \omega m \hbar \omega} a} + \del{\frac{\sqrt{2 \omega m \hbar}}{2 \omega m \hbar \omega}}^2 + k_2 \\
  \Leftrightarrow\quad
  &\frac{1}{2} = \del{\frac{\sqrt{2 \omega m \hbar}}{2 \omega m \hbar \omega}}^2 + k_2 \\
  \Leftrightarrow\quad
  &k_2 = \frac{1}{2} - \del{\frac{\sqrt{2 \omega m \hbar}}{2 \omega m \hbar \omega}}^2
\end{align*}

So we get
\begin{equation*}
  H = k_1 \del{b^\dagger b + k_2}
\end{equation*}

with
\begin{align*}
  k_1 &= \hbar \omega \\
  k_2 &= \frac{1}{2} - \del{\frac{\sqrt{2 \omega m \hbar}}{2 \omega m \hbar \omega}}^2 \\
  b   &= a + \lambda \frac{\sqrt{2 \omega m \hbar}}{2 \omega m \hbar \omega}
\end{align*}

Because $b$ is just $a$ plus a constant, they satisfy the same commutator
relations as $a$ and $a^\dagger$ and so are real creation and anhiliation
operators.

\item
  Because $H = k_1\del{N + k_2}$ its obvious that $H$ and $N$ have the same eigenvectors.
\begin{align*}
  &k_1\del{N + k_2} \ket{\psi} = E \ket{\psi} \\
  \Leftrightarrow\quad&
  \del{N + k_2} \ket{\psi} = \frac{E}{k_1} \ket{\psi} \\
  \Leftrightarrow\quad&
  N \ket{\psi} = \frac{E}{k_1} \ket{\psi} - k_2 \ket{\psi}\\
  \Leftrightarrow\quad&
  N \ket{\psi} = \frac{E - k_2k_1}{k_1} \ket{\psi}
\end{align*}

\begin{enumerate}[i)]
  \item
    We see that
    \begin{equation*}
      \sbr{b,b^\dagger} = \sbr{a+c,a^\dagger+c} = \sbr{a,a^\dagger} = 1
    \end{equation*}

    And therefore
    \begin{align*}
      &bb^\dagger = bb^\dagger - b^\dagger b + b^\dagger b = 1 + b^\dagger b \\
      \Leftrightarrow\quad &b^\dagger b = bb^\dagger - 1
    \end{align*}

    Using these relations, we see that if $N\ket{\xi} = \eta \ket{\xi}$, then
    \begin{align*}
      N b \ket{\xi} &= \del{bN -b} \ket{\xi} = \del{\eta - 1} b \ket{\xi} \\
      N b^\dagger \ket{\xi} &= \del{b^\dagger N -b^\dagger} \ket{\xi} = \del{\eta + 1} b^\dagger \ket{\xi} \\
    \end{align*}
  \item
    We show that $N$ is positive definite. Let $\ket{\xi}$ be an eigenvector of
    $N$ with eigenvalue $\eta$.

    \begin{align*}
      \underbrace{\envert{b \ket{\xi}2}^2}_{>0} = \braket{\xi|a^\dagger a|\xi} = \braket{\xi|N|\xi} = \eta \underbrace{\braket{\xi|\xi}}_{> 0}
    \end{align*}
    Therefore $\eta > 0$.

    Let $\ket{\phi}$ be an eigenvector of $N$ with eigenvalue $\phi$, where
    $\phi$ is not integer. Let $n$ be the smallest integer greater than $\phi$.
    Then $b^m \ket{\phi}$ is an eigenvector of $N$ with eigenvalue $\phi - n <
    0$. But because $N$ is positive definite, its eigenvalues best be greater
    or equal zero. This is a contraduction. Therefore all eigenvalues must be
    integer. Therefore the smallest eigenvalue is $0$.

\end{enumerate}

\item 
\begin{enumerate}[i)]
  \item
    Insert long and boring proof by induction here.
  \item
    \begin{align*}
      b \ket{n}
      &= b \frac{1}{\sqrt{n!}} \del{b^\dagger}^n \ket{0} 
      = bb^\dagger \frac{1}{\sqrt{n}} \del{\frac{1}{\sqrt{(n-1)!}} \del{b^\dagger}^{n-1} \ket{0}} 
      = \frac{1}{\sqrt{n}} bb^\dagger \ket{n-1} \\
      &= \frac{1}{\sqrt{n}} \del{1 + b^\dagger b} \ket{n-1}
      = \frac{1}{\sqrt{n}} \del{1 + N} \ket{n-1}
      = \frac{1}{\sqrt{n}} \del{1 + (n-1)} \ket{n-1} \\
      &= \frac{n}{\sqrt{n}} \ket{n-1}
      = \sqrt{n} \ket{n-1}
    \end{align*}

    \begin{align*}
      b^\dagger \ket{n} 
      &= b^\dagger \frac{1}{\sqrt{n!}} \del{b^\dagger}^n \ket{0}
      = \sqrt{n+1} \frac{1}{\sqrt{(n+1)!}} \del{b^\dagger}^{n+1} \ket{0}
      = \sqrt{n+1} \ket{n+1}
    \end{align*}

    \begin{align*}
      N \ket{n} = b^\dagger b \ket{n} = \sqrt{n} b^\dagger \ket{n-1} = \sqrt{n} \sqrt{(n-1)+1} \ket{n} = n \ket{n}
    \end{align*}

\end{enumerate}
\item Trivial.

\end{enumerate}
\end{document}
