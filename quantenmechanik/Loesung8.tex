\documentclass[a4paper,german,12pt,smallheadings]{scrartcl}
\usepackage[T1]{fontenc}
\usepackage[utf8]{inputenc}
\usepackage{babel}
\usepackage{geometry}
\usepackage{amsmath}
\usepackage{amssymb}
\usepackage{float}
\usepackage{enumerate}
\usepackage{braket} % Teh quantum stuff
\usepackage{commath} % http://tex.stackexchange.com/questions/14821/whats-the-proper-way-to-typeset-a-differential-operator
\usepackage{cancel}
%\usepackage{wrapfig}
\usepackage[thinspace,thinqspace,squaren,textstyle]{SIunits}

% New command for color underlining
\usepackage{xcolor}

\newsavebox\MBox
\newcommand\colul[2][red]{{\sbox\MBox{$#2$}%
  \rlap{\usebox\MBox}\color{#1}\rule[-1.2\dp\MBox]{\wd\MBox}{0.5pt}}}

\restylefloat{table}
\geometry{a4paper, top=15mm, left=20mm, right=40mm, bottom=20mm, headsep=10mm, footskip=12mm}
\linespread{1.5}
\setlength\parindent{0pt}
\DeclareMathOperator{\Tr}{Tr}
\DeclareMathOperator{\Var}{Var}
\begin{document}
\allowdisplaybreaks % Seitenumbrüche in Formeln erlauben
\begin{center}
\bfseries % Fettdruck einschalten
\sffamily % Serifenlose Schrift
\vspace{-40pt}
Quantum Mechanics, winter term 2013/2014, exercise sheet 7+1

Markus Fenske, Luis Herrmann, Tutor: Adam Nagy
\vspace{-10pt}
\end{center}

\section*{Exercise 1: Disturbed harmonic oscillator}

We can write
\begin{equation*}
  H = \hbar \omega \del{a^\dagger a + \frac{1}{2}} + \lambda x
\end{equation*}

Using
\begin{align*}
  &a + a^\dagger = 2 \frac{\omega m}{\sqrt{2 \omega m \hbar}} x \\
  \Leftrightarrow\quad& x = \frac{\sqrt{2 \omega m \hbar}}{2 \omega m} \del{a + a^\dagger}
\end{align*}

We get
\begin{equation*}
  H = \hbar \omega \del{a^\dagger a + \frac{1}{2}} + \lambda \frac{\sqrt{2 \omega m \hbar}}{2 \omega m} \del{a + a^\dagger}
\end{equation*}
\end{document}
