\documentclass[a4paper,german,12pt,smallheadings]{scrartcl}
\usepackage[T1]{fontenc}
\usepackage[utf8]{inputenc}
\usepackage{babel}
\usepackage{geometry}
\usepackage{amsmath}
\usepackage{amssymb}
\usepackage{float}
\usepackage{enumerate}
\usepackage{braket} % Teh quantum stuff
\usepackage{commath} % http://tex.stackexchange.com/questions/14821/whats-the-proper-way-to-typeset-a-differential-operator
\usepackage{cancel}
%\usepackage{wrapfig}
\usepackage[thinspace,thinqspace,squaren,textstyle]{SIunits}

% New command for color underlining
\usepackage{xcolor}

\newsavebox\MBox
\newcommand\colul[2][red]{{\sbox\MBox{$#2$}%
  \rlap{\usebox\MBox}\color{#1}\rule[-1.2\dp\MBox]{\wd\MBox}{0.5pt}}}

\restylefloat{table}
\geometry{a4paper, top=15mm, left=10mm, right=20mm, bottom=20mm, headsep=10mm, footskip=12mm}
\linespread{1.5}
\setlength\parindent{0pt}
\DeclareMathOperator{\Tr}{Tr}
\DeclareMathOperator{\Var}{Var}
\begin{document}
\allowdisplaybreaks % Seitenumbrüche in Formeln erlauben
\begin{center}
\bfseries % Fettdruck einschalten
\sffamily % Serifenlose Schrift
\vspace{-40pt}
Quantum Mechanics, winter term 2013/2014, exercise sheet 12

Markus Fenske, Tutor: Adam Nagy
\vspace{-10pt}
\end{center}

\section*{Exercise 1: Perturbation theory}
\begin{enumerate}[a)]
  \item
    When setting $\lambda = 0$, we get the Hamiltonian for the harmonic
    oscillator where $m = \hbar = \omega = 1$.
    \begin{equation*}
      H^{(0)} = - \frac{1}{2} \pd[2]{}{x} + \frac{1}{2} x^2
    \end{equation*}

    The eigenstates are know as
    \begin{equation*}
      \psi_n^{(0)}(x) = \frac{1}{\sqrt[4]{\pi}} \frac{1}{\sqrt{2^n n!}} H_n(x) x e^{-\frac{x^2}{2}}
    \end{equation*}

    The eigenenergies are
    \begin{equation*}
      E^{(0)}_n = n + \frac{1}{2}
    \end{equation*}

  \item
    The pertubed Hamiltonian is
    \begin{equation*}
      H = H^{(0)} + \lambda V \; \text{ where } \; V = x^4
    \end{equation*}

    So the first order energy is
    \begin{align*}
      \braket{\psi_n^{(0)}|V|\psi_n^{(0)}} &= \braket{n|x^4|n} \\
      \intertext{The $x$-operator is in this case $x = \frac{1}{\sqrt{2}} (a+a^\dagger)$, so we get}
      &= \frac{1}{4} \braket{n|(a+a^\dagger)^4|n} \\
      &= \frac{1}{4} \braket{n|(a^2 + 2N + 1 +{a^\dagger}^2)^2|n} \\
      &= \frac{1}{4} \bra{n}(a^4 + 2a^2N + a^2 + a^2{a^\dagger}^2 + 2Na^2 +4N^2 + 2N +\\ & \quad\; 2N{a^\dagger}^2 + a^2 + 2N+1+{a^\dagger}^2 + {a^\dagger}^2a^2 + 2{a^\dagger}^2N + {a^\dagger}^2 + {a^\dagger}^4)\ket{n}
      \intertext{To clear this up, we can remove all terms that contain an unbalanced number of $a$ and $a^\dagger$, because they will vanish.}
      &= \frac{1}{4} \braket{n|(a^2{a^\dagger}^2 + 4 N^2 + 2N + 2N + 1 + {a^\dagger}^2a^2)|n}
    \end{align*}

    Now for some operator transformations using the known relations.

    \begin{align*}
      a^2{a^\dagger}^2 &= aaa^\dagger a^\dagger \\
                       &= a(N+1)a^\dagger \\
                       &= aNa^\dagger + aa^\dagger \\
                       &= (Na + a)a^\dagger + N + 1 \\
                       &= Naa^\dagger + aa^\dagger + N + 1 \\
                       &= N(N+1) + (N+1) + N + 1 \\
                       &= N^2 + 3N + 2
    \end{align*}

    \begin{align*}
      {a^\dagger}^2 a^2 &= a^\dagger N a \\
                        &= a^\dagger (aN - a) \\
                        &= N^2 - N
    \end{align*}

    Substituting this into the original equation leads
    \begin{align*}
      \frac{1}{4} \braket{n|(a^2{a^\dagger}^2 + 4 N^2 + 4N + 1 + {a^\dagger}^2a^2)|n} 
      &= \frac{1}{4} \braket{n|(N^2 + 3N + 2 + 4N^2 + 4N + 1 + N^2 - N)|n} \\
      &= \frac{1}{4} \braket{n|(6N^2 + 6N + 3)|n} \\
      &= \frac{6n^2 + 6n +3}{4}
    \end{align*}

    So the first-order energy correction is
    \begin{align*}
      E_n^{(1)}(\lambda) = \lambda \frac{6n^2 + 6n + 3}{4}
    \end{align*}

    To get the corrected energy, one would still have to add the $E_n^{(0)}$
    term.

    For the second order energies we have to calculate
    \begin{align*}
      E_n^{(2)} = \sum_{m (\neq n)} \frac{\envert{\braket{m|V|n}}^2}{E^{(0)}_n - E^{(0)}_m}
    \end{align*}

    In order to shorten the notation, we define the \textit{falling factorial}
    denoted by an underlined exponent:
    \begin{equation*}
      n^{\underline{m}} := \prod_{i=0}^{m-1} (n-i)
    \end{equation*}

    Examples: $n^{\underline{4}} = n (n-1) (n-2) (n-3)$, or for the anhilation
    operator $a^m \ket{n} = \sqrt{n^{\underline{m}}} \ket{n}$

    We also define the rising factorial, denoted by an overlined exponent.
    \begin{equation*}
      n^{\overline{m}} := \prod_{i=0}^{m-1} (n+i)
    \end{equation*}

    Using this, we calculate $\braket{m|V|n}$, where we know that $m \neq n$.
    So we get to the same point as above, but remove the balanced terms
    containing the same number of anhilation as creation operators, because
    they will vanish.

    \begin{align*}
      \braket{m|x^4|n} &= \frac{1}{4} \braket{m|(a^4 + 2a^2N + 2a^2 + 2Na^2 + 2N{a^\dagger}^2 + 2{a^\dagger}^2 N + 2{a^\dagger}^2 + {a^\dagger}^4)|n} \\
                                   &= \frac{1}{4} \left(\sqrt{n^{\underline{4}}} \braket{m|n-4} + 2 n \sqrt{n^{\underline{2}}} \braket{m|n-2} \right.
      + 2 \sqrt{n^{\underline{2}}} \braket{m|n-2} + 2 (n-2) \sqrt{n^{\underline{2}}} \braket{m|n-2} \\
    & \quad + 2n \sqrt{(n+1)^{\overline 2}} \braket{m|n+2}+ 2(n+2) \sqrt{(n+1)^{\overline 2}} \braket{m|n+2} + 2 \sqrt{(n+1)^{\overline 2}} \braket{m|n+2} \\
    &\quad + \left. \sqrt{(n+1)^{\overline 4}} \braket{m|n+4} \right) \\
    &=  \frac{\sqrt{n^{\underline 4}}}{4}            \braket{m|n-4}
      + \frac{\sqrt{n^{\underline 2}}}{2} \del{n-1}  \braket{m|n-2} \\
      &\quad+ \frac{\sqrt{(n+1)^{\overline  2}}}{2} \del{2n+3} \braket{m|n+2}
      + \frac{\sqrt{(n+1)^{\overline  4}}}{4}            \braket{m|n+4}
    \end{align*}

    The only non-zero terms are for $m=n+4$, $m=n+2$, $m=n-2$ and $m=n-4$, so
    we can write the sum as

    \begin{align*}
      E_n^{(2)} &=
      \frac{|\braket{n|V|n+4}|^2}{E^{(0)}_{n+4} - E^{(0)}_{m}} +
      \frac{|\braket{n|V|n+2}|^2}{E^{(0)}_{n+2} - E^{(0)}_{m}} +
      \frac{|\braket{n|V|n-2}|^2}{E^{(0)}_{n-2} - E^{(0)}_{m}} +
      \frac{|\braket{n|V|n-4}|^2}{E^{(0)}_{n-4} - E^{(0)}_{m}} \\
      &=
      \frac{|\braket{n|V|n+4}|^2}{4} +
      \frac{|\braket{n|V|n+2}|^2}{2}
      -\frac{|\braket{n|V|n-2}|^2}{2}
      -\frac{|\braket{n|V|n-4}|^2}{4} \\
      &=
      \frac{n^{\underline 4}}{64}
      +\frac{n^{\underline 2}(n-1)^2}{8}
      -\frac{(n+1)^{\overline 2}(2n+3)^2}{8}
      -\frac{(n+1)^{\overline 4}}{64} \\
      &=
      \frac{n(n-1)(n-2)(n-3) + 8 n(n-1)^3 - 8(n+1)(n+2)(2n+3)^2 - (n+1)(n+2)(n+3)(n+4)}{64} \\
      &= -\frac{3}{8}n^4 - \frac{29}{8}n^3 - \frac{53}{8}n^2 - \frac{59}{8} n - \frac{21}{8}
    \end{align*}

    So the second order energy is
    \begin{equation*}
      E_n^{(2)}(\lambda)  = -\lambda^2 \del{\frac{3}{8}n^4 + \frac{29}{8}n^3 + \frac{53}{8}n^2 + \frac{59}{8} n + \frac{21}{8}}
    \end{equation*}

  \item
    The first order eigenstates are defined by
    \begin{align*}
      \ket{\psi_n^{(1)}} &= \sum_{k (\neq n)} \frac{\braket{k|V|n}}{E_n - E_k} \ket{k} \\
      &=
      \frac{\braket{n+4|V|n}}{E^{(0)}_{n+4} - E^{(0)}_{m}} \ket{n+4} +
      \frac{\braket{n+2|V|n}}{E^{(0)}_{n+2} - E^{(0)}_{m}} \ket{n+2} +
      \frac{\braket{n-2|V|n}}{E^{(0)}_{n-2} - E^{(0)}_{m}} \ket{n-2} +
      \frac{\braket{n-4|V|n}}{E^{(0)}_{n-4} - E^{(0)}_{m}} \ket{n-4} \\
      &= 
      \frac{\sqrt{n^{\underline 4}}}{16} \ket{n+4}
      +\frac{\sqrt{n^{\underline 2}}(n-1)}{8} \ket{n+2}
      -\frac{\sqrt{(n+1)^{\overline 2}}(2n+3)}{8} \ket{n-2}
      -\frac{\sqrt{(n+1)^{\overline 4}}}{16} \ket{n-4} \\\
    \end{align*}

\end{enumerate}

\section*{Exercise 2: Time-dependent perturbation theory}
\begin{enumerate}[a)]
  \item
    See script, 6.2.2
  \item
    The transition amplitude $A$ is
    \begin{align*}
      A = \frac{\lambda}{i \hbar} \int_0^t e^{\frac{i \omega_0 s}{2}} \cos \del{\omega_L s} \dif s =
      \lambda
      \del{
        -\frac{2 \omega_0 e^\frac{i \omega_0 t}{2} \cos\del{\omega_L t}}{\hbar\del{\omega_0^2 - 4 \omega_L^2}}
        +i\frac{4 \omega_L e^\frac{i \omega_0 t}{2} \sin\del{\omega_L t}}{\hbar\del{\omega_0^2 - 4 \omega_L^2}}
      }
    \end{align*}

    The transition probability is therefore
    \begin{align*}
      P &= \envert{A}^2 \\
       &= \frac{4 \omega_0^2 \cos^2\del{\omega_L t} + 16 \omega_L^2 \sin\del{\omega_L t}}{\hbar^2\del{\omega_0^2 - 4 \omega_L^2}^2}
    \end{align*}
  \item
    We obtain a result by deriving $\pd{P}{\omega_L}$ and $\pd{P}{t}$ to get a maximum.
  \item
\end{enumerate}
\section*{Exercise 3: Coupling of spin and angular momentum}
\begin{enumerate}[a)]
  \item
    Because we know that $L$ and $S$ operate on different parts of the Hilbert
    space, they and all of their parts commute. So we can do this:
    \begin{align*}
      &L^2 + S^2 + 2 L_3 S_3 + L_+ S_- + L_- S_+ \\
      &\quad = L^2 + S^2 + 2 L_3 S_3 + (L_1+iL_2) (S_1-iS_2) + (L_1-iL_2)(S_1 + iS_2) \\
      &\quad = L^2 + S^2 + 2 L_3 S_3 + (L_1S_1 -iL_1S_2 +iL_2S_1 +L_2S_2) + (L_1S_1 - iL_2S_1 + iL_1S_2 + L_2S_2) \\
      &\quad = L^2 + S^2 + 2 L_3 S_3 + 2L_2S_2 + 2L_1S_1 \\
      &\quad = (L+S)^2 \\
      &\quad = J^2 \\
    \end{align*}
  \item
    We follow the hint given in the exercise sheet and check if
    $\ket{\frac{3}{2}, \frac{3}{2}}$ is an eigenvector of $J_z$ (and thus of $J^2$, because
    they commute). We see that $J_z = L_z + S_z$. So we have
    \begin{align*}
      J_z \ket{\tfrac{3}{2}, \tfrac{3}{2}} 
      &= (L_z \otimes I + I \otimes S_z) \del{\ket{1,1} \otimes \ket{\tfrac{1}{2}, \tfrac{1}{2}}} \\
      &= \del{L_z \ket{1,1} \otimes \ket{\tfrac{1}{2}, \tfrac{1}{2}}} + \del{\ket{1,1} \otimes S_z \ket{\tfrac{1}{2}, \tfrac{1}{2}}} \\
      &= \del{\hbar \ket{1,1} \otimes \ket{\tfrac{1}{2}, \tfrac{1}{2}}} + \del{\ket{1,1} \otimes \tfrac{\hbar}{2} \ket{\tfrac{1}{2}, \tfrac{1}{2}}} \\
      &= \tfrac{3\hbar}{2} \del{\ket{1,1} \otimes \ket{\tfrac{1}{2}, \tfrac{1}{2}}} \\
      &= \tfrac{3\hbar}{2} \ket{\tfrac{3}{2}, \tfrac{3}{2}}
    \end{align*}

    So we see, that the given vector is an eigenvector. Using the ladder
    operators, we can construct all 4 more eigenvectors.

    From the lecture we know the ladder operators.

    \begin{align*}
      J_+ \ket{j, m} &= \hbar \sqrt{(j-m)(j+m+1)} \ket{j, m+1} \\
      J_- \ket{j, m} &= \hbar \sqrt{(j+m)(j-m+1)} \ket{j, m-1} \\
    \end{align*}

    We see that
    \begin{align*}
      J_+ \ket{\tfrac{3}{2}, \tfrac{3}{2}} &= 0 \\
      J_- \ket{\tfrac{3}{2}, \tfrac{3}{2}} &= \hbar \sqrt{3} \ket{\tfrac{3}{2}, \tfrac{1}{2}} \\
      J_- \ket{\tfrac{3}{2}, \tfrac{1}{2}} &= \hbar 2 \ket{\tfrac{3}{2}, -\tfrac{1}{2}} \\
      J_- \ket{\tfrac{3}{2}, -\tfrac{1}{2}} &= \hbar \sqrt{3} \ket{\tfrac{3}{2}, -\tfrac{3}{2}} \\
      J_- \ket{\tfrac{3}{2}, -\tfrac{3}{2}} &= 0
    \end{align*}

    Analogously we proove that $\ket{\tfrac{1}{2}, \tfrac{1}{2}}$ is an
    eigenector of $J_z$.
    \begin{align*}
      J_z \ket{\tfrac{1}{2}, \tfrac{1}{2}} 
      &= (L_z \otimes I + I \otimes S_z) \del{
        \sqrt{\tfrac{2}{3}}\ket{1,1} \otimes \ket{\tfrac{1}{2}, -\tfrac{1}{2}} -
        \sqrt{\tfrac{1}{3}}\ket{1,0} \otimes \ket{\tfrac{1}{2}, \tfrac{1}{2}}
      }\\
      &=
        \sqrt{\tfrac{2}{3}}L_z\ket{1,1} \otimes \ket{\tfrac{1}{2}, -\tfrac{1}{2}} -
        \sqrt{\tfrac{1}{3}}L_z\ket{1,0} \otimes \ket{\tfrac{1}{2}, \tfrac{1}{2}} +
        \sqrt{\tfrac{2}{3}}\ket{1,1} \otimes S_z\ket{\tfrac{1}{2}, -\tfrac{1}{2}} -
        \sqrt{\tfrac{1}{3}}\ket{1,0} \otimes S_z\ket{\tfrac{1}{2}, \tfrac{1}{2}} \\
      &=
        \sqrt{\tfrac{2}{3}} \hbar \ket{1,1} \otimes \ket{\tfrac{1}{2}, -\tfrac{1}{2}} -
        \sqrt{\tfrac{2}{3}}\ket{1,1} \otimes -\tfrac{\hbar}{2} \ket{\tfrac{1}{2}, -\tfrac{1}{2}} -
        \sqrt{\tfrac{1}{3}}\ket{1,0} \otimes \tfrac{\hbar}{2} \ket{\tfrac{1}{2}, \tfrac{1}{2}} \\
      &= \tfrac{\hbar}{2} \del{
        \sqrt{\tfrac{2}{3}}\ket{1,1} \otimes \ket{\tfrac{1}{2}, -\tfrac{1}{2}} -
        \sqrt{\tfrac{1}{3}}\ket{1,0} \otimes \ket{\tfrac{1}{2}, \tfrac{1}{2}}
      }\\
    \end{align*}

    We see that
    \begin{align*}
      J_+ \ket{\tfrac{1}{2}, \tfrac{1}{2}} &= 0 \\
      J_- \ket{\tfrac{1}{2}, \tfrac{1}{2}} &= \hbar \ket{\tfrac{1}{2}, -\tfrac{1}{2}} \\
      J_- \ket{\tfrac{1}{2}, -\tfrac{1}{2}} &= 0
    \end{align*}

    Thus we have constructed all eigenvectors.

  \item
    We know that
    \begin{align*}
      S_i = \frac{\hbar}{2} \sigma_i
    \end{align*}

    where $\sigma_i$ are the Pauli matrices. So we already know the matrix
    representation of $S_x$, $S_y$, $S_z$ in the basis $\{\ket{\tfrac{1}{2},
    \tfrac{1}{2}}, \ket{\tfrac{1}{2}, -\tfrac{1}{2}}\}$:

    \begin{align*}
      S_x = \frac{\hbar}{2} \begin{pmatrix}
        0 & 1 \\
        1 & 0
      \end{pmatrix}, \quad
      S_y = \frac{\hbar}{2} \begin{pmatrix}
        0 & -i \\
        i & 0
      \end{pmatrix}, \quad
      S_z = \frac{\hbar}{2} \begin{pmatrix}
        1 & 0 \\
        0 & -1
      \end{pmatrix},
    \end{align*}

    For $L_z$ we directly see that $L_z \ket{j, m} = \hbar m \ket{j,m}$. So we
    have in basis $\{\ket{1,1}, \ket{1,0}, \ket{1,-1}\}$:

    \begin{align*}
      L_z = \hbar \begin{pmatrix}
        1 & 0 & 0 \\
        0 & 0 & 0 \\
        0 & 0 & 1
      \end{pmatrix}
    \end{align*}

    We also know the ladder operators and calculate
    \begin{align*}
      L_+ \ket{1,1} &= 0 \\
      L_+ \ket{1,0} &= \hbar \sqrt{2} \ket{1,1} \\
      L_+ \ket{1,-1} &= \hbar \sqrt{2} \ket{1,0} \\
      L_- \ket{1,1} &= \hbar \sqrt{2} \ket{1,0} \\
      L_- \ket{1,0} &= \hbar \sqrt{2} \ket{1,-1} \\
      L_- \ket{1,-1} &= 0 \\
    \end{align*}

    So the matricies are
    \begin{align*}
      L_+ = \hbar \sqrt{2} \begin{pmatrix}
        0 & 1 & 0 \\
        0 & 0 & 1 \\
        0 & 0 & 0 
      \end{pmatrix}, \quad
      L_- = \hbar \sqrt{2} \begin{pmatrix}
        0 & 0 & 0 \\
        1 & 0 & 0 \\
        0 & 1 & 0 
      \end{pmatrix}
    \end{align*}

    From $L_x = \frac{1}{2} \del{L_+ + L_-}$ and $L_y = \frac{1}{2i} \del{L_+ -
    L_-}$ we construct
    \begin{align*}
      L_x = \hbar \frac{\sqrt{2}}{2} \begin{pmatrix}
        0 & 1 & 0 \\
        1 & 0 & 1 \\
        0 & 1 & 0
      \end{pmatrix}, \quad
      L_y = \hbar \frac{\sqrt{2}}{2i} \begin{pmatrix}
        0 & 1 & 0 \\
        -1 & 0 & 1 \\
        0 & -1 & 0
      \end{pmatrix}
    \end{align*}

    So in the basis $
    \{
      \ket{1,1} \otimes \ket{\tfrac{1}{2}, \tfrac{1}{2}},
      \ket{1,0} \otimes \ket{\tfrac{1}{2}, \tfrac{1}{2}},
      \ket{1,-1} \otimes \ket{\tfrac{1}{2}, \tfrac{1}{2}},
      \ket{1,1} \otimes \ket{\tfrac{1}{2}, -\tfrac{1}{2}},
      \ket{1,0} \otimes \ket{\tfrac{1}{2}, -\tfrac{1}{2}},
      \ket{1,-1} \otimes \ket{\tfrac{1}{2}, -\tfrac{1}{2}}
    \}
    $ we can write the matrices as

    \begin{align*}
      L_x \otimes S_x &= \frac{\hbar^2}{2} \sqrt{2}
        \begin{pmatrix}
          0
            \begin{pmatrix}
              0 & 1 & 0 \\
              1 & 0 & 1 \\
              0 & 1 & 0
            \end{pmatrix}
           &
          1
            \begin{pmatrix}
              0 & 1 & 0 \\
              1 & 0 & 1 \\
              0 & 1 & 0
            \end{pmatrix}
            \\
          1
            \begin{pmatrix}
              0 & 1 & 0 \\
              1 & 0 & 1 \\
              0 & 1 & 0
            \end{pmatrix}
           &
          0
            \begin{pmatrix}
              0 & 1 & 0 \\
              1 & 0 & 1 \\
              0 & 1 & 0
            \end{pmatrix}
          \end{pmatrix}
          =
          \frac{\hbar^2}{2} \sqrt{2}
          \begin{pmatrix}
            0 & 0 & 0 & 0 & 1 & 0 \\
            0 & 0 & 0 & 1 & 0 & 1 \\
            0 & 0 & 0 & 0 & 1 & 0 \\
            0 & 1 & 0 & 0 & 0 & 0 \\
            1 & 0 & 1 & 0 & 0 & 0 \\
            0 & 1 & 0 & 0 & 0 & 0
          \end{pmatrix} \\
      L_y \otimes S_y &= \frac{\hbar^2}{2} \sqrt{2}
        \begin{pmatrix}
          0
            \begin{pmatrix}
              0 & 1 & 0 \\
              -1 & 0 & 1 \\
              0 & -1 & 0
            \end{pmatrix}
           &
          -1
            \begin{pmatrix}
              0 & 1 & 0 \\
              -1 & 0 & 1 \\
              0 & -1 & 0
            \end{pmatrix}
            \\
          1
            \begin{pmatrix}
              0 & 1 & 0 \\
              -1 & 0 & 1 \\
              0 & -1 & 0
            \end{pmatrix}
           &
          0
            \begin{pmatrix}
              0 & 1 & 0 \\
              -1 & 0 & 1 \\
              0 & -1 & 0
            \end{pmatrix}
          \end{pmatrix}
          =
          \frac{\hbar^2}{2} \sqrt{2}
          \begin{pmatrix}
            0 & 0 & 0 & 0 & -1 & 0 \\
            0 & 0 & 0 & 1 & 0 & -1 \\
            0 & 0 & 0 & 0 & 1 & 0 \\
            0 & 1 & 0 & 0 & 0 & 0 \\
            -1 & 0 & 1 & 0 & 0 & 0 \\
            0 & -1 & 0 & 0 & 0 & 0
          \end{pmatrix} \\
      L_z \otimes S_z &= \frac{\hbar^2}{2}
        \begin{pmatrix}
          1
            \begin{pmatrix}
              1 & 0 & 0 \\
              0 & 0 & 0 \\
              0 & 0 & -1
            \end{pmatrix}
           &
          0
            \begin{pmatrix}
              1 & 0 & 0 \\
              0 & 0 & 0 \\
              0 & 0 & -1
            \end{pmatrix}
            \\
          0
            \begin{pmatrix}
              1 & 0 & 0 \\
              0 & 0 & 0 \\
              0 & 0 & -1
            \end{pmatrix}
           &
          -1
            \begin{pmatrix}
              1 & 0 & 0 \\
              0 & 0 & 0 \\
              0 & 0 & -1
            \end{pmatrix}
          \end{pmatrix}
          =
          \frac{\hbar^2}{2}
          \begin{pmatrix}
            1 & 0 & 0 & 0 & 0 & 0 \\
            0 & 0 & 0 & 0 & 0 & 0 \\
            0 & 0 & -1 & 0 & 0 & 0 \\
            0 & 0 & 0 & 1 & 0 & 0 \\
            0 & 0 & 0 & 0 & 0 & 0 \\
            0 & 0 & 0 & 0 & 0 & -1
          \end{pmatrix}
    \end{align*}



\end{enumerate}
\end{document}
