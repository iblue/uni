\documentclass[a4paper,german,12pt,smallheadings]{scrartcl}
\usepackage[T1]{fontenc}
\usepackage[utf8]{inputenc}
\usepackage{babel}
\usepackage{geometry}
\usepackage{amsmath}
\usepackage{amssymb}
\usepackage{float}
\usepackage{enumerate}
\usepackage{braket} % Teh quantum stuff
\usepackage{commath} % http://tex.stackexchange.com/questions/14821/whats-the-proper-way-to-typeset-a-differential-operator
\usepackage{cancel}
%\usepackage{wrapfig}
\usepackage[thinspace,thinqspace,squaren,textstyle]{SIunits}

% New command for color underlining
\usepackage{xcolor}

\newsavebox\MBox
\newcommand\colul[2][red]{{\sbox\MBox{$#2$}%
  \rlap{\usebox\MBox}\color{#1}\rule[-1.2\dp\MBox]{\wd\MBox}{0.5pt}}}

\restylefloat{table}
\geometry{a4paper, top=15mm, left=20mm, right=40mm, bottom=20mm, headsep=10mm, footskip=12mm}
\linespread{1.5}
\setlength\parindent{0pt}
\DeclareMathOperator{\Tr}{Tr}
\DeclareMathOperator{\Var}{Var}
\begin{document}
\allowdisplaybreaks % Seitenumbrüche in Formeln erlauben
\begin{center}
\bfseries % Fettdruck einschalten
\sffamily % Serifenlose Schrift
\vspace{-40pt}
Quantum Mechanics, winter term 2013/2014, exercise sheet 7

Markus Fenske, Tutor: Adam Nagy
\vspace{-10pt}
\end{center}

\section*{Exercise 1: Continuity equation}
We know that $\rho = \envert{\psi}^2 = \psi^*\psi$. So the time derivative is
\begin{align*}
  \od{}{t} \rho &= \od{}{t} \psi^*\psi \\
                         &= \del{\od{}{t} \psi^*} \psi + \psi^* \del{\od{}{t} \psi} \\
  \intertext{Substituting the Schrödinger equation $i \hbar \od{}{t} \psi = H \psi$:}
  &= \del{-\frac{1}{i\hbar} H \psi^*} \psi + \psi^* \del{\frac{1}{i \hbar} H \psi} \\
  \intertext{Substituting the Hamiltonian $H = \frac{p^2}{2m} + V = -\frac{\hbar^2}{2m} \nabla^2 + V$:}
  &= \del{-\frac{1}{i\hbar} \del{-\frac{\hbar^2}{2m} \nabla^2 + V} \psi^*} \psi + \psi^* \del{\frac{1}{i \hbar} \del{-\frac{\hbar^2}{2m} \nabla^2 + V} \psi} \\
  &= \del{\frac{\hbar}{2 mi} \nabla^2 \psi^*}\psi \cancel{- \del{\frac{1}{i \hbar} V \psi^*}\psi} - \psi^* \del{\frac{\hbar}{2 mi} \nabla^2 \psi} \cancel{+ \psi^* \del{\frac{1}{i \hbar} V \psi}} \\
  &= \frac{\hbar}{2 mi} \del{\del{\nabla^2 \psi^*}\psi - \psi^* \del{\nabla^2 \psi}}
\end{align*}

The gradient of the probability current density is
\begin{align*}
  \nabla j &= \nabla \frac{\hbar}{2mi} \del{\psi^* \del{\nabla \psi} - \del{\nabla \psi^*} \psi} \\
           &= \frac{\hbar}{2mi} \nabla \del{\psi^* \del{\nabla \psi} - \del{\nabla \psi^*} \psi} \\
           &= \frac{\hbar}{2mi} \del{\del{\del{\cancel{\nabla \psi^* \del{\nabla \psi}} + \psi^* \del{\nabla^2 \psi}} - \del{\del{\nabla^2 \psi^*} \psi + \cancel{\del{\nabla \psi^*} \nabla \psi}}}} \\
           &= \frac{\hbar}{2mi} \del{\psi^* \del{\nabla^2 \psi} - \del{\nabla^2 \psi^*} \psi} \\
           &= -\frac{\hbar}{2mi} \del{\del{\nabla^2 \psi^*}\psi - \psi^* \del{\nabla^2 \psi}} \\
           &= -\od{}{t} \rho
\end{align*}

Therefore
\begin{equation*}
  \od{}{t} \rho + \nabla j = 0
\end{equation*}

\section*{Exercise 2: Uncertainty principle}
\begin{enumerate}[a)]
  \item
    To be shown
    \begin{equation*}
      \Var\del{A} = \braket{A^2} - \braket{A}^2
    \end{equation*}
    It is
    \begin{equation*}
      \braket{A} = \braket{\psi|A|\psi}
    \end{equation*}
    Therefore
    \begin{align*}
      \Var\del{A} &= \braket{\del{A - \braket{A}}^2} \\
                  &= \braket{A^2 - A\braket{A} - \braket{A}A + \braket{A}^2} \\
                  &= \braket{A^2 - 2A\braket{A} + \braket{A}^2} \\
                  &= \braket{\psi|\del{A^2 - 2A\braket{A} + \braket{A}^2}|\psi} \\
                  &= \braket{\psi|A^2|\psi} + \braket{\psi|\del{-2A\braket{A}}|\psi}+ \braket{\psi|\braket{A}^2|\psi} \\
                  &= \braket{A^2} - 2\braket{A}\braket{\psi|A|\psi} + \braket{\psi|\braket{A}^2|\psi} \\
                  &= \braket{A^2} - 2\braket{A}\braket{A} + \braket{\psi|\braket{A}^2|\psi} \\
                  &= \braket{A^2} - 2\braket{A}\braket{A} + \braket{A}^2 \cancelto{1}{\braket{\psi|\psi}} \\
                  &= \braket{A^2} - 2\braket{A}^2 + \braket{A}^2 \\
                  &= \braket{A^2} - \braket{A}^2
    \end{align*}
  \item
    \begin{align*}
      \sbr{x,p} \ket{\psi} &= \del{xp - px}\ket{\psi} \\
                          &= \del{x \frac{\hbar}{i} \od{}{x} - \frac{\hbar}{i} \od{}{x} x} \ket{\psi} \\
                          &= \del{-x \hbar i \od{}{x} + \hbar i \od{}{x} x} \ket{\psi} \\
                          &= -x \hbar i \od{}{x} \ket{\psi} + \hbar i \od{}{x} x \ket{\psi} \\
                          &= -x \hbar i \od{}{x} \ket{\psi} + \hbar i \del{\ket{\psi} + x \od{}{x}\ket{\psi}} \\
                          &= \del{\cancel{-x \hbar i \od{}{x}} + \hbar i \cancel{+ x \hbar i \od{}{x}}} \ket{\psi} \\
                          &= i \hbar \ket{\psi}
    \end{align*}
\end{enumerate}

\section*{Exercise 3: Time dependence in the Heisenberge picture}
\begin{enumerate}[a)]
  \item
    Using mathematical induction.

    For $n=1$
    \begin{equation*}
      [x,p^n] = [x,p] = i \hbar = i \hbar n p^{n-1}
    \end{equation*}

    For the inductive step, we use the identitity from excercise sheet 4:
    \begin{equation*}
      [A,BC] = B[A,C] + [A,B]C
    \end{equation*}

    We have to show that if $[x,p^n] = i \hbar n p^{n-1}$, then $[x,p^{n+1}] =
    i \hbar \del{n+1} p^n$.

    \begin{align*}
      [x,p^{n+1}] &= p^n[x,p] + [x,p^n]p \\
                  &= p^n i \hbar + \del{i \hbar n p^{n-1}} p \\
                  &= i \hbar \del{p^n + n p^{n-1} p} \\
                  &= i \hbar (n+1) p^n
    \end{align*}

    Therefore the relation is true.
  \item
    \begin{align*}
      \sbr{x, e^{-iHt}} &= x \exp\del{-iHt} - \exp\del{-iHt}x \\
                        &= x \sum_{k=0}^\infty \frac{\del{-it}^k}{k!} H^k - \sum_{k=0}^\infty \frac{\del{-it}^k}{k!} H^k x \\
                        &= \sum_{k=0}^\infty \frac{\del{-it}^k}{k!} xH^k - \sum_{k=0}^\infty \frac{\del{-it}^k}{k!} H^k x \\
                        &= \sum_{k=0}^\infty \frac{\del{-it}^k}{k!} \del{xH^k - H^kx} \\
                        &= \sum_{k=0}^\infty \frac{\del{-it}^k}{k!} \del{x \del{\frac{p^2}{2m}}^k - \del{\frac{p^2}{2m}}^k x} \\
                        &= \sum_{k=0}^\infty \frac{\del{-it}^k}{k!} \frac{1}{\del{2m}^k} \del{x p^{2k} - p^{2k} x} \\
                        &= \sum_{k=0}^\infty \frac{\del{-it}^k}{k!} \frac{1}{\del{2m}^k} \sbr{x,p^{2k}} \\
                        &= \sum_{k=0}^\infty \frac{\del{-it}^k}{k!} \frac{1}{\del{2m}^k} i \hbar 2k p^{2k-1} \\
                        &= 2 \hbar i \sum_{k=0}^\infty \frac{\del{-it}^k}{\del{k-1}!} \frac{p^{2k-1}}{\del{2m}^k} \\
                        &= \frac{\del{2 \hbar i}\del{-it}}{2m} \sum_{k=0}^\infty \frac{\del{-it}^{k-1}}{\del{k-1}!} \frac{p^{2(k-1)}}{\del{2m}^{k-1}} p \\
                        &= \frac{\hbar t}{m} p \exp\del{-it \frac{p^2}{2m}} \\
                        &= \frac{\hbar t}{m} p \exp\del{-it H}
    \end{align*}
  \item
    \begin{align*}
      x(t) &= \exp\del{iHt} x \exp\del{-iHt} \\
           &= \exp\del{iHt} \del{\sbr{x, \exp\del{-iHt}} + \exp\del{-iHt}x} \\
           &= \exp\del{iHt} \sbr{x, \exp\del{-iHt}} + \cancel{\exp\del{iHt}\exp\del{-iHt}}x \\
           &= \exp\del{iHt} \frac{\hbar t}{m} p \exp\del{-iHt} + x \\
           &= \frac{\hbar t}{m} \exp\del{iHt} p \exp\del{-iHt} + x \\
           &= \frac{\hbar t}{m} p(t) + x
    \end{align*}
  \item
    In the assignment, the time evolution operator is $\exp iHt$, therefore we
    assume that per definition $\hbar = 1$. The correct time evoltion operator would be
    $\exp \frac{iHt}{\hbar}$, so in the derivative an $\hbar$ would appear. We
    just add the $\hbar$ in the middle.

    \begin{align*}
      \od{}{t} x(t) &= \od{}{t} \del{\exp\del{iHt} x \exp\del{-iHt}} \\
                    &= \del{\pd{}{t} \exp\del{iHt}} x \exp\del{-iHt} +
                       \exp\del{iHt} \underbrace{\del{\pd{}{t} x}}_{=0} \exp\del{-iHt} +
                       \exp\del{iHt} x \del{\pd{}{t} \exp\del{-iHt}} \\
                    &= \exp\del{iHt} \del{iH} x \exp\del{-iHt} + \exp\del{iHt}x \del{-iH} \exp\del{-iHt} \\
                    &= -\frac{i}{\hbar}\del{\exp\del{iHt} H x \exp\del{-iHt} - \exp{iHt}xH\exp\del{-iHt}} \\
                    &= -\frac{i}{\hbar}\del{H(t)x(t) - x(t)H(t)} \\
                    &= -\frac{i}{\hbar} \sbr{H(t), x(t)}
    \end{align*}
  \item
    \begin{align*}
      \braket{x}(t) &= \braket{\psi(t)|x|\psi(t)} \\
                    &= \braket{\psi_0|x(t)|\psi_0} \\
                    &= \braket{\psi_0|\frac{\hbar t}{m} p(t) + x|\psi_0} \\
                    &= \braket{\psi_0|\frac{\hbar t}{m} p(t)|\psi_0} + \braket{\psi_0|x|\psi_0} \\
                    &= \frac{\hbar t}{m} \braket{p(t)} + \braket{x}
    \end{align*}
\end{enumerate}
\end{document}
