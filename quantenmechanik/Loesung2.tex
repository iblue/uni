\documentclass[a4paper,german,12pt,smallheadings]{scrartcl}
\usepackage[T1]{fontenc}
\usepackage[utf8]{inputenc}
\usepackage{babel}
\usepackage{tikz}
\usepackage{geometry}
\usepackage{amsmath}
\usepackage{amssymb}
\usepackage{float}
\usepackage{enumerate}
%\usepackage{wrapfig}
\usepackage[thinspace,thinqspace,squaren,textstyle]{SIunits}
\restylefloat{table}
\geometry{a4paper, top=15mm, left=20mm, right=40mm, bottom=20mm, headsep=10mm, footskip=12mm}
\linespread{1.5}
\setlength\parindent{0pt}
\begin{document}
\begin{center}
\bfseries % Fettdruck einschalten
\sffamily % Serifenlose Schrift
\vspace{-40pt}
Quantum Mechanics, winter term 2013/2014, exercise sheet 1

Markus Fenske, Tutor: Earl Campbell
\vspace{-10pt}
\end{center}

\section*{Exercise 2: Compton effect}

\begin{enumerate}[a)]
  \item
    The Compton wavelength is calculated by
    \begin{equation*}
      \lambda = \frac{h}{mc}
    \end{equation*}

    The electron mass is not given. I will use the literature value for now,
    because deriving the electron mass is beyond the scope of this exercise
    sheet.

    With $m \approx 9.11 \cdot 10^{-31} \; \kilogram$ I get

    \begin{equation*}
      \lambda \approx 2.4 \; \pico\meter
    \end{equation*}
\end{enumerate}
\end{document}
