\documentclass[a4paper,german,12pt,smallheadings]{scrartcl}
\usepackage[T1]{fontenc}
\usepackage[utf8]{inputenc}
\usepackage{babel}
\usepackage{tikz}
\usepackage{geometry}
\usepackage{amsmath}
\usepackage{amssymb}
\usepackage{float}
\usepackage{enumerate}
%\usepackage{wrapfig}
\usepackage[thinspace,thinqspace,squaren,textstyle]{SIunits}
\restylefloat{table}
\geometry{a4paper, top=15mm, left=20mm, right=40mm, bottom=20mm, headsep=10mm, footskip=12mm}
\linespread{1.5}
\setlength\parindent{0pt}
\begin{document}
\begin{center}
\bfseries % Fettdruck einschalten
\sffamily % Serifenlose Schrift
\vspace{-40pt}
Quantum Mechanics, winter term 2013/2014, exercise sheet 2

Markus Fenske, Tutor: Earl Campbell
\vspace{-10pt}
\end{center}

\section*{Exercise 1: Photoelectric effect}

\begin{enumerate}[a)]
  \item
    The maximum kinetic energy of an ejected electron is
    \begin{equation*}
      E_\text{kin} = h \nu - W_A = \frac{hc}{\lambda} - W_A \approx 1.44 \; \electronvolt
    \end{equation*}

    Because of $E_\text{kin} = q U$, the voltage $U$ required to stop the electron is
    \begin{equation*}
      \frac{E_\text{kin}}{e} = 1.44 \; \volt
    \end{equation*}

    Electrons with different energies are emitted, because not only movable
    electrons can be emitted, but also electrons bound to atoms. These need
    more energy to be ejected, therefore have less kinetic energy.

  \item
    Each photon has an energy of
    \begin{equation*}
      E = \frac{hc}{\lambda}
    \end{equation*}

    The amount of electrons ejected is
    \begin{equation*}
      n = \frac{S A t}{E} = \frac{S A t \lambda}{hc} \approx 5.437
    \end{equation*}

    The current is
    \begin{equation*}
      I = \frac{\Delta Q}{\Delta t} = \frac{5.437 \; \text{e}}{1 \; \milli\second} \approx 8.71 \cdot 10^{-16} \; \ampere
    \end{equation*}

  \item
    An electron cannot absorb a photon, because the electron has a spin of $
    \frac{\sqrt{3}}{2} \hbar$, while a photon has a spin of $\pm \hbar$. Because spin
    is conserved, the photon cannot be destroyed.

    Unfortunately the assignment is to show this using conservation of energy
    and momentum, which takes a few more lines.

    Assuming non-relativistic speeds. Before the interaction, energy and
    momentum of photon and electron are:
    \begin{align*}
      \gamma: \quad \quad & p = \frac{E_\gamma}{c}  &\quad e^-: \quad \quad & p = 0 \\
                          & E = E_\gamma &                & E_\text{kin} = 0 \\
    \end{align*}

    After the interaction, the photon has vanished. Energy and momentum of the
    electron are:
    \begin{align*}
      &p = mv \\
      &E = \frac{1}{2} mv^2
    \end{align*}

    Using energy and momentum conservation:
    \begin{align*}
      \frac{E_\gamma}{c} &= mv\\
      E_\gamma &= \frac{1}{2} mv^2
    \end{align*}

    From momentum, we get:
    \begin{equation*}
      v = \frac{E_\gamma}{mc}
    \end{equation*}

    From energy:
    \begin{equation*}
      v = \sqrt{\frac{2 E_\gamma}{m}}
    \end{equation*}

    These two equations are not compatible ($v \neq v$). Therefore, a free
    electron cannot absorb a photon.



\end{enumerate}

\section*{Exercise 2: Compton effect}

\begin{enumerate}[a)]
  \item
    The Compton wavelength is calculated by
    \begin{equation*}
      \lambda = \frac{h}{mc}
    \end{equation*}

    The electron mass is not given. I will use the literature value for now,
    because deriving the electron mass is beyond the scope of this exercise
    sheet.

    With $m \approx 9.11 \cdot 10^{-31} \; \kilogram$ I get

    \begin{equation*}
      \lambda \approx 2.4 \; \pico\meter
    \end{equation*}

    The energy is calculated by $E = h \nu$ where $\nu = \frac{c}{\lambda}$,
    which leads to $E = \frac{hc}{\lambda}$. With the Compton wavelength, this
    results in $E=mc^2$.

    So the Compton energy \textbf{equals the rest energy} and is
    \begin{equation*}
      E \approx 511 \; \kilo\electronvolt
    \end{equation*}

  \item
    Using
    \begin{equation*}
      \Delta \lambda = \lambda_{\text{C}} \left( 1 - \cos \phi \right)
    \end{equation*}

    we get
    \begin{equation*}
      \Delta \lambda \approx 710.6 \; \femto\meter
    \end{equation*}

    The change of photon energy is calculated by
    \begin{equation*}
      p = \frac{E_\text{after}}{E_\text{before}} - 1 = - E_{\text{before}} \frac{\Delta\lambda}{hc} \approx - 5.73 \cdot 10^{-4}
    \end{equation*}

    so the photon energy decreases by $0.0573$ \%.

    With $E_{\text{before}} = 10^3 \; \kilo\electronvolt$, the energy would
    decrease by $57.3$ \% (as seen from the formula, percentage increases
    proportionally).

    According to energy conservation, the energy lost by the photon results in
    a kinetic energy of the electron. With the values calculated above, the
    kinetic energy of the electron is

    \begin{equation*}
      E_\text{kin} = 0.573 \; \electronvolt
    \end{equation*}
\end{enumerate}

\section*{Exercise 3: de Broglie wavelength}

\begin{enumerate}[a)]
  \item
    The energy of the photon is
    \begin{equation*}
      E = h \nu = \frac{hc}{\lambda} \Leftrightarrow \lambda = \frac{hc}{E}
    \end{equation*}

    The de Broglie wavelength for the electron is (in case of non-relativistic particle speeds):
    \begin{equation*}
      \lambda = \frac{h}{p} = \frac{h}{m_e v}
    \end{equation*}

    Which leads to
    \begin{equation*}
      \frac{hc}{E} = \frac{h}{m_e v} \Leftrightarrow v = \frac{E}{m_e c}
    \end{equation*}

    (With the given values this results in $v \approx 0.0978 c$, so a
    non-relativistic calculation is appropriate, because $\sqrt{1 -
    \frac{v^2}{c^2}} \approx 1$)

    The kinetic energy of the electrons then must be
    \begin{equation*}
      E = \frac{1}{2} m_ev^2 = \frac{1}{2} \frac{E_\gamma^2}{m_e c^2} \approx 4.89 \; \kilo\electronvolt
    \end{equation*}
  \item
    The momentum of a particle with kinetic energy $E$ is:
    \begin{equation*}
      E = \frac{p^2}{2m} \Leftrightarrow p = \sqrt{2 E m}
    \end{equation*}

    The average kinetic energy $E$ of a particle at temperatur $T$ is:
    \begin{equation*}
      E = \pi k_B T
    \end{equation*}

    This leads to a momentum $p$:
    \begin{equation*}
      p = \sqrt{2 \pi k_B T m}
    \end{equation*}

    And a de Broglie wavelength:
    \begin{equation*}
      \lambda = \frac{h}{p} = \frac{h}{\sqrt{2 \pi k_B T m}} \approx 50.38 \; \pico\meter
    \end{equation*}

  \item
    A FIFA approved football ball weights between 420 and 445 grams. According
    to the Guinness book of world records web page, the fastest football kick
    ($129 \;\kilo\meter\per\second$) was achieved by Francisco Javier Galan
    Màrin at the studios of El Show de los Récords in Madrid on 29 October
    2001.

    The slowest ball would probably be a ball cooled by liquid helium
    to approximately $4 \; kelvin$.

    The fastest, heaviest ball has a de Broglie wavelength of
    \begin{equation*}
      \lambda = \frac{h}{mv} \approx 4 \cdot 10^{-35} \; \meter \approx 2.5 l_P
    \end{equation*}

    where $l_P$ is Planck's length.

    The slowest, lightest balls wavelength can be calculated by the thermal de
    Broglie formula derived earlier (assuming the ball is a particle).

    \begin{equation*}
      \lambda = \frac{h}{p} = \frac{h}{\sqrt{2 \pi k_B T m}} \approx 5.5 \cdot 10^{-23} \; \meter
    \end{equation*}

    The wavelengths are by orders of magnitude to small to be detected, so
    assigning a de Broglie wavelength to a ball does not make sense. A ball can
    be described by classical mechanics.
\end{enumerate}
\end{document}
