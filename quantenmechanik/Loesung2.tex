\documentclass[a4paper,german,12pt,smallheadings]{scrartcl}
\usepackage[T1]{fontenc}
\usepackage[utf8]{inputenc}
\usepackage{babel}
\usepackage{tikz}
\usepackage{geometry}
\usepackage{amsmath}
\usepackage{amssymb}
\usepackage{float}
\usepackage{enumerate}
%\usepackage{wrapfig}
\usepackage[thinspace,thinqspace,squaren,textstyle]{SIunits}
\restylefloat{table}
\geometry{a4paper, top=15mm, left=20mm, right=40mm, bottom=20mm, headsep=10mm, footskip=12mm}
\linespread{1.5}
\setlength\parindent{0pt}
\begin{document}
\begin{center}
\bfseries % Fettdruck einschalten
\sffamily % Serifenlose Schrift
\vspace{-40pt}
Quantum Mechanics, winter term 2013/2014, exercise sheet 2

Markus Fenske, Tutor: Earl Campbell
\vspace{-10pt}
\end{center}

\section*{Exercise 2: Compton effect}

\begin{enumerate}[a)]
  \item
    The Compton wavelength is calculated by
    \begin{equation*}
      \lambda = \frac{h}{mc}
    \end{equation*}

    The electron mass is not given. I will use the literature value for now,
    because deriving the electron mass is beyond the scope of this exercise
    sheet.

    With $m \approx 9.11 \cdot 10^{-31} \; \kilogram$ I get

    \begin{equation*}
      \lambda \approx 2.4 \; \pico\meter
    \end{equation*}

    The energy is calculated by $E = h \nu$ where $\nu = \frac{c}{\lambda}$,
    which leads to $E = \frac{hc}{\lambda}$. With the Compton wavelength, this
    results in $E=mc^2$.

    So the Compton energy \textbf{equals the rest energy} and is
    \begin{equation*}
      E \approx 511 \; \kilo\electronvolt
    \end{equation*}

  \item
    Using
    \begin{equation*}
      \Delta \lambda = \lambda_{\text{C}} \left( 1 - \cos \phi \right)
    \end{equation*}

    we get
    \begin{equation*}
      \Delta \lambda \approx 710.6 \; \femto\meter
    \end{equation*}

    The change of photon energy is calculated by
    \begin{equation*}
      p = \frac{E_\text{after}}{E_\text{before}} - 1 = - E_{\text{before}} \frac{\Delta\lambda}{hc} \approx - 5.73 \cdot 10^{-4}
    \end{equation*}

    so the photon energy decreases by $0.0573$ \%.

    With $E_{\text{before}} = 10^3 \; \kilo\electronvolt$, the energy would
    decrease by $57.3$ \% (as seen from the formula, percentage increases
    proportionally).

    According to energy conservation, the energy lost by the photon results in
    a kinetic energy of the electron. With the values calculated above, the
    kinetic energy of the electron is

    \begin{equation*}
      E_\text{kin} = 0.573 \; \electronvolt
    \end{equation*}
\end{enumerate}

\section*{Exercise 3: de Broglie wavelength}

\begin{enumerate}[a)]
  \item
    The energy of the photon is
    \begin{equation*}
      E = h \nu = \frac{hc}{\lambda} \Leftrightarrow \lambda = \frac{hc}{E}
    \end{equation*}

    The de Broglie wavelength for the electron is (in case of non-relativistic particle speeds):
    \begin{equation*}
      \lambda = \frac{h}{p} = \frac{h}{m_e v}
    \end{equation*}

    Which leads to
    \begin{equation*}
      \frac{hc}{E} = \frac{h}{m_e v} \Leftrightarrow v = \frac{E}{m_e c}
    \end{equation*}

    (With the given values this results in $v \approx 0.0978 c$, so a
    non-relativistic calculation is appropriate, because $\sqrt{1 -
    \frac{v^2}{c^2}} \approx 1$)

    The kinetic energy of the electrons then must be
    \begin{equation*}
      E = \frac{1}{2} m_ev^2 = \frac{1}{2} \frac{E_\gamma^2}{m_e c^2} \approx 4.89 \; \kilo\electronvolt
    \end{equation*}
  \item
    The momentum of a particle with kinetic energy $E$ is:
    \begin{equation*}
      E = \frac{p^2}{2m} \Leftrightarrow p = \sqrt{2 E m}
    \end{equation*}

    The average kinetic energy $E$ of a particle at temperatur $T$ is:
    \begin{equation*}
      E = \pi k_B T
    \end{equation*}

    This leads to a momentum $p$:
    \begin{equation*}
      p = \sqrt{2 \pi k_B T m}
    \end{equation*}

    And a de Broglie wavelength:
    \begin{equation*}
      \lambda = \frac{h}{p} = \frac{h}{\sqrt{2 \pi k_B T m}} \approx 50.38 \; \pico\meter
    \end{equation*}


\end{enumerate}
\end{document}
