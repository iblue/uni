\documentclass[a4paper,german,12pt,smallheadings]{scrartcl}
\usepackage[T1]{fontenc}
\usepackage[utf8]{inputenc}
\usepackage{babel}
\usepackage{geometry}
\usepackage{amsmath}
\usepackage{amssymb}
\usepackage{float}
\usepackage{enumerate}
\usepackage{braket} % Teh quantum stuff
\usepackage{commath} % http://tex.stackexchange.com/questions/14821/whats-the-proper-way-to-typeset-a-differential-operator
\usepackage{cancel}
%\usepackage{wrapfig}
\usepackage[thinspace,thinqspace,squaren,textstyle]{SIunits}

% New command for color underlining
\usepackage{xcolor}

\newsavebox\MBox
\newcommand\colul[2][red]{{\sbox\MBox{$#2$}%
  \rlap{\usebox\MBox}\color{#1}\rule[-1.2\dp\MBox]{\wd\MBox}{0.5pt}}}

\restylefloat{table}
\geometry{a4paper, top=15mm, left=20mm, right=40mm, bottom=20mm, headsep=10mm, footskip=12mm}
\linespread{1.5}
\setlength\parindent{0pt}
\DeclareMathOperator{\Tr}{Tr}
\DeclareMathOperator{\Var}{Var}
\begin{document}
\allowdisplaybreaks % Seitenumbrüche in Formeln erlauben
\begin{center}
\bfseries % Fettdruck einschalten
\sffamily % Serifenlose Schrift
\vspace{-40pt}
Quantum Mechanics, winter term 2013/2014, exercise sheet 11

Markus Fenske, Tutor: Adam Nagy
\vspace{-10pt}
\end{center}

\section*{Exercise 1: Squeezed states}
\begin{enumerate}[a)]
  \item
    From $S(\xi) = \exp \frac{1}{2} \del{\xi^* a^2 - \xi {a^\dagger}^2}$ we get:
    \begin{align*}
      S(\xi)^\dagger = \exp \frac{1}{2} \del{\xi {a^\dagger}^2 - \xi^* a^2} = S(-\xi)
    \end{align*}

    We immediatly see that $S(\xi)S(-\xi) = I$ (because $e^a e^{-a} = 1$).
    Therefore $S(\xi)^\dagger = S(\xi)^{-1}$.
  \item
    First of all, since we are lazy we define a combined position and momentum
    operator $U_\pm$, so we do not have to do the same thing twice.

    \begin{align*}
      U_+ &= k_+ \del{a + a^\dagger} \qquad \text{(where $k_+ := \sqrt{\frac{\hbar}{2 m \omega}}$)} \\
      U_- &= k_- \del{a - a^\dagger} \qquad \text{(where $k_- := -i\sqrt{\frac{m \omega \hbar}{2}}$)} \\
    \end{align*}

    So we have $U_+ = X$, $U_- = P$.

    As given in the excerise sheet, we know that $\xi \in \mathbb{R}$, so
    $\xi^* = \xi$. This leads to

    \begin{align*}
      S(\xi) = \exp\del{\frac{1}{2} \xi \del{a^2 + {a^\dagger}^2}}
    \end{align*}

    We define
    \begin{align*}
      B := \frac{1}{2} \del{a^2 + {a^\dagger}^2}
    \end{align*}

    Therefore
    \begin{align*}
      F(\xi) = S^\dagger(\xi) U_\pm S(\xi) = e^{-\xi B} U_\pm e^{\xi B}
    \end{align*}

    The derivative is
    \begin{equation}
      \od{F}{\xi} = -Be^{-\xi B}U_\pm e^{\xi B} + e^{-\xi B}U_\pm B e^{\xi B}
      \label{deriv}
    \end{equation}

    In order to rewrite (\ref{deriv}) we solve the following commutator using
    the identities $[A,B] = -[B,A]$ and $[AB,C] = A[B,C] + [A,C]B$.
    \begin{align*}
      [U_\pm,B] &= \frac{1}{2} k_\pm \left[\del{a \pm a^\dagger}, \del{a^2 - {a^\dagger}^2}\right] \\
            &= \frac{1}{2} k_\pm \del{\underbrace{[a,a^2]}_{=0} - [a,{a^\dagger}^2] \pm [a^\dagger, a^2] \mp \underbrace{[a^\dagger, {a^\dagger}^2]}_{=0}} \\
            &= \frac{1}{2} k_\pm \del{[{a^\dagger}^2, a] \pm [a^2, a^\dagger]} \\
            &= \frac{1}{2} k_\pm \del{a^\dagger \underbrace{[a^\dagger,a] }_{=-1}+ \underbrace{[a^\dagger, a]}_{=-1}a^\dagger \pm a\underbrace{[a,a^\dagger]}_{=1} \pm \underbrace{[a,a^\dagger]}_{=1}a} \\
            &= \frac{1}{2} k_\pm \del{\pm2a^\dagger -2a} \\
            &= \mp k_\pm \del{a^\dagger \pm a} \\
            &= \mp U_\pm
    \end{align*}

    Therefore $U\pm B = BU_\pm \mp U_\pm$. Inserting this into (\ref{deriv}) leads
    \begin{equation}
      \od{F}{\xi} = -Be^{-\xi B}U_\pm e^{\xi B} + e^{-\xi B} \del{B U_\pm \mp U_\pm} e^{\xi B}
    \end{equation}

    Also note that $[B,e^{\pm B}] = 0$ (can be seen if you wrote $e^{\pm B}$ as series). Therefore

    \begin{align*}
      \od{F}{\xi} &= -B e^{-\xi B}U_\pm e^{\xi B} + e^{-\xi B} \del{BU_\pm \mp U_\pm} e^{\xi B} \\
                  &= -B e^{-\xi B}U_\pm e^{\xi B} + B e^{-\xi B} U_\pm e^{\xi B} \mp e^{-\xi B}U_\pm e^{\xi B} \\
                  &= \mp e^{-\xi B}U_\pm e^{\xi B} \\
                  &= \mp F(\xi)
    \end{align*}

    We now write $F(\xi)$ as Taylor series in $\xi = 0$

    \begin{equation*}
      F(\xi) = \sum_{n=0}^\infty \frac{1}{n!} \od[n]{F}{\xi}(0) \del{\xi - 0}^n
    \end{equation*}

    From the differential equation, we immediatly see that $\od[n]{F}{\xi} = (\mp1)^n F(\xi)$.

    \begin{equation*}
      F(\xi) = \sum_{n=0}^\infty \frac{(\mp 1)^n}{n!} F(0) \xi^n
    \end{equation*}

    We also see that $F(0) = e^0 U_\pm e^0 = U_\pm$. Therefore

    \begin{equation*}
      F(\xi) = U_\pm \sum_{n=0}^\infty \frac{(\mp \xi)^n}{n!}
    \end{equation*}

    The series is just $e^{-\xi}$, so the solution is

    \begin{equation}
      F(\xi) = U_\pm e^{\mp \xi}
    \end{equation}

    Because $U_+ = X, U_- = P$ we get

    \begin{equation}
      S^\dagger X S = Xe^{-\xi}
    \end{equation}

    and
    \begin{equation}
      S^\dagger P S = Pe^{\xi}
    \end{equation}

  \item
    We get the following an expectation values.

    \begin{align*}
      \braket{X}
      &= \braket{\xi|X|\xi}
      = \braket{0|S^\dagger(\xi)XS(\xi)|0}
      = \braket{0|X e^{-\xi}|0}
      = 0
      \\
      \braket{P}
      &= \braket{\xi|P|\xi}
      = \braket{0|S^\dagger(\xi)PS(\xi)|0}
      = \braket{0|P e^{\xi}|0}
      = 0
    \end{align*}

    And the following variances
    \begin{align*}
      \operatorname{Var}(X)
      &= \braket{X^2} - \braket{X}^2
      = \braket{X^2}
      = \braket{\xi|X^2|\xi}
      = \braket{0|(\del{^\dagger(\xi)XS(\xi)}^2|0}
      = \braket{0|\del{Xe^{-\xi}}^2|0} \\
      &= \braket{0|X^2|0} e^{-2\xi}
      = \frac{\hbar}{2 m \omega} \braket{0|(a^\dagger + a)^2|0} e^{-2\xi}
      = \frac{\hbar}{2 m \omega} \braket{0|{a^\dagger}^2 + a^\dagger a + aa^\dagger + a^2|0} e^{-2\xi} \\
      &= \frac{\hbar}{2 m \omega} \del{\braket{0|{a^\dagger}^2|0} + \braket{0|N|0} + \braket{0|N+1|0} + \braket{0|a^2|0}} e^{-2\xi}
      = \frac{\hbar}{2 m \omega} e^{-2 \xi} \\
      \operatorname{Var}(P)
      &= \braket{P^2} - \braket{P}^2
      = \braket{P^2}
      = \braket{\xi|P^2|\xi}
      = \braket{0|(\del{^\dagger(\xi)PS(\xi)}^2|0}
      = \braket{0|\del{Pe^{\xi}}^2|0} \\
      &= \braket{0|P^2|0} e^{2\xi}
      = -\frac{m \omega \hbar}{2} \braket{0|(a^\dagger - a)^2|0} e^{2\xi}
      = -\frac{m \omega \hbar}{2} \braket{0|{a^\dagger}^2 - a^\dagger a - aa^\dagger + a^2|0} e^{2\xi} \\
      &= -\frac{m \omega \hbar}{2} \del{\braket{0|{a^\dagger}^2|0} - \braket{0|N|0} - \braket{0|N+1|0} + \braket{0|a^2|0}} e^{2\xi}
      = \frac{m \omega \hbar}{2} e^{2 \xi} \\
    \end{align*}

    Which leads to the following uncertainties
    \begin{align*}
      \Delta x = \sqrt{\operatorname{Var}(X)} = \sqrt{\frac{\hbar}{2 m \omega}} e^{-\xi}
      \qquad\qquad
      \Delta p = \sqrt{\operatorname{Var}(P)} = \sqrt{\frac{m \omega \hbar}{2}} e^{\xi}
    \end{align*}

    So obviously these states are termed \textit{squeezed states}, because the
    are squeezed in space, which leads to a spread in momentum, so the
    uncertainty relation is not violated.

    \begin{align*}
      \Delta x \Delta p = \frac{\hbar}{2}
    \end{align*}
\end{enumerate}
\section*{Exercise 2: Rotational symmetric problems and Bohr model}
\begin{enumerate}[a)]
  \item
  \item
    It follows that
    \begin{equation*}
    [H, L^2] = \frac{1}{2m} \underbrace{[H, P^2]}_{=0} + V \underbrace{[H, X^2]}_{=0} = 0
    \end{equation*}

    Using Ehrenfest's theorem, we see that
    \begin{equation*}
      \od{}{t} \braket{L^2} = -\frac{i}{\hbar} \braket{\underbrace{[L^2, H]}_{=0}} + \braket{\underbrace{\pd{L}{t}}_{=0}} = 0
    \end{equation*}

    So the expectation value of $L^2$ is constant in time.

    Also if two operators commute they can be diagonlized simultaniously and
    thus have the same eigenvectors.
  \item
    Bohr postulated a quantization of the angualr momentum: $L = n \hbar$. The
    angular momentum is defined as $L = r \times p = r \times mv$. We assume,
    that electrons move on circular orbits, where radius vector and velocity
    vector are orthogonal. So $r || v$ leads to $L = mvr$. So we get from $n
    \hbar = mvr$.

    \begin{equation*}
      v = \frac{n \hbar}{mr}
    \end{equation*}

    We assume a Coulomb potential which can be written using $e_0^2 =
    \frac{e^2}{4 \pi \epsilon_0}$ as
    \begin{equation*}
      V(r) = - \frac{e_0^2}{r}
    \end{equation*}

    Which leads to a Coulomb force
    \begin{equation*}
      F_{el} = \frac{e_0^2}{r^2}
    \end{equation*}

    There is also a centripetal force of
    \begin{equation*}
      F_{z} = \frac{mv^2}{r}
    \end{equation*}

    For the electron to be in a stable orbit, they have to be equal, so
    \begin{equation*}
      \frac{e_0^2}{r^2} = \frac{m}{r} \del{\frac{n \hbar}{m r}}^2
    \end{equation*}

    Leads to a orbit radius of
    \begin{equation*}
      r = \frac{n^2\hbar^2}{me_0^2}
    \end{equation*}

    Using this, we can calculate the possible energy levels
    \begin{align*}
      E_n &= V + T \\
          &= -\frac{e_0^2}{r^2} + \frac{1}{2} m v^2 \\
          &= -e_0^2 \frac{me_0^2}{n^2 \hbar^2} + \frac{1}{2} \frac{n^2 \hbar^2}{m^2 r^2} \\
          &= -\frac{me_0^4}{n^2 \hbar^2} + \frac{1}{2} \frac{n^2 \hbar^2}{m^2} \frac{m^2e_0^4}{n^2 \hbar^2} \\
          &= -\frac{me_0^4}{n^2 \hbar^2} + \frac{1}{2} \frac{me_0^4}{n^2 \hbar^2} \\
          &= -\frac{1}{2} \frac{me_0^4}{n^2 \hbar^2}
    \end{align*}

    Which is exactly the result that can be obtained quantum mechanically (see
    script 5.97).

  \item
    The idea is that electrons create standing waves around the orbit. The
    orbit circumference is $2 \pi r$ and it must be multiples of the wavelength
    $\lambda$.
    \begin{equation*}
      n \lambda = 2 \pi r
    \end{equation*}

    Substituting the De-Broglie wavelength $\lambda = \frac{h}{p}$ gives
    \begin{align*}
                           &n \frac{h}{mv} = 2 \pi r \\
      \Leftrightarrow\quad &n \hbar = mvr = L
    \end{align*}

    Which is Bohrs ansatz. So the quantization of the angular momentum can be
    explained using De-Broglies wavelength.

\end{enumerate}
\section*{Exercise 3: Rotation operator}
\begin{enumerate}[a)]
  \item
     An operator that reverses a rotation by angle $\theta$ has the same effect
     as an operator that rotates by angle $-\theta$. Therefore
     $R(\mathbf{n},\theta)^{-1} = R(\mathbf{n},-\theta)$.

     A rotation by angle $\beta$, then by angle $\alpha$ is the same as a
     rotation by angle $\alpha + \beta$. Therefore $R(\mathbf{n}, \alpha) R(\mathbf{n}, \beta) =
     R(\mathbf{n}, \alpha + \beta)$.

   \item
     As given in the exercise sheet, we assume a rotation around the $z$ axis:
     $\mathbf{n} = \begin{pmatrix} 0 \\ 0 \\ 1\end{pmatrix}$

     \begin{align*}
       R(\mathbf{n}, \epsilon) \mathbf{x} &= \del{1 - \frac{i}{\hbar} \epsilon \mathbf{n} \cdot \mathbf{L} + \mathcal{O}(\epsilon^2)} \mathbf{x} \\
                                          &= \mathbf{x} - \frac{i}{\hbar} \epsilon L_z \mathbf{x} + \mathcal{O}(\epsilon^2)
       \intertext{Inserting $L_z = - i \hbar \del{x_1 \partial_2 - x_2 \partial_1}$:}
                                          &= \mathbf{x} - \epsilon \del{x_1 \partial_2 - x_2 \partial_1} \mathbf{x} + \mathcal{O}(\epsilon^2) \\
                                          &= \mathbf{x} - \epsilon \begin{pmatrix} - x_2 \\ x_1 \\ 0 \end{pmatrix}  + \mathcal{O}(\epsilon^2) \\
                                          &= \mathbf{x} - \epsilon \mathbf{n} \times \mathbf{x} + \mathcal{O}(\epsilon^2)
     \end{align*}

     Despite the sign mistake in the excercise sheet this is a rotation matrix
     for small rotations with angle $\epsilon$.

   \item
     To rotate by angle $\theta$, we just need $N$ small rotations by
     $\epsilon$, where $N = \frac{\theta}{\epsilon}$. Since we have a formula
     for infinetismally small rotations, we must have $\epsilon \to 0$, so $N
     \to \infty$. So we can write:

     \begin{align*}
       R(\mathbf{n}, \theta) &= \lim_{N \to \infty} R\del{\mathbf{n}, \frac{\theta}{N}}^N \\
                             &= \lim_{N \to \infty} \del{1 - \frac{i}{\hbar} \frac{\theta}{N} \mathbf{n} \mathbf{L}}^N \\
       \intertext{Analogous to $\lim_{n \to \infty} (1 - \frac{x}{n})^n = e^{-x}$, we can solve this limit as}
                               &= e^{-\frac{i}{\hbar} \theta \mathbf{n} \mathbf{L}}
     \end{align*}
\end{enumerate}
\end{document}
