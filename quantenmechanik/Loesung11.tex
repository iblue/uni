\documentclass[a4paper,german,12pt,smallheadings]{scrartcl}
\usepackage[T1]{fontenc}
\usepackage[utf8]{inputenc}
\usepackage{babel}
\usepackage{geometry}
\usepackage{amsmath}
\usepackage{amssymb}
\usepackage{float}
\usepackage{enumerate}
\usepackage{braket} % Teh quantum stuff
\usepackage{commath} % http://tex.stackexchange.com/questions/14821/whats-the-proper-way-to-typeset-a-differential-operator
\usepackage{cancel}
%\usepackage{wrapfig}
\usepackage[thinspace,thinqspace,squaren,textstyle]{SIunits}

% New command for color underlining
\usepackage{xcolor}

\newsavebox\MBox
\newcommand\colul[2][red]{{\sbox\MBox{$#2$}%
  \rlap{\usebox\MBox}\color{#1}\rule[-1.2\dp\MBox]{\wd\MBox}{0.5pt}}}

\restylefloat{table}
\geometry{a4paper, top=15mm, left=20mm, right=40mm, bottom=20mm, headsep=10mm, footskip=12mm}
\linespread{1.5}
\setlength\parindent{0pt}
\DeclareMathOperator{\Tr}{Tr}
\DeclareMathOperator{\Var}{Var}
\begin{document}
\allowdisplaybreaks % Seitenumbrüche in Formeln erlauben
\begin{center}
\bfseries % Fettdruck einschalten
\sffamily % Serifenlose Schrift
\vspace{-40pt}
Quantum Mechanics, winter term 2013/2014, exercise sheet 11

Markus Fenske, Tutor: Adam Nagy
\vspace{-10pt}
\end{center}

\section*{Exercise 1: Squeezed states}
\begin{enumerate}[a)]
  \item
    From $S(\xi) = \exp \frac{1}{2} \del{\xi^* a^2 - \xi {a^\dagger}^2}$ we get:
    \begin{align*}
      S(\xi)^\dagger = \exp \frac{1}{2} \del{\xi {a^\dagger}^2 - \xi^* a^2} = S(-\xi)
    \end{align*}

    We immediatly see that $S(\xi)S(-\xi) = I$ (because $e^a e^{-a} = 1$).
    Therefore $S(\xi)^\dagger = S(\xi)^{-1}$.
  \item
    As given in the excerise sheet, we know that $\xi \in \mathbb{R}$, so
    $\xi^* = \xi$. This leads to

    \begin{align*}
      S(\xi) = \exp\del{\frac{1}{2} \xi \del{a^2 + {a^\dagger}^2}}
    \end{align*}

    We define
    \begin{align*}
      B := \frac{1}{2} \xi \del{a^2 + {a^\dagger}^2}
    \end{align*}

    Therefore
    \begin{align*}
      F(\xi) = S^\dagger(\xi) X S(\xi) = e^{-B} X e^B
    \end{align*}

    The derivative is
    \begin{equation}
      \od{F}{\xi} = -\frac{B}{\xi}e^{-B}Xe^B + e^{-B}X \frac{B}{\xi}e^B
      \label{deriv}
    \end{equation}

    We define $k := \sqrt{\frac{\hbar}{2 m \omega}}$, so the $X$ operator can be written as
    \begin{align*}
      X = k\del{a + a^\dagger}
    \end{align*}

    In order to rewrite (\ref{deriv}) we solve the following commutator using
    the identities $[A,B] = -[B,A]$ and $[AB,C] = A[B,C] + [A,C]B$.
    \begin{align*}
      [X,B] &= \frac{1}{2} \xi k \left[\del{a+a^\dagger}, \del{a^2 - {a^\dagger}^2}\right] \\
            &= \frac{1}{2} \xi k \del{\underbrace{[a,a^2]}_{=0} - [a,{a^\dagger}^2] + [a^\dagger, a^2] - \underbrace{[a^\dagger, {a^\dagger}^2]}_{=0}} \\
            &= \frac{1}{2} \xi k \del{[{a^\dagger}^2, a] - [a^2, a^\dagger]} \\
            &= \frac{1}{2} \xi k \del{a^\dagger \underbrace{[a^\dagger,a] }_{=-1}+ \underbrace{[a^\dagger, a]}_{=-1}a^\dagger - a\underbrace{[a,a^\dagger]}_{=1} - \underbrace{[a,a^\dagger]}_{=1}a} \\
            &= \frac{1}{2} \xi k \del{-2a^\dagger -2a} \\
            &= -\xi k \del{a^\dagger + a} \\
            &= -\xi X
    \end{align*}

    Therefore $XB = BX - \xi X$. Inserting this into (\ref{deriv}) leads
    \begin{equation}
      \od{F}{\xi} = -\frac{B}{\xi}e^{-B}Xe^B + e^{-B} \del{BX - \xi X} \frac{1}{\xi}e^B
    \end{equation}

    Also note that $[B,e^{-B}] = 0$ (can be seen if you wrote $e^{-B}$ as series). Therefore
    \begin{align*}
      \od{F}{\xi} &= -\frac{B}{\xi}e^{-B}Xe^B + \frac{1}{\xi} e^{-B} \del{BX - \xi X} e^B \\
                  &= -\frac{B}{\xi}e^{-B}Xe^B + \frac{B}{\xi} e^{-B}Xe^B - e^{-B}Xe^{B} \\
                  &= -F(\xi)
    \end{align*}

    A solution for this differential equation is $F(\xi) = A e^{-\xi}$ where $A$
    is constant. So we can write $F(\xi) = Xe^{-\xi}$.

    For the $P$ operator, the solution is equivalent. We define $k := -i
    \sqrt{\frac{m \omega \hbar}{2}}$. Because $P = k(a^\dagger - a)$, we get

    \begin{align*}
      [P,B] &= \frac{1}{2} \xi k \left[\del{a-a^\dagger}, \del{a^2 - {a^\dagger}^2}\right] \\
            &= \frac{1}{2} \xi k \del{\underbrace{[a,a^2]}_{=0} - [a,{a^\dagger}^2] + [a^\dagger, a^2] - \underbrace{[a^\dagger, {a^\dagger}^2]}_{=0}} \\
            &= \frac{1}{2} \xi k \del{[{a^\dagger}^2, a] + [a^2, a^\dagger]} \\
            &= \frac{1}{2} \xi k \del{a^\dagger \underbrace{[a^\dagger,a] }_{=-1}+ \underbrace{[a^\dagger, a]}_{=-1}a^\dagger + a\underbrace{[a,a^\dagger]}_{=1} + \underbrace{[a,a^\dagger]}_{=1}a} \\
            &= \frac{1}{2} \xi k \del{-2a^\dagger+2a} \\
            &= \xi k \del{a - a^\dagger} \\
            &= \xi P
    \end{align*}

    The rest stays the same, so the solution is to the differential equation is $F(\xi) = P e^\xi$.

  \item
    We get the following an expectation values.

    \begin{align*}
      \braket{X}
      &= \braket{\xi|X|\xi}
      = \braket{0|S^\dagger(\xi)XS(\xi)|0}
      = \braket{0|X e^{-\xi}|0}
      = 0
      \\
      \braket{P}
      &= \braket{\xi|P|\xi}
      = \braket{0|S^\dagger(\xi)PS(\xi)|0}
      = \braket{0|P e^{\xi}|0}
      = 0
    \end{align*}

    And the following variances
    \begin{align*}
      \operatorname{Var}(X)
      &= \braket{X^2} - \braket{X}^2
      = \braket{X^2}
      = \braket{\xi|X^2|\xi}
      = \braket{0|(\del{^\dagger(\xi)XS(\xi)}^2|0}
      = \braket{0|\del{Xe^{-\xi}}^2|0} \\
      &= \braket{0|X^2|0} e^{-2\xi}
      = \frac{\hbar}{2 m \omega} \braket{0|(a^\dagger + a)^2|0} e^{-2\xi}
      = \frac{\hbar}{2 m \omega} \braket{0|{a^\dagger}^2 + a^\dagger a + aa^\dagger + a^2|0} e^{-2\xi} \\
      &= \frac{\hbar}{2 m \omega} \del{\braket{0|{a^\dagger}^2|0} + \braket{0|N|0} + \braket{0|N+1|0} + \braket{0|a^2|0}} e^{-2\xi}
      = \frac{\hbar}{2 m \omega} e^{-2 \xi} \\
      \operatorname{Var}(P)
      &= \braket{P^2} - \braket{P}^2
      = \braket{P^2}
      = \braket{\xi|P^2|\xi}
      = \braket{0|(\del{^\dagger(\xi)PS(\xi)}^2|0}
      = \braket{0|\del{Pe^{\xi}}^2|0} \\
      &= \braket{0|P^2|0} e^{2\xi}
      = -\frac{m \omega \hbar}{2} \braket{0|(a^\dagger - a)^2|0} e^{2\xi}
      = -\frac{m \omega \hbar}{2} \braket{0|{a^\dagger}^2 - a^\dagger a - aa^\dagger + a^2|0} e^{2\xi} \\
      &= -\frac{m \omega \hbar}{2} \del{\braket{0|{a^\dagger}^2|0} - \braket{0|N|0} - \braket{0|N+1|0} + \braket{0|a^2|0}} e^{2\xi}
      = \frac{m \omega \hbar}{2} e^{2 \xi} \\
    \end{align*}

    Which leads to the following uncertainties
    \begin{align*}
      \Delta x = \sqrt{\operatorname{Var}(X)} = \sqrt{\frac{\hbar}{2 m \omega}} e^{-\xi}
      \qquad\qquad
      \Delta p = \sqrt{\operatorname{Var}(P)} = \sqrt{\frac{m \omega \hbar}{2}} e^{\xi}
    \end{align*}

    So obviously these states are termed \textit{squeezed states}, because the
    are squeezed in space, which leads to a spread in momentum, so the
    uncertainty relation is not violated.

    \begin{align*}
      \Delta x \Delta p = \frac{\hbar}{2}
    \end{align*}
\end{enumerate}
\section*{Exercise 2: Rotational symmetric problems and Bohr model}
\begin{enumerate}[a)]
  \item
  \item
  \item
    Bohr postulated a quantization of the angualr momentum: $L = n \hbar$. The
    angular momentum is defined as $L = r \times p = r \times mv$. We assume,
    that electrons move on circular orbits, where radius vector and velocity
    vector are orthogonal. So $r || v$ leads to $L = mvr$. So we get from $n
    \hbar = mvr$.

    \begin{equation*}
      v = \frac{n \hbar}{mr}
    \end{equation*}

    We assume a Coulomb potential which can be written using $e_0^2 =
    \frac{e^2}{4 \pi \epsilon_0}$ as
    \begin{equation*}
      V(r) = - \frac{e_0^2}{r}
    \end{equation*}

    Which leads to a Coulomb force
    \begin{equation*}
      F_{el} = \frac{e_0^2}{r^2}
    \end{equation*}

    There is also a centripetal force of
    \begin{equation*}
      F_{z} = \frac{mv^2}{r}
    \end{equation*}

    For the electron to be in a stable orbit, they have to be equal, so
    \begin{equation*}
      \frac{e_0^2}{r^2} = \frac{m}{r} \del{\frac{n \hbar}{m r}}^2
    \end{equation*}

    Leads to a orbit radius of
    \begin{equation*}
      r = \frac{n^2\hbar^2}{me_0^2}
    \end{equation*}

    Using this, we can calculate the possible energy levels
    \begin{align*}
      E_n &= V + T \\
          &= -\frac{e_0^2}{r^2} + \frac{1}{2} m v^2 \\
          &= -e_0^2 \frac{me_0^2}{n^2 \hbar^2} + \frac{1}{2} \frac{n^2 \hbar^2}{m^2 r^2} \\
          &= -\frac{me_0^4}{n^2 \hbar^2} + \frac{1}{2} \frac{n^2 \hbar^2}{m^2} \frac{m^2e_0^4}{n^2 \hbar^2} \\
          &= -\frac{me_0^4}{n^2 \hbar^2} + \frac{1}{2} \frac{me_0^4}{n^2 \hbar^2} \\
          &= -\frac{1}{2} \frac{me_0^4}{n^2 \hbar^2}
    \end{align*}

    Which is exactly the result that can be obtained quantum mechanically (see
    script 5.97).

  \item
    The idea is that electrons create standing waves around the orbit. The
    orbit circumference is $2 \pi r$ and it must be multiples of the wavelength
    $\lambda$.
    \begin{equation*}
      n \lambda = 2 \pi r
    \end{equation*}

    Substituting the De-Broglie wavelength $\lambda = \frac{h}{p}$ gives
    \begin{align*}
                           &n \frac{h}{mv} = 2 \pi r \\
      \Leftrightarrow\quad &n \hbar = mvr = L
    \end{align*}

    Which is Bohrs ansatz. So the quantization of the angular momentum can be
    explained using De-Broglies wavelength.

\end{enumerate}
\section*{Exercise 3: Rotation operator}
\begin{enumerate}[a)]
  \item
     An operator that reverses a rotation by angle $\theta$ has the same effect
     as an operator that rotates by angle $-\theta$. Therefore
     $R(\mathbf{n},\theta)^{-1} = R(\mathbf{n},-\theta)$.

     A rotation by angle $\beta$, then by angle $\alpha$ is the same as a
     rotation by angle $\alpha + \beta$. Therefore $R(\mathbf{n}, \alpha) R(\mathbf{n}, \beta) =
     R(\mathbf{n}, \alpha + \beta)$.
\end{enumerate}
\end{document}
