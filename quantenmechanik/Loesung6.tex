\documentclass[a4paper,german,12pt,smallheadings]{scrartcl}
\usepackage[T1]{fontenc}
\usepackage[utf8]{inputenc}
\usepackage{babel}
\usepackage{tikz}
\usepackage{geometry}
\usepackage{amsmath}
\usepackage{amssymb}
\usepackage{float}
\usepackage{enumerate}
\usepackage{braket} % Teh quantum stuff
\usepackage{commath} % http://tex.stackexchange.com/questions/14821/whats-the-proper-way-to-typeset-a-differential-operator
\usepackage{cancel}
%\usepackage{wrapfig}
\usepackage[thinspace,thinqspace,squaren,textstyle]{SIunits}

% New command for color underlining
\usepackage{xcolor}

\newsavebox\MBox
\newcommand\colul[2][red]{{\sbox\MBox{$#2$}%
  \rlap{\usebox\MBox}\color{#1}\rule[-1.2\dp\MBox]{\wd\MBox}{0.5pt}}}

\restylefloat{table}
\geometry{a4paper, top=15mm, left=20mm, right=40mm, bottom=20mm, headsep=10mm, footskip=12mm}
\linespread{1.5}
\setlength\parindent{0pt}
\DeclareMathOperator{\Tr}{Tr}
\begin{document}
\begin{center}
\bfseries % Fettdruck einschalten
\sffamily % Serifenlose Schrift
\vspace{-40pt}
Quantum Mechanics, winter term 2013/2014, exercise sheet 6

Markus Fenske, Tutor: Adam Nagy
\vspace{-10pt}
\end{center}

\section*{Exercise 1: spin $1/2$ particle in magnetic field}
\begin{enumerate}[a)]
  \item
    Using
    \begin{equation*}
    \sigma_x = \begin{pmatrix}0 & 1 \\ 1 & 0\end{pmatrix} \qquad \sigma_z = \begin{pmatrix}1 & 0 \\ 0 & -1\end{pmatrix}
    \end{equation*}

    We get
    \begin{equation*}
    H = B \begin{pmatrix} \cos \theta & \sin \theta \\ \sin \theta & -\cos \theta \end{pmatrix}
    \end{equation*}

    The eigenvalues are
    \begin{align*}
    &B \begin{vmatrix} \cos(\theta) - \lambda & \sin \theta \\ \sin \theta & -\cos(\theta) - \lambda \end{vmatrix} \overset{!}{=} 0 \\
      \Leftrightarrow\quad&B \del{-\del{\cos\del{\theta} - \lambda}^2-\sin^2\del{\theta}}  = 0\\
      \Leftrightarrow\quad&B \del{-\cos^2\del\theta \cancel{-\lambda\cos\del\theta}\cancel{+\lambda\cos\del\theta}+\lambda^2-\sin^2\del\theta} = 0\\
      \Leftrightarrow\quad&B \del{-\cancelto{1}{\del{\cos^2\del\theta+\sin^2\del\theta}}+\lambda^2} = 0\\
      \Leftrightarrow\quad&B \del{-1+\lambda^2} = 0\\
      \Rightarrow\quad &\lambda_+ = 1,\quad \lambda_- = -1
    \end{align*}

    Calculatring the eigenvectors (assuming $\sin\theta \neq 0$)
    \begin{align*}
      &\begin{pmatrix}
        \cos\del\theta - \lambda_\pm & \sin \theta \\
        \sin\theta & -\cos\del\theta - \lambda_\pm \\
      \end{pmatrix}
      &\xrightarrow[\text{leading elements}]{\text{Crossdivide by}}&
      \begin{pmatrix}
        1 & \frac{\sin \theta}{\cos\del\theta-\lambda_\pm} \\
        1 & \frac{-\cos\del\theta - \lambda_\pm}{\sin\theta} \\
      \end{pmatrix}
      &\xrightarrow[\text{I from II}]{\text{Subtract}} \\
      &\begin{pmatrix}
        1 & \frac{\sin \theta}{\cos\del\theta-\lambda_\pm} \\
        0 & \frac{-\cos\del\theta - \lambda_\pm}{\sin\theta} - \frac{\sin \theta}{\cos\del\theta-\lambda_\pm} \\
      \end{pmatrix}
      &=&
      \begin{pmatrix}
        1 & \frac{\sin \theta}{\cos\del\theta-\lambda_\pm} \\
        0 & \frac{-\del{\cos\del\theta - \lambda_\pm}^2 - \sin^2\del\theta}{\sin\theta\del{\cos\del\theta-\lambda_\pm}} \\
      \end{pmatrix}
      &= \\
      &\begin{pmatrix}
        1 & \frac{\sin \theta}{\cos\del\theta-\lambda_\pm} \\
        0 & \frac{-\cos^2\theta\cancel{+\cos\theta}\cancel{-\cos\theta}+1-\sin^2\theta}{\sin\theta\del{\cos\del\theta-\lambda_\pm}} \\
      \end{pmatrix}
      &=&
      \begin{pmatrix}
        1 & \frac{\sin \theta}{\cos\del\theta-\lambda_\pm} \\
        0 & \frac{-\del{\cos^2\theta+\sin^2\theta}+1}{\sin\theta\del{\cos\del\theta-\lambda_\pm}} \\
      \end{pmatrix}
      &= \\
      &\begin{pmatrix}
        1 & \frac{\sin \theta}{\cos\del\theta-\lambda_\pm} \\
        0 & 0 \\
      \end{pmatrix}
      &\xrightarrow[x_1 = 1]{\text{Choose}}&
      \quad x_2 = \frac{\sin\theta}{\cos\del\theta - \lambda_\pm}
    \end{align*}

    So the eigenvectors are (by multiplying with $\cos\del\delta \mp 1$):
    \begin{equation*}
    \ket{\psi_\pm} = \begin{pmatrix} \cos\del\theta \mp 1 \\ \sin\theta \end{pmatrix}
    \end{equation*}

    The projections are? \textbf{TODO}

\end{enumerate}
\end{document}
