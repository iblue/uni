\documentclass[a4paper,german,12pt,smallheadings]{scrartcl}
\usepackage[T1]{fontenc}
\usepackage[utf8]{inputenc}
\usepackage{babel}
\usepackage{tikz}
\usepackage{geometry}
\usepackage{amsmath}
\usepackage{amssymb}
\usepackage{float}
\usepackage{enumerate}
\usepackage{braket} % Teh quantum stuff
%\usepackage{wrapfig}
\usepackage[thinspace,thinqspace,squaren,textstyle]{SIunits}

% New command for color underlining
\usepackage{xcolor}

\newsavebox\MBox
\newcommand\colul[2][red]{{\sbox\MBox{$#2$}%
  \rlap{\usebox\MBox}\color{#1}\rule[-1.2\dp\MBox]{\wd\MBox}{0.5pt}}}

\restylefloat{table}
\geometry{a4paper, top=15mm, left=20mm, right=40mm, bottom=20mm, headsep=10mm, footskip=12mm}
\linespread{1.5}
\setlength\parindent{0pt}
\DeclareMathOperator{\Tr}{Tr}
\begin{document}
\begin{center}
\bfseries % Fettdruck einschalten
\sffamily % Serifenlose Schrift
\vspace{-40pt}
Quantum Mechanics, winter term 2013/2014, exercise sheet 5

Markus Fenske, Tutor: Adam Nagy
\vspace{-10pt}
\end{center}

\section*{Exercise 1: Measurements}
\begin{enumerate}[a)]
  \item
    \begin{equation*}
      \braket{\psi|A|\psi} =
      \frac{1}{\sqrt{18}}
      \begin{pmatrix}
        2 & 1 & 2 & 3
      \end{pmatrix}
      \begin{pmatrix}
         9 & -6 & -1 &  0 \\
        -6 &  9 &  0 & -1 \\
        -1 &  0 &  9 & -6 \\
         0 & -1 & -6 &  9
      \end{pmatrix}
      \frac{1}{\sqrt{18}}
      \begin{pmatrix}
       2 \\
       1 \\
       2 \\
       3
      \end{pmatrix}
      = \frac{52}{\sqrt{18}^2} = \frac{26}{9}
    \end{equation*}
  \item
    In order to diagonalize the matrix, we first need to calculate the
    eigenvalues by solving the characteristic polynomial. 
    \begin{align*}
      &\begin{vmatrix}
        9-\lambda & -6        & -1        &  0         \\
        -6        & 9-\lambda & 0         & -1         \\
        -1        & 0         & 9-\lambda & -6         \\
        0         & -1        & -6        &  9-\lambda
      \end{vmatrix} 
      \quad\text{(Expanding the determinant along the first row)}
      \\
      = &(9-\lambda)
      \begin{vmatrix}
        9-\lambda & 0         & -1       \\
        0         & 9-\lambda & -6       \\
        -1        & -6        & 9-\lambda
      \end{vmatrix}
      +6
      \begin{vmatrix}
       -6 & 0         & -1         \\
       -1 & 9-\lambda & -6         \\
       0  & -6        & 9 - \lambda
      \end{vmatrix}
      -1
      \begin{vmatrix}
        -6 & 9 - \lambda & -1        \\
        -1 & 0           & -6        \\
        0  & -1          & 9-\lambda
      \end{vmatrix} \\
      &\text{(Using rule of Saurus)} \\
      = &(9-\lambda)((9-\lambda)^3+0+0-(9-\lambda)-36(9-\lambda)-0) \\
        &+6(-6(9-\lambda)^2+0-6-0+6^3-0) \\
        &-1(0+9-1-0+36+(9-\lambda)^2) \\
      = &(9-\lambda)^4-37(9-\lambda)^2
        -6^2(9-\lambda)^2-6^2+6^4
        -35-(9-\lambda)^2 \\
      = &(9-\lambda)^4-74(9-\lambda)^2+1225 
        \qquad\text{(Expanding the polynomial)} \\
      = &9^4-4\cdot9^3 \lambda + 6\cdot9^2\lambda^2-4\cdot9\lambda^3+\lambda^4-74(81-18\lambda+\lambda)^2+1225 \\
      = &6561-2916\lambda+486\lambda^2-36\lambda^3+\lambda^4-5994+1332\lambda-74\lambda^2+1225 \\
      = &\lambda^4-36\lambda^3+412\lambda^2-1584\lambda+1792
    \end{align*}

    The roots of this polynomial are the eigenvalues of the matrix. Because
    this is the characteristic polynomial of a $4 \times 4$ hermitian matrix,
    it has 4 real roots.

    We are calculating the roots by clever guessing. This polynomial is normalized
    (no coefficient in front of the largest exponent), therefore the
    coefficient in front of the second largest exponent ($x^3$) is the sum of
    all roots, while the $x^0$ coefficient is the sum of the roots.
    This trick works, because
    \begin{equation*}
      (x-a)(x-b)(x-c)(x-d) = x^4 -(a+b+c+d)x^3 + \cdots + abcd
    \end{equation*}


    Therefore the sum of the roots is $a+b+c+d = 36$, while the product is
    $abcd = 1792$. Because $abcd$ factorizes to $2^8 \cdot 7$, we begin
    guessing with small numbers: $a=2$, $b=4$, which is correct:
    \begin{align*}
      2^4 -36\cdot2^3+ 412\cdot2^2 - 1584\cdot2 + 1792 = 0 \\
      4^4 -36\cdot4^3+ 412\cdot4^2 - 1584\cdot4 + 1792 = 0 \\
    \end{align*}

    The remaining roots must fulfill $cd = 2^5 \cdot 7$, $c+d = 30$. We guess
    $c=2^4$ and $d=2\cdot7$ and we are done.
    \begin{align*}
      16^4 -36\cdot16^3+ 412\cdot16^2 - 1584\cdot16 + 1792 = 0 \\
      14^4 -36\cdot14^3+ 412\cdot14^2 - 1584\cdot14 + 1792 = 0 \\
    \end{align*}

    % FIXME
    % use \xrightarrow[below]{above} to make to look nicer, rewrite II-= ... as
    % II = II - ...

    So the eigenvalues of the matrix are $\{2,4,14,16\}$.

    The eigenvectors can be calculated by solving $A - \lambda I = 0$. For $\lambda_1 = 2$
    \begin{alignat*}{2}
      &\begin{pmatrix}
        7 & -6 & -1 & 0 \\
        -6 & 7 & 0 & -1 \\
        -1 & 0 & 7 & -6 \\
        0 & -1 & -6 & 7 \\
      \end{pmatrix}
      &\xrightarrow{\text{reorder columns}}
      &\begin{pmatrix}
        -1 & 0 & 7 & -6 \\
        0 & -1 & -6 & 7 \\
        -6 & 7 & 0 & -1 \\
        7 & -6 & -1 & 0 \\
      \end{pmatrix}
      \xrightarrow[IV = IV - 7 \cdot I]{III = III + 6 \cdot I}\\
      &\begin{pmatrix}
        -1 & 0 & 7 & -6 \\
        0 & -1 & -6 & 7 \\
        0  & 7 & -42 & 35 \\
        0 & -6 & 48 & -42 \\
      \end{pmatrix}
      &\xrightarrow[IV=IV-6\cdot II]{III=III+7 \cdot II}
      &\begin{pmatrix}
        -1 & 0 & 7 & -6 \\
        0 & -1 & -6 & 7 \\
        0 & 0 & 84 & -84 \\
        0  & 0 & -84 & 84 \\
      \end{pmatrix}
      \xrightarrow[\text{normalize}]{IV = IV + III} \\
      &\begin{pmatrix}
        -1 & 0 & 7 & -6 \\
        0 & -1 & -6 & 7 \\
        0 & 0 & 1 & -1 \\
        0  & 0 & 0 & 0 \\
      \end{pmatrix}
      &\xrightarrow{\hspace{1.3cm}}
      &\begin{matrix}
        \text{choose } x_4 = 1 \\
        -x_3 - 1x_4 = x_3 + 1 = 0 \Rightarrow x_3 = 1 \\
        -x_2 - 6x_3 + 7x_4 = -x_2 -6 +7 = 0 \Rightarrow x_2 = 1 \\
        -x_1 + 7x_3 - 6x_4 = x_1 +7 -6 = 0 \Rightarrow x_1 = 1
      \end{matrix}
    \end{alignat*}

    This results in the normalized Eigenvector
    \begin{equation*}
      \vec{v}_1 = \frac{1}{2} \begin{pmatrix}1 \\ 1 \\ 1 \\ 1\end{pmatrix}
    \end{equation*}

    For $\lambda_2 = 4$:

    \begin{align*}
      &\begin{pmatrix}
        5 & -6 & -1 & 0 \\
        -6 & 5 & 0 & -1 \\
        -1 & 0 & 5 & -6 \\
        0 & -1 & -6 & 5 \\
      \end{pmatrix}
      \xrightarrow{II -= 6 \cdot III, I += 5 \cdot III}
      \begin{pmatrix}
        0 & -6 & 24 & -30 \\
        0 & 5 & -30 & 35 \\
        -1 & 0 & 5 & -6 \\
        0 & -1 & -6 & 5\\
      \end{pmatrix}
      \xrightarrow{\text{exchange III, I}} \\
      &\begin{pmatrix}
        -1 & 0 & 5 & -6 \\
        0 & 5 & -30 & 35 \\
        0 & -6 & 24 & -30 \\
        0 & -1 & -6 & 5\\
      \end{pmatrix}
      \xrightarrow{II += 5 \cdot IV, III-=6 \cdot IV}
      \begin{pmatrix}
        -1 & 0 & 5 & -6 \\
        0 & 0 & -60 & 60 \\
        0 & 0 & 60 & -60 \\
        0 & -1 & -6 & 5\\
      \end{pmatrix}
    \end{align*}

    From this, we get the normalized eigenvector
    \begin{equation*}
    \vec{v}_2 = \frac{1}{2} \begin{pmatrix} -1 \\ -1 \\ 1 \\ 1 \end{pmatrix}
    \end{equation*}

    For $\lambda_3 = 14$
    \begin{align*}
      &\begin{pmatrix}
        -5 & -6 & -1 & 0 \\
        -6 & -5 & 0 & -1 \\
        -1 & 0 & -5 & -6 \\
        0 & -1 & -6 & -5 \\
      \end{pmatrix}
      \xrightarrow{\cdot (-1),\text{ reorder}}
      \begin{pmatrix}
        1 & 0 & 5 & 6 \\
        0 & 1 & 6 & 5 \\
        5 & 6 & 1 & 0 \\
        6 & 5 & 0 & 1
      \end{pmatrix}
      \xrightarrow{III-= 5 \cdot I, IV -= 6 \cdot I} \\
      &\begin{pmatrix}
        1 & 0 & 5 & 6 \\
        0 & 1 & 6 & 5 \\
        0 & 6 & -24 & -30 \\
        0 & 5 & -30 & -35
      \end{pmatrix}
      \xrightarrow{III -= 6 \cdot II, IV -= 5 \cdot II}
      \begin{pmatrix}
        1 & 0 & 5 & 6 \\
        0 & 1 & 6 & 5 \\
        0 & 0 & -60 & -60 \\
        0 & 0 & -60 & -60
      \end{pmatrix}
      \xrightarrow{IV -= II,\text{ normalize}} \\
      &\begin{pmatrix}
        1 & 0 & 5 & 6 \\
        0 & 1 & 6 & 5 \\
        0 & 0 & 1 & 1 \\
        0 & 0 & 0 & 0
      \end{pmatrix}
      \xrightarrow{\hspace{3cm}}
      \begin{matrix}
        \text{choose } x_4 = 1 \\
        x_3 + 1x_4 = x_3 + 1 = 0 \Rightarrow x_3 = -1 \\
        x_2 + 6x_3 + 5x_2 = x_2 -6 +5 = 0 \Rightarrow x_2 = 1 \\
        x_1 + 5x_3 + 6x_4 = x_1 -5 +6 = 0 \Rightarrow x_1 = -1
      \end{matrix}
    \end{align*}

    Results in the normalized eigenvector
    \begin{equation*}
      \vec{v}_3 = \frac{1}{2} \begin{pmatrix} -1 \\ 1 \\ -1 \\ 1 \end{pmatrix}
    \end{equation*}

    For $\lambda_4 = 16$:

    \begin{align*}
      &\begin{pmatrix}
        -7 & -6 & -1 & 0 \\
        -6 & -7 & 0 & -1 \\
        -1 & 0 & -7 & -6 \\
        0 & -1 & -6 & -7 \\
      \end{pmatrix}
      \xrightarrow{\cdot (-1),\text{ exchange I, III}}
      \begin{pmatrix}
        1 & 0 & 7 & 6 \\
        6 & 7 & 0 & 1 \\
        7 & 6 & 1 & 0 \\
        0 & 1 & 6 & 7 \\
      \end{pmatrix}
      \xrightarrow{II-=6\cdot I, III -= 7 \cdot I} \\
      &\begin{pmatrix}
        1 & 0 & 7 & 6 \\
        0 & 7 & -42 & -35 \\
        0 & 6 & -48 & -42 \\
        0 & 1 & 6 & 7 \\
      \end{pmatrix}
      \xrightarrow{II-=7 \cdot IV, III -= 6 \cdot III}
      \begin{pmatrix}
        1 & 0 & 7 & 6 \\
        0 & 0 & -84 & -84 \\
        0 & 0 & -84 & -84 \\
        0 & 1 & 6 & 7
      \end{pmatrix}
    \end{align*}

    Therefore we get
    \begin{equation*}
      \vec{v}_4 = \frac{1}{2} \begin{pmatrix} 1 \\ -1 \\ -1 \\ 1\end{pmatrix}
    \end{equation*}

    Using the eigenvectors, we can determine the transformation matrix as
    \begin{equation*}
      S = \frac{1}{2}
      \begin{pmatrix}
        1 & -1 & -1 & 1 \\
        1 & -1 & 1 & -1 \\
        1 & 1 & -1 & -1 \\
        1 & 1 & 1 & 1
      \end{pmatrix}
    \end{equation*}

    Using the Gauß-Jordan algorithm, we calculate the inverse $S^{-1}$.

    \begin{align*}
      &\left(
        \begin{array}{cccc|cccc}
          1 & -1 & -1 &  1   &   1 & 0 & 0 & 0 \\
          1 & -1 &  1 & -1   &   0 & 1 & 0 & 0 \\
          1 &  1 & -1 & -1   &   0 & 0 & 1 & 0 \\
          1 &  1 &  1 &  1   &   0 & 0 & 0 & 1
        \end{array}
      \right)
      \xrightarrow[\text{and reorder}]{II, III, IV - I}
      \left(
        \begin{array}{cccc|cccc}
          1 & -1 & -1 &  1   &    1 & 0 & 0 & 0 \\
          0 &  2 &  0 & -2   &   -1 & 1 & 0 & 0 \\
          0 &  2 &  2 &  0   &   -1 & 0 & 1 & 0 \\
          0 &  0 &  2 & -2   &   -1 & 0 & 0 & 1
        \end{array}
      \right)
      \xrightarrow{III = III - II}
      \\
      &\left(
        \begin{array}{cccc|cccc}
          1 & -1 & -1 &  1   &    1 & 0 & 0  & 0 \\
          0 &  2 &  0 & -2   &   -1 & 1 & 0  & 0 \\
          0 &  0 &  2 &  2   &    0 & 0 & -1 & 1 \\
          0 &  0 &  2 & -2   &   -1 & 0 & 0  & 1
        \end{array}
      \right)
      \xrightarrow{IV - III}
      \left(
        \begin{array}{cccc|cccc}
          1 & -1 & -1 &  1   &    1 & 0 & 0  & 0 \\
          0 &  2 &  0 & -2   &   -1 & 1 & 0  & 0 \\
          0 &  0 &  2 &  2   &    0 & 0 & -1 & 1 \\
          0 &  0 &  0 & -4   &   -1 & 1 & 1  & -1
        \end{array}
      \right)
      \xrightarrow[IV \cdot (-1)]{I \cdot 4, II, III \cdot 2} \\
      &\left(
        \begin{array}{cccc|cccc}
          4 & -4 & -4 &  4   &    4 &  0 &  0  & 0 \\
          0 &  4 &  0 & -4   &   -2 &  2 &  0  & 0 \\
          0 &  0 &  4 &  4   &    0 &  0 & -2  & 2 \\
          0 &  0 &  0 &  4   &    1 & -1 & -1  & 1
        \end{array}
      \right)
      \xrightarrow{III - IV}
      \left(
        \begin{array}{cccc|cccc}
          4 & -4 & -4 &  4   &    4 &  0 &  0  & 0 \\
          0 &  4 &  0 & -4   &   -2 &  2 &  0  & 0 \\
          0 &  0 &  4 &  0   &   -1 &  1 & -1  & 1 \\
          0 &  0 &  0 &  4   &    1 & -1 & -1  & 1
        \end{array}
      \right)
      \xrightarrow{II + IV} \\
      &\left(
        \begin{array}{cccc|cccc}
          4 & -4 & -4 &  4   &    4 & 0  &  0  & 0 \\
          0 &  4 &  0 &  0   &   -1 & -1 &  1  & 1 \\
          0 &  0 &  4 &  0   &   -1 &  1 & -1  & 1 \\
          0 &  0 &  0 &  4   &    1 & -1 & -1  & 1
        \end{array}
      \right)
      \xrightarrow{I + II}
      \left(
        \begin{array}{cccc|cccc}
          4 &  0 & -4 &  4   &    3 & -1 &  1 & 1 \\
          0 &  4 &  0 &  0   &   -1 & -1 &  1 & 1 \\
          0 &  0 &  4 &  0   &   -1 &  1 & -1 & 1 \\
          0 &  0 &  0 &  4   &    1 & -1 & -1 & 1
        \end{array}
      \right)
      \xrightarrow{I + III} \\
      &\left(
        \begin{array}{cccc|cccc}
          4 &  0 & 0 & 4  &    2 &  0 &  0 & 2 \\
          0 &  4 & 0 & 0  &   -1 & -1 &  1 & 1 \\
          0 &  0 & 4 & 0  &   -1 &  1 & -1 & 1 \\
          0 &  0 & 0 & 4  &    1 & -1 & -1 & 1
        \end{array}
      \right)
      \xrightarrow{I - IV}
      \left(
        \begin{array}{cccc|cccc}
          4 &  0 & 0 & 0  &    1 &  1 &  1 & 1 \\
          0 &  4 & 0 & 0  &   -1 & -1 &  1 & 1 \\
          0 &  0 & 4 & 0  &   -1 &  1 & -1 & 1 \\
          0 &  0 & 0 & 4  &    1 & -1 & -1 & 1
        \end{array}
      \right)
    \end{align*}

    Therefore the inverse matrix is
    \begin{equation*}
      S^{-1} = \frac{1}{2}\begin{pmatrix}
         1 &  1 &  1 & 1 \\
        -1 & -1 &  1 & 1 \\
        -1 &  1 & -1 & 1 \\
         1 & -1 & -1 & 1
      \end{pmatrix}
    \end{equation*}

    Using the transformation matrices and the eigenvectors, we can now diagonalize the matrix as
    \begin{equation*}
      SAS^{-1} =
      \frac{1}{2}
      \begin{pmatrix}
        1 & -1 & -1 & 1 \\
        1 & -1 & 1 & -1 \\
        1 & 1 & -1 & -1 \\
        1 & 1 & 1 & 1
      \end{pmatrix}
      \begin{pmatrix}
        2 & 0 & 0 & 0 \\
        0 & 4 & 0 & 0 \\
        0 & 0 & 14 & 0 \\
        0 & 0 & 0 & 16
      \end{pmatrix}
      \frac{1}{2}\begin{pmatrix}
         1 &  1 &  1 & 1 \\
        -1 & -1 &  1 & 1 \\
        -1 &  1 & -1 & 1 \\
         1 & -1 & -1 & 1
      \end{pmatrix}
    \end{equation*}
  \item
    The probabilities for the values $\lambda_n = \{2, 4, 14, 16\}$ are
    \begin{align*}
      &p_1
      = |\braket{\psi|v_1}|^2
      = \left| \frac{1}{\sqrt{18}}
        \begin{pmatrix} 2 & 1 & 2 & 3 \end{pmatrix}
         \frac{1}{2} \begin{pmatrix}1 \\ 1 \\ 1 \\ 1\end{pmatrix} \right|^2
      = \frac{1}{18} \frac{1}{4} (2+1+2+3)^2
      = \frac{16}{18} \\
      &p_2
      = |\braket{\psi|v_2}|^2
      = \left| \frac{1}{\sqrt{18}}
        \begin{pmatrix} 2 & 1 & 2 & 3 \end{pmatrix}
         \frac{1}{2} \begin{pmatrix}-1 \\ -1 \\ 1 \\ 1\end{pmatrix} \right|^2
      = \frac{1}{18} \frac{1}{4} (-2-1+2+3)^2
      = \frac{1}{18} \\
      &p_3
      = |\braket{\psi|v_3}|^2
      = \left| \frac{1}{\sqrt{18}}
        \begin{pmatrix} 2 & 1 & 2 & 3 \end{pmatrix}
         \frac{1}{2} \begin{pmatrix}-1 \\ 1 \\ -1 \\ 1\end{pmatrix} \right|^2
      = \frac{1}{18} \frac{1}{4} (-2+1-2+3)^2
      = 0 \\
      &p_4
      = |\braket{\psi|v_4}|^2
      = \left| \frac{1}{\sqrt{18}}
        \begin{pmatrix} 2 & 1 & 2 & 3 \end{pmatrix}
         \frac{1}{2} \begin{pmatrix}1 \\ -1 \\ -1 \\ 1\end{pmatrix} \right|^2
      = \frac{1}{18} \frac{1}{4} (2-1-2+3)^2
      = \frac{1}{18}
    \end{align*}

    The average expected value is therefore
    \begin{equation*}
      \sum \lambda_n p_n = 2 \cdot \frac{16}{18} + 4 \cdot \frac{1}{18} + 14 \cdot 0 + 16 \cdot \frac{1}{18} = \frac{26}{9}
    \end{equation*}

    Which is exactly the same result as calculated in a).

  \item
    After the measurement, the system is in a state corresponding to the
    eigenvector. So for a result $\lambda_n$, the state is $\ket{\psi_n}$.

    \begin{align*}
      \lambda_1 =  2 \qquad &\Rightarrow \qquad \ket{\psi_1} =  \phantom{-}\ket{1} + \ket{2} + \ket{3} + \ket{4} \\
      \lambda_2 =  4 \qquad &\Rightarrow \qquad \ket{\psi_2} = -\ket{1} - \ket{2} + \ket{3} + \ket{4} \\
      \lambda_3 = 14 \qquad &\Rightarrow \qquad \ket{\psi_3} = -\ket{1} + \ket{2} - \ket{3} + \ket{4} \\
      \lambda_4 = 16 \qquad &\Rightarrow \qquad \ket{\psi_4} =  \phantom{-}\ket{1} - \ket{2} - \ket{3} + \ket{4} \\
    \end{align*}

\end{enumerate}

\section*{Exercise 2: Repeated measurements}
\begin{enumerate}[a)]
  \item
    By rewriting $\ket{\psi}$ in a canonical basis, we see that
    \begin{equation*}
      \ket{\psi} 
      = \frac{1}{\sqrt{2}} \left( \ket{+} + \ket{-} \right)
      = \frac{1}{\sqrt{2}} \left( \frac{1}{\sqrt{2}} \begin{pmatrix} 1 \\ 1\end{pmatrix} + \frac{1}{\sqrt{2}} \begin{pmatrix} 1 \\ -1 \end{pmatrix} \right) = \begin{pmatrix} 1 \\ 0 \end{pmatrix} = \ket{0}
    \end{equation*}

    The Pauli matrix $\sigma_z$ can be written as a $\sigma_z = \ket{0}\bra{0}
    - \ket{1}\bra{1}$. So the average value of the measurement is
    \begin{equation*}
      \braket{0|\sigma_z|0} 
      = \bra{0}(\ket{0}\bra{0} - \ket{1}\bra{1})\ket{0}
      = \braket{0|0}\braket{0|0} - \braket{0|1}\braket{1|0}
      = 1\cdot1 - 0 \cdot 0
      = 1
    \end{equation*}

    The eigenvectors of $\sigma_z$ are the canonical base vectors. Using these,
    we calculate the probabilities as
    \begin{align*}
      p_0 = \left|\braket{0|0}\right|^2 = 1 \\
      p_1 = \left|\braket{1|0}\right|^2 = 0
    \end{align*}

    The measurement does not modify the state of the system
    \begin{equation*}
      \ket{\psi_0} = \frac{1}{\sqrt{1}} \braket{0|0} \ket{0} = \ket{0}
    \end{equation*}

  \item
    We write $A$ as a matrix
    \begin{equation*}
      A = \sigma_x + \sigma_y + \sigma_z = \begin{pmatrix} 1 & 1-i \\ 1+i & -1\end{pmatrix}
    \end{equation*}

    The eigenvalues and corresponding eigenvectors are
    \begin{align*}
    \lambda_0 = \sqrt{3} \qquad
    &\Rightarrow \qquad \ket{\phi_0'} = \begin{pmatrix} \frac{\sqrt{3}+1}{2}(1-i) \\ 1\end{pmatrix} \\
    \lambda_1 = -\sqrt{3} \qquad
    &\Rightarrow \qquad \ket{\phi_1'} = \begin{pmatrix} \frac{\sqrt{3}-1}{2}(i-1) \\ 1\end{pmatrix}
    \end{align*}

    The eigenvectors need to be normalized.
    \begin{align*}
      || \ket{\phi_0'} ||^2 = 3 + \sqrt{3} \qquad
    &\Rightarrow \qquad \ket{\phi_0} = \begin{pmatrix} \frac{\sqrt{3}+1}{2\sqrt{3+\sqrt{3}}}(1-i) \\ \frac{1}{\sqrt{3+\sqrt{3}}} \end{pmatrix} \\
      || \ket{\phi_1'} ||^2 = 3 - \sqrt{3} \qquad
    &\Rightarrow \qquad \ket{\phi_1} = \begin{pmatrix} \frac{\sqrt{3}-1}{2\sqrt{3-\sqrt{3}}}(i-1) \\ \frac{1}{\sqrt{3-\sqrt{3}}} \end{pmatrix} \\
    \end{align*}

    Now we can calculate the probabilities.
    \begin{align*}
      p_0 &= |\braket{0|\phi_0}|^2 = \left|\frac{\sqrt{3}+1}{2\sqrt{3+\sqrt{3}}}(1-i)\right|^2 = \frac{\sqrt{3}}{6} + \frac{1}{2} \approx 0.79 \\
      p_1 &= |\braket{0|\phi_1}|^2 = \left|\frac{\sqrt{3}-1}{2\sqrt{3-\sqrt{3}}}(i-1)\right|^2 = \frac{1}{2} - \frac{\sqrt{3}}{6} \approx 0.21
    \end{align*}

    The average expected value is
    \begin{equation*}
      p_0 \lambda_0 + p_1\lambda_1 = 
      \left(\frac{\sqrt{3}}{6}+\frac{1}{2}\right)\sqrt{3} + 
      \left(\frac{1}{2} - \frac{\sqrt{3}}{6}\right)\left(-\sqrt{3}\right)
      = 1
    \end{equation*}
  \item
    When the measured value is $-\sqrt{3}$, the system is now in the state of
    the corresponding eigenvector.

    \begin{equation*}
      \ket{\psi} = \ket{\phi_1}
    \end{equation*}

    The eigenvectors of $\sigma_z$ again are $\ket{0}$ and $\ket{1}$, so the
    probabilities for each of these states are

    \begin{align*}
      p_0 &= |\braket{0|\psi}|^2 = \frac{1}{2} - \frac{\sqrt{3}}{6} \approx 0.21 \qquad \text{(as calculated above)} \\
      p_1 &= |\braket{1|\psi}|^2 = 1-p_0 = \frac{\sqrt{3}}{6} + \frac{1}{2} \approx 0.79 \qquad \text{(because I am lazy)}
    \end{align*}

    The probabilities are the same as above, but the eigenvalues differ. So the
    average expected value is
    \begin{equation*}
      p_0 \lambda_0 + p_1\lambda_1 = 
      \left(\frac{\sqrt{3}}{6}+\frac{1}{2}\right)\cdot1 + 
      \left(\frac{1}{2} - \frac{\sqrt{3}}{6}\right) \cdot (-1) =
      \frac{\sqrt{3}}{3}
    \end{equation*}
  \item
    Because we measure the same observable again, the system is already in
    a state that is an eigenvector of this observable. The result of this
    measurement would be the same as the previous result.
\end{enumerate}

\section*{Exercise 3: Repeated measurements}
\begin{enumerate}[a)]
  \item
    The wave functions are orthogonal if $\braket{\psi_0|\psi_1} = 0$.

    \begin{equation*}
      \braket{\psi_0|\psi_1} 
      = \int\limits_{- \infty}^{\infty} \overline{\psi_0} \psi_1 \; \text{d}x
      = \int\limits_{- 1}^{1} \cos\left(\frac{\pi}{2}x\right) \sin(\pi x) \; \text{d}x
    \end{equation*}

    Using the trigonometric identity $\sin a \cos b = \frac{1}{2} \left( \sin(a
    -b) + \sin(a+b) \right)$ we can write this as

    \begin{equation*}
      = \int\limits_{-1}^{1} \left(\frac{1}{2} \sin\left(\frac{\pi}{2} x\right) + \sin \left( \frac{3 \pi}{2} x\right)\right) \;\text{d}x
    \end{equation*}

    Because $\sin$ is an odd function ($\sin(x) = -\sin(x)$) this integral is
    obviously zero. Therefore $\ket{0}$ and $\ket{1}$ are orthogonal.
  \item
    If the state is $\ket{0}$, then the probability to find the particle on the
    left side ($x \in [-1, 0]$) is

    \begin{align*}
      p
      &= \int_{-1}^0 \text{d}x \; |\ket{0}|^2
      = \int_{-1}^0 \text{d}x \; \cos\left(\frac{\pi}{2} x\right)^2
      = \frac{2}{\pi} \int_{-\frac{\pi}{2}}^0 \text{d}u \; \cos^2(u) \\
      &= \frac{2}{\pi} \int_{-\frac{\pi}{2}}^0 \text{d}u \; \frac{1}{2} \left( 1+ \cos(2u) \right)
      = \frac{1}{\pi} \int_{-\frac{\pi}{2}}^0 \text{d}u \; \left(1+ \cos(2u) \right) \\
      &= \frac{1}{\pi} \left[u\right]_{u=-\frac{\pi}{2}}^0 + \frac{1}{\pi} \int_{-\frac{\pi}{2}}^0 \text{d}u \; \cos(2u)
      = \frac{1}{2} + \frac{1}{2 \pi} \underbrace{\int_{-\pi}^0 \text{d}v \cos(v)}_{= 0 \text{ (geometrically)}} \\
      &= \frac{1}{2}
    \end{align*}

    If the state is $\ket{1}$, then the probability to find the particle on the
    left side is
    \begin{align*}
      p
      &= \int_{-1}^0 \text{d}x \; |\ket{1}|^2
      = \int_{-1}^0 \text{d}x \; \sin(\pi x)^2
      = \frac{1}{\pi} \int_{-\pi}^0 \text{d}u \; \sin^2(u) \\
      &= \frac{1}{\pi} \int_{-\pi}^0 \text{d}u \; \frac{1}{2} \left(1- \cos(2u) \right)
      = \frac{1}{\pi} \int_{-\frac{\pi}{2}}^0 \text{d}u \; \left(1+ \cos(2u) \right) \\
      &= \frac{1}{2\pi} \left[u\right]_{u=-\pi}^0 - \frac{1}{\pi} \underbrace{\int_{-\frac{\pi}{2}}^0 \text{d}u \; \cos(2u)}_{=0\text{ (as above)}}
      = \frac{1}{2}
    \end{align*}
  \item
    Let $\ket{\psi} = \alpha \ket{0} + \beta \ket{1}$. The probability is now

    \begin{align*}
      p
      &= \int_{-1}^0 \text{d}x \; |\ket{\psi}|^2
      = \int_{-1}^0 \text{d}x \; \left|\alpha\cos\left(\frac{\pi}{2} x\right) + \beta\sin(\pi x)\right|^2 \\
      &= |\alpha|^2\int_{-1}^0 \text{d}x \; \cos^2\left(\frac{\pi}{2} x\right)
       + |\alpha||\beta| \left|\int_{-1}^0 \text{d}x \; \cos\left(\frac{\pi}{2} x\right)\sin(\pi x)\right|
       + |\beta|^2\int_{-1}^0 \text{d}x \; \sin^2\left(\pi x\right) \\
     &= \frac{1}{2} |\alpha|^2 + \frac{4\pi}{3} |\alpha| |\beta| + \frac{1}{2} |\beta|^2
    \end{align*}

  \item
    % FIXME: Something is odd here. p(xi) may be greater than 1, which must be
    % wrong...
    In order to get valid superpositions, we must demand that $|\alpha|^2 + |\beta|^2
    = 1$, which results in
    \begin{equation*}
      |\beta|^2 = 1-|\alpha|^2
    \end{equation*}

    When substituting into $p(\alpha, \beta)$, we can eliminate one variable
    \begin{equation*}
      p(\alpha) = \frac{1}{2} |\alpha|^2 + \frac{4 \pi}{3} |\alpha| \sqrt{1-|\alpha|^2} + \frac{1}{2}(1 - |\alpha|^2)
    \end{equation*}

    To simplify things even more, we can substitute $\xi = |\alpha|^2$
    \begin{equation*}
      p(\xi) = \frac{\xi}{2} + \frac{4 \pi}{3} \sqrt{\xi - \xi^2} + \frac{1- \xi}{2}
    \end{equation*}

    To search for a possible maximum, set $\text{d}p / \text{d} \xi \overset{!}{=} 0$
    \begin{align*}
      &\frac{\text{d}p}{\text{d} \xi} = \frac{4 \pi}{3} \frac{\text{d}p}{\text{d} \xi} \sqrt{\xi - \xi^2} \overset{!}{=} 0 \\
      \Leftrightarrow\qquad
      &\frac{1 - 2 \xi}{2\sqrt{\xi - \xi^2}} = 0 \\
      \Leftrightarrow\qquad
      &1 - 2 \xi = 0 \\
      \Leftrightarrow\qquad
      &\xi = \frac{1}{2} \\
    \end{align*}

    So the maximum is at $|\alpha|^2 = \frac{1}{2}$, which results in
    \begin{equation*}
      |\alpha| = \frac{1}{\sqrt{2}}, |\beta| = \frac{1}{\sqrt{2}}
    \end{equation*}


\end{enumerate}
\end{document}
