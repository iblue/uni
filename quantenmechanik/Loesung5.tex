\documentclass[a4paper,german,12pt,smallheadings]{scrartcl}
\usepackage[T1]{fontenc}
\usepackage[utf8]{inputenc}
\usepackage{babel}
\usepackage{tikz}
\usepackage{geometry}
\usepackage{amsmath}
\usepackage{amssymb}
\usepackage{float}
\usepackage{enumerate}
\usepackage{braket} % Teh quantum stuff
%\usepackage{wrapfig}
\usepackage[thinspace,thinqspace,squaren,textstyle]{SIunits}

% New command for color underlining
\usepackage{xcolor}

\newsavebox\MBox
\newcommand\colul[2][red]{{\sbox\MBox{$#2$}%
  \rlap{\usebox\MBox}\color{#1}\rule[-1.2\dp\MBox]{\wd\MBox}{0.5pt}}}

\restylefloat{table}
\geometry{a4paper, top=15mm, left=20mm, right=40mm, bottom=20mm, headsep=10mm, footskip=12mm}
\linespread{1.5}
\setlength\parindent{0pt}
\DeclareMathOperator{\Tr}{Tr}
\begin{document}
\begin{center}
\bfseries % Fettdruck einschalten
\sffamily % Serifenlose Schrift
\vspace{-40pt}
Quantum Mechanics, winter term 2013/2014, exercise sheet 5

Markus Fenske, Tutor: Adam Nagy
\vspace{-10pt}
\end{center}

\section*{Exercise 1: Measurements}
\begin{enumerate}[a)]
  \item
    \begin{equation*}
      \braket{\psi|A|\psi} =
      \frac{1}{\sqrt{18}}
      \begin{pmatrix}
        2 & 1 & 2 & 3
      \end{pmatrix}
      \begin{pmatrix}
         9 & -6 & -1 &  0 \\
        -6 &  9 &  0 & -1 \\
        -1 &  0 &  9 & -6 \\
         0 & -1 & -6 &  9
      \end{pmatrix}
      \frac{1}{\sqrt{18}}
      \begin{pmatrix}
       2 \\
       1 \\
       2 \\
       3
      \end{pmatrix}
      = \frac{52}{\sqrt{18}^2} = \frac{26}{9}
    \end{equation*}
\end{enumerate}

\end{document}
