\documentclass[a4paper,german,12pt,smallheadings]{scrartcl}
\usepackage[T1]{fontenc}
\usepackage[utf8]{inputenc}
\usepackage{babel}
\usepackage{tikz}
\usepackage{geometry}
\usepackage{amsmath}
\usepackage{amssymb}
\usepackage{float}
\usepackage{enumerate}
\usepackage{braket} % Teh quantum stuff
%\usepackage{wrapfig}
\usepackage[thinspace,thinqspace,squaren,textstyle]{SIunits}
\restylefloat{table}
\geometry{a4paper, top=15mm, left=20mm, right=40mm, bottom=20mm, headsep=10mm, footskip=12mm}
\linespread{1.5}
\setlength\parindent{0pt}
\DeclareMathOperator{\Tr}{Tr}
\begin{document}
\begin{center}
\bfseries % Fettdruck einschalten
\sffamily % Serifenlose Schrift
\vspace{-40pt}
Quantum Mechanics, winter term 2013/2014, exercise sheet 3

Markus Fenske, Tutor: Adam Nagy
\vspace{-10pt}
\end{center}

\section*{Exercise 1: Wave packets and uncertainty principle}

\begin{enumerate}[a)]
  \item
    Because the wave functions looks like the Fourier transformation of a
    rectangular function of width $\sigma_k$ and height $C_0$ I try

    \begin{equation*}
    C(k,t) = \begin{cases}
      C_0 &\text{if } - \frac{\sigma_k}{2} \le k \le \frac{\sigma_k}{2} \\
      0   &\text{elsewhere}
      \end{cases}
    \end{equation*}

    The wave function with these amplitudes is
    \begin{align*}
      \psi(x,t) &= \int_{- \infty}^{+ \infty} C(k) \cdot e^{iku} \; dk \\
      &= C_0 \int_{- \frac{\sigma_k}{2}}^{+\frac{\sigma_k}{2}} e^{iku} \; dk \\
      &= \frac{C_0}{iu} \left[ e^{iu \frac{\sigma_k}{2}} - e^{-iu \frac{\sigma_k}{2}} \right] \\
      &= 2 C_0\frac{\sin \left(u \frac{\sigma_k}{2}\right)}{u}
    \end{align*}

    Which is exactly, what I was looking for

  \item
    A normalized wave function must meet the following condition.

    \begin{equation*}
      \int_{-\infty}^{\infty} |\psi(x,t)|^2 \; dx = 1
    \end{equation*}

    I solve for $C_0$ (using $dx = -du$, because $u = v_gt - x$):

    \begin{align*}
      &\int_{-\infty}^{\infty} -\left|2C_0 \frac{\sin\left(u \frac{\sigma_k}{2}\right)}{u}\right|^2 \; du = 1 \\
      \Leftrightarrow\quad&
      -4C_0^2 \int_{-\infty}^{\infty} \frac{\sin^2\left(u \frac{\sigma_k}{2}\right)}{u^2} \; du = 1 \\
      \Leftrightarrow\quad&
      -4C_0^2 \pi \frac{\sigma_k}{2} = 1 \\
      \Leftrightarrow\quad&
      C_0^2 = -\frac{1}{2\pi \sigma_k}  \\
      \Leftrightarrow\quad&
      C_0 = i \sqrt{\frac{1}{2 \pi \sigma_k}} \\
    \end{align*}

  \item
    The spatial uncertainty is $\Delta p = \Delta k \hbar = \sigma_k \hbar$
    (because the width of the rectangle is $\sigma_k$).

    According to the uncertainty priciple $\Delta p \Delta x \ge h$, so
    \begin{equation*}
      \Delta x \ge \frac{h}{\Delta p} = \frac{h}{\hbar \sigma_k} = \frac{2 \pi}{\sigma_k}
    \end{equation*}

  \item
    The spatial and momentum dimensions are linked by a Fourier transformation.
    When a small pulse is Fourier transformed, it leads to wide frequency
    distribution and vice versa. Therefore a small uncertainty in the spatial
    dimension leads to a big uncertainty in the momentum and vice versa.

\end{enumerate}

\section*{Exercise 2: Bra-Ket Notation}

\begin{enumerate}[a)]
  \item
    \begin{equation*}
      \ket{\alpha} = \begin{pmatrix} 7i \\ -1 \\ 3\end{pmatrix}
      \qquad\qquad
      \ket{\beta} = \begin{pmatrix} 4i \\ 2 \\ 0\end{pmatrix}
    \end{equation*}

    \begin{equation*}
      \bra{\alpha} = (-7i, -1, 3)
      \qquad
      \qquad
      \bra{\beta} = (-4i, 2, 0)
    \end{equation*}

  \item
    \begin{equation*}
      \braket{\alpha|\beta} \overset{\text{(in this case)}}{=} \braket{\beta|\alpha} = 26
    \end{equation*}

    Let $\ket{\alpha}'$ be the normalized version of $\ket{\alpha}$.

    \begin{equation*}
      \ket{\alpha}' = \frac{7i}{59}\ket{1} -\frac{1}{59}\ket{2} + \frac{3}{59}\ket{3}
    \end{equation*}

  \item
    \begin{equation*}
      \ket{\alpha}\bra{\beta} =
      \begin{pmatrix} 7i \\ -1 \\ 3 \end{pmatrix}
        (-4i, 2, 0)
        =
        \begin{pmatrix}
          28 & 14i & 0 \\
          4i & -2 & 0 \\
          -12i & 6 & 0
        \end{pmatrix}
    \end{equation*}

    This matrix is neither symmetric, nor hermitian.

\end{enumerate}

\section*{Exercise 3: Hilbert spaces and trace}
\begin{enumerate}[a)]
  \item
    \begin{align*}
      \braket{x, \alpha y + z} &=
      \sum_{k=1}^d \overline{x}_k (\alpha y + z)_k =
      \sum_{k=1}^d \overline{x}_k \alpha y_k + \overline{x}_k z_k \\
      &= \alpha \sum_{k=1}^d \overline{x}_k y_k + \sum_{k=1}^d \overline{x}_k z_k =
      \alpha \braket{x,y} + \braket{x,z}
    \end{align*}
  \item
    \begin{equation*}
      \overline{\braket{y,x}} =
      \overline{\sum_{k=1}^d \overline{y_k} x_k} =
      \sum_{k=1}^d \overline{\overline{y_k}}\; \overline{x_k} =
      \sum_{k=1}^d \overline{x_k} y_k =
      \braket{x,y}
    \end{equation*}
  \item
    Let $x_n = a_n+b_ni$ where all $a_n,b_n \in \mathbb{R}$.

    \begin{equation*}
      \braket{x,x} =
      \sum_{k=1}^d \overline{x_k} x_k =
      \sum_{k=1}^d (a_k-b_ki)(a_k+b_ki) =
      \sum_{k=1}^d a_k^2 - b_k^2i^2 =
      \sum_{k=1}^d a_k^2 + b_k^2 \ge 0
    \end{equation*}

    If $x = 0$, then $a_n = b_n = 0$, which results in $\braket{x, x} = 0$.
\end{enumerate}

\section*{Exercise 4: Matrices and trace}
Using Einstein summation convention.

\begin{enumerate}[a)]
  \item
    \begin{equation*}
      \braket{A^\dagger x, y} =
      \braket{(\overline{A_{ji}} x_j), y} =
      \overline{\overline{A_{ji}} x_j} y_i =
      A_{ji} \overline{x_j} y_i =
      \overline{x_j} A_{ji} y_i =
      \braket{x, Ay}
    \end{equation*}
  \item
    \begin{equation*}
      (AB)^\dagger = 
      (A_{ij}B_{jk})^\dagger =
      \overline{A_{kj}B_{ji}} =
      \overline{B_{ji}}\;\overline{A_{kj}} =
      B^\dagger_{ij} A^\dagger_{jk} =
      B^\dagger A^\dagger
    \end{equation*}
  \item
    It's obvious that
    \begin{equation*}
      \Tr(AB) = A_{ij}B_{ji} = \Tr(BA)
    \end{equation*}

    Which leads to
    \begin{equation*}
      \Tr(S(AS^{-1})) = \Tr(A (S^{-1}S)) = \Tr(A)
    \end{equation*}

    Because $SAS^{-1}$ is the base transformation of $A$, the trace is
    invariant under base transformations.
\end{enumerate}
\end{document}
