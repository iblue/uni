\documentclass[a4paper,german,12pt,smallheadings]{scrartcl}
\usepackage[T1]{fontenc}
\usepackage[utf8]{inputenc}
\usepackage{babel}
\usepackage{geometry}
\usepackage{amsmath}
\usepackage{amssymb}
\usepackage{float}
\usepackage{enumerate}
\usepackage{braket} % Teh quantum stuff
\usepackage{commath} % http://tex.stackexchange.com/questions/14821/whats-the-proper-way-to-typeset-a-differential-operator
\usepackage{cancel}
%\usepackage{wrapfig}
\usepackage[thinspace,thinqspace,squaren,textstyle]{SIunits}

% New command for color underlining
\usepackage{xcolor}

\newsavebox\MBox
\newcommand\colul[2][red]{{\sbox\MBox{$#2$}%
  \rlap{\usebox\MBox}\color{#1}\rule[-1.2\dp\MBox]{\wd\MBox}{0.5pt}}}

\restylefloat{table}
\geometry{a4paper, top=15mm, left=20mm, right=40mm, bottom=20mm, headsep=10mm, footskip=12mm}
\linespread{1.5}
\setlength\parindent{0pt}
\DeclareMathOperator{\Tr}{Tr}
\DeclareMathOperator{\Var}{Var}
\begin{document}
\allowdisplaybreaks % Seitenumbrüche in Formeln erlauben
\begin{center}
\bfseries % Fettdruck einschalten
\sffamily % Serifenlose Schrift
\vspace{-40pt}
Quantum Mechanics, winter term 2013/2014, exercise sheet 10

Markus Fenske, Tutor: Adam Nagy
\vspace{-10pt}
\end{center}

\section*{Exercise 1: Time evolution of coherent state}
\begin{enumerate}[a)]
  \item
    It is to be shown that if $\ket{\psi(0)} = \ket{\alpha}$ (where
    $\ket{\alpha}$ is a coherent state) it follows that
    \begin{equation*}
      \ket{\psi(t)} = e^{-\frac{i \omega t}{2}} \ket{\alpha e^{-i \omega t}}
    \end{equation*}

    From the excercise sheet, we know that
    \begin{equation}
      \ket{\alpha} = \exp\del{-\frac{1}{2} \envert{\alpha}^2} \sum_{n=0}^\infty \frac{\alpha^n}{\sqrt{n!}} \ket{n}
      \label{from_ex}
    \end{equation}

    Inserting
    \begin{equation*}
      \ket{n} = \frac{1}{\sqrt{n!}} \del{a^\dagger}^n \ket{0}
    \end{equation*}

    we get
    \begin{align*}
      \ket{\alpha} &= \exp\del{-\frac{1}{2} \envert{\alpha}^2} \sum_{n=0}^\infty \frac{\del{\alpha a^\dagger}^n}{n!} \ket{0} \\
                   &= \exp\del{-\frac{1}{2} \envert{\alpha}^2} \exp\del{\alpha a^\dagger} \ket{0} \\
                   % &= \exp\del{-\frac{1}{2} \envert{\alpha}^2 + \alpha a^\dagger} \ket{0}
    \end{align*}

    And therefore
    \begin{equation}
      \ket{\alpha} = \exp\del{-\frac{1}{2} \envert{\alpha}^2 + \alpha a^\dagger} \ket{0}
      \label{coherent_state}
    \end{equation}

    Now for the time evolution of $\ket{\alpha}$:

    \begin{equation*}
      \ket{\psi(t)} = U(t,0) \ket{\alpha} = \exp\del{-\frac{i H t}{\hbar}} \ket{\alpha}
    \end{equation*}

    Using equation (\ref{from_ex}):

    \begin{align*}
      \ket{\psi(t)} &= \exp\del{-\frac{i H t}{\hbar}} \exp\del{-\frac{1}{2} \envert{\alpha}^2} \sum_{n=0}^\infty \frac{\alpha^n}{\sqrt{n!}} \ket{n} \\
      &= \exp\del{-\frac{1}{2}  \envert{\alpha}^2} \exp\del{-\frac{i H t}{\hbar}} \sum_{n=0}^\infty \frac{\alpha^n}{\sqrt{n!}} \ket{n} \\
      &= \exp\del{-\frac{1}{2}  \envert{\alpha}^2}  \sum_{n=0}^\infty \frac{\alpha^n}{\sqrt{n!}} \exp\del{-\frac{i H t}{\hbar}} \ket{n} \\
      &= \exp\del{-\frac{1}{2}  \envert{\alpha}^2}  \sum_{n=0}^\infty \frac{\alpha^n}{\sqrt{n!}} \sum_{m=0}^\infty \del{\frac{-iHt}{\hbar}}^m \frac{1}{m!} \ket{n} \\
      &= \exp\del{-\frac{1}{2}  \envert{\alpha}^2}  \sum_{n=0}^\infty \frac{\alpha^n}{\sqrt{n!}} \sum_{m=0}^\infty \del{\frac{-it}{\hbar}}^m \frac{1}{m!} H^m \ket{n} \\
      \intertext{
        We know that $\ket{n}$ are eigenvectors of the Hamiltonian satisfying $H
        \ket{n} = \hbar \omega \del{n + \frac{1}{2}} \ket{n}$. This leads to
      }
      &= \exp\del{-\frac{1}{2}  \envert{\alpha}^2}  \sum_{n=0}^\infty \frac{\alpha^n}{\sqrt{n!}} \sum_{m=0}^\infty \del{\frac{-it}{\hbar}}^m \frac{1}{m!} \del{\hbar \omega \del{n + \frac{1}{2}}}^m \ket{n} \\
      &= \exp\del{-\frac{1}{2}  \envert{\alpha}^2}  \sum_{n=0}^\infty \frac{\alpha^n}{\sqrt{n!}} \exp\del{-i \omega t \del{n + \frac{1}{2}}}\ket{n} \\
      &= \exp\del{-\frac{1}{2}  \envert{\alpha}^2}  \exp\del{-\frac{i \omega t}{2}} \sum_{n=0}^\infty \frac{\alpha^n}{\sqrt{n!}} \exp\del{-i \omega t n}\ket{n} \\
      \intertext{
        Inserting (\ref{from_ex}), we get
      }
      &= \exp\del{-\frac{1}{2}  \envert{\alpha}^2}  \exp\del{-\frac{i \omega t}{2}} \sum_{n=0}^\infty \frac{\alpha^n}{\sqrt{n!}} \exp\del{-i \omega t n} \frac{1}{\sqrt{n!}} \del{a^\dagger}^n \ket{0} \\
      &= \exp\del{-\frac{1}{2}  \envert{\alpha}^2}  \exp\del{-\frac{i \omega t}{2}} \sum_{n=0}^\infty \frac{\del{\alpha e^{-i \omega t} a^\dagger}^n}{n!} \ket{0} \\
      &= \exp\del{-\frac{1}{2}  \envert{\alpha}^2}  \exp\del{-\frac{i \omega t}{2}} \exp \del{\alpha e^{-i \omega t} a^\dagger} \ket{0} \\
      &= \exp\del{-\frac{i \omega t}{2}} \underbrace{\exp\del{-\frac{1}{2}  \envert{\alpha}^2  + \alpha e^{-i \omega t} a^\dagger} \ket{0}}_{\text{Looks similar to equation (\ref{coherent_state})}} \\
      &= \exp\del{-\frac{i \omega t}{2}} \ket{\alpha e^{-i \omega t}}
    \end{align*}
  \item
    From the definition of the anhilation and creation operators
    \begin{equation*}
      a = \frac{1}{\sqrt{2 m \omega \hbar}} \del{m \omega x + ip} \qquad
      a^\dagger = \frac{1}{\sqrt{2 m \omega \hbar}} \del{m \omega x - ip}
    \end{equation*}

    we get
    \begin{equation*}
      x = \sqrt{\frac{\hbar}{2 m \omega}} \del{a + a^\dagger} \qquad
      p = -i\sqrt{\frac{m \omega \hbar}{2}} \del{a - a^\dagger}
    \end{equation*}

    So the expected values are
    \begin{align*}
      \braket{x} &= \sqrt{\frac{\hbar}{2 m \omega}} \braket{\alpha e^{-i \omega t}|\del{a + a^\dagger}|\alpha e^{-i \omega t}} \\
                 &= \sqrt{\frac{\hbar}{2 m \omega}} \del{ \braket{\alpha e^{-i \omega t}|a|\alpha e^{-i \omega t}} +
                                                          \braket{\alpha e^{-i \omega t}|a^\dagger|\alpha e^{-i \omega t}}} \\
                 &= \sqrt{\frac{\hbar}{2 m \omega}} \del{ \alpha e^{-i \omega t} + \alpha^* e^{i \omega t}} \\
    \end{align*}

    and (as above)
    \begin{align*}
      \braket{p} &= -i\sqrt{\frac{m \hbar \omega}{2}} \del{\alpha e^{-i \omega t} - \alpha^* e^{i \omega t}}
    \end{align*}

    The variance of the position operator is
    \begin{equation*}
      \operatorname{Var}(x) = \braket{x^2} - \braket{x}^2
    \end{equation*}

    where
    \begin{align*}
      \braket{x^2} &= \frac{\hbar}{2 m \omega} \braket{\alpha e^{-i \omega t}|\del{a + a^\dagger}^2|\alpha e^{-i \omega t}} \\
                   &= \frac{\hbar}{2 m \omega} \del{\braket{a^2} + \braket{aa^\dagger} + \braket{a^\dagger a} + \braket{\del{a^\dagger}^2}}
    \end{align*}

    Evaluating the operator values
    \begin{align*}
      \braket{a^2} &= \braket{\alpha e^{-i \omega t}|a^2|\alpha e^{-i \omega t}} = \alpha^2 e^{-2i \omega t} \\
      \braket{{a^\dagger}^2} &= \braket{\alpha e^{-i \omega t}|{a^\dagger}^2|\alpha e^{-i \omega t}} = {\alpha^*}^2 e^{2i \omega t} \\
      \braket{a^\dagger a} &= \alpha^* e^{i \omega t} \alpha e^{-i \omega t} = \envert{\alpha}^2 \\
      \braket{aa^\dagger} &= \braket{a^\dagger a + 1} = \envert{\alpha}^2 + 1
      \quad \text{(Because $[a, a^\dagger] = 1$)} \\
    \end{align*}

    Inserting into the original equation, we get
    \begin{align*}
      \braket{x^2} &= \frac{\hbar}{2 m \omega} \del{\alpha^2 e^{-2i \omega t} + 2\envert{\alpha}^2 + 1 + {\alpha^*}^2 e^{2 i \omega t}}
    \end{align*}

    This leads to
    \begin{align*}
      \operatorname{Var}(x) &= \braket{x^2} - \braket{x}^2 \\
                            &= \frac{\hbar}{2 m \omega} \del{\alpha^2 e^{-2i \omega t} + 2\envert{\alpha}^2 + 1 + {\alpha^*}^2 e^{2 i \omega t}} -
                               \frac{\hbar}{2 m \omega} \del{\alpha^2 e^{-2i \omega t} + 2\envert{\alpha}^2 + {\alpha^*}^2 e^{2 i \omega t}} \\
                            &= \frac{\hbar}{2 m \omega}
    \end{align*}

\end{enumerate}
\end{document}
