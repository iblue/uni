\documentclass[a4paper,german,12pt,smallheadings]{scrartcl}
\usepackage[T1]{fontenc}
\usepackage[utf8]{inputenc}
\usepackage{babel}
\usepackage{tikz}
\usepackage{geometry}
\usepackage{amsmath}
\usepackage{amssymb}
\usepackage{float}
\usepackage{enumerate}
\usepackage{braket} % Teh quantum stuff
%\usepackage{wrapfig}
\usepackage[thinspace,thinqspace,squaren,textstyle]{SIunits}

% New command for color underlining
\usepackage{xcolor}

\newsavebox\MBox
\newcommand\colul[2][red]{{\sbox\MBox{$#2$}%
  \rlap{\usebox\MBox}\color{#1}\rule[-1.2\dp\MBox]{\wd\MBox}{0.5pt}}}

\restylefloat{table}
\geometry{a4paper, top=15mm, left=20mm, right=40mm, bottom=20mm, headsep=10mm, footskip=12mm}
\linespread{1.5}
\setlength\parindent{0pt}
\DeclareMathOperator{\Tr}{Tr}
\begin{document}
\begin{center}
\bfseries % Fettdruck einschalten
\sffamily % Serifenlose Schrift
\vspace{-40pt}
Quantum Mechanics, winter term 2013/2014, exercise sheet 4

Markus Fenske, Tutor: Adam Nagy
\vspace{-10pt}
\end{center}

\section*{Exercise 1: Uncertainty principle}
\begin{enumerate}[a)]
  \item
    When the electrons passes through a slit of width $d$, the uncertainty on the $x$-Axis is

    \begin{equation*}
      \Delta x = d
    \end{equation*}

    Therefore the uncertainty in momentum is
    \begin{equation*}
      \Delta p_x = \frac{\hbar}{2 \Delta x}
    \end{equation*}

    Using the non-relativistic momentum, we obtain for $\Delta v \ll v$

    \begin{equation*}
      \Delta p_x = \gamma m_e \Delta v_x
    \end{equation*}

    Therefore the uncertainty in speed is
    \begin{equation*}
      \Delta v_x = \frac{\hbar}{2 \gamma m_e \Delta x}
    \end{equation*}

    If the electrons need a time $t$ from the slit to the wall, the spatial uncertainty will be
    \begin{equation*}
      b = \Delta v_x \cdot t = \frac{\hbar}{2 \gamma m_e \Delta x} t
    \end{equation*}

    The time needed for a distance $d$ at speed $v$ is
    \begin{equation*}
      t = \frac{d}{v}
    \end{equation*}

    Using the relativistic equations for energy and momentum ($E_0 = mc^2$)
    \begin{equation*}
      E_\text{kin} = (\gamma - 1) E_0
    \end{equation*}

    We obtain
    \begin{equation*}
      v = c \sqrt{1- \frac{E_0^2}{(E_\text{kin} - E_0)^2}}
    \end{equation*}

    Therefore the complete equation is
    \begin{equation*}
      b = \frac{\hbar}{2 m_e \Delta x} \frac{d}{c \alpha} \sqrt{1 - \alpha},\qquad \alpha = \sqrt{1 - \frac{E_0^2}{(E_\text{kin} + E_0)^2}}
    \end{equation*}

    We get $\alpha \approx 0.4127$ ($v = 0.4127 c$) and therefore an additional beam width of
    \begin{equation*}
      b = 0.0251 \;\micro \meter
    \end{equation*}

    Therefore the total beam width should be
    \begin{equation*}
      w = 50 \pm 0.0251 \; \micro \meter
    \end{equation*}


  \item
    %\textbf{TODO:} Relativistic calculation, because of the huge energy.

    Using the uncertainty principle
    \begin{equation*}
      \Delta x \Delta p \ge \frac{\hbar}{2}
    \end{equation*}

    We get for the uncertainty of momentum:
    \begin{equation*}
      \Delta p \ge \frac{\hbar}{2 \Delta x}
    \end{equation*}

    The kinetic energy is
    \begin{equation*}
      E_\text{kin} = \frac{1}{2} mv^2 = \frac{p_0^2}{2m}
    \end{equation*}

    With the uncertain momentum $p_0 = p + \Delta p$ this leads to
    \begin{equation*}
      E_\text{kin} \ge \frac{(p + \frac{\hbar}{2 \Delta x})^2}{2m}
    \end{equation*}

    The minimal momentum is $p=0$, which results in the zero point energy
    \begin{equation*}
      E_\text{kin} \ge \frac{\hbar}{8 \Delta x^2 m}
    \end{equation*}

    With the given value of $\Delta x = 10^{-14} \; \meter$, this results in a
    minimum energy of $E_\text{kin,min} \approx 598 \; \mega\electronvolt$

  \item
    Using energy-time uncertainty
    \begin{equation*}
      \Delta E \Delta t \ge \frac{\hbar}{2}
    \end{equation*}

    And the photon energy
    \begin{equation*}
      E = h \nu
    \end{equation*}

    We obtain
    \begin{equation*}
      \Delta \nu \ge \frac{\hbar}{h} \frac{1}{2 \Delta t} = \frac{1}{4 \pi \Delta t}
    \end{equation*}

    For a pulse of $\Delta t = 1 \; \nano\second$ this results in a minium frequency width $\Delta \nu_\text{min}$ of
    \begin{equation*}
      \Delta \nu_\text{min} \approx 79.58 \; \mega\hertz
    \end{equation*}
  \item
    Using above equation with $\Delta t = 0.2 \; \second$, we obtain
    \begin{equation*}
      \Delta \nu_\text{min} \approx 0.398 \; \hertz
    \end{equation*}

    It would be easier to determine the frequency of a tone of higher pitch,
    because the percentual uncertainty is lower.
\end{enumerate}
\section*{Exercise 3: Commutators and eigenvalues of hermitian $2 \times 2$ matricies}

\begin{enumerate}[a)]
  \item
    (i)
    \begin{align*}
      &[A, BC] =
      ABC - BCA =
      ABC - BCA + (BAC-BAC) =
      BAC - BCA + ABC - BAC = \\
      &B(AC-CA) + (AB-BA)C =
      B[A,C] + [A,B]C
    \end{align*}
    (ii)
    \begin{align*}
      &[A, [B,C]] + [B, [C,A]] + [C, [A,B]] = \\
      &[A, (BC-CB)] + [B, (CA-AC)] + [C, (AB-BA)] = \\
      &A(BC-CB)-(BC-CB)A + B(CA-AC)-(CA-AC)B + C(AB-BA)-(AB-BA)C = \\
      &\colul[red]{ABC}\colul[green]{-ACB}\colul[blue]{-BCA}\colul[yellow]{+CBA} \;\;
      \colul[blue]{+BCA}\colul[magenta]{-BAC}\colul[cyan]{-CAB}\colul[green]{+ACB} \;\;
      \colul[cyan]{+ CAB}\colul[yellow]{-CBA}\colul[red]{-ABC}\colul[magenta]{+BAC} = 0
    \end{align*}
  \item
    Determinant:
    \begin{equation*}
      \det M = \alpha\beta - \gamma\gamma^* = \alpha\beta - |\gamma|^2
    \end{equation*}

    Trace:
    \begin{equation*}
      \operatorname{tr} M = \alpha\beta
    \end{equation*}

    Eigenvalues:
    \begin{align*}
      &\det(M-\lambda I) = (\alpha - \lambda)(\beta - \lambda) - |\gamma|^2 \overset{!}{=} 0 \\
      \Leftrightarrow\quad& \alpha\beta - \alpha\lambda - \beta\lambda + \lambda^2 - |\gamma|^2 = 0 \\
      \Leftrightarrow\quad& \lambda^2 - \lambda(\alpha + \beta) - |\gamma|^2 + \alpha\beta = 0 \\
      \Leftrightarrow\quad& \lambda_{1,2} = \frac{\alpha+\beta}{2} \pm \sqrt{\frac{(\alpha + \beta)^2}{4} + |\gamma|^2 - \alpha\beta} \\
      & \quad\;\;\, = \frac{\alpha+\beta}{2} \pm \sqrt{\frac{\alpha^2 + 2\alpha\beta + \beta^2}{4} + |\gamma|^2 - \alpha\beta} \\
      & \quad\;\;\, = \frac{\alpha+\beta}{2} \pm \sqrt{\frac{\alpha^2 - 2\alpha\beta + \beta^2}{4} + |\gamma|^2} \\
      & \quad\;\;\,= \frac{\alpha+\beta}{2} \pm \sqrt{\frac{(\alpha - \beta)^2}{4} + |\gamma|^2} \\
    \end{align*}

    The term under the square root is $\ge 0$, because $\alpha, \beta,
    |\gamma|$ are real numbers. So the eigenvalues must be real numbers.
\end{enumerate}

\section*{Exercise 4: Pauli matricies}
\begin{enumerate}[a)]
  \item
    Traces:
    \begin{equation*}
      \operatorname{tr} \sigma_0 = 1, \qquad
      \operatorname{tr} \sigma_1 = 0, \qquad
      \operatorname{tr} \sigma_2 = 0, \qquad
      \operatorname{tr} \sigma_3 = 0
    \end{equation*}

    Determinants:
    \begin{equation*}
      \det \sigma_0 = 0, \qquad
      \det \sigma_1 = -1, \qquad
      \det \sigma_2 = -1, \qquad
      \det \sigma_1 = -1
    \end{equation*}

    Eigenvalues (using the formula from 3b):
    \begin{equation*}
      \lambda_{\sigma_0} = \{1,1\}, \qquad
      \lambda_{\sigma_1} = \{-1,1\}, \qquad
      \lambda_{\sigma_2} = \{-1,1\}, \qquad
      \lambda_{\sigma_3} = \{-1,1\}
    \end{equation*}

    The matricies are obviously hermitian (because $\sigma_n = \sigma_n^\dagger$).
  \item
    \begin{align*}
      \delta_{12} = 0, \epsilon_{123} = 1 \qquad\Rightarrow\qquad
      &\sigma_1\sigma_2 =
      \begin{pmatrix} 0 & 1 \\ 1 & 0 \end{pmatrix}
      \begin{pmatrix} 0 & -i \\ i & 0 \end{pmatrix}
      =
      \begin{pmatrix} i & 0 \\ 0 & -i \end{pmatrix} = i\sigma_3 \\
      \delta_{23} = 0, \epsilon_{231} = 1 \qquad\Rightarrow\qquad
      &\sigma_2\sigma_3 =
      \begin{pmatrix} 0 & -i \\ i & 0 \end{pmatrix}
      \begin{pmatrix} 1 & 0 \\ 0 & -1\end{pmatrix}
      =
      \begin{pmatrix} 0 & i \\ i & 0\end{pmatrix} = i\sigma_1 \\
      \delta_{23} = 0, \epsilon_{132} = -1 \qquad\Rightarrow\qquad
      &\sigma_1\sigma_3 =
      \begin{pmatrix} 0 & 1 \\ 1 & 0 \end{pmatrix}
      \begin{pmatrix} 1 & 0 \\ 0 & -1\end{pmatrix}
      =
      \begin{pmatrix} 0 & -1 \\ 1 & 0\end{pmatrix}
      = -i\sigma_2
    \end{align*}

    Using the given equation:

    \begin{equation*}
      [\sigma_i, \sigma_j] = \sigma_i\sigma_j - \sigma_j\sigma_i =
      \delta_{ij}\sigma_0 + i \sum_{k=1}^3 \epsilon_{ijk} \sigma_k -
      \left(\delta_{ji}\sigma_0 + i \sum_{k=1}^3 \epsilon_{jik} \sigma_k\right)
    \end{equation*}

    \textbf{Case 1:} For $i \neq j$, the Kronecker deltas will vanish.  Because $\epsilon_{ijk}
    = -\epsilon_{jik}$, this leads to

    \begin{equation*}
      [\sigma_i, \sigma_j] = 2i \sum_{k=1}^3 \epsilon_{ijk} \sigma_k
    \end{equation*}

    \textbf{Case 2:} If $i=j$, the commutator will be $[\sigma_i, \sigma_i] =
    0$, therefore the sum should vanish. The Levi-Civita symbol will
    become $\epsilon_{iik} = 0$. Therefore the equation is true for $i = j$ as
    well.
  \item
    \textbf{TODO:} Calculate dot product of matricies, to prove orthogonality

    To prove that the Pauli matrices are a orthogonal base of $\mathbb{C}^{2
    \times 2}$ is to show that there exists a unique solution to this equation
    for every $M$:

    \begin{equation*}
    M = \sum_{i=0}^3 c_i \sigma_i
    \end{equation*}

    This can be written as:
    \begin{equation*}
      \begin{pmatrix}
        \alpha & \beta \\ \gamma & \delta
      \end{pmatrix}
      =
      c_0
      \begin{pmatrix}
        1 & 0 \\
        0 & 1
      \end{pmatrix}
      +
      c_1
      \begin{pmatrix}
        0 & 1 \\
        1 & 0
      \end{pmatrix}
      +
      c_2
      \begin{pmatrix}
        0 & -i \\
        i & 0
      \end{pmatrix}
      +
      c_3
      \begin{pmatrix}
        1 & 0 \\
        0 & -1
      \end{pmatrix}
    \end{equation*}

    Which is just a system of linear equations and can be solved to
    \begin{equation*}
      c_0 = \frac{\alpha + \delta}{2}, \qquad
      c_1 = \frac{\beta + i \gamma}{2}, \qquad
      c_2 = \frac{\beta - i \gamma}{2}, \qquad
      c_3 = \frac{\alpha - \delta}{2}
    \end{equation*}

    Therefore the Pauli matrices create a orthgonal basis of $\mathbb{C}^{2
    \times 2}$. This also solves the second part of the exercise.
\end{enumerate}

\end{document}
