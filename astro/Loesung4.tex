\documentclass[a4paper,german,12pt,smallheadings]{scrartcl}
\usepackage[T1]{fontenc}
\usepackage[utf8]{inputenc}
\usepackage{babel}
\usepackage{tikz}
\usetikzlibrary{calc}
\usepackage{tkz-euclide}
\usetkzobj{all}
\usepackage{pgfplots}
\usepackage{geometry}
\usepackage[fleqn]{amsmath}
\usepackage{amssymb}
\usepackage{float}
\usepackage{lscape}
\usepackage{enumerate}
\usepackage{commath} % http://tex.stackexchange.com/questions/14821/whats-the-proper-way-to-typeset-a-differential-operator
\usepackage{cancel}
\usepackage{gnuplottex}

% Number only referenced equations
\usepackage[fleqn]{mathtools}
\mathtoolsset{showonlyrefs}

%\usepackage{wrapfig}
\usepackage{siunitx}
\sisetup{locale = DE}

% New command for color underlining
\usepackage{xcolor}

\newsavebox\MBox
\newcommand\colul[2][red]{{\sbox\MBox{$#2$}%
  \rlap{\usebox\MBox}\color{#1}\rule[-1.2\dp\MBox]{\wd\MBox}{0.5pt}}}

% Stupid units for astronomy
\DeclareSIUnit\day{d}
\DeclareSIUnit\year{yr}
\DeclareSIUnit\au{AU}
% Astronomy-Style angles (hours, minutes, seconds)
\newcommand*{\ra}[2][]{{
  \def\SIUnitSymbolDegree{\textsuperscript{h}}%
  \def\SIUnitSymbolArcminute{\textsuperscript{m}}%
  \def\SIUnitSymbolArcsecond{\textsuperscript{s}}%
  \ang[#1]{#2}}%
}

\restylefloat{table}
\geometry{a4paper, top=15mm, left=10mm, right=20mm, bottom=20mm, headsep=10mm, footskip=12mm}
\linespread{1.2}
\setlength\parindent{0pt}
\DeclareMathOperator{\Tr}{Tr}
\DeclareMathOperator{\Var}{Var}

\begin{document}
\allowdisplaybreaks % Seitenumbrüche in Formeln erlauben
\begin{center}
\bfseries % Fettdruck einschalten
\sffamily % Serifenlose Schrift
\vspace{-40pt}
Einführung in die Astronomie und Astrophysik, Sommersemester 2014, Aufgabenblatt 4

Markus Fenske, Julia Schuch, Tutor: Daniel Härdt
\vspace{-10pt}
\end{center}
\section*{Aufgabe 3: Stefan-Boltzmann-Gesetz}

Die Plancksche Strahlungsformel gibt die Strahldichte für jede Frequenz an. Das
heißt, wir müssen über alle Frequenzen und dann über die Strahldichte integrieren.

\begin{align}
  \omega = \int_0^\infty \frac{8 \pi h \nu^3}{c^3} \frac{1}{\exp\del{\frac{h \nu}{k T}}} \dif \nu
\end{align}

Substitution $u = \frac{h \nu}{k T}$ liefert:

\begin{align}
  &= \int_0^\infty \frac{8 \pi h}{c^3} \del{ \frac{kT}{h} u}^3 \frac{1}{e^u - 1} \frac{kT}{h} \dif u \\
  &= \frac{8 \pi}{c^3} \frac{k^4 T^4}{h^3} \underbrace{\int_0^\infty \frac{u^3}{e^u - 1} \dif u}_{\mathclap{\Gamma(4) \zeta(4) = \frac{\pi^4}{15}}} \\
  &= \frac{8 \pi^5 k^4}{15 c^3 h^3} T^4
\end{align}

Für die Strahlungsdichte der vom Oberflächenelement $\dif A$ in das
Raumwinkelelement $\dif \Omega$ unter dem Winkel $\theta$ senkrecht zur Fläche
gilt $S = \frac{c}{4 \pi} \omega \cos \theta$. Die Strahlung wird nur in eine
Richtung der Kugel abgegeben (auf der anderen Seite ist der Strahler), also
$\dif \Omega = 2 \pi$.

Insgesamt gilt für die abgegebene Leistung pro Fläche also
\begin{align}
  P &= \int S \dif \Omega \\
    &= \omega \frac{c}{4 \pi} \int\limits_0^{\frac{\pi}{2}} \dif \theta \int\limits_0^{2 \pi} \dif \phi \cos \theta \sin \theta \\
    &= \omega \frac{c}{4} \\
    &= \frac{2 \pi^5 k^4}{15c^2h^3} T^4
\end{align}

Daraus folgt für die auf eine Fläche $A$ abgegebene Leistung

\begin{equation}
  P = \sigma A T^4
\end{equation}

Mit
\begin{equation}
  \sigma = \frac{2 \pi^5 k^4}{15 c^2 h^3} \approx \SI{5.67e-8}{W/(m^2 K^4)}
\end{equation}

\section*{Aufgabe 4: 2-Körper-Problem im Schwerpunktsystem}

\begin{align*}
  \vec{L}_1 + \vec{L}_2
  &= \vec{r}_1 \times \vec{p}_1 + \vec{r}_2 \times \vec{p}_2 \\
  &= m_1 \vec{r}_1 \times \dot{\vec{r}}_1 +
   m_2 \vec{r}_2 \times \dot{\vec{r}}_2
\end{align*}

Sei $\vec{r}$ der Verbindungsvektor zwischen den beiden Massen. Da der Ursprung
im Schwerpunkt liegt, ist $\vec{r}_1 = \dfrac{m_2}{m_1 + m_2} \vec{r} =
\dfrac{\mu}{m_1} \vec{r}$ und analog $\vec{r}_2 = -\dfrac{\mu}{m_2} \vec{r}$,
wobei $\mu$ die reduzierte Masse $\dfrac{m_1 m_2}{M}$ ist.

\begin{align*}
  &= \frac{\mu^2}{m_1} \vec{r} \times \dot{\vec{r}} + \frac{\mu^2}{m_2} \vec{r} \times \dot{\vec{r}}
\end{align*}

Wobei
\begin{align*}
  \frac{\mu^2}{m_1} + \frac{\mu^2}{m_2} = \mu^2 \del{\frac{1}{m_1} + \frac{1}{m_2}} = \mu^2 \frac{1}{\mu} = \mu
\end{align*}

Somit
\begin{equation}
  \vec{L}_1 + \vec{L}_2 = \mu \vec{r} \times \dot{\vec{r}} = \vec{L}_s
\end{equation}

\end{document}
