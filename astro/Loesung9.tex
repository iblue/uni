\documentclass[a4paper,german,12pt,smallheadings]{scrartcl}
\usepackage[T1]{fontenc}
\usepackage[utf8]{inputenc}
\usepackage{babel}
\usepackage{tikz}
\usetikzlibrary{calc}
\usepackage{tkz-euclide}
\usetkzobj{all}
\usepackage{pgfplots}
\usepackage{geometry}
\usepackage[fleqn]{amsmath}
\usepackage{amssymb}
\usepackage{float}
\usepackage{lscape}
\usepackage{enumerate}
\usepackage{commath} % http://tex.stackexchange.com/questions/14821/whats-the-proper-way-to-typeset-a-differential-operator
\usepackage{cancel}
\usepackage{gnuplottex}

% Number only referenced equations
\usepackage[fleqn]{mathtools}
\mathtoolsset{showonlyrefs}

%\usepackage{wrapfig}
\usepackage{siunitx}
\sisetup{locale = DE}

% New command for color underlining
\usepackage{xcolor}

\newsavebox\MBox
\newcommand\colul[2][red]{{\sbox\MBox{$#2$}%
  \rlap{\usebox\MBox}\color{#1}\rule[-1.2\dp\MBox]{\wd\MBox}{0.5pt}}}

% Stupid units for astronomy
\DeclareSIUnit\day{d}
\DeclareSIUnit\year{yr}
\DeclareSIUnit\au{AU}
% Astronomy-Style angles (hours, minutes, seconds)
\newcommand*{\ra}[2][]{{
  \def\SIUnitSymbolDegree{\textsuperscript{h}}%
  \def\SIUnitSymbolArcminute{\textsuperscript{m}}%
  \def\SIUnitSymbolArcsecond{\textsuperscript{s}}%
  \ang[#1]{#2}}%
}

\restylefloat{table}
\geometry{a4paper, top=15mm, left=10mm, right=20mm, bottom=20mm, headsep=10mm, footskip=12mm}
\linespread{1.2}
\setlength\parindent{0pt}
\DeclareMathOperator{\Tr}{Tr}
\DeclareMathOperator{\Var}{Var}

\begin{document}
\allowdisplaybreaks % Seitenumbrüche in Formeln erlauben
\begin{center}
\bfseries % Fettdruck einschalten
\sffamily % Serifenlose Schrift
\vspace{-40pt}
Einführung in die Astronomie und Astrophysik, Sommersemester 2014, Aufgabenblatt 9

Markus Fenske, Julia Schuch, Tutor: Daniel Härdt
\vspace{-10pt}
\end{center}
\section*{Aufgabe 3: Potentielle Energie einer Molekülwolke}
Die potentielle Energie ist die Energie, die ich erhalte, wenn ich die Wolke
zusammenbaue.

Angenommen es gäbe bereits eine kugelförmige Masse $M$ und ich hole aus
unendlicher Entfernung eine kleine Masse $dM$ die ich sphärisch um die Kugel
verteile, dann wird die potentielle Energie
\begin{equation}
  dV = - \frac{G M dM}{r}
\end{equation}
frei.

Wenn ich die Wolke aus solchen sphärischen Schichten der Dichte $\rho$ aufbaue,
erhalte ich jeweils eine zusätzliche Kugelschale der Dichte $\rho$ mit dem
Radius $dr$. Diese hat jeweils die Masse
\begin{equation}
  dM = 4 \pi r^2 \rho dr
\end{equation}

Angenommen, ich habe solche Schichten bis zum Radius $r$ gestapelt und die
Dichte sei konstant, dann ist die Masse
\begin{equation}
  M(r) = \rho V = \rho \frac{4}{3} \pi r^3
\end{equation}

Wenn wir über die einzelnen Schichten integrieren, erhalten wir
\begin{equation}
  V = -G \int_0^R \frac{M(r) dM(r)}{r} = -G \frac{16 \pi^2}{4} \rho^2 \int_0^R \dif \; r^4 = -G\frac{16 \pi^2 \rho^2}{4} \frac{1}{5} R^5
\end{equation}

Mit
\begin{equation}
  \rho = \frac{M}{V} = \frac{3 M}{4 \pi R^3}
\end{equation}

also
\begin{equation}
  V = -G \frac{16 \pi^2}{4} \frac{1}{5} R^5 \frac{9 M^2}{16\pi^2 R^6} = - \frac{3}{5} \frac{G M^2}{R}
\end{equation}


\end{document}
