\documentclass[a4paper,german,12pt,smallheadings]{scrartcl}
\usepackage[T1]{fontenc}
\usepackage[utf8]{inputenc}
\usepackage{babel}
\usepackage{tikz}
\usetikzlibrary{calc}
\usepackage{tkz-euclide}
\usetkzobj{all}
\usepackage{pgfplots}
\usepackage{geometry}
\usepackage[fleqn]{amsmath}
\usepackage{amssymb}
\usepackage{float}
\usepackage{lscape}
\usepackage{enumerate}
\usepackage{commath} % http://tex.stackexchange.com/questions/14821/whats-the-proper-way-to-typeset-a-differential-operator
\usepackage{cancel}
\usepackage{gnuplottex}

% Number only referenced equations
\usepackage[fleqn]{mathtools}
\mathtoolsset{showonlyrefs}

%\usepackage{wrapfig}
\usepackage{siunitx}
\sisetup{locale = DE}

% New command for color underlining
\usepackage{xcolor}

\newsavebox\MBox
\newcommand\colul[2][red]{{\sbox\MBox{$#2$}%
  \rlap{\usebox\MBox}\color{#1}\rule[-1.2\dp\MBox]{\wd\MBox}{0.5pt}}}

% Stupid units for astronomy
\DeclareSIUnit\day{d}
\DeclareSIUnit\year{yr}
\DeclareSIUnit\au{AU}
% Astronomy-Style angles (hours, minutes, seconds)
\newcommand*{\ra}[2][]{{
  \def\SIUnitSymbolDegree{\textsuperscript{h}}%
  \def\SIUnitSymbolArcminute{\textsuperscript{m}}%
  \def\SIUnitSymbolArcsecond{\textsuperscript{s}}%
  \ang[#1]{#2}}%
}

\restylefloat{table}
\geometry{a4paper, top=15mm, left=10mm, right=20mm, bottom=20mm, headsep=10mm, footskip=12mm}
\linespread{1.2}
\setlength\parindent{0pt}
\DeclareMathOperator{\Tr}{Tr}
\DeclareMathOperator{\Var}{Var}

\begin{document}
\allowdisplaybreaks % Seitenumbrüche in Formeln erlauben
\begin{center}
\bfseries % Fettdruck einschalten
\sffamily % Serifenlose Schrift
\vspace{-40pt}
Einführung in die Astronomie und Astrophysik, Sommersemester 2014, Aufgabenblatt 6

Markus Fenske, Julia Schuch, Tutor: Daniel Härdt
\vspace{-10pt}
\end{center}
\section*{Aufgabe 1: Klassifikation von Sternspektren aus Objektivprismenaufnahmen}
Im Tutorium bearbeitet.

\section*{Aufgabe 3: Roche-Flächen in Doppelsternsystem}
\begin{enumerate}[a)]
  \item
    Als Koordinatensystem wählen wir ein mitrotierendes
    Schwerpunktkoordinatensystem. Dabei setzen wir aufgrund der
    Rotationssymmetrie um die $x$-Achse direkt $z=0$. Die Sterne befinden sich
    dann an den Orten
    % solve([d=M_A/(M_A+M_B)*x_A + M_B/(M_A+M_B)*x_B, x_A+x_B=d], [x_A, x_B]);
    \begin{equation}
      x_A = \frac{aq}{q-1}, x_B = -\frac{a}{q-1}
    \end{equation}

    Sie haben die Massen
    % solve([q=M_A/M_B, M=M_A+M_B], [M_A, M_B])
    \begin{equation}
      M_A = \frac{qM}{q+1}, M_B = \frac{M}{q+1}
    \end{equation}

    Aus den Gravitationskräften erhalten wir die Potentiale
    \begin{equation}
      \phi_{A,B} = -G \frac{M_{A,B}}{\sqrt{(x - x_{A,B})^2 + y^2}}
    \end{equation}

    Das System hat einen einen konstanten Abstand, rotiert also nach
    Keplerschem Gesetz mit konstanter Winkelgeschwindigkeit $\omega = \sqrt{\dfrac{GM}{d}}$. Daraus
    resultiert das Zentrifugalpotential
    \begin{equation}
      \phi_Z = -\frac{\omega^2 r^2}{2} = -\frac{GM(x^2+y^2)}{2d}
    \end{equation}

    Setzt man das alles ein, normiert die Längen auf $a$, die Massen auf $M$
    und setzt der Einfachheit halber auch $G=1$, erhält man
    \begin{equation}
      \phi =
      \frac{q - 1}{\sqrt{\dfrac{y^2}{a^2} + \del{\dfrac{x}{a} - q}^2}}
      - \frac{q}{\sqrt{\dfrac{y^2}{a^2} + \del{\dfrac{x}{a} - q + 1}^2}}
      - \frac{x^2 + y^2}{2a^2}
    \end{equation}

    %FIXME: Welchen Einfluss haben a und q auf das System? => Keine Ahnung?


  \item
    \begin{figure}[h!]
      \includegraphics[width=\textwidth]{lagrange.pdf}
      \label{fig:pot}
      \caption{Effektives Potential eines Doppelsternsystems}
    \end{figure}
    Skizze siehe nächste Seite.

    Die Linien entsprechen den Äquipotentialflächen. Die Extrema wurden jeweils mit
    einem Punkt markiert, sie entsprechen den Lagrangepunkten, denn in den
    Extrempunkten des Potentials wirkt keine Kraft mehr (auf ein Teilchen
    infinitesimaler Masse).

    Ein Materietausch kann stattfinden, sobald Materie des einen Sterns das
    Potentialextremum (Lagrangepunkt) überschreitet, denn dort kehrt die Kraft
    ihr Vorzeichen um. Einer der Sterne muss also größer sein, als der
    entsprechende Radius.
  \item
    Wenn für Stern A der Radius gegeben ist durch
    \begin{equation}
      R_{R,A} \approx 0{,}44 \frac{q^{0{,}33}}{(q+1)^{0{,}2}}
    \end{equation}

    Das Problem ist symmetrisch, die Massen sind vertauschbar. Wegen $q =
    \frac{M_A}{M_B}$ muss für Stern B einfach nur $q$ durch $q^{-1} =
    \frac{M_B}{M_A}$ ersetzt werden. Somit
    \begin{equation}
      R_{R,B} \approx 0{,}44 \frac{q^{-0{,}33}}{(\frac{1}{q}+1)^{0{,}2}}
    \end{equation}

    Daraus folgt
    \begin{align*}
      \frac{R_{R,A}}{R_{R,B}} = \frac{q^{0,33}}{q^{-0{,}33}} \frac{(\frac{1}{q}+1)^{0{,}2}}{(q+1)^{0{,}2}} = q^{0{,}66} \frac{\del{\frac{1}{q}}^{0{,}2} (q+1)^{0{,}2}}{(q+1)^{0{,}2}} = q^{0{,}46}
    \end{align*}


\end{enumerate}
\end{document}
