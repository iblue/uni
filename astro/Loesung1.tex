\documentclass[a4paper,german,12pt,smallheadings]{scrartcl}
\usepackage[T1]{fontenc}
\usepackage[utf8]{inputenc}
\usepackage{babel}
\usepackage{tikz}
\usepackage{pgfplots}
\usepackage{geometry}
\usepackage[fleqn]{amsmath}
\usepackage{amssymb}
\usepackage{float}
\usepackage{enumerate}
\usepackage{commath} % http://tex.stackexchange.com/questions/14821/whats-the-proper-way-to-typeset-a-differential-operator
\usepackage{cancel}

% Number only referenced equations
\usepackage[fleqn]{mathtools}
\mathtoolsset{showonlyrefs}

%\usepackage{wrapfig}
\usepackage{siunitx}
\sisetup{locale = DE}

% New command for color underlining
\usepackage{xcolor}

\newsavebox\MBox
\newcommand\colul[2][red]{{\sbox\MBox{$#2$}%
  \rlap{\usebox\MBox}\color{#1}\rule[-1.2\dp\MBox]{\wd\MBox}{0.5pt}}}

% Stupid units for astronomy
\DeclareSIUnit\day{d}
\DeclareSIUnit\year{yr}
% Astronomy-Style angles (hours, minutes, seconds)
\newcommand*{\ra}[2][]{{
  \def\SIUnitSymbolDegree{\textsuperscript{h}}%
  \def\SIUnitSymbolArcminute{\textsuperscript{m}}%
  \def\SIUnitSymbolArcsecond{\textsuperscript{s}}%
  \ang[#1]{#2}}%
}

\restylefloat{table}
\geometry{a4paper, top=15mm, left=10mm, right=20mm, bottom=20mm, headsep=10mm, footskip=12mm}
\linespread{1.5}
\setlength\parindent{0pt}
\DeclareMathOperator{\Tr}{Tr}
\DeclareMathOperator{\Var}{Var}

\begin{document}
\allowdisplaybreaks % Seitenumbrüche in Formeln erlauben
\begin{center}
\bfseries % Fettdruck einschalten
\sffamily % Serifenlose Schrift
\vspace{-40pt}
Einführung in die Astronomie und Astrophysik, Sommersemester 2014, Aufgabenblatt 1

Markus Fenske, Julia Schuch, Tutor: Daniel Härdt
\vspace{-10pt}
\end{center}
\section*{Aufgabe 1: Beobachtung am Teleskop}

\begin{enumerate}[a)]
  \item
    Wir rechnen die Zeit in Jahre um:
    \begin{equation}
      t = 2014 + \frac{\SI{31}{\day} + \SI{28}{\day} + \SI{31}{\day} + \SI{30}{\day}}{\SI{365,2422}{\day\per\year}} \approx \SI{2014.33}{\year}
    \end{equation}

    und sehen, dass für Mars keine Präzessionskorrekturen notwendig sind. Für
    Beteigeuze rechnen wir zuerst die Koordinaten in Grad um:

    \begin{equation}
      \alpha = \ra{5;55;5.4} = \SI{15}{\degree\per\hour} \del{
        \SI{5}{\hour} +
        \frac{\SI{55}{\minute}}{\SI{60}{\minute\per\hour}} +
        \frac{\SI{5.4}{\second}}{\SI{60}{\minute\per\hour}\SI{60}{\second\per\minute}}
      } \approx \ang{88.77}
    \end{equation}

    \begin{equation}
      \delta = \ang{7;24;25} = \ang{7} + \frac{\SI{24}{\arcminute}}{\SI{60}{\arcminute\per\degree}} + \frac{\SI{25}{\arcsecond}}{\SI{60}{\arcsecond\per\arcminute} \SI{60}{\arcminute\per\degree}} \approx \ang{7.41}
    \end{equation}

    Anhand der gegebenen Formeln
    \begin{equation}
      \Delta \alpha(t) = \del{
        \SI{46}{\arcsecond\per\year} +
        \SI{20}{\arcsecond\per\year} \tan \delta \sin \alpha
      } \Delta t
    \end{equation}
    und
    \begin{equation}
      \Delta \delta(t) = \SI{20}{\arcsecond\per\year} \cos \alpha \cdot \Delta t
    \end{equation}

    Erhalten wir mit $\Delta t = \SI{2014.33}{\year} - \SI{1998.50}{\year} = \SI{15.83}{\year}$ die Korrekturen
    \begin{equation}
      \Delta \alpha \approx \SI{769.3}{\arcsecond} \cdot \frac{1}{\SI{15}{\arcsecond\per\second}} \approx \SI{51.29}{\second}
    \end{equation}
    und
    \begin{equation}
      \Delta \delta \approx \SI{6.80}{\arcsecond}
    \end{equation}
  \item
    Wir sparen uns ab hier die ausführlichen Umrechnungen der Gradmaße.

    Der Längengrad der Takustr. 3a ist im Skript angegeben mit
    \begin{equation}
      \lambda_P = \ang{-13;17;48.8} \approx \ang{-13.297} \approx \ra{-0.8864}
    \end{equation}

    Die GMST ist für den 30. April 2014 tabelliert als
    \begin{equation}
      \Theta(\lambda_G, \ra{0} \text{ UT}) = \ra{14;31;26.5} = \ra{14.5240}
    \end{equation}

    Die Zeitkorrektur ist aufgrund der Sommerzeit
    \begin{equation}
      \Delta T = \ra{-2}
    \end{equation}

    Die Uhrzeit ist
    \begin{equation}
      t = \ra{23}
    \end{equation}

    Eingesetzt in die gegebene Formel erhält man
    \begin{equation}
      \Theta(\lambda_B, t) = \Theta(\lambda_G, \ra{0} \text{ UT}) - \lambda_P + (t + \Delta T) \cdot 1{,}00274 \approx \ra{36.6376} = \ra{12.4104} = \ra{12;24;37}
    \end{equation}

  \item
    Die Stundenwinkel berechnen sich durch
    \begin{equation}
      \tau = \Theta(\lambda_P, t) - \alpha(t)
    \end{equation}

    Unter Berücksichtigung der Präzessionskorrekturen ist
    \begin{equation}
      \tau = \Theta(\lambda_P, t) - \alpha - \Delta \alpha(t)
    \end{equation}

    Somit ergibt sich für Mars
    \begin{equation}
      \tau = \ra{-0.3334} = \ra{23.6666} = \ra{23;40;0}
    \end{equation}

    Und für Beteigeuze
    \begin{equation}
      \tau = \ra{6.4936} = \ra{6;29;37}
    \end{equation}
  \item
    Durch Einsetzen der vorher berechneten Werte und der Breite $\Phi_P =
    \ang{52.4539}$ in die gegebene Formel
    \begin{equation}
      h = \arcsin \del{\sin \Phi_P \sin \delta + \cos \Phi_P \cos \delta \cos \tau}
    \end{equation}

    erhält man für Mars
    \begin{equation}
      h = \ang{34.307} = \ang{34;18;25.2}
    \end{equation}

    und für Beteigeuze
    \begin{equation}
      h = \ang{1.4} = \ang{1;24;0.0}
    \end{equation}

\end{enumerate}
\end{document}
