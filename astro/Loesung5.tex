\documentclass[a4paper,german,12pt,smallheadings]{scrartcl}
\usepackage[T1]{fontenc}
\usepackage[utf8]{inputenc}
\usepackage{babel}
\usepackage{tikz}
\usetikzlibrary{calc}
\usepackage{tkz-euclide}
\usetkzobj{all}
\usepackage{pgfplots}
\usepackage{geometry}
\usepackage[fleqn]{amsmath}
\usepackage{amssymb}
\usepackage{float}
\usepackage{lscape}
\usepackage{enumerate}
\usepackage{commath} % http://tex.stackexchange.com/questions/14821/whats-the-proper-way-to-typeset-a-differential-operator
\usepackage{cancel}
\usepackage{gnuplottex}

% Number only referenced equations
\usepackage[fleqn]{mathtools}
\mathtoolsset{showonlyrefs}

%\usepackage{wrapfig}
\usepackage{siunitx}
\sisetup{locale = DE}

% New command for color underlining
\usepackage{xcolor}

\newsavebox\MBox
\newcommand\colul[2][red]{{\sbox\MBox{$#2$}%
  \rlap{\usebox\MBox}\color{#1}\rule[-1.2\dp\MBox]{\wd\MBox}{0.5pt}}}

% Stupid units for astronomy
\DeclareSIUnit\day{d}
\DeclareSIUnit\year{yr}
\DeclareSIUnit\au{AU}
% Astronomy-Style angles (hours, minutes, seconds)
\newcommand*{\ra}[2][]{{
  \def\SIUnitSymbolDegree{\textsuperscript{h}}%
  \def\SIUnitSymbolArcminute{\textsuperscript{m}}%
  \def\SIUnitSymbolArcsecond{\textsuperscript{s}}%
  \ang[#1]{#2}}%
}

\restylefloat{table}
\geometry{a4paper, top=15mm, left=10mm, right=20mm, bottom=20mm, headsep=10mm, footskip=12mm}
\linespread{1.2}
\setlength\parindent{0pt}
\DeclareMathOperator{\Tr}{Tr}
\DeclareMathOperator{\Var}{Var}

\begin{document}
\allowdisplaybreaks % Seitenumbrüche in Formeln erlauben
\begin{center}
\bfseries % Fettdruck einschalten
\sffamily % Serifenlose Schrift
\vspace{-40pt}
Einführung in die Astronomie und Astrophysik, Sommersemester 2014, Aufgabenblatt 5

Markus Fenske, Julia Schuch, Tutor: Daniel Härdt
\vspace{-10pt}
\end{center}
\section*{Aufgabe 1: Energiehaushalt eines Hot Jupiter}
\begin{enumerate}[a)]
  \item
    Aus der Grafik kann man 9 Perioden im Zeitraum von 80 Tagen ablesen. Daraus
    ergibt sich die Periode $T = \dfrac{80}{9}\,\mathrm{d} = 8{,}\overline{8}\,\mathrm{d}$.

    Aus dem 3. Keplerschen Gesetz erhalten wir
    % WA: ((gravitational constant) * (1.05*(mass of sun) + 2.96*(mass of
    % jupiter))*(80/9 days)^2/(4 pi^2))^(1/3) in astronomical units
    \begin{equation}
      a = \sqrt[3]{
        \frac{G(M+m)T^2}{4 \pi^2}
      } \approx \SI{0.0854}{AU}
    \end{equation}
  \item
    Die Strahlungsgleichgewichtstemperatur erhalten wir aus
    \begin{equation}
      T_\text{Pl} = \sqrt[4]{
        (1-A) \frac{F_\text{abs}}{F_\text{em}} \frac{R_\star^2}{a^2} T_\star^4
      }
    \end{equation}

    Dabei ist $F_\text{abs} = \pi R_\text{Pl}^2$ die absorbierende Fläche (also
    die Größe des Schattens des Planeten).

    %Aus der gegebenen Masse und der
    %Dichte erhalten wir den Radius, wenn wir den Planeten als kugelförmig
    %annehmen
    %\begin{equation}
    % V = \frac{M_\text{Pl}}{\rho_\text{Pl}} = \frac{4}{3} \pi R_\text{Pl}^3
    % \Leftrightarrow R_\text{Pl} = \sqrt[3]{\frac{3M}{4\pi \rho}}
    %\end{equation}

    \begin{enumerate}[(i)]
      \item
        Im Fall eines schnell rotierenden Planetens ist die emmitierende
        Fläche die gesamte Oberfläche $F_\text{em} = 4 \pi R_\text{Pl}^2$. Die
        Temperatur ist dann
        % WA: ((1-0.5)*1/4*(1.025 radius of sun)^2/(0.08543 AU)^2 (6050
        % kelvin)^4)^(1/4)
        \begin{equation}
          T_\text{Pl} = \sqrt[4]{
            \del{1-A} \frac{1}{4} \frac{R_\star^2}{a^2} T_\star^4
          } = \SI{849.6}{\kelvin}
        \end{equation}

      \item
        Im Fall eines langsam rotierenden Planeten halbiert sich die
        emmitierende Fläche.
        % WA: ((1-0.5)*1/2*(1.025 radius of sun)^2/(0.08543 AU)^2 (6050
        % kelvin)^4)^(1/4)
        \begin{equation}
          T_\text{Pl} = \sqrt[4]{
            \del{1-A} \frac{1}{2} \frac{R_\star^2}{a^2} T_\star^4
          } = \SI{1010}{\kelvin}
        \end{equation}

      %TODO: Zeug mit innerer Energiequelle!
    \end{enumerate}
  \item
    Da die direkte Beobachtung nicht möglich ist, werden zur Suche von
    Exoplaneten indirekte Entdeckungsmethoden angewandt. Zwei erfolgreiche
    Methoden nutzen dabei aus, dass sich ein System aus Stern und Planet um den
    gemeinsamen Schwerpunkt bewegt. Bei der Radialgeschwindigkeitsmethode misst
    man die Rot- und Blauverschiebung des Sternenspektrums durch die Bewegung
    vom Beobachter weg und auf den Beobachter zu (Dopplereffekt), um die
    Geschwindigkeitskomponente in Radialrichtung zu bestimmen. Bei der
    astrometrischen Methode misst man durch direkte Beobachtung über einen
    Zeitraum die Rotation des Sterns um den Schwerpunkt.

    Je massereicher der Stern ist, desto ausgeprägter sind die Effekte. Je
    kürzer seine Umlaufzeit ist, umso kleiner ist die Periode der Effekte.
    Beides erhöht die Erkennungswahrscheinlichkeit, so dass man bevorzugt
    Planeten mit größer Masse und geringen Umlaufzeiten entdeckt.
\end{enumerate}

\section*{Aufgabe 2: Gezeitenkräfte}
\begin{enumerate}[a)]
  \item
    Die Gezeitenbeschleunigung an der Oberfläche ist in erster Näherung
    \begin{equation}
      a_g = \mp 2 R \frac{GM}{r^3}
    \end{equation}
    dabei ist $R$ der Erdradius, $M$ die Masse des gezeitenverursachenden
    Himmelskörpers (Mond, Sonne) und $r$ sein Abstand.

    Für den Mond ist $a_g \approx \mp \SI{1.1e-6}{m/s^2}$. Für die Sonne $a_g
    \approx \mp \SI{5.05e-7}{m/s^2}$. Die Gezeitenkraft der Sonne ist als
    $\num{0.46}$-mal so stark wie die des Mondes.

    % "Welche Konsequenz hat dies für die Bewohner der Erde?". Bescheuerte
    % Frage...
    Die Konsequenz für die Bewohner der Erde -- zumindest für diejenige, die
    rechnen können -- ist, dass auch die Gezeitenwirkung der Sonne in die
    Berechnung der Gezeiten einbezogen werden muss.

  \item
    Die Roche-Grenze ist
    \begin{equation}
      d \approx 2{,}423 R \sqrt[3]{\frac{\rho_M}{\rho_m}} \approx \SI{10.8e6}{m}
    \end{equation}
  \item
    Die Roche-Grenze gibt an, ab welchem Abstand ein Körper ohne innere Kräfte
    von der Gezeitenkraft ausseinandergerissen wird. Phobos liegt innerhalb des
    Grenze und könnte daher zerfallen, so dass Mars ein Ringsystem ausbildet.
  \item
    Künstliche Satelliten bestehen aus Festkörpern, die nicht durch die
    Gravitation, sondern durch elektromagnetische Kräfte gebunden sind. Die
    Gezeitenkräfte sind zu schwach um sie zu zerreißen.
\end{enumerate}

\section*{Aufgabe 3: Mondbedeckungen von Jupiter}
\begin{enumerate}[a)]
  \item
    In erster Näherung (Kreisförmige Bahnen, Jupiter im Perigäum) ist der
    Abstand $\SI{4.2}{AU}$. Zusammen mit dem Jupiterradius ergibt sich der
    scheinbare Durchmesser
    % 2*arctan((2*radius of jupiter)/(4.2 au))
    \begin{equation}
      \delta_J = 2 \arctan \frac{R_\text{J}}{r} \approx 45''
    \end{equation}

    Stellarium gibt hingegen den genaueren Wert $\delta_J \approx 43''$ an,
    mit dem wir weiterrechnen (insbesondere weil $\SI{1}{cm}$ auf dem Bild dann
    $10''$ entsprechen).
  \item
    Jupiter ist weit genug von der Sonne entfernt, um die Sonnenstrahlen als
    parallel anzunehmen. Daraus folgt, dass der scheinbare Durchmesser von Io
    gleich dem scheinbaren Durchmesser seines Kernschattens ist. Durch
    Einsetzen des Radius von Io in obige Formel erhalten wir $\delta = 1{,}1''$. Eine
    Messung mit dem Lineal ergibt einen gemessenen scheinbaren Durchmesser von
    $\delta = 1''$. Die Abweichung entsteht aus der geringen Messgenauigkeit
    und der geringen Auflösung des Bildes.
  \item
    % Zeichnen Sie ... Nö. Keine Lust.
    Nicht bearbeitet.

\end{enumerate}
\end{document}
