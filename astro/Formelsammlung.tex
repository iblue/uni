\documentclass[a4paper,german,12pt,smallheadings]{scrartcl}
\usepackage[T1]{fontenc}
\usepackage[utf8]{inputenc}
\usepackage{babel}
\usepackage{tikz}
\usetikzlibrary{calc}
\usepackage{tkz-euclide}
\usetkzobj{all}
\usepackage{pgfplots}
\usepackage{geometry}
\usepackage[fleqn]{amsmath}
\usepackage{amssymb}
\usepackage{wasysym}
\usepackage{float}
\usepackage{lscape}
\usepackage{enumerate}
\usepackage{commath} % http://tex.stackexchange.com/questions/14821/whats-the-proper-way-to-typeset-a-differential-operator
\usepackage{cancel}
\usepackage{gnuplottex}

% Number only referenced equations
\usepackage[fleqn]{mathtools}
\mathtoolsset{showonlyrefs}

%\usepackage{wrapfig}
\usepackage{siunitx}
\sisetup{locale = DE}

% New command for color underlining
\usepackage{xcolor}

\newsavebox\MBox
\newcommand\colul[2][red]{{\sbox\MBox{$#2$}%
  \rlap{\usebox\MBox}\color{#1}\rule[-1.2\dp\MBox]{\wd\MBox}{0.5pt}}}

% Stupid units for astronomy
\DeclareSIUnit\day{d}
\DeclareSIUnit\year{yr}
\DeclareSIUnit\au{AU}
% Astronomy-Style angles (hours, minutes, seconds)
\newcommand*{\ra}[2][]{{
  \def\SIUnitSymbolDegree{\textsuperscript{h}}%
  \def\SIUnitSymbolArcminute{\textsuperscript{m}}%
  \def\SIUnitSymbolArcsecond{\textsuperscript{s}}%
  \ang[#1]{#2}}%
}

\restylefloat{table}
\geometry{a4paper, top=15mm, left=10mm, right=20mm, bottom=20mm, headsep=10mm, footskip=12mm}
\linespread{1.2}
\setlength\parindent{0pt}
\DeclareMathOperator{\Tr}{Tr}
\DeclareMathOperator{\Var}{Var}

\begin{document}
\allowdisplaybreaks % Seitenumbrüche in Formeln erlauben
\begin{center}
\bfseries % Fettdruck einschalten
\sffamily % Serifenlose Schrift
\vspace{-40pt}
Einführung in die Astronomie und Astrophysik, Sommersemester 2014, Formelsammlung
\vspace{-10pt}
\end{center}
\section*{Koordinatensysteme}

\section*{Sterne und Entfernung}

Entfernungsmodul (Differenz zwischen scheinbarer ($m$) und absoluter Helligkeit
($M$)):
\begin{equation}
  m - M = -5 + 5 \log_{10}(r \text{ [pc]})
\end{equation}

Entfernung $r$ bei trigonometrischer Parallaxe $p$:
\begin{equation}
  r\text{ [pc]} = \frac{1}{p\text{ ['']}}
\end{equation}

Rotverschiebung $z$ wenn eine Spektrallinie $\lambda_0$ bei der Wellenlänge
$\lambda$ erscheint und sich daraus ergebende Radialgeschwindigkeit $v_r$
(Dopplereffekt):
\begin{equation}
  z = \frac{\lambda_0}{\lambda} - 1 \approx \frac{v_r}{c}
\end{equation}

Tangentialgeschwindigkeit bei beobachteter Bewegung $\mu$ am Himmel:
\begin{equation}
  v_t \text{ [km/s]} = 4{,}74 \cdot r \text{ [pc]} \cdot \mu \text{ [''/a]}
\end{equation}

Wiensches Verschiebungsgesetz: Bei Temperatur $T$ Emmissionsmaximum bei $\lambda$:
\begin{equation}
  T_\text{eff} \cdot \lambda_\text{max} = \text{const.} = 2897{,}4 \; \mathrm{\mu m\, K}
\end{equation}

Leuchtkraft eines Sterns:
\begin{equation}
  L_\star = 4 \pi R_\star^2 \sigma T^4_\text{eff}
\end{equation}

Absolute Helligkeit:
\begin{equation}
  M_\star = M_\odot - 2{,}5 \log_{10} \del{\frac{L_\star}{L_\odot}} \text{ mit } M_\odot = 4{,}74^\text{mag}
\end{equation}

Virialsatz für gravitativ gebundene Systeme:
\begin{equation}
  2 \left<E_\text{kin}\right> + \left<E_\text{pot}\right> = 0
\end{equation}

Geschwindigkeit aus Hubble-Konstante (für kosmologische Entfernungen):
\begin{equation}
  v = H_0 r
\end{equation}

Massenverhältnis zu Leuchtkraftverhältnis für Hauptreihensterne
\begin{equation}
  \frac{L_\star}{L_\odot} = \begin{cases}
    4 \log \frac{M_\star}{M_\odot} & \text{wenn} M_\star > 0{,}6 M_\odot \\
    2 \log \frac{M_\star}{M_\odot} - 0{,}4 & \text{sonst}
  \end{cases}
\end{equation}

\section*{Planetensysteme: Kepler und Zeug}
Herleitung aus Zentrifugalkraft gleich Gravitationskraft
\begin{equation}
  F_G = - G \frac{Mm}{r^2} = m \omega^2 r = F_Z
\end{equation}
1. Keplersches Gesetz (Bahnen sind Ellipsen in deren Brennpunkt die Sonne steht). $a$: Große Halbachse, $\epsilon$: Exzentrizität.
\begin{equation}
  r(\phi) = \frac{a(1-\epsilon)}{1 + \epsilon \cos \phi}
\end{equation}

2. Keplersches Gesetz: Fahrstrahl überstreicht in gleichen Zeiten gleiche
Flächen
\begin{equation}
  A(t, t + \dif t) = \frac{L}{2m} \dif t
\end{equation}

Drehimpulserhaltung bei Kepler
\begin{equation}
  L = \vec{r} \times \vec{p} = r m v \sin \alpha = r m v = \text{const}
\end{equation}

3. Keplersches Gesetz: Quadrate der Perioden verhalten sich wie Kuben der
großen Halbachsen.
\begin{equation}
  \frac{T^2}{a^3} = \frac{4 \pi^2}{G(M+m)} \approx \frac{4 \pi^2}{GM}
\end{equation}

Masse eines Exoplaneten aus RV-Kurve ($i$: Inklination, $\kappa$: Amplitude)
\begin{equation}
  M_\text{Pl} \sin(i) = \frac{\kappa M_\star}{\sqrt[3]{\frac{2 \pi G M_\star}{T}}}
\end{equation}



\end{document}
