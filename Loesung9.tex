\documentclass[a4paper,german,12pt,smallheadings]{scrartcl}
\usepackage[T1]{fontenc}
\usepackage[utf8]{inputenc}
\usepackage{babel}
\usepackage{tikz}
\usepackage{geometry}
\usepackage{amsmath}
\usepackage{amssymb}
\usepackage{float}
\usepackage{wrapfig}
\usepackage[thinspace,thinqspace,squaren,textstyle]{SIunits}
\restylefloat{table}
\geometry{a4paper, top=15mm, left=20mm, right=40mm, bottom=20mm, headsep=10mm, footskip=12mm}
\linespread{1.5}
\setlength\parindent{0pt}
\begin{document}
\begin{center}
\bfseries % Fettdruck einschalten
\sffamily % Serifenlose Schrift
\vspace{-40pt}
Mathematik für Physiker I, Wintersemester 2012/2013, 9. Übungsblatt

Florian Neumeyer und Markus Fenske, Tutor: Stephan Schwartz
\vspace{-10pt}
\end{center}


\section*{Aufgabe 9.1}
\subsection*{Teil a}

Sei $v \in V$, dann gilt aufgrund der Idempotenz des Endomorphismus $P$
\begin{align*}
  P^2(v) &= P(v) \\
  P^2(v) - P(v) &= 0
\end{align*}

Da $P$ ein Vektorraum-Endomorphismus, damit ein Vektorraum-Holomorphismus und
damit eine lineare Abbildung ist, folgt aus muss aus der Homogenität linearer
Abbildungen ($P(x) + P(y) = P(x+y)$ mit $x{,}y \in V$) folgen, dass

\begin{align*}
  P(P(v) - v) &= 0
\end{align*}

Der Kern von $P$ ist definiert als $\ker P := \{v \in V: P(v) = 0\}$, somit
muss $P(v) - v \in \ker P$ sein.

Das Bild von $P$ ist per definitionem $\operatorname{Im} P := \{v \in V: P(w) = v, w \in V\}$.
Sei also $u \in V$, dann lässt es sich darstellen durch

\begin{align*}
  u &= u \\
  u &= u - P(u) + P(u) \\
  u &= x + y, x \in \ker P, y \in \operatorname{Im} P
\end{align*}

Womit bewiesen wäre, dass $V = \ker P + \operatorname{Im} P$. Allerdings ist
dies nur der Beweis für die Summe, nicht für die direkte Summe. Für die direkte
Summe ist zu zeigen, dass sich jedes Element auf genau eine Weise aus $\ker P$
und $\operatorname{Im} P$ schreiben lässt, was äquivalent ist zu $\ker P \cap
\operatorname{Im} P = \{0\}$ (siehe Skript, 6.8.6).

Gegegeben sei $x \in \ker P$ und gleichzeitig $x \in \operatorname{Im} P$. Zu
zeigen ist, dass daraus $x = 0$ folgt.

Wegen $x \in \operatorname{Im} P$, ist $x$ darstellbar ist durch $x =
P(z), z \in V$.

\begin{align*}
  x &= P(z)
\end{align*}

Da $P$ eine lineare Abbildung ist, müssen bei gleichen Werten auch gleiche
Bilder herrauskommen. Somit

\begin{align*}
  P(x) &= P(P(z))
\end{align*}

Da $P$ idempotent ($P^2 = P$) ist:

\begin{align*}
  P(x) &= P(z)
\end{align*}

Aus $x \in \ker P$ folgt, dass $P(x) = 0$:
\begin{align*}
  0 &= P(z)
\end{align*}

Am Anfang war allerdings bereits gegeben, dass $x = P(z)$:
\begin{align*}
  0 &= x
\end{align*}

Somit $\ker P \cap \operatorname{Im} P = \{0\}$, und zusammen mit $V = \ker P +
\operatorname{Im} P$ ergibt das:

\begin{equation}
  V = \ker P \oplus \operatorname{Im} P
\end{equation}

Und jetzt darf ich so ein Kästchen machen, oder? $\Box$

\subsection*{Teil b}

Es ist gegeben, dass $P$ nicht alles in den Nullpunkt schickendarf ($O$), noch
untätig sein darf ($I$). Eine Projektion auf eine Achse ist eine Möglichkeit.

\begin{equation} 
  P\left(\begin{pmatrix} \alpha \\ \beta \end{pmatrix}\right) =
    \begin{pmatrix}\alpha \\ 0\end{pmatrix}
\end{equation}

\section*{Aufgabe 9.2}
Die Aufgabe wäre langweilig, wenn wir von der Differentiation nach etwas
anderem als $t$ ausgingen. Nehmen wir uns also die Basisvektoren von $A$
(\textbf{Rausfinden, wie man solche fancy Buchstaben wir in der
Aufgabenstellung macht!}) und wenden $D$ an.

\textbf{Achtung: Textsatz ist hier kaputt, Ausrichtung scheiße}
\begin{align*}
  D(a_1) &= D(\sin t)  = \frac{d}{dt} \sin  t = \cos t    &= a_2 \\
  D(a_2) &= D(\cos t)  = \frac{d}{dt} \cos  t = -\sin t   &= -a_1 \\
  D(a_3) &= D(\sin 2t) = \frac{d}{dt} \sin 2t = 2\cos 2t  &= 2a_4 \\
  D(a_4) &= D(\cos 2t) = \frac{d}{dt} \cos 2t = -2\sin 2t &= -2a_3
\end{align*}

Wenn man dies in der Basis $A$ (\textbf{Hier auch fancy Buchstabe!}) schreibt,
dann ist das

\begin{align*}
  D_A\left(\begin{pmatrix} 1 \\ 0 \\ 0 \\ 0\end{pmatrix}\right) &= \begin{pmatrix} 0 \\ 1 \\ 0 \\ 0\end{pmatrix} \\
  D_A\left(\begin{pmatrix} 0 \\ 1 \\ 0 \\ 0\end{pmatrix}\right) &= \begin{pmatrix} -1 \\ 0 \\ 0 \\ 0\end{pmatrix} \\
  D_A\left(\begin{pmatrix} 0 \\ 0 \\ 1 \\ 0\end{pmatrix}\right) &= \begin{pmatrix} 0 \\ 0 \\ 0 \\ 2\end{pmatrix} \\
  D_A\left(\begin{pmatrix} 0 \\ 0 \\ 0 \\ 1\end{pmatrix}\right) &= \begin{pmatrix} 0 \\ 0 \\ -2 \\ 0\end{pmatrix} \\
\end{align*}

Da die darstellende Matrix blabla (\textbf{Hier den superwichtigen Satz aus dem Tutorium eintragen})

\begin{equation}
  M_A(D) = \begin{pmatrix}0 & 0 & -1 & 0 \\ 0 & 0 & 0 & 1 \\ 0 & 0 & 0 & 2 \\ 0 & 0 & -2 & 0\end{pmatrix}
\end{equation}

Jetzt nur noch darstellende Matrix aufstellen und fertig. Selbe Aktion für $D^2$. $D$ ist invertierbar, weil Matrix invertierbar ist. Invertierte Matrix ausrechnen.

\section*{Aufgabe 9.4}

Die Spaltenvektoren der darstellenden Matrix sind die Basisvektoren in der Basis (\textbf{oder so? Nochmal in den Notizen nachsehen!}).

\end{document}
