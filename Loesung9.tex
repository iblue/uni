\documentclass[a4paper,german,12pt,smallheadings]{scrartcl}
\usepackage[T1]{fontenc}
\usepackage[utf8]{inputenc}
\usepackage{babel}
\usepackage{tikz}
\usepackage{geometry}
\usepackage{amsmath}
\usepackage{amssymb}
\usepackage{float}
\usepackage{wrapfig}
\usepackage[thinspace,thinqspace,squaren,textstyle]{SIunits}
\restylefloat{table}
\geometry{a4paper, top=15mm, left=20mm, right=40mm, bottom=20mm, headsep=10mm, footskip=12mm}
\linespread{1.5}
\setlength\parindent{0pt}
\begin{document}
\begin{center}
\bfseries % Fettdruck einschalten
\sffamily % Serifenlose Schrift
\vspace{-40pt}
Mathematik für Physiker I, Wintersemester 2012/2013, 9. Übungsblatt

Florian Neumeyer und Markus Fenske, Tutor: Stephan Schwartz
\vspace{-10pt}
\end{center}


\section*{Aufgabe 9.1}
\subsection*{Teil a}

Sei $v \in V$, dann gilt aufgrund der Idempotenz des Endomorphismus $P$
\begin{align*}
  P^2(v) &= P(v) \\
  P^2(v) - P(v) &= 0
\end{align*}

Da $P$ ein Vektorraum-Endomorphismus, damit ein Vektorraum-Holomorphismus und
damit eine lineare Abbildung ist, folgt aus muss aus der Homogenität linearer
Abbildungen ($P(x) + P(y) = P(x+y)$ mit $x{,}y \in V$) folgen, dass

\begin{align*}
  P(P(v) - v) &= 0
\end{align*}

Der Kern von $P$ ist definiert als $\ker P := \{v \in V: P(v) = 0\}$, somit
muss $P(v) - v \in \ker P$ sein.

Das Bild von $P$ ist per definitionem $\operatorname{Im} P := \{v \in V: P(w) = v, w \in V\}$.
Sei also $u \in V$, dann lässt es sich darstellen durch

\begin{align*}
  u &= u \\
  u &= u - P(u) + P(u) \\
  u &= x + y, x \in \ker P, y \in \operatorname{Im} P
\end{align*}

Womit bewiesen wäre, dass $V = \ker P + \operatorname{Im} P$. Allerdings ist
dies nur der Beweis für die Summe, nicht für die direkte Summe. Für die direkte
Summe ist zu zeigen, dass sich jedes Element auf genau eine Weise aus $\ker P$
und $\operatorname{Im} P$ schreiben lässt, was äquivalent ist zu $\ker P \cap
\operatorname{Im} P = \{0\}$ (siehe Skript, 6.8.6).

Gegegeben sei $x \in \ker P$ und gleichzeitig $x \in \operatorname{Im} P$. Zu
zeigen ist, dass daraus $x = 0$ folgt.

Wegen $x \in \operatorname{Im} P$, ist $x$ darstellbar ist durch $x =
P(z), z \in V$.

\begin{align*}
  x &= P(z)
\end{align*}

Da $P$ eine lineare Abbildung ist, müssen bei gleichen Werten auch gleiche
Bilder herrauskommen. Somit

\begin{align*}
  P(x) &= P(P(z))
\end{align*}

Da $P$ idempotent ($P^2 = P$) ist:

\begin{align*}
  P(x) &= P(z)
\end{align*}

Aus $x \in \ker P$ folgt, dass $P(x) = 0$:
\begin{align*}
  0 &= P(z)
\end{align*}

Am Anfang war allerdings bereits gegeben, dass $x = P(z)$:
\begin{align*}
  0 &= x
\end{align*}

Somit $\ker P \cap \operatorname{Im} P = \{0\}$, und zusammen mit $V = \ker P +
\operatorname{Im} P$ ergibt das:

\begin{equation}
  V = \ker P \oplus \operatorname{Im} P
\end{equation}

Und jetzt darf ich so ein Kästchen machen, oder? $\Box$

\subsection*{Teil b}

Es ist gegeben, dass $P$ nicht alles in den Nullpunkt schicken darf ($O$), noch
untätig sein darf ($I$). Eine Projektion auf eine Achse ist eine Möglichkeit.

\begin{equation} 
  P\left(\begin{pmatrix} \alpha \\ \beta \end{pmatrix}\right) =
    \begin{pmatrix}\alpha \\ 0\end{pmatrix}
\end{equation}

\section*{Aufgabe 9.2}
Die Aufgabe wäre langweilig, wenn wir von der Differentiation nach etwas
anderem als $t$ ausgingen. Nehmen wir uns also die Basisvektoren von
$a_1, \dotsb, a_4 \in \mathcal{A}$ und wenden $D$ an.

\begin{alignat*}{5}
  D(a_1) &= D(\sin t)  &&= \frac{d}{dt} \sin  t &&= \cos t    &&= a_2 \\
  D(a_2) &= D(\cos t)  &&= \frac{d}{dt} \cos  t &&= -\sin t   &&= -a_1 \\
  D(a_3) &= D(\sin 2t) &&= \frac{d}{dt} \sin 2t &&= 2\cos 2t  &&= 2a_4 \\
  D(a_4) &= D(\cos 2t) &&= \frac{d}{dt} \cos 2t &&= -2\sin 2t &&= -2a_3
\end{alignat*}

Wenn man die Basisvektoren direkt in der Basis $\mathcal{A}$ schreibt, dann
bedeutet das, dass sich für die Bilder der Basisvektoren jeweils ergibt:

\begin{align*}
  D_A\left(\begin{pmatrix} 1 \\ 0 \\ 0 \\ 0\end{pmatrix}\right) &= \begin{pmatrix} 0 \\ 1 \\ 0 \\ 0\end{pmatrix}, &
  D_A\left(\begin{pmatrix} 0 \\ 1 \\ 0 \\ 0\end{pmatrix}\right) &= \begin{pmatrix} -1 \\ 0 \\ 0 \\ 0\end{pmatrix} \\
  D_A\left(\begin{pmatrix} 0 \\ 0 \\ 1 \\ 0\end{pmatrix}\right) &= \begin{pmatrix} 0 \\ 0 \\ 0 \\ 2\end{pmatrix}, &
  D_A\left(\begin{pmatrix} 0 \\ 0 \\ 0 \\ 1\end{pmatrix}\right) &= \begin{pmatrix} 0 \\ 0 \\ -2 \\ 0\end{pmatrix} \\
\end{align*}

In den Spalten der darstellenden Matrix stehen die Bilder der Basisvektoren
ausgedrückt in der Basis. Da dort oben die Basisvektoren und ihre Bilder
stehen, ergibt sich auch schon direkt die darstellende Matrix.

\begin{equation}
  M_{\mathcal{A}}(D) = \begin{pmatrix}0 & -1 & 0 & 0 \\ 1 & 0 & 0 & 0 \\ 0 & 0 & 0 & -2 \\ 0 & 0 & 2 & 0\end{pmatrix}
\end{equation}

$M_{\mathcal{A}}(D^2)$ ist dann die darstellende Matrix für zweimaliges
Anwenden des Differentiationsoperators. Dorthin gibt es zwei Wege, entweder
zweimaliges Ableiten oder die Multiplikation der Matrix mit sich selbst.
Letzteres ergibt (\textbf{TODO: muss ich die Rechnung abtippen?}):

\begin{equation}
  M_{\mathcal{A}}(D^2) = \begin{pmatrix}-1 & 0 & 0 & 0 \\ 0 & -1 & 0 & 0 \\ 0 & 0 & -4 & 0 \\ 0 & 0 & 0 & -4\end{pmatrix}
\end{equation}

Was mit dem Ableitungen übereinstimmen würde, würde man sie ausrechnen. Denn es
ergeben sich jeweils die selben trignomometrischen Funktionen, allerdings mit
einem Minuszeichen davor bzw\. dem Vorfaktor $2 \cdot 2$ für die Terme mit der
Frequenz $2t$.

An $M_{\mathcal{A}}(D)$ sieht man, dass $D$ invertierbar ist, denn $\det
M_{\mathcal{A}}(D) \neq 0$. Da die Determinante sich durch Vertauschen der
Zeilen nicht ändert, lässt sich die Matrix in Diagonalform bringen. Durch
Multiplikation der Diagonalelemente ergibt sich dann die Determinante, die
ungleich Null ist. $M_{\mathcal{A}}(D^{-1}) = M_{\mathcal{A}}(D)^{-1}$  Das
Inverse der Matrix (\textbf{TODO: sind die Rechenschritte erforderlich?}) ist:

\begin{equation}
  M_{\mathcal{A}}(D^{-1}) = \begin{pmatrix}0 & -1 & 0 & 0 \\ 1 & 0 & 0 & 0 \\ 0 & 0 & 0 & -\frac{1}{2} \\ 0 & 0 & \frac{1}{2} & 0\end{pmatrix}
\end{equation}

\section*{Aufgabe 9.3}
\subsection*{Teil a}
Wenn man $z = a+bi$ setzt, ist die Abbildung:

\begin{align*}
  z &\mapsto w_1z + w_2\overline{z} \\
  z &\mapsto (\alpha_1 + \beta_1i)(a+bi) + (\alpha_2+\beta_2i)(a-bi) \\
  z &\mapsto (\alpha_1a-\beta_1b) + (\alpha_1b + \beta_1a)i + (\alpha_2a+\beta_2b)+(-\alpha_2b+\beta_2a)i
\end{align*}

Durch Auftrennen der Real- und Imaginärteile erhält man die Vektorform

\begin{align*}
  \begin{pmatrix} a \\ b \end{pmatrix} &\mapsto \begin{pmatrix} (\alpha_1a-\beta_1b) + (\alpha_2a+\beta_2b) \\ (-\alpha_2b+\beta_2a)i + (\alpha_1b + \beta_1a)i \end{pmatrix} \\
  \begin{pmatrix} a \\ b \end{pmatrix} &\mapsto \begin{pmatrix} (\alpha_1+\alpha_2)a + (\beta_2 - \beta_1)b \\ (\beta_1+\beta_2)a + (\alpha_1-\alpha_2)b \end{pmatrix}
\end{align*}

Damit kann man die darstellende Matrix eigentlich schon direkt ablesen, denn in
den Spalten der darstellenden Matrix stehen die Bilder der Basisvektoren
ausgedrückt in der Basis. Explizit:

\begin{align*}
  \begin{pmatrix} 1 \\ 0 \end{pmatrix} &\mapsto \begin{pmatrix} (\alpha_1+\alpha_2) \\ (\beta_1+\beta_2) \end{pmatrix} \\
  \begin{pmatrix} 0 \\ 1 \end{pmatrix} &\mapsto \begin{pmatrix} (\beta_2 - \beta_1) \\ (\alpha_1-\alpha_2) \end{pmatrix}
\end{align*}

Somit ist die darstellende Matrix in der kanonischen Basis:

\begin{equation}
  M_{\mathcal{B}}^{\mathcal{A}} = \begin{pmatrix} \alpha_1+\alpha_2 & \beta_2 - \beta_1 \\ \beta_1+\beta_2 & \alpha_1-\alpha_2 \end{pmatrix}
\end{equation}
\subsection*{Teil b}

Jeder Vektorraum-Endomorphismus ist durch eine darstellende Matrix darstellbar.
Solange die Komponenten der darstellenden Matrix beliebig gewählt werden
können, kann auch jeder Endomorphismus abgebildet werden. Wir zeigen also, dass
sich die obige darstellende Matrix mit beliebigen Komponenten füllen lässt.

\begin{align*}
  \begin{pmatrix} \alpha_1+\alpha_2 & \beta_2 - \beta_1 \\ \beta_1+\beta_2 & \alpha_1-\alpha_2 \end{pmatrix} = \begin{pmatrix} a & b \\ c & d\end{pmatrix}
\end{align*}

Wenn diese Gleichung eindeutig lösbar ist, lässt sich jeder Endomorphismus
darstellen.

\begin{align*}
  a &= \alpha_1 + \alpha_2 \\
  b &= \beta_2 - \beta_1 \\
  c &= \beta_1+\beta_2 \\
  d &= \alpha_1 - \alpha_2
\end{align*}

Durch Addition der zweiten und dritten Zeile erhält man $b+c = 2\beta_2$, somit
$\beta_2 = (b+c)/2$. Durch Einsetzen in die dritten Zeile $\beta_1 = c -
(b+c)/2$. Durch Addition der ersten und vierten Zeile analog $a+d = 2\alpha_1$,
woraus $\alpha_1 = (a+d)/2$ folgt. Und durch Einsetzen in die erste Zeile
$\alpha_2 = a - (a+d)/2$.

Zusammengefasst:

\begin{align*}
  \alpha_1 &= \frac{a+d}{2} \\
  \alpha_2 &= a - \frac{a+d}{2} \\
  \beta_1  &= c - \frac{b+c}{2} \\
  \beta_2  &= \frac{b+c}{2}
\end{align*}

Das bedeutet, dass jede Darstellungsmatrix eines Endomorphismus im gegebenen
Vektorraum durch die darstellende Matrix der gegebenen Abbildung darstellbar
ist, damit jeder Endomorphismus durch $z \mapsto w_1z + w_2\overline{z}$, wenn
man $w_1$ und $w_2$ entsprechend wählt.

\section*{Aufgabe 9.4}
$M^{\mathcal{A}}_{\mathcal{B}}(F)$ besteht aus drei Teilen.
\begin{enumerate}
  \item Transformation des Wertes von $\mathcal{A}$ in die kanonische Basis. Die darstellende Matrix dieser Transformation sei $T^{\mathcal{A}}_E$.
  \item Anwenden der linearen Abbildung, die darstellende Matrix ist $A$, wir nennen sie $A^E_E$.
  \item Transformation des Bildes von der kanonischen Basis nach $\mathcal{B}$. Die darstellende Matrix wollen wir $T^{E}_{\mathcal{B}}$ nennen.
\end{enumerate}

Durch die Verkettung der drei Abbildungen erhalten wir die
geforderte Abbildung. Die darstellende Matrix ist dann die Multiplikation der drei darstellenden Matrizen.

\begin{equation}
  M^{\mathcal{A}}_{\mathcal{B}}(F) = F T^{\mathcal{A}}_E A^E_E T^{E}_{\mathcal{B}}
\end{equation}


Die erste Matrix $T^{\mathcal{A}}_E$ lässt sich berechnen. Denn das Inverse der
darstellenden Matrix ist die darstellende Matrix der Rücktransformation
(\textbf{TODO: Inverses ausrechnen!}).

\begin{equation}
  T^{E}_{\mathcal{B}} = (T^{E}_{\mathcal{B}})^{-1} = \begin{pmatrix} \dotsb \end{pmatrix}
\end{equation}

Die zweite Matrix $A^E_E$ ist gegeben.

Die letzte Matrix $T^{E}_{\mathcal{B}}$ kann man ablesen, denn in
den Spalten der darstellenden Matrix stehen die Bilder der Basisvektoren
ausgedrückt in der Basis.

\begin{equation}
  T^{E}_{\mathcal{B}} = \begin{pmatrix} 1 & -2 \\ 2 & 0 \end{pmatrix}
\end{equation}

Da wir nun alle Matrizen zusammen haben, können wir
$M^{\mathcal{A}}_{\mathcal{B}}$ berechnen (\textbf{TODO: Berechnen!}).

\begin{equation}
  M^{\mathcal{A}}_{\mathcal{B}} =
\begin{pmatrix}
  \dotsb
\end{pmatrix}
\begin{pmatrix}
  1 &  2 & 3 & 4 \\
  0 & -1 & 2 & 1
\end{pmatrix}
\begin{pmatrix}
  1 & -2 \\
  2 & 0
\end{pmatrix}
=
\begin{pmatrix}
  \dotsb?\dotsb
\end{pmatrix}
\end{equation}

\end{document}
