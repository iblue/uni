\documentclass[a4paper,german,12pt,smallheadings]{scrartcl}
\usepackage[T1]{fontenc}
\usepackage[utf8]{inputenc}
\usepackage{babel}
\usepackage{tikz}
\usepackage{geometry}
\usepackage{amsmath}
\usepackage{amssymb}
\usepackage{float}
\usepackage{wrapfig}
\usepackage[thinspace,thinqspace,squaren,textstyle]{SIunits}
\restylefloat{table}
\geometry{a4paper, top=15mm, left=20mm, right=40mm, bottom=20mm, headsep=10mm, footskip=12mm}
\linespread{1.5}
\setlength\parindent{0pt}
\begin{document}
\begin{center}
\bfseries % Fettdruck einschalten
\sffamily % Serifenlose Schrift
\vspace{-40pt}
Experiementalphysik I, Wintersemester 2012/2013, 10. Übungsblatt

Normen Peulecke und Markus Fenske, Tutor: Alex Krüger
%Markus Fenske, Tutor: Alex Krüger
\vspace{-10pt}
\end{center}


\section*{38. Innere Energie}

Die durchschnittliche kinetische Energie eines Teilchens ist gegeben durch
$\epsilon = \frac{3}{2} kT$. Die Anzahl der Teilchen durch $N = \frac{PV}{kT}$.
Damit ergibt sich für die Gesamtenergie:

\begin{align*}
  E &= N \cdot \epsilon \\
    &= \frac{PV}{kT} \frac{3}{2} kT \\
    &= \frac{3}{2} PV
\end{align*}

Der Druck ist über die Zeit konstant, dann durch das Schlüsselloch findet ein
Druckausgleich statt. Das Volumen des Zimmers ändert sich ebenfalls nicht. Die
innere Energie der Luft, die im Zimmer verbleibt, ist konstant. Ich bin mir
jedoch nicht sicher, ob das die Frage war.

Sollte sich die Frage auf die Energie beziehen, die die Luft hat, die vor der
Erwärmung im Zimmer war, dann lässt sich diese Energie über die Volumenänderung
berechnen.

\begin{align*}
  \frac{PV_2}{NkT_2} &= \frac{PV_1}{NkT_1} \\
  \frac{V_2}{T_2} &= \frac{V_1}{T_1} \\
  V_2 &= V_1 \frac{T_2}{T_1}
\end{align*}

Somit erhöht sich die Energie um

\begin{align*}
  \delta E &= \frac{3}{2} (PV_2 - PV_1) \\
  \delta E &= \frac{3}{2} \left(PV_1 \frac{T_2}{T_1} - PV_1\right) \\
  \delta E &= \frac{3}{2} PV_1 \left(\frac{T_2}{T_1} - 1\right)
\end{align*}

Mit $P = 101300 \pascal$, $V_1 = 50 \cubic\meter$, $T_1 = 278 \kelvin$, $T_2 =
295 \kelvin$ ergibt sich eine Erhöhung um $\approx 9292 \joule$.

\end{document}
