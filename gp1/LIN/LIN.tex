\documentclass[a4paper,german,12pt,smallheadings]{scrartcl}
\usepackage[T1]{fontenc}
\usepackage[utf8]{inputenc}
\usepackage{babel}
\usepackage{geometry}
\usepackage[fleqn]{amsmath}
\usepackage{amssymb}
\usepackage{float}
\usepackage{enumerate}
\usepackage{listings} % Source code
\usepackage{lscape} % landscape
\usepackage{commath} % http://tex.stackexchange.com/questions/14821/whats-the-proper-way-to-typeset-a-differential-operator
\usepackage{cancel}

\usepackage[fleqn]{mathtools}
% Number only referenced equations
%\mathtoolsset{showonlyrefs}

%\usepackage{wrapfig}
\usepackage[thinspace,thinqspace,squaren,textstyle]{SIunits}

% New command for color underlining
\usepackage{xcolor}

\newsavebox\MBox
\newcommand\colul[2][red]{{\sbox\MBox{$#2$}%
  \rlap{\usebox\MBox}\color{#1}\rule[-1.2\dp\MBox]{\wd\MBox}{0.5pt}}}

\restylefloat{table}
\geometry{a4paper, top=15mm, left=20mm, right=10mm, bottom=20mm, headsep=10mm, footskip=12mm}
\linespread{1.5}
\setlength\parindent{0pt}
\DeclareMathOperator{\Tr}{Tr}
\DeclareMathOperator{\Var}{Var}
\begin{document}
\allowdisplaybreaks % Seitenumbrüche in Formeln erlauben
\begin{center}
\bfseries % Fettdruck einschalten
\sffamily % Serifenlose Schrift
\vspace{-40pt}
Physikalisches Grundpraktikum 1, Sommersemester 2014

Markus Fenske, Paul Rahmann, Tutor: Jonathan Heidkamp

Lineare Bewegungen
\vspace{-10pt}
\end{center}

\section*{Physikalische Grundlagen}

Nach den Newtonschen Axiomen führt eine Kraft zur Änderung des Impulses
($\vec{F} = \dot{\vec{p}}$), während der Impuls der Masse multipliziert mit der
Geschwindigkeit ist ($\vec{p} = m \vec{v}$), und die Geschwindigkeit die
Änderung des Ortes ($\vec{v} = \dot{\vec{x}}$). Im Fall konstanter Massen erhalten wir
\begin{equation}
  \vec{F} = m \ddot{\vec{x}}
\end{equation}

Für lineare Bewegungen berücksichtigen wir Bewegungen entlang einer Achse, so
dass die Gleichung die skalare Form annimmt.

\begin{equation}
  F = m \ddot{x}
\end{equation}

Im Fall einer konstanten Kraft $F$, die um eine geschwindigkeitsabhängige
Reibung $\delta \dot{x}$ gemindert wird, erhalten wir die lineare, homogene
Differentialgleichung 2. Ordnung.

\begin{equation}
  F - \delta \dot{x} = m \ddot{x}
\end{equation}

Diese lösen wir nun. Durch Substitution $v = \dot{x}$ erhalten wir eine
lineare, homogene Differentialgleichung 1. Ordnung.

\begin{equation}
  F - \delta v = m \dot{v}
\end{equation}

Diese Gleichung ist durch Separation der Variablen zu lösen

\begin{align}
  &F - \delta v = m \frac{\dif v}{\dif t} \\
  \Leftrightarrow\quad
  &\int \dif t \; \frac{1}{m} = \int \dif v \; \frac{1}{F - \delta v} \\
\end{align}

Wir substituieren $u(v) = F - \delta v$, somit $\dif u = - \delta \dif v
\Leftrightarrow \dif v = -\frac{1}{\delta} \dif u$.

\begin{align}
  \Leftrightarrow\quad
  &\frac{t}{m} + C_1 = - \frac{1}{\delta} \int \dif u \; \frac{1}{u} \\
  \Leftrightarrow\quad
  &\frac{t}{m} + C_1 = - \frac{1}{\delta} \ln(u) + C_2 \qquad \text{(Hier $C = C_1 - C_2$  und Resubstitution)} \\
  \Leftrightarrow\quad
  &\frac{t}{m} + C = - \frac{1}{\delta} \ln(F - \delta v) \\
  \Leftrightarrow\quad
  &-\frac{\delta}{m} t -\delta C = \ln(F - \delta v) \\
  \Leftrightarrow\quad
  &F - \delta v = \exp \del{-\frac{\delta}{m} t - \delta C} \\
  \Leftrightarrow\quad
  &- \delta v = \exp \del{-\frac{\delta}{m} t - \delta C} - F \\
  \Leftrightarrow\quad
  &v = -\frac{1}{\delta} \exp \del{-\frac{\delta}{m} t - \delta C} + \frac{F}{\delta} \\
\end{align}

Durch die Anfangsbedingung $v(0) = 0$ erhalten wir
\begin{equation}
  0 = -\frac{1}{\delta} \exp \del{- \delta C} + \frac{F}{\delta}
\end{equation}

So dass wir $\exp\del{-\delta C} = F$ sehen. Somit
\begin{align}
  &v = -\frac{F}{\delta} \exp \del{-\frac{\delta}{m} t} + \frac{F}{\delta} \\
  \Leftrightarrow\quad
  &v(t) = \frac{F}{\delta} \del{1 - \exp{\frac{\delta}{m} t}}
\end{align}

Im Fall $t \to \infty$ strebt $v(t)$ natürlich gegen eine Grenzgeschwindigkeit
$\frac{F}{\delta}$.  Das heißt, dass ein Körper, der mit einer konstanten Kraft
$F$ beschleunigt wird, unter Einfluss einer geschwindigkeitsabhängigen Reibung
seine Geschwindigkeit einer Konstante annähert. Je größer dabei die
Reibungskonstante $\delta$ ist, desto geringer ist diese Grenzgeschwindigkeit.
Je größer die Kraft $F$ ist, desto höher ist die Grenzgeschwindigkeit. Ist die
Kraft $F = 0$, bleibt die Geschwindigkeit des Körpers $0$. Dies alles stimmt
mit der Alltagserfahrung überein.

\section*{Aufgaben}
\begin{enumerate}[1.]
  \item
    Justierung der Fahrbahn und des Messsystems.
  \item
    Untersuchung von Bewegungen bei konstanter Kraft: Aufnahme von
    Weg-/Geschwindigkeits-/Beschleunigungs-Zeit-Messungen für verschiedene
    Kombinationen von Masse (Gleiter) und Zugkraft (Anhängegewicht) und
    Überprüfung des Bewegungsgesetzes in Abhängigkeit dieser Parameter.
  \item
    Untersuchung von Bewegungen unter Berücksichtigung einer
    geschwindigkeitsproportionalen Dämpfungskraft (magnetische
    Wirbelstromdämpfung). Berechnung der Dämpfungskonstanten aus der
    Zeitkonstanten der Bewegung und der Geschwindigkeit.
\end{enumerate}

% Auswertung hier
\newpage

\section*{Auswertung und Diskussion: Aufgabe 2}
Für diese Aufgabe wurden die Messungen 1 bis 6 aufgenommen. Dabei wurden
jeweils keine Messdaten, sondern nur Plots ausgedruckt.

Für die Geschwindigkeit $v(t)$ wurden jeweils Ausgleichsgeraden eingezeichnet
um aus der Steigung die Beschleunigung $a$ zu ermitteln. Aufgrund des starken
Rauschens wurde die Grenzgerade durch dieses bestimmt, statt durch den von
CASSY-Lab angegebenen Fehler $\Delta x = 1 \milli\meter$.

Für die Auswertung nutzen wir die hergeleitete Formel. Dabei nehmen wir an,
dass das System in etwa reibungsfrei ist, also $\delta = 0$.

Die Kraft $F$ bestimmt sich dann durch die angehangenen Gewichte, den Haken und
den Faden, auf den im Verlauf immer mehr Gewichtskraft wirkt. Da dies für die
theoretische Herleitung nicht berücksichtigt wurde und unserer Meinung nicht im
Rahmen der Messgenauigkeit liegt, vernachlässigen wir den Faden für die
theoretischen Werte.

Die wirkende Kraft ist
\begin{equation}
  F = \del{m_\text{Anhängemasse} + m_\text{Haken}} g
\end{equation}

Die beschleunigte Masse ist
\begin{equation}
  m = m_\text{Gleiter+Faden} + m_\text{Gleitergewichte} + m_\text{Anhängemasse} + m_\text{Haken}
\end{equation}

Die erwartete Beschleunigung ist somit
\begin{equation}
  a = g \frac{m_\text{Anhängemasse} + m_\text{Haken}}{m_\text{Gleiter+Faden} + m_\text{Gleitergewichte} + m_\text{Anhängemasse} + m_\text{Haken}} =
  \frac{g}{1 + \dfrac{m_\text{Anhängemasse} + m_\text{Haken}}{{m_\text{Gleiter+Faden} + m_\text{Gleitergewichte}}}}
\end{equation}

Sei $m_\text{G} = m_\text{Gleiter+Faden} + m_\text{Gleitergewichte}$ und $m_\text{Z} = m_\text{Anhängemasse} + m_\text{Haken}$, dann ist
\begin{equation}
  a = \frac{g}{1 + \dfrac{m_\text{G}}{m_\text{Z}}}
\end{equation}

Per Gaußscher Fehlerfortplanzung ergibt sich dann
\begin{align}
  \Delta a &= \sqrt{
    \underbrace{\del{\frac{\partial a}{\partial g} \Delta g}^2}_{\approx 0} +
    \del{
      g \frac{m_Z}{\del{m_G + m_Z}^2} \Delta m_G
    }^2
    +
    \del{
      g \frac{m_G}{\del{m_G + m_Z}^2} \Delta m_Z
    }^2
  } \\
  &= \frac{g}{(m_G + m_Z)^2} \sqrt{\del{m_Z \Delta m_G}^2 + \del{m_G \Delta m_Z}^2}
\end{align}

Da der Fehler der Massen gleich ist ($\Delta m = 0{,}1 \gram$) und $m_G$ und
$m_Z$ aus der Addition mehrer Massen entstehten, ist der Fehler der Massen
jeweils $\Delta m_G = \sqrt{n_G} \Delta m$ und $\Delta m_Z = \sqrt{n_Z} \Delta
m$, wobei $n$ die Anzahl der Massen beschreibt, aus denen sich die $m$
zusammensetzen.

\begin{equation}
  \Delta a = \frac{\sqrt{\del{n_Z m_Z^2 + n_G m_G^2}}}{\del{m_Z + m_G}^2} \; g \Delta m
\end{equation}

Für $g$ verwenden wir den Wert, den wir im letzten Versuch (Schwerependel) dem
Platzskript entnommen haben.
\begin{equation}
  g = 9{,}81282\pm0{,}00001 \meter\second^{-2}
\end{equation}

Der relative Fehler von $g$ ist um mehrere Nachkommastellen geringer als der
relative Fehler der Massen und daher vernachlässigbar.

Damit können wir nun die theoretisch erwarteten Werte berechnen und mit den
experimentell ermittelten Werten aus der grafischen Auswertung vergleichen.

Dabei sind folgende Massen relevant. Zusatzgewichte für den Gleiter: $m_{G1} =
100{,}2$, $m_{G2} = 100{,}9$, $m_{G3} = 100{,}3$. Anhängegewichte: $m_1 =
1{,}1$, $m_2 = 1{,}8$, $m_3 = 14{,}9$. Das Grundgewicht des Gleiters ist
$91{,}4$. Für die Anhängemasse wird zusätzlich eine Büroklammer mit der Masse
$0{,}3$ berücksichtig. Alle Fehler $\Delta m = 0{,}1$, alle Angaben in Gramm.

% TODO: Nobody cares? - Wenn einkommentieren, dann Skript in den Anhang.
%Die Werte wurden der Einfachheit halber mit dem im Anhang abgedruckten
%Ruby-Script berechnet, welches die obigen Überlegungen implementiert.

\vspace{0.7cm}

\begin{tabular}{l|l|l|r|r|l}
  Msg. & Anhängegewichte & Gleitergewichte & $a$ (theo.) & $a$ (exp.) & Kommentar \\
  \hline
  1 & $m_3$      & $m_{G1}$                  & $0{,}721\pm0{,}007$ & $0{,}70\pm0{,}03$ & Werte identisch \\
  2 & $m_3$      & $m_{G1}, m_{G2}$          & $0{,}484\pm0{,}006$ & $0{,}45\pm0{,}02$ & Werte verträglich \\
  3 & $m_3$      & $m_{G1}, m_{G2}, m_{G3}$  & $0{,}366\pm0{,}005$ & $0{,}33\pm0{,}01$ & Werte verträglich \\
  4 & $m_2$      & $m_{G1}$                  & $0{,}106\pm0{,}008$ & -- & Nicht auswertbar \\
  5 & $m_2$      & $m_{G1}$                  & $0{,}106\pm0{,}008$ & $0{,}10\pm0{,}09$ & Werte identisch \\
  6 & $m_1, m_2$ & $m_{G1}$                  & $0{,}161\pm0{,}008$ & $0{,}152\pm0{,}007$ & Werte identisch
\end{tabular}

\vspace{0.7cm}

Für Messung 6 haben wir zusätzlich $x$ über $t^2$ aufgetragen. Aufgrund der
Grobheit des vom Programm ausgegebenen Rasters ist zwar keine Steigung
ablesbar, jedoch entspricht die Form näherungsweise einer Gerade.

Es ist eine leichte Abweichung erkennbar, deren Ursachen vielfältig sein
könnten. In Frage kommen:
\begin{itemize}
  \item Einflüsse durch die vernachlässigten Kräfte: Gewichtskraft des
  Zugfadens, Drehmoment des Messrades
  \item Störeinflüsse: Hochblasen des Fadens durch die Luftkissenschiene
  oder allgemeine Luftbewegungen im Raum
  \item Thermischer Verzug des Papiers beim Ausdrucken auf dem Laserdrucker.
\end{itemize}

Ohne die numerischen Messwerte können wir letzteres nicht ausschließen, können
also keine Aussage machen. Die Abweichung ist allerdings so gering, dass der
Zusammenhang $x \propto t^2$ in sehr guter Näherung gilt.

Die aufgetragenen Werte für $a$ sind stark verrauscht, konnten jedoch aus $v$
rekonstruiert werden. Ohne numerische Messdaten ist eine weitere Auswertung
nicht möglich, die Werte können als konstant angesehen werden.

Insgesamt haben wir das hergeleitete Bewegungsgesetz damit für den Fall $\delta
= 0$ im Rahmen der Messgenauigkeit bestätigt.

\section*{Aufgabe 3}

% FIXME: Plot in den Anhang!

Hier wurden Magnete an den Gleiter angebracht um eine Reibung durch
Wirbelströme zu erzielen.

Die Masse des Gleiters mit den Magneten wurde neu bestimmt, wobei die Waage
durch die Magneten beeinflusst wurde. Hält man den Gleiter über die Waage ohne
diese zu berühren, zeigt diese ein Gewicht von $-0{,}5 \gram$ bis $-0{,}9
\gram$  an. Dies haben wir mehrfach wiederholt und schätzen den benötigten
Korrekturterm auf $-0{,}7 \pm 0{,}3 \gram$.

Der beim Auflegen des Gleiters (inklusive Zugfaden) auf der Waage angezeigte
Wert ist $146{,}5\pm0{,}1 \gram$, so dass wir von einer wahren Masse von
$m_\text{GMF} = 145{,}80\pm0{,}31 \gram$ ausgehen.

% Probleme: Gleiter wackelt. Magnetische Streben in der Fahrbahn.

Es wurden zwei Messungen mit Anhängemassen durchgeführt um die
Grenzgeschwindigkeit zu bestimmen. Für Messung 7 wurde der Gleiter losgelassen,
für Messung 8 wurde er zusätzlich angestoßen, um sich der Grenzgeschwindigkeit
von oben zu nähern.

Für beide Messungen haben wir numerischen Messdaten (Position $x$ und
Geschwindigkeit $v$) ausgedruckt. Die Werte für $v$ sind dabei nicht
verwertbar, da sie vom Programm auf eine Nachkommastelle gerundet wurden.

Wir haben daher die Werte eingescannt, mittels OCR-Verfahren in Text konvertiert,
triviale Erkennungsfehler durch Reguläre Ausdrücke automatisch korrigiert und
abschließende Korrekturen manuell vorgenommen. Die Daten haben wir anschließend
mit einem Ruby-Script (siehe Anhang) numerisch differenziert. Dies führt zu
einer Genauigkeitsverbesserung um eine Nachkommastelle. Dann haben wir Werte
vom Anfang und Ende entfernt, die aufgenommen wurden, bevor der Gleiter
losgelassen wurde bzw. nachdem er am Ende aufgeprallt ist.

Vom GP1-Skript wird nun vorgeschlagen unter Ausnutzung der Beziehung
\begin{equation} v(t) - \frac{F}{\delta} = -\exp \del{- \frac{\delta}{m} t}
\end{equation} zuerst die Grenzgeschwindigkeit $v_G = \frac{F}{\delta}$ zu
raten und dann $v(t) - v_G$ halblogarithmisch gegen den Exponentialterm
aufzutragen. Dieses Verfahren soll iterativ zeichnerisch durchgeführt werden,
bis sich ein Gerade ergibt wird. Diese Vorgehensweise ist zwar möglich, hat
aber einige Nachteile
\begin{itemize}
  \item Ungenau, da $v_G$ nur geschätzt wird und dann unter Benutzung der
    Steigung geprüft wird. Der Wert wird nur geraten.
  \item Zu hoher Zeitaufwand
\end{itemize}

Stattdessen erschien es uns lohnenswerter, die Werte unter Benutzung der
Methode der kleinsten Quadrate computergestützt gegen den Graphen $v(t) = a (1
- e^{-bt})$ zu fitten. Wir benutzen dazu das Programm GNUplot (Script siehe
Anhang).

Die Ungenauigkeit der Ortsmessung wird von CASSY-Lab mit $\Delta x = 1
\milli\meter$ angegeben. Ein erster Blick auf die Werte zeigt aber, dass sich
durch die numerische Differentiation ein starkes Rauschen ergibt, so dass der
Messfehler darin untergeht. Der Algorithmus gibt Fehlerwerte für die gefitteten
Parameter an, so dass wir mit diesen Werten arbeiten. Der Plot befindet sich im
Anhang.

Wir erhalten $\frac{F}{\delta} = v_G = 0{,}3051\pm0{,}0026
\;\meter\second^{-1}$ und $b = 2{,}941\pm0{,}079$.

Die gesamte beschleunigte Masse ist $m_1 + m_2 + m_3 + m_\text{GMF} +
m_\text{Haken} = 163{,}60 \gram$. Der Fehler ist $\Delta m = \sqrt{4 \cdot
0{,}1^2 + 0{,}31^2} = 0{,}37 \gram$. Somit ist $m = 163{,}60 \pm 0{,}37 \gram$.

Die angehängte Masse ist $m_1 + m_2 + m_3 = 17{,}8 \pm 0{,}02 \;\gram$. Den
Fehler von $g$ vernachlässigen wir (siehe oben) und erhalten so $F = (m_1 + m_2
+ m_3) g = 0{,}17467\pm 0{,}00019 \;\newton$

Aus $\delta = \frac{F}{v_G}$ bestimmen wir $\delta = 0{,}5725$ mit dem Fehler
$\Delta \delta = \sqrt{\del{\frac{\Delta F}{v_G}}^2 + \del{\frac{F}{v_G^2}
\Delta v_G}^2} = 0{,}0049$. Somit $\delta =  0{,}573\pm0{,}005 \;
\kilogram\second^{-1}$.

Beim Fitten des Plots wurde nicht berücksichtigt, dass der Gleiter erst zur
Zeit $t_0 = 0{,}50 \pm 0{,}05$ (Fehler wegen Messauflösung von $50
\milli\second$) losgelassen wurde. Dieser Wert muss also noch zu $b$ addiert
werden, um den wahren Wert zu erhalten. Damit ist $b' = 3{,}44 \pm 0{,}50$. Es
ergibt sich $b' \cdot m = 0{,}56278$ mit dem Fehler $\Delta \delta = \sqrt{(b'
\Delta m)^2 + (m \Delta b')^2} = 0{,}562\pm0{,}82 \; \kilogram\second^{-1}$.
Dieser Wert ist mit dem ersten identisch, jedoch viel ungenauer. Es zeigt, dass
unsere Auswertungsmethode gültig ist und hervorragende Resultate liefert.

\section*{Diskussion}
% FIXME
Abgesehen
von dem durch die numerische Differentiation verursachten Rauschens ist dabei
keine Abweichung von einem linearen Verlauf erkennbar. Die Werte $a(t)$ könnten
möglicherweise konstant sein, aufgrund des hohen Rauschens können wir
allerdings diese Daten kaum verwerten. Es könnte sich so ziemlich jeder
beliebige Verlauf darin verbergen.

Die grafisch ermittelten Beschleunigungen $a$ können wir mit den theoretischen
Werten vergleichen. Es wird dabei die Gesamtmasse des Gleiters und der
Anhängelasten beschleunigt, bestehend aus Gleiter, Verbindungsfaden,
Anhängehaken, Zusatzmassen und Anhängemassen. Dass der Verbindungsfaden im
Verlauf der Bewegung graduell mehr als Anhängemasse wirkt vernachlässigen wir
aufgrund des geringen Gewichtes des Fadens ($0{,}6 \pm 0{,}2 \gram$). Auch das
Trägheitsmoment des Umlenkrades vernachlässigen wir.

Die Werte für Beschleunigung und Geschwindigkeit wurden Software durch
numerische Differentiation gewonnen. Dieses Verfahren hat den Nachteil eines
relativ großen Jitters, so dass die Werte für die Beschleunigung quasi
unbrauchbar sind. Die Werte für den Ort haben die Form einer Parabel, sind aber
nicht gegen $t^2$ aufgetragen und damit ebenfalls unbrauchbar.

Trotz der vergleichsweise hohen Messgenauigkeit der Aparatur gehen uns also
viele Informationen verloren. Hier wäre es Wünschenswert gewesen, die
numerischen Messwerte zu haben, was daran scheiterte, dass Druckerpapier
vergleichsweise knapp war und rationiert wurde. Da der Laborcomputer auch
leider keinen Anschluss an das Internet hatte (oder dieser nicht freigeschaltet
war), konnten wir auch keine Messdaten zur späteren Auswertung übertragen.

Insgesamt konnten wir mit den Messdaten trotzdem die Gültigkeit der Newtonschen
Bewegungsgleichungen und des damit aufgestellten theoretischen Modells
innerhalb der Messgenauigkeit zeigen, so dass der Versuch trotzdem als
erfolgreich angesehen werden kann.

\newpage
\begin{landscape}
  % Generieren mit gnuplot ./plot/messung7.gnuplot
  \input{plot-messung7.tex}
\end{landscape}

\newpage
\section*{Anhang}
\lstset{basicstyle=\footnotesize\ttfamily,breaklines=true}
\lstset{numbers=left, frame=single}

Programm \texttt{error-calc.rb} zur Berechnung der theoretischen Werte und
theoretischen Fehler der Messungen 1-6.

\lstinputlisting[language=Ruby]{error-calc.rb}

Programm \texttt{numdiff.rb} zur numerischen Differenzierung der Messwerte und
konvertierung zu GNUplot.
\lstinputlisting[language=Ruby]{numdiff.rb}

GNUplot script \texttt{plot/messung7.gnuplot} zum fitten der Daten in Messung 7.
\lstinputlisting{plot/messung7.gnuplot}



\end{document}
