\documentclass[a4paper,german,12pt,smallheadings]{scrartcl}
\usepackage[T1]{fontenc}
\usepackage[utf8]{inputenc}
\usepackage{babel}
\usepackage{geometry}
\usepackage[fleqn]{amsmath}
\usepackage{amssymb}
\usepackage{float}
\usepackage{enumerate}
\usepackage{commath} % http://tex.stackexchange.com/questions/14821/whats-the-proper-way-to-typeset-a-differential-operator
\usepackage{cancel}

\usepackage[fleqn]{mathtools}
% Number only referenced equations
%\mathtoolsset{showonlyrefs}

%\usepackage{wrapfig}
\usepackage[thinspace,thinqspace,squaren,textstyle]{SIunits}

% New command for color underlining
\usepackage{xcolor}

\newsavebox\MBox
\newcommand\colul[2][red]{{\sbox\MBox{$#2$}%
  \rlap{\usebox\MBox}\color{#1}\rule[-1.2\dp\MBox]{\wd\MBox}{0.5pt}}}

\restylefloat{table}
\geometry{a4paper, top=15mm, left=20mm, right=10mm, bottom=20mm, headsep=10mm, footskip=12mm}
\linespread{1.5}
\setlength\parindent{0pt}
\DeclareMathOperator{\Tr}{Tr}
\DeclareMathOperator{\Var}{Var}
\begin{document}
\allowdisplaybreaks % Seitenumbrüche in Formeln erlauben
\begin{center}
\bfseries % Fettdruck einschalten
\sffamily % Serifenlose Schrift
\vspace{-40pt}
Physikalisches Grundpraktikum 1, Sommersemester 2014

Markus Fenske, Paul Rahmann, Tutor: Jonathan Heidkamp

Lineare Bewegungen
\vspace{-10pt}
\end{center}

\section*{Physikalische Grundlagen}

Nach den Newtonschen Axiomen führt eine Kraft zur Änderungs des Impulses
($\vec{F} = \dot{\vec{p}}$), während der Impuls der Masse multipliziert mit der
Geschwindigkeit ist ($\vec{p} = m \vec{v}$), und die Geschwindigkeit die
Änderung des Ortes ($\vec{v} = \dot{\vec{x}}$). Im Fall konstanter Massen erhalten wir
\begin{equation}
  \vec{F} = m \ddot{\vec{x}}
\end{equation}

Für lineare Bewegungen berücksichtigen wir Bewegungen entlang einer Achse, so
dass die Gleichung die skalare Form annimmt.

\begin{equation}
  F = m \ddot{x}
\end{equation}

Im Fall einer konstanten Kraft $F$, die um eine geschwindigkeitsabhängige
Reibung $\delta \dot{x}$ gemindert wird, erhalten wir die lineare, homogene
Differentialgleichung 2. Ordnung.

\begin{equation}
  F - \delta \dot{x} = m \ddot{x}
\end{equation}

Diese lösen wir nun. Durch Substitution $v = \dot{x}$ erhalten wir eine
lineare, homegene Differentialgleichung 1. Ordnung.

\begin{equation}
  F - \delta v = m \dot{v}
\end{equation}

Diese Gleichung ist durch Separation der Variablen zu lösen

\begin{align}
  &F - \delta v = m \frac{\dif v}{\dif t} \\
  \Leftrightarrow\quad
  &\int \dif t \; \frac{1}{m} = \int \dif v \; \frac{1}{F - \delta v} \\
\end{align}

Wir substituieren $u(v) = F - \delta v$, somit $\dif u = - \delta \dif v
\Leftrightarrow \dif v = -\frac{1}{\delta} \dif u$.

\begin{align}
  \Leftrightarrow\quad
  &\frac{t}{m} + C_1 = - \frac{1}{\delta} \int \dif u \; \frac{1}{u} \\
  \Leftrightarrow\quad
  &\frac{t}{m} + C_1 = - \frac{1}{\delta} \ln(u) + C_2 \qquad \text{(Hier $C = C_1 - C_2$  und Resubstitution)} \\
  \Leftrightarrow\quad
  &\frac{t}{m} + C = - \frac{1}{\delta} \ln(F - \delta v) \\
  \Leftrightarrow\quad
  &-\frac{\delta}{m} t -\frac{C}{\delta}  = \ln(F - \delta v) \\
  \Leftrightarrow\quad
  &F - \delta v = \exp \del{-\frac{\delta}{m} t - \frac{C}{\delta}} \\
  \Leftrightarrow\quad
  &- \delta v = \exp \del{-\frac{\delta}{m} t - \frac{C}{\delta}} - F \\
  \Leftrightarrow\quad
  &v = -\frac{1}{\delta} \exp \del{-\frac{\delta}{m} t - \frac{C}{\delta}} + \frac{F}{\delta} \\
\end{align}

Durch die Anfangsbedingung $v(0) = 0$ erhalten wir
\begin{equation}
  0 = -\frac{1}{\delta} \exp \del{- \frac{C}{\delta}} + \frac{F}{\delta}
\end{equation}

So dass wir $\exp\del{-\frac{C}{\delta}} = F$ sehen. Somit
\begin{align}
  &v = -\frac{F}{\delta} \exp \del{-\frac{\delta}{m} t} + \frac{F}{\delta} \\
  \Leftrightarrow\quad
  &v(t) = \frac{F}{\delta} \del{1 - \exp{\frac{\delta}{m} t}}
\end{align}

Im Grenzfall $t \to \infty$ folgt natürlich eine konstante
Geschwindigkeit $\frac{F}{\delta}$. Das heißt, dass ein Körper, der mit einer
konstanten Kraft $F$ beschleunigt wird unter Einfluss einer
geschwindigkeitsabhängigen seine Geschwindigkeit einer Konstante annhähert. Je
stärker dabei die Reibung $\delta$ ist, desto geringer ist diese
Grenzgeschwindigkeit. Je größer die Kraft $F$ ist, desto höher ist diese
Geschwindigkeit. Ist die Kraft $F = 0$, bleibt die Geschwindkeit des Körpers
$0$.



\end{document}
