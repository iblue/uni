\documentclass[a4paper,german,12pt,smallheadings]{scrartcl}
\usepackage[T1]{fontenc}
\usepackage[utf8]{inputenc}
\usepackage{babel}
\usepackage{geometry}
\usepackage[fleqn]{amsmath}
\usepackage{amssymb}
\usepackage{float}
\usepackage{enumerate}
\usepackage{listings} % Source code
\usepackage{lscape} % landscape
\usepackage{commath} % http://tex.stackexchange.com/questions/14821/whats-the-proper-way-to-typeset-a-differential-operator
\usepackage{cancel}
\usepackage[fleqn]{mathtools}
% Number only referenced equations
%\mathtoolsset{showonlyrefs}

%\usepackage{wrapfig}
\usepackage{siunitx}
\sisetup{separate-uncertainty=true,locale=DE}

% New command for color underlining
\usepackage{xcolor}

\newsavebox\MBox
\newcommand\colul[2][red]{{\sbox\MBox{$#2$}%
  \rlap{\usebox\MBox}\color{#1}\rule[-1.2\dp\MBox]{\wd\MBox}{0.5pt}}}

\restylefloat{table}
\geometry{a4paper, top=15mm, left=20mm, right=10mm, bottom=20mm, headsep=10mm, footskip=12mm}
\linespread{1.5}
\setlength\parindent{0pt}
\DeclareMathOperator{\Tr}{Tr}
\DeclareMathOperator{\Var}{Var}
\begin{document}
\allowdisplaybreaks % Seitenumbrüche in Formeln erlauben
\begin{center}
\bfseries % Fettdruck einschalten
\sffamily % Serifenlose Schrift
\vspace{-40pt}
Physikalisches Grundpraktikum 1, Sommersemester 2014

Markus Fenske \texttt{<iblue@zedat.fu-berlin.de>}

Paul Rahmann \texttt{<paulrahmann@zedat.fu-berlin.de>}

Tutor: Ramona Kositzki
\vspace{-10pt}
\end{center}

\section*{Physikalische Grundlagen}

Auf einen Körper in einer Flüssigkeit wirken drei verschiedene Kräfte. Zuerst
einmal die Gewichtskraft des Körpers. Wenn wir das Problem als eindimensional
annehmen und die $x$-Achse nach unten zeigt ($x = 0$ also zu Messbeginn), ist
die Gewichtskraft
\begin{equation}
  F_G = mg
\end{equation}

Dem entgegen wirkt eine Auftriebskraft, die der Masse der vom Körper verdrängten
Flüssigkeit entspricht: $m_\text{verdrängt} = \rho V$. Im Fall einer Kugel mit
Radius $r$ ist das Volumen $3V = 4 \pi r^3$, somit

\begin{equation}
  F_A = -\frac{4}{3} \pi r^3 \rho g
\end{equation}

Bewegt sich die Kugel durch die Flüssigkeit, entsteht bei eine laminare (also
wirbelfreie -- die Flüssigkeit bewegt sich in Schichten) oder turbulente (also
nicht wirbelfreie) Ströumung. Welche Art der Strömung vorliegt, lässt sich
durch die Reynolds-Zahl bestimmen

\begin{equation}
  \text{Re} = \frac{\rho v r}{\eta}
\end{equation}

Je kleiner die Reynolds-Zahl ist, desto schneller zerfallen auftretende
Turbulenzen. Für kleine Werte kann also von einer ideal laminaren Strömung
ausgegangen werden, sie tritt auf bei langsamen Bewegungen kleiner Körper
($r,v$ klein), in Flüssigkeiten die im Vergleich zu ihrer Dichte $\rho$ eine
hohe Viskosität $\eta$ aufweisen.

Im Fall laminarer Strömungen entsteht eine geschwindigkeitsabhängige
Reibungskraft. Sie wird hauptsächlich durch Bindungskräfte der Moleküle der
Flüssigkeit untereinander verursacht. Je höher diese Bindungskräfte, desto
höher ist die Viskosität und damit auch die Reibung. Die Reibungskraft, die der
Bewegung entgegengesetzt ist, ergibt sich dann aus dem Stokesschen Gesetz.

\begin{equation}
  F_R = -6 \pi r \eta v
\end{equation}

Die Summe aller Kräfte verschwindet bei Ausbildung eines Gleichgewichts ($v$
konstant), also

\begin{equation}
  mg - \frac{4}{3} \pi r^3 \rho g - 6 \pi r \eta v = 0
\end{equation}

Umgestellt kann man daraus, sobald die Grenzgeschwindigkeit $v$ erreicht ist,
die Viskosität $\eta$ berechnen, wenn die übrigen Parameter bekannt sind.

\begin{equation}
  \eta = \frac{3mg - 4 \pi \rho r^3 g}{18 \pi r v}
  \label{visco}
\end{equation}

Wichtig ist auch, dass die Viskosität $\eta$ temperaturabhängig ist, denn eine
höhere Temperatur entspricht einer schnelleren Bewegung der Moleküle der
Flüssigkeit. Man kann sich überlegen, dass die Abstände zwischen den Molekülen
größer werden (das sieht man makroskopisch auf daran, dass die Flüssigkeit sich
ausdehnt). Bei größeren Abständen zwischen den Molekülen sind die
Wechselwirkungskräfte entsprechend geringer. Die Temperaturabhängigkeit der
Viskosität ist dabei

\begin{equation}
  \eta(T) = A \exp \frac{B}{T}
  \label{temp_dep}
\end{equation}

Um bei bekanntem $\eta$ den Radius $r$ zu ermitteln, muss man eine kubische
Gleichung lösen. Einfacher ist es, im Fall kleiner Radien $r^3 \approx 0$ zu
nähern. Man erhält dann

\begin{equation}
  r \approx \frac{mg}{6 \pi \eta v}
\end{equation}

\section*{Aufgaben}

\begin{enumerate}[1.]
  \item Messung der Fallgeschwindigkeit von Stahlkugeln mit bekanntem und
    unbekanntem Radius in Abhängigkeit von der Temperatur.
  \item Untersuchung der funktionalen Abhängigkeit der Viskosität des
    Rizinusöls von der Temperatur. Bestimmung der Viskosität des Öles bei $20
    \,^{\circ}\mathrm{C}$ und Vergleich mit dem Literaturwert.
  \item
    Bestimmung der Radien der unbekannten Stahlkugeln aus den Messungen und
    Vergleich mit einer direkten Messung mit einer Mikrometerschraube.
  \item Aufstellung der Lösung der Bewegungsgleichung mit den Randbedingungen
    $v(t=0) = v_0 = 0$ und $v(t \to \infty) = v_\infty$ und Abschätzung der
    Zeit bzw. Wegstrecke, ab der die Kugeln mit praktisch konstanter
    Geschwindigkeit sinken.
\end{enumerate}

\newpage
\begin{center}
\bfseries % Fettdruck einschalten
\sffamily % Serifenlose Schrift
\vspace{-40pt}
Physikalisches Grundpraktikum 1, Sommersemester 2014

Markus Fenske \texttt{<iblue@zedat.fu-berlin.de>}

Paul Rahmann \texttt{<paulrahmann@zedat.fu-berlin.de>}

Tutor: Ramona Kositzki

Kugelfallviskosimeter am 16. Mai 2014
\vspace{-10pt}
\end{center}

\section*{Aufgabe 1}
Sofern nicht anders angegeben wird Gaußsche Fehlerrechnung verwendet.


Die Fallzeiten der Kugeln wurden über eine feste Fallstrecke von $s =
\SI{0.2060+-0.0005}{m}$ bei verschiedenen Temperaturen gemessen.

Es wurde jeweils die Anfangs- und die Endtemperatur gemessen und davon der
Mittelwert als konstante Temperatur angenommen. Aufgrund der Ablesegenauigkeit
des Thermometers von $\Delta T_{1,2} = \SI{0.2}{K}$ ergibt sich bei Mittelwertbildung
$T = \frac{T_2 - T_1}{2}$ der Fehler
\begin{equation}
  \Delta T = \frac{\sqrt{2}}{2} \Delta T_{1,2} \approx \SI{0.14}{K}
\end{equation}

Die Zeit wurde mit einer mechanischen Stoppuhr mit einer Ablesengenauigkeit von
$ \SI{0.1}{s}$ gemessen. Der Ablesefehler tritt nur einmal pro Messung auf.
Zusätzliche berücksichtigen wir eine Reaktionszeit von $\SI{0.3}{s}$, die
jeweils am Anfang und am Ende der Messung auftritt. Der Gesamtfehler ist somit

\begin{equation}
  \Delta t = \sqrt{\del{\SI{0.1}{s}}^2 + 2 \cdot \del{\SI{0.3}{s}}^2} \approx \SI{0.44}{s}
\end{equation}

Die entsprechenden Messwerte finden sich in der Messwerttabelle.

\section*{Aufgabe 2}


Aufgrund des exponentiellen Zusammenhangs (Gleichung \ref{temp_dep}) tragen wir
$t$ halblogarithmisch gegen $\frac{1}{T}$ auf. Der Fehler von $\frac{1}{T}$ ist

\begin{equation}
  \Delta \frac{1}{T} = \frac{\Delta T}{T^2}
\end{equation}

Den exponentiellen Zusammenhang können wir im Rahmen der Messgenauigkeit
bestätigen. Durch Projektion der Ausgleichs- und Grenzgeraden auf die Gerade $T
= 20\,^{\circ} \mathrm{C}$ kann man die Fallzeiten für diese Temperatur
ablesen.

Wir setzen $s = vt$ und $3m = 4 \pi \rho_\text{K} r^3$ in Gleichung \ref{visco}
ein und erhalten

\begin{equation}
  \eta = \frac{2 t r^2 g \del{\rho_\text{K} - \rho_\text{Ö}}}{9 s}
  \label{visco2}
\end{equation}

dabei ist $t$ die Fallzeit, $r$ der Kugelradius, $g$ die Erdbeschleunigung,
$\rho_\text{K}$ die Dichte der Stahlkugel, $\rho_\text{Ö}$ die Dichte des Öls
und $s$ die Fallstrecke.

Wir nutzen die folgenden Werte:
\begin{align*}
  &t = \SI{14.9+-1.6}{s}     && \qquad \text{(siehe grafische Auswertung)} \\
  &r = \SI{1.000+-0.002}{mm} && \qquad \text{(Quelle: Platzskript)} \\
  &g = \SI{9.812777+-0.000005}{m/s^2} && \qquad \text{(Quelle: Skript GP1, Seite 61)} \\
  &\rho_\text{K} = \SI{7.81+-0.02e3}{kg/m^3} && \qquad \text{(Quelle: Platzskript)} \\
  &\rho_\text{Ö} = \SI{0.975+-0.005e3}{kg/m^3} && \qquad \text{(Quelle: Platzskript)} \\
  &s = \SI{20.6+-0.05}{cm} && \qquad \text{(gemessen)}
\end{align*}

Man sieht sofort, dass der relative Fehler von $t$ um mehrere Nachkommastellen
größer ist, als die relativen Fehler der übrigen Werte. Daher sind die Fehler
der anderen Werte vernachlässigbar. Wir erhalten

\begin{equation}
  \Delta \eta = \frac{\eta}{t} \Delta t
\end{equation}

Damit ergibt sich als Endergebnis eine Viskosität von

\begin{equation*}
  \eta = \SI{1.1+-0.2}{\pascal \second}
\end{equation*}

In Marcelo Alonso, Edward J. Finn: Physik auf Seite 117 wird die Viskosität von
Rizinusöl bei $20\,^{\circ}\mathrm{C}$ mit $\SI{9.86e-3}{\pascal \second}$
angegeben, leider auch ohne Fehlerintervall. Die Abweichung zum Literaturwert
erklären wir mit einem Fehler in der Quelle. Wikipedia hingegen wird leider
nicht als zitierfähig angesehen (ironischerweise wegen der angeblichen
Unzuverlässigkeit), übernimmt jedoch im englischen Artikel \textit{Viscosity}
aus dem CRC Handbook of Chemistry and Physics, 86. Edition (die uns natürlich
nicht vorliegt) einen Wert, der um den Faktor $10^2$ höher ist.

Wir vermuten daher, der wahre Wert der Viskosität ist (unter Annahme der
Fehlergrenzen von einer Nachkommastelle bei ``ansonsten zuverlässigen
Quellen'' -- siehe GP1-Skript):

\begin{equation}
  \eta = \SI{0.986+-0.001}{\pascal \second}
\end{equation}

Unsere Messungen stimmen also mit dem Literaturwert überein.

\section*{Aufgabe 3}

Aus Gleichung \ref{visco2} sieht man, dass bei gleichbleibenden anderen
Parametern $t \cdot r^2$ konstant bleibt. Daraus erhalten wir

\begin{equation}
  \frac{t_1}{t_2} = \frac{r_1^2}{r_2^2} \qquad \Rightarrow \qquad r_2 = \sqrt{\frac{t_1}{t_2}} r_1
\end{equation}

Für den Fehler gilt
\begin{align*}
  \Delta r_2 &= \sqrt{
    \del{\frac{\partial r_2}{\partial t_1} \Delta t_1}^2 +
    \del{\frac{\partial r_2}{\partial t_2} \Delta t_2}^2 +
    \del{\frac{\partial r_2}{\partial r_1} \Delta r_1}^2
  } \\
  &= \sqrt{
    \del{ \frac{r_1}{2 \sqrt{t_1 t_2}} \Delta t_1 }^2 +
    \del{ -\frac{r_1}{2} \sqrt{ \frac{t_1}{t_2^3}} \Delta t_2 }^2 +
    \del{ \sqrt{ \frac{t_1}{t_2}} \Delta r_1 }^2
  } \\
  &= \sqrt{
    \frac{r_1^2}{4 t_1 t_2} \Delta t_1^2 + \frac{t_1 r_1^2}{4 t_2^3} \Delta t_2^2 + \frac{t_1}{t_2} \Delta r_1^2
  }
\end{align*}

Somit erhalten wir die Werte

\vspace{0.5cm}
\begin{tabular}{l|r|r|r}
  Kugel & Fallzeit $t$ [s] & Berechneter Radius $r$ [mm] & Gemessener Radius $r'$ [mm] \\
  \hline
  A & $\num{52.9+-3.3}$ & $\num{0.52+-0.04}$ & $\num{0.495+-0.005}$ \\
  B & $\num{22.6+-1.9}$ & $\num{0.81+-0.06}$ & $\num{0.780+-0.005}$ \\
  C & $\num{10.0+-1.2}$ & $\num{1.2+-0.1}$ & $\num{1.235+-0.005}$ \\
\end{tabular}
\vspace{0.5cm}

Die berechneten Werte stimmen innerhalb der Messgenauigkeit mit den
gemessenen überein.

\section*{Aufgabe 4}

Die Summe der Kräfte  $F_G + F_A + F_R$ ist

\begin{equation}
  mg - \frac{4}{3} \pi r^3 \rho g - 6 \pi r \eta v
\end{equation}

Hier wird die Stahlkugel der Masse $m$ nach unten beschleunigt, während das Öl,
was sich an dieser Stelle befindet, nach oben beschleunigt wird. Diese hat die
Masse $\frac{4}{3} \pi r^3 \rho$. Es liegt nahe, dies zu einer reduzierten
Masse $\mu = m - \frac{4}{3} \pi r^3 \rho$ zusammenzufassen. Die
Bewegungsgleichung muss dementsprechend auch das Öl enthalten. Sie lautet also

\begin{equation}
  \mu g - 6 \pi r \eta \dot{x} = \mu \ddot{x}
\end{equation}

Wir definieren $k := 6 \pi r \eta$ und erhalten

\begin{equation}
  \mu g - k \dot{x} = \mu \ddot{x}
\end{equation}

Durch $v = \dot{x}$ können wir diese Differentialgleichung durch Trennung der
Variablen lösen:

\begin{align*}
  &\mu g - k v = \mu \frac{\dif v}{\dif t} \\
  \Leftrightarrow \quad &
  \int \dif v \; \frac{1}{\mu g - kv} = \int \dif t \; \frac{1}{\mu} \\
  \Leftrightarrow \quad &
  -\frac{1}{k} \ln \del{\mu g - kv} = \frac{t}{\mu} + C_1 \\
  \Leftrightarrow \quad &
  \mu g - kv = C_2 \exp \del{- \frac{kt}{\mu} }, \quad C_2 > 0 \\
  \Leftrightarrow \quad &
  v(t) = -C \exp \del{- \frac{kt}{\mu} } + \frac{\mu g}{k}, \quad C > 0
\end{align*}

Aus der Anfangsbedingung $v(0) = 0$ sehen wir sofort $C = \dfrac{\mu g}{k}$.
Entsprechend ist $v_\infty = \dfrac{\mu g}{k}$.

\begin{equation}
  v(t) = v_\infty \del{1 - \exp\del{-\frac{k t}{\mu}}}
\end{equation}


Bevor wir die reduzierte Masse $\mu$ wieder separieren und dann aus
Erhaltungssätzen die Geschwindigkeiten der einzelnen Komponenten bestimmen,
erscheint es sinnvoll, sich zunächst zu fragen, ob dies nicht vernachlässigt
werden kann. Die Aufgabenstellung ist eine \textit{Abschätzung} der Zeit bzw.
Wegstrecke, ab der die Kugeln mit \textit{praktisch} konstanter Geschwindigkeit
sinken. Da die Fallzeitenauswertung nach grafischer Auswertung auf gerade
einmal 2 signifikante Stellen genau ist, wäre eine Geschwindigkeitsabweichung
von $1 \%$ unterhalb der Messbarkeitsgrenze. Da die Dichte der Kugeln etwa dem
8-fachen des Öls entspricht, müsste dieser Wert auf $0{,}01 \cdot \dfrac{8}{9}$
korrigiert werden, was auch nicht weiter ins Gewicht fällt. Für die
\textit{praktische Abschätzung} können wir es also als ausreichend ansehen,
wenn die reduzierte Masse $0{,}99 \cdot v_\infty$ erreicht hat.

Dies ist der Fall, wenn

\begin{align*}
  &0{,}99 = \del{1 - \exp\del{-\frac{k t}{\mu}}} \\
  \Leftrightarrow \quad &
  0{,}01 = \exp\del{-\frac{k t}{\mu}} \\
  \Leftrightarrow \quad &
  \ln{0{,}01} = -\frac{k t}{\mu} \\
  \Leftrightarrow \quad &
  t = \frac{\mu \ln{100}}{k}\\
\end{align*}

Die zurückgelegte Wegstrecke ist

\begin{align}
  x(t) &= \int_0^t \dif t' v_{\infty} \del{1 - \exp\del{-\frac{k t}{\mu}}} \\
  &= v_\infty t + \frac{\mu}{k} \exp\del{-\frac{k t}{\mu}} \\
  &= \frac{\mu}{k} g t + \frac{\mu}{k} \exp\del{-\frac{k t}{\mu}} \\
  &= \frac{\mu^2}{k^2} g \ln(100) + \frac{\mu}{k} \exp\del{-\ln(100)} \\
  &= \frac{\mu^2g}{k^2} \ln(100) + 0{,}01 \frac{\mu}{k}
\end{align}

Aus den substituierten Werten $\mu = \frac{4}{3} \pi r^3 \del{\rho_\text{K} -
\rho_\text{Ö}}$ und $k = 6 \pi r \eta$ erhalten wir

\begin{equation}
  \frac{\mu}{k} = \frac{2}{9} \frac{r^2}{\eta} \del{ \rho_\text{K} - \rho_\text{Ö} }
\end{equation}

Ausgehend von den gemessenen Radien können wir nun relativ einfach die Zeiten
und Strecken bestimmen. Wir gehen von einer Viskosität von $\eta =
\SI{1}{\pascal \second}$ aus -- denn dieser Wert ist der Literaturwert,
gerundet auf unsere Messgenauigkeit, was für die Abschätzung ausreichend ist.
Die Abschätzung erfolgt nur auf eine Stelle genau und ohne Fehlerrechnung, den
interessant sind hier nur die Größenordnungen

\vspace{0.5 cm}
\begin{tabular}{l|r|r}
  Kugel & Zeit [ms] & Weg [mm] \\
  \hline
  A & 2 & 0{,}01 \\
  B & 4 & 0{,}05 \\
  0 & 7 & 0{,}1 \\
  C & 11 & 0{,}3
\end{tabular}
\vspace{0.5 cm}

Man sieht, dass die Endgeschwindigkeit bereits nach sehr kurzer Zeit und sehr
kurzen Strecken erreicht wird.

\section*{Zusammenfassung und Diskussion}

Wir konnten aus der Messung der Fallzeit von Stahlkugeln in Rizinusöl bei
verschiedenen Temperaturen erfolgreich die Viskosität bei Normalbedingen
extrapolieren. Auch kpnnten wir den exponentiellen Zusammenhang zwischen
Viskosität und Temperatur zeigen.

Die relativ geringe Genauigkeit hätte man durch eine konstante Temperierung des
Öls steigern können, was jedoch die Versuchsdurchführung erheblich
verkompliziert hätte.

Die Bestimmung des Radius der Kugeln mit der Mikrometerschraube ist
% schief gelaufen, weil nicht klar war, wie fest man die anzieht und wie man
% die benutzt, aber das macht nichts, weil der Radius eh nicht so genau
% gemessen werden kann bla.

Aus Aufgabe 4 konnten wir ersehen, dass die Annahme einer konstanten
Fallgeschwindigkeit gerechtfertigt ist, dann die Beschleunigungsphase ist sehr
kurz gerechtfertigt war, denn wir konnten zeigen, dass eine konstante
Geschwindigkeit bereits nach sehr kurzer Strecke erreicht wird.

\begin{landscape}
  % Generieren mit gnuplot ./plot/plot.gnuplot
  \input{plot.tex}
\end{landscape}
\end{document}
