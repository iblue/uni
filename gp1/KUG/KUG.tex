\documentclass[a4paper,german,12pt,smallheadings]{scrartcl}
\usepackage[T1]{fontenc}
\usepackage[utf8]{inputenc}
\usepackage{babel}
\usepackage{geometry}
\usepackage[fleqn]{amsmath}
\usepackage{amssymb}
\usepackage{float}
\usepackage{enumerate}
\usepackage{listings} % Source code
\usepackage{lscape} % landscape
\usepackage{commath} % http://tex.stackexchange.com/questions/14821/whats-the-proper-way-to-typeset-a-differential-operator
\usepackage{cancel}

\usepackage[fleqn]{mathtools}
% Number only referenced equations
%\mathtoolsset{showonlyrefs}

%\usepackage{wrapfig}
\usepackage[thinspace,thinqspace,squaren,textstyle]{SIunits}

% New command for color underlining
\usepackage{xcolor}

\newsavebox\MBox
\newcommand\colul[2][red]{{\sbox\MBox{$#2$}%
  \rlap{\usebox\MBox}\color{#1}\rule[-1.2\dp\MBox]{\wd\MBox}{0.5pt}}}

\restylefloat{table}
\geometry{a4paper, top=15mm, left=20mm, right=10mm, bottom=20mm, headsep=10mm, footskip=12mm}
\linespread{1.5}
\setlength\parindent{0pt}
\DeclareMathOperator{\Tr}{Tr}
\DeclareMathOperator{\Var}{Var}
\begin{document}
\allowdisplaybreaks % Seitenumbrüche in Formeln erlauben
\begin{center}
\bfseries % Fettdruck einschalten
\sffamily % Serifenlose Schrift
\vspace{-40pt}
Physikalisches Grundpraktikum 1, Sommersemester 2014

Markus Fenske \texttt{<iblue@zedat.fu-berlin.de>}

Paul Rahmann \texttt{<paulrahmann@zedat.fu-berlin.de>}

Tutor: Ramona Kositzki
\vspace{-10pt}
\end{center}

\section*{Physikalische Grundlagen}

Auf einen Körper in einer Flüssigkeit wirken drei verschiedene Kräfte. Zuerst
einmal die Gewichtskraft des Körpers. Wenn wir das Problem als eindimensional
annehmen und die $x$-Achse nach unten zeigt ($x = 0$ also zu Messbeginn), ist
die Gewichtskraft
\begin{equation}
  F_G = mg
\end{equation}

Dem entgegen wirkt eine Auftriebskraft, die der Masse der vom Körper verdrängten
Flüssigkeit entspricht: $m_\text{verdrängt} = \rho V$. Im Fall einer Kugel mit
Radius $r$ ist das Volumen $3V = 4 \pi r^3$, somit

\begin{equation}
  F_A = -\frac{4}{3} \pi r^3 \rho
\end{equation}

Bewegt sich die Kugel durch die Flüssigkeit, entsteht bei eine laminare (also
wirbelfreie -- die Flüssigkeit bewegt sich in Schichten) oder turbulente (also
nicht wirbelfreie) Ströumung. Welche Art der Strömung vorliegt, lässt sich
durch die Reynolds-Zahl bestimmen

\begin{equation}
  \text{Re} = \frac{\rho v r}{\eta}
\end{equation}

Je kleiner die Reynolds-Zahl ist, desto schneller zerfallen auftretende
Turbulenzen. Für kleine Werte kann also von einer ideal laminaren Strömung
ausgegangen werden, sie tritt auf bei langsamen Bewegungen kleiner Körper
($r,v$ klein), in Flüssigkeiten die im Vergleich zu ihrer Dichte $\rho$ eine
hohe Viskosität $\eta$ aufweisen.

Im Fall laminarer Strömungen entsteht eine geschwindigkeitsabhängige
Reibungskraft. Sie wird hauptsächlich durch Bindungskräfte der Moleküle der
Flüssigkeit untereinander verursacht. Je höher diese Bindungskräfte, desto
höher ist die Viskosität und damit auch die Reibung. Die Reibungskraft, die der
Bewegung entgegengesetzt ist, ergibt sich dann aus dem Stokesschen Gesetz.

\begin{equation}
  F_R = -6 \pi r \eta v
\end{equation}

Die Summe aller Kräfte verschwindet bei Ausbildung eines Gleichgewichts ($v$
konstant), also

\begin{equation}
  mg - \frac{4}{3} \pi r^3 \rho - 6 \pi r \eta v = 0
\end{equation}

Umgestellt kann man daraus, sobald die Grenzgeschwindigkeit $v$ erreicht ist,
die Viskosität $\eta$ berechnen, wenn die übrigen Parameter bekannt sind.

\begin{equation}
  \eta = \frac{3mg - 4 \pi \rho r^3}{18 \pi r v}
\end{equation}

Wichtig ist auch, dass die Viskosität $\eta$ temperaturabhängig ist, denn eine
höhere Temperatur entspricht einer schnelleren Bewegung der Moleküle der
Flüssigkeit. Man kann sich überlegen, dass die Abstände zwischen den Molekülen
größer werden (das sieht man makroskopisch auf daran, dass die Flüssigkeit sich
ausdehnt). Bei größeren Abständen zwischen den Molekülen sind die
Wechselwirkungskräfte entsprechend geringer. Die Temperaturabhängigkeit der
Viskosität ist dabei

\begin{equation}
  \eta(T) = A \exp \frac{B}{T}
\end{equation}

Um bei bekanntem $\eta$ den Radius $r$ zu ermitteln, muss man eine kubische
Gleichung lösen. Einfacher ist es, im Fall kleiner Radien $r^3 \approx 0$ zu
nähern. Man erhält dann

\begin{equation}
  r \approx \frac{mg}{6 \pi \eta v}
\end{equation}

\section*{Aufgaben}

\begin{enumerate}[1.]
  \item Messung der Fallgeschwindigkeit von Stahlkugeln mit bekanntem und
    unbekanntem Radius in Abhängigkeit von der Temperatur.
  \item Untersuchung der funktionalen Abhängigkeit der Viskosität des
    Rizinusöls von der Temperatur. Bestimmung der Viskosität des Öles bei $20
    \deg C$ und Vergleich mit dem Literaturwert.
  \item
    Bestimmung der Radien der unbekannten Stahlkugeln aus den Messungen und
    Vergleich mit einer direkten Messung mit einer Mikrometerschraube.
  \item Aufstellung der Lösung der Bewegungsgleichung mit den Randbedingungen
    $v(t=0) = v_0 = 0$ und $v(t \to \infty) = v_\infty$ und Abschätzung der
    Zeit bzw. Wegstrecke, ab der die Kugeln mit praktisch konstanter
    Geschwindigkeit sinken.
\end{enumerate}
\end{document}
