\documentclass[a4paper,german,12pt,smallheadings]{scrartcl}
\usepackage[T1]{fontenc}
\usepackage[utf8]{inputenc}
\usepackage{babel}
\usepackage{geometry}
\usepackage[fleqn]{amsmath}
\usepackage{amssymb}
\usepackage{float}
\usepackage{enumerate}
\usepackage{commath} % http://tex.stackexchange.com/questions/14821/whats-the-proper-way-to-typeset-a-differential-operator
\usepackage{cancel}

\usepackage[fleqn]{mathtools}
% Number only referenced equations
%\mathtoolsset{showonlyrefs}

%\usepackage{wrapfig}
\usepackage[thinspace,thinqspace,squaren,textstyle]{SIunits}

% New command for color underlining
\usepackage{xcolor}

\newsavebox\MBox
\newcommand\colul[2][red]{{\sbox\MBox{$#2$}%
  \rlap{\usebox\MBox}\color{#1}\rule[-1.2\dp\MBox]{\wd\MBox}{0.5pt}}}

\restylefloat{table}
\geometry{a4paper, top=15mm, left=10mm, right=20mm, bottom=20mm, headsep=10mm, footskip=12mm}
\linespread{1.5}
\setlength\parindent{0pt}
\DeclareMathOperator{\Tr}{Tr}
\DeclareMathOperator{\Var}{Var}
\begin{document}
\allowdisplaybreaks % Seitenumbrüche in Formeln erlauben
\begin{center}
\bfseries % Fettdruck einschalten
\sffamily % Serifenlose Schrift
\vspace{-40pt}
Physikalisches Grundpraktikum 1, Sommersemester 2014

Markus Fenske, Tutor: X. Chen

Gravity Pendulum
\vspace{-10pt}
\end{center}

\section*{Physical Basis}

We describe an arbitrarily shaped rigid body swinging by a pivot.

The equation of motion for the rotation of a rigid body around a fixed axis is

\begin{equation}
  M = I \frac{d^2 \phi}{d t^2}
  \label{eq_motion}
\end{equation}

where $M$ is the torque, $I$ the moment of inertia and $\phi$ the angle of rotation.

The torque is generated by gravity. If assume the rigid body to be small
compared to the size of the planet, the gravitational force ($F = -mg$) can be
assumed to be constant in magnitude and direction  and the torque can be
described as

\begin{equation}
  M = -mg s \sin \phi
  \label{torque}
\end{equation}

where $m$ is the mass of the body, $g$ the gravitational constant, $s$ the
distance between the pivot and the center of mass.

Substituting eq. \ref{torque} into eq. \ref{eq_motion} leads the equation of
motion

\begin{equation}
  -mg s \sin \phi = I \frac{d^2 \phi}{d t^2}
  \label{diff_eq}
\end{equation}

$\sin \phi$ can be written as Taylor series

\begin{equation}
  \sin \phi = \phi - \frac{\phi^3}{6} + \frac{x^5}{120} - \dots
\end{equation}

If we assume $\phi$ to be small, we can thus approximate $\sin \phi \approx
\phi$, which then leads the differential equation

\begin{equation}
  \frac{d^2 \phi}{d t^2} = \phi \frac{-mgs}{I}
\end{equation}

Using the ansatz $\phi(t) = \phi_0 \sin \omega t$, where $\phi_0$ is the
amplitude at $t=0$ and $\omega$ is the angular frequenzcy, we can solve this
equation as

\begin{equation}
  \phi(t) = \phi_0 \sin \sqrt{\frac{mgs}{I} t}
\end{equation}

We see that the period $T_0$ (that is where $\phi(t) = \phi(t + T_0)$) of the motion is

\begin{equation}
  T_0 = 2 \pi \sqrt{\frac{I}{mgs}}
\end{equation}

By introducting the reduced pendulum length $L = \frac{I}{ms}$, the
equation becomes analogous to the mathematical pendulum

\begin{equation}
  T_0 = 2 \pi \sqrt{\frac{L}{g}}
\end{equation}

% FIXME: Das Herleiten, wenn ich Zeit dazu habe. Entsprechende Literatur zitieren.

If the angle $\phi$ is not small, we need to use eq. \ref{diff_eq}. To solve
this equation, one needs more advanced knowledge of differential equations,
which is beyond the scope of this lab course. From the GP1 script we obtain

\begin{equation}
  T = T_0 \del{1 + \frac{1}{4} \sin^2 \frac{\phi_0}{2} + \del{\frac{1 \cdot 3}{2 \cdot 4}}^2 \sin^4 \frac{\phi_0}{2} + \dots}
\end{equation}

Again using the first order expansion of $\sin$ and using only the first term, we can approximate

\begin{equation}
  T \approx T_0 \del{1 + \frac{\phi_0^2}{16}}
\end{equation}


\end{document}
