\documentclass[a4paper,german,12pt,smallheadings]{scrartcl}
\usepackage[T1]{fontenc}
\usepackage[utf8]{inputenc}
\usepackage{babel}
\usepackage{geometry}
\usepackage[fleqn]{amsmath}
\usepackage{amssymb}
\usepackage{float}
\usepackage{enumerate}
\usepackage{commath} % http://tex.stackexchange.com/questions/14821/whats-the-proper-way-to-typeset-a-differential-operator
\usepackage{cancel}

\usepackage[fleqn]{mathtools}
% Number only referenced equations
%\mathtoolsset{showonlyrefs}

%\usepackage{wrapfig}
\usepackage[thinspace,thinqspace,squaren,textstyle]{SIunits}

% New command for color underlining
\usepackage{xcolor}

\newsavebox\MBox
\newcommand\colul[2][red]{{\sbox\MBox{$#2$}%
  \rlap{\usebox\MBox}\color{#1}\rule[-1.2\dp\MBox]{\wd\MBox}{0.5pt}}}

\restylefloat{table}
\geometry{a4paper, top=15mm, left=20mm, right=10mm, bottom=20mm, headsep=10mm, footskip=12mm}
\linespread{1.5}
\setlength\parindent{0pt}
\DeclareMathOperator{\Tr}{Tr}
\DeclareMathOperator{\Var}{Var}
\begin{document}
\allowdisplaybreaks % Seitenumbrüche in Formeln erlauben
\begin{center}
\bfseries % Fettdruck einschalten
\sffamily % Serifenlose Schrift
\vspace{-40pt}
Physikalisches Grundpraktikum 1, Sommersemester 2014

Markus Fenske, Paul Rahmann, Tutor: X. Chen

Gravity Pendulum
\vspace{-10pt}
\end{center}

\section*{Physical Basis}

We describe an arbitrarily shaped rigid body swinging by a pivot.

The equation of motion for the rotation of a rigid body around a fixed axis is

\begin{equation}
  M = I \frac{d^2 \phi}{d t^2}
  \label{eq_motion}
\end{equation}

where $M$ is the torque, $I$ the moment of inertia and $\phi$ the angle of rotation.

The torque is generated by gravity. If assume the rigid body to be small
compared to the size of the planet, the gravitational force ($F = -mg$) can be
assumed to be constant in magnitude and direction  and the torque can be
described as

\begin{equation}
  M = -mg s \sin \phi
  \label{torque}
\end{equation}

where $m$ is the mass of the body, $g$ the gravitational constant, $s$ the
distance between the pivot and the center of mass.

Substituting eq. \ref{torque} into eq. \ref{eq_motion} leads the equation of
motion

\begin{equation}
  -mg s \sin \phi = I \frac{d^2 \phi}{d t^2}
  \label{diff_eq}
\end{equation}

$\sin \phi$ can be written as Taylor series

\begin{equation}
  \sin \phi = \phi - \frac{\phi^3}{6} + \frac{x^5}{120} - \dots
\end{equation}

If we assume $\phi$ to be small, we can thus approximate $\sin \phi \approx
\phi$, which then leads the differential equation

\begin{equation}
  \frac{d^2 \phi}{d t^2} = \phi \frac{-mgs}{I}
\end{equation}

Using the ansatz $\phi(t) = \phi_0 \sin \omega t$, where $\phi_0$ is the
amplitude at $t=0$ and $\omega$ is the angular frequenzcy, we can solve this
equation as

\begin{equation}
  \phi(t) = \phi_0 \sin \sqrt{\frac{mgs}{I} t}
\end{equation}

We see that the period $T_0$ (that is where $\phi(t) = \phi(t + T_0)$) of the motion is

\begin{equation}
  T_0 = 2 \pi \sqrt{\frac{I}{mgs}}
  \label{period}
\end{equation}

By introducting the reduced pendulum length $L = \frac{I}{ms}$, the
equation becomes analogous to the mathematical pendulum

\begin{equation}
  T_0 = 2 \pi \sqrt{\frac{L}{g}}
\end{equation}

% FIXME: Das Herleiten, wenn ich Zeit dazu habe. Entsprechende Literatur zitieren.

If the angle $\phi$ is not small, we need to use eq. \ref{diff_eq}. To solve
this equation, one needs more advanced knowledge of differential equations,
which is beyond the scope of this lab course. From the GP1 script we obtain

\begin{equation}
  T = T_0 \del{1 + \frac{1}{4} \sin^2 \frac{\phi_0}{2} + \del{\frac{1 \cdot 3}{2 \cdot 4}}^2 \sin^4 \frac{\phi_0}{2} + \dots}
\end{equation}

Again using the first order expansion of $\sin$ and using only the first term, we can approximate

\begin{equation}
  T \approx T_0 \del{1 + \frac{\phi_0^2}{16}}
\end{equation}

In order to calculate the gravitational constant we need to measure the initial
amplitude $\phi_0$ (or keep it relatively small and neglect it), the period
$T$, and the reduced pendulum length $L$. So we would need to determining the
moment of inertia $I$, the mass $m$, and $s$ (for which we need the center of
mass its distance from the pivot).

If we knew the mass $m$ of the pendulum and its volume and it had been
manifactured of a homogenious material with density $\rho$, we could calculate
its moment of inertia by

\begin{equation}
  I = \int_V r^2 \rho \dif V
\end{equation}

where $r^2$ is the perpendicular distance from the pivot. For this however, we
need to know the exact form of the pendulum. Even a regularily shaped body (say
a steel bar) would have some production tolerance, so this is not practical.
Instead we use the setup known as Kater's pendulum.

By substituting the parallel axis theorem into eq. \ref{period}, we obtain

\begin{equation}
  T_{0} = 2 \pi \sqrt{\frac{I_s + ms^2}{mgs}}
\end{equation}

where $I_s$ is the moment of inertia at the center of mass and $s$ is the
distance from it. Given the pendulum has exchangeble pivots at distances $s_1$
and $s_2$ and given the periods are the same for those points, we get a
quadratic equation with the solutions

\begin{equation}
  s_1 = s_2 \quad \cup \quad s_2 = \frac{I_s}{m s_1}
\end{equation}

The first solution is not useful, but using the second solution, we obtain the
reduced pendulum length

\begin{equation}
  s_1 + s_2 = \frac{I_s}{ms_1} + s_1 = \frac{I_s + ms_1^2}{ms_1} = \frac{I}{ms} = L
\end{equation}

So if we have two pivots on a line through the center of mass, each with a
different distance to the center of mass and a swing around both pivots gives
the period, the distance between the pivots equals the reduced length $L$.

\section*{Exercises}
\begin{enumerate}
  \item Measuring the period of the gravity pendulum as a function of the
    amplitude (using the stop watch and the quartz light barrier watch).
  \item Measuring the local gravitational acceleration using Kater's pendulum.
\end{enumerate}
\newpage

\section*{Measurement protocol}
\subsection*{Exercise 1}
Legend: L: light barrier, S: stop watch.

\begin{tabular}{l|l|l|l}
  Amplitude [\quad\quad] & Number of periods & Time [s] \qquad\qquad\qquad & Comments \qquad\qquad\qquad\qquad\qquad\qquad \\
  \hline
   &  &  &  \\
  \hline
   &  &  &  \\
  \hline
   &  &  &  \\
  \hline
   &  &  &  \\
  \hline
   &  &  &  \\
  \hline
   &  &  &  \\
  \hline
   &  &  &  \\
  \hline
   &  &  &  \\
  \hline
   &  &  &  \\
  \hline
   &  &  &  \\
  \hline
   &  &  &  \\
  \hline
   &  &  &  \\
  \hline
   &  &  &  \\
  \hline
   &  &  &  \\
  \hline
   &  &  &  \\
  \hline
   &  &  &  \\
  \hline
   &  &  &  \\
  \hline
   &  &  &  \\
  \hline
   &  &  &  \\
  \hline
   &  &  &  \\
  \hline
   &  &  &  \\
  \hline
   &  &  &  \\
  \hline
   &  &  &  \\
  \hline
   &  &  &  \\
  \hline
   &  &  &  \\
  \hline
   &  &  &  \\
  \hline
   &  &  &  \\
  \hline
   &  &  &  \\
  \hline
   &  &  &  \\
\end{tabular}

\newpage

\subsection*{Exercise 2}

\begin{tabular}{l|l|l|l}
  Position of $m_1$ [\quad\quad] & Period [s] \qquad\qquad\qquad & Pivot & Comments \qquad\qquad\qquad\qquad\qquad\qquad\qquad\qquad \\
  \hline
   &  &  &  \\
  \hline
   &  &  &  \\
  \hline
   &  &  &  \\
  \hline
   &  &  &  \\
  \hline
   &  &  &  \\
  \hline
   &  &  &  \\
  \hline
   &  &  &  \\
  \hline
   &  &  &  \\
  \hline
   &  &  &  \\
  \hline
   &  &  &  \\
  \hline
   &  &  &  \\
  \hline
   &  &  &  \\
  \hline
   &  &  &  \\
  \hline
   &  &  &  \\
  \hline
   &  &  &  \\
  \hline
   &  &  &  \\
  \hline
   &  &  &  \\
  \hline
   &  &  &  \\
  \hline
   &  &  &  \\
  \hline
   &  &  &  \\
  \hline
   &  &  &  \\
  \hline
   &  &  &  \\
  \hline
   &  &  &  \\
  \hline
   &  &  &  \\
  \hline
   &  &  &  \\
  \hline
   &  &  &  \\
  \hline
   &  &  &  \\
  \hline
   &  &  &  \\
  \hline
   &  &  &  \\
  \hline
   &  &  &  \\
  \hline
   &  &  &  \\
  \hline
   &  &  &  \\
\end{tabular}

\newpage


\section*{Things to be done}
\begin{enumerate}
  \item Take pictures of aparatus!
  \item Write down key points like measurement accuracy of the watches, length,
    masses and anything that could be of importance about the environment like
    exact location, temperature, etc.
  \item Measure reaction time by stop watch: Try to stop at exact 3 sec and
    write down values.
  \item Exercise 1a: Measure period of gravity pendulum by stop watch. Probably find out
    that the measurement has high uncertainty.
  \item Exercise 1b: Measure period using light barrier watch. Period variance is expected
    to be low, so try to measure multiple periods if possible.
  \item Exercise 2: Measure period by pivot 1 and pivot 2 in turns. Move $m_1$,
    after each two measurements.
  \item Graphical analysis 1a/1b: Plot amplitude (in radians) against period to get $T_0$?
  \item Graphical analysis 2: Plot pivot a and pivot b in same diagram.
    Intersection leads gravitational constant.
\end{enumerate}

\end{document}
