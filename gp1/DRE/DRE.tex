\documentclass[a4paper,german,12pt,smallheadings]{scrartcl}
\usepackage[T1]{fontenc}
\usepackage[utf8]{inputenc}
\usepackage{babel}
\usepackage{geometry}
\usepackage[fleqn]{amsmath}
\usepackage{amssymb}
\usepackage{float}
\usepackage{enumerate}
\usepackage{listings} % Source code
\usepackage{lscape} % landscape
\usepackage{commath} % http://tex.stackexchange.com/questions/14821/whats-the-proper-way-to-typeset-a-differential-operator
\usepackage{cancel}
\usepackage[fleqn]{mathtools}
% Number only referenced equations
%\mathtoolsset{showonlyrefs}

%\usepackage{wrapfig}
\usepackage{siunitx}
\sisetup{separate-uncertainty=true,locale=DE}

% New command for color underlining
\usepackage{xcolor}

\newsavebox\MBox
\newcommand\colul[2][red]{{\sbox\MBox{$#2$}%
  \rlap{\usebox\MBox}\color{#1}\rule[-1.2\dp\MBox]{\wd\MBox}{0.5pt}}}

\restylefloat{table}
\geometry{a4paper, top=15mm, left=20mm, right=10mm, bottom=20mm, headsep=10mm, footskip=12mm}
\linespread{1.5}
\setlength\parindent{0pt}
\DeclareMathOperator{\Tr}{Tr}
\DeclareMathOperator{\Var}{Var}
\begin{document}

\begin{titlepage}
\newcommand{\HRule}{\rule{\linewidth}{0.5mm}}

\begin{center}
  \textsc{\Large Physikalisches Grundpraktkum 1}
  \HRule\\[0.4 cm]
  {\huge \bfseries Gleichmäßig beschleunigte Drehbewegungen}
  \HRule\\[0.4 cm]

  \begin{minipage}{0.65\textwidth}
  \begin{flushleft}
    Markus Fenske \texttt{<iblue@zedat.fu-berlin.de>} \\
    Paul Rahmann \texttt{<paulrahmann@zedat.fu-berlin.de>}
  \end{flushleft}
  \end{minipage}
  \hfill
  \begin{minipage}{0.30\textwidth}
  \begin{flushright}
    Tutor: Christian Hindermann \\
    Versuchstag: 6. Juni 2014
  \end{flushright}
  \end{minipage}

  \vspace{1cm}

  \tableofcontents


  %{\large \today}
  \vfill
\end{center}
\newpage

\end{titlepage}

\allowdisplaybreaks % Seitenumbrüche in Formeln erlauben
\begin{center}
\bfseries % Fettdruck einschalten
\sffamily % Serifenlose Schrift
\vspace{-40pt}
Physikalisches Grundpraktikum 1, Sommersemester 2014

Markus Fenske \texttt{<iblue@zedat.fu-berlin.de>}

Paul Rahmann \texttt{<paulrahmann@zedat.fu-berlin.de>}

Drehbewegungen, Tutor: Christian Hindermann
\vspace{-10pt}
\end{center}

\section*{Physikalische Grundlagen}
Das Trägheitsmoment wird durch $F = \dot{p} = m \dot{v}$ motiviert. Wenn auf eine Masse $m$
im Abstand $r$ von der Drehachse eine Kraft wirkt, folgt

\begin{equation}
  \vec{r} \times \vec{F}
  = \vec{r} \times m \dot{\vec{v}}
\end{equation}

mit $v = r \omega$ und $r$ konstant und weil die Drehachse senkrecht auf Kraft
und Ortsvektor steht

\begin{equation}
  = \dot{\vec{\omega}} r^2 m
\end{equation}

Durch Überführung von $m$ in ein Massenelement $\dif m$ folgt

\begin{equation}
  \underbrace{\vec{r} \times \vec{F}}_{=: \vec{M}} = \dot{\vec{\omega}} \underbrace{\int \dif m \; r^2}_{=: I}
\end{equation}

Wir verwenden ab hier nur die Beträge, da alle Vektoren jeweils senkrecht
aufeinander stehen. Die Richtungen können dann aus der Rechte-Hand-Regel
ermittelt werden.

Die Bewegungsgleichung für die Drehbewegung eines starren Körpers lautet also gemäß der Herleitung
\begin{equation}
  M = I \dot{\omega} = I \ddot{\phi}
\end{equation}

Analog zur Newtonschen Bewegungsgleichung für lineare Bewegungen unter Einfluss
einer konstanten Kraft ist die Lösung dieser Differentialgleichung unter
Annahme eines konstanten Drehmoments $M$
\begin{equation}
  \phi(t) = \frac{M}{2I}t^2 + \omega_0 t + \phi_0
  \label{eq_of_motion}
\end{equation}

Dabei ist $M$ das konstante Drehmoment, $I$ das Trägheitsmoment, $\omega_0$ die
Anfangswinkelgeschwindigkeit und $\phi_0$ der Anfangswinkel. Als Konsequenz
ergibt sich unter Einfluss einer konstanten Kraft also eine lineare Zunahme der
Winkelgeschwindigkeit
\begin{equation}
  \omega(t) = \frac{M}{I} t + \omega_0
\end{equation}

Das Drehmoment im Schwerpunktkoordinatensystem ist

\begin{equation}
  I = \int \dif m \; r^2
  \label{speed}
\end{equation}

Sei die Drehachse oBdA $\omega \hat{z}$. Verschieben wir die Drehachse um
$(a_x, a_y)$ erhalten wir

\begin{align*}
  I_a &= \int \dif m \; \del{(x - a_x)^2 + (y - a_y)^2} \\
      &= \int \dif m \; \del{x^2 - 2xa_x + a_x^2 + y^2 - 2ya_y + a_y^2} \\
      &= \int \dif m \; \del{x^2 + y^2}
       + 2 a_x \int \dif m \; x
       + 2 a_y \int \dif m \; y
       + \del{a_x^2 + a_y^2} \int \dif m
\end{align*}

Der erste Term ist das Drehmoment um den Schwerpunkt, die nächsten beiden
Integrale sind die $x$- und die $y$-Koordinate des Schwerpunktes im
Schwerpunktsystem. Da in diesem der Schwerpunkt im Ursprung liegt, verschwinden
sie. Das nächste Integral ist einfach die Gesamtmasse. Somit

\begin{equation}
  I_a = I + ma^2
\end{equation}

Dies ist der Steinersche Satz.

Für die Aufgabenstellung ist außerdem die Berechnung des Trägheitsmoments eines
starren Zylinders gefordert. Ein Vollzylinder wird dem experimentellen Aufbau
nicht gerecht, denn die Masse konzentriert sie unseren Erachtens nach auf den
Reifen, nicht auf die Speichen. Gehen wir also von einer Massenverteilung $\rho
= \frac{m}{2 \pi R} \delta(r-R)\delta(z)$ aus, erhalten wir

\begin{equation}
  I = \int \dif V \; r^2 \rho = m R^2
\end{equation}

\section*{Aufgaben}
%Untersuchung gleichmäßig beschleunigter Drehbewegungen
%\begin{itemize}
%  \item Messung von Weg-Zeit-Abhängigkeiten
%  \item Messung von Drehmoment-Zeit-Abhängigkeiten
%  \item Messung der Reibungsverluste
%\end{itemize}
%für unterschiedliche Trägheitsmomente (mit und ohne Zusatzmassen).
\begin{enumerate}[1.]
  \item Qualitative und quantitative Überprüfung des Bewegungsgesetzes. Messung
    der Zeit in Abhängigkeit des Drehwinkels (bei festem Drehmoment) und in
    Abhängigkeit des Drehmoments (bei festem Drehwinkel). Bestimmung der
    Trägheitsmomente (mit und ohne Zusatzmassen) des Rades aus den Messungen
    und Vergleich mit dem berechneten Wert aus dem \textit{Steinerschen Satz}.
  \item
    Diskussion von Reibungseinflüssen und Reibungsmodellen (Abhängigkeit von
    Reibungskräften bzw. Reibungsmomenten von den verschiedenen
    Bewegungsparametern) aus den Ergebnissen der Messungen.
\end{enumerate}

\newpage
\section*{Versuchsaufbau}
% TODO: Text
% TODO: Bonuspunkte für Skizze mit PGF/Tikz? Zweite Wahl: Foto


\section*{Versuchsdurchführung}

Wir haben jeweils mittels Stoppuhr die Zeiten gemessen, die die angehängte
Masse bis zum Tiefpunkt und zum nächsten Hochpunkt braucht. Der Tiefpunkt ist
jeweils dort, wo der im Rad angebrachte Zeiger seinen Nulldurchgang hat, der
Hochpunkt dort, wo er seinen Stillstand erreicht.

Dies haben wir jeweils für jede der 5 Masses, die beiden verschiedenen Drehrichtungen
und jeweils 1,2 und 5 Umdrehungen und in den Konfigurationen ohne und mit
zusätzlicher Trägheitsmasse, so dass sich insgesamt 60 Messwertsätze mit 120
Zeiten ergeben.

Den Hochpunkt des Gewichtes haben wir abgeschätzt indem wir die noch
aufgewickelten Umdrehungen und die ungefähre Position des Zeigers (Angaben von
0 bis 12 Uhr) abgeschätzt haben.

Außerdem haben wir jede Messung mit der Handykamera (Samsung Galaxy S3,
mp4-Format, variable Framerate) aufgenommen um die abgeschätzten Positionen und
evtl. Zeiten zu verfeinern oder vielleicht weitere Daten extrahieren zu können.

\section*{Datenanalyse}
Zur Analyse der Videos hatten wir mehrere Optionen zur Auswahl. Entweder die 60
Videos manuell auszuwerten oder eine Software zu entwickeln, die dies
automatisiert macht.

Da die zweite Variante herrausforderner erschien, haben wir diese gewählt. Die
Software extrahiert automatisch aus den Videos die jeweilige Position des
Zeigers (Winkel) in Abhängigkeit von der Zeit. Wird die Position verloren, wird
anhand von Feature Matching die Position abgeschätzt, was leider nur bei
geringen Geschwindigkeiten gut funktioniert. Wird die Position komplett
verloren, werden Werte mit großen Fehlern ($\pm \pi$) angegeben, und weiter
über Feature Matching geschätzt, bis die Position wiedergefunden wird.

Insgesamt funktioniert der Ansatz gut, so dass wir für 33 der 60 Messungen eine
sehr genaue Ort-Zeit-Kurve extrahieren konnten.

Im Anschluss haben wir manuell aus den Videos die jeweiligen Zeiten bestimmt,
bei denen das Rad losgelassen wurde. Eine automatische Bestimmung war leider
nicht möglich.

Die Bestimmung des Umkehrpunktes, also des Punktes an dem der Faden von der
Rolle komplett abgewickelt war, war mit der Postprocessing-Software möglich
(Quellcode siehe Anhang oder DVD) % FIXME
Die Software hat dann die Daten zwischen Start- und Umkehrpunkt selektiert und
mit neuen Nullpunkten $t' = t - t_\text{Start}$ und $\phi' = \phi -
\phi_\text{Start}$ versehen. Diese Daten nutzen wir in der ersten Aufgabe.

% TODO: Das machen!
Die genaue Funktionsweise der Software ist im Anhang näher beschrieben. Die
analysierten Daten und Ergebnisse und der Quellcode liegen dem Protokoll auf
DVD bei.

\section*{Ergänzte Messwerttabelle}
% TODO: Tabelle

\section*{Aufgabe 2}
Wir bearbeiten zuerst Aufgabe 2, weil man anhand der Messwerte sofort sehen
kann, dass der reibungsfreie Fall überhaupt nicht zutrifft.

Aus der Bewegungsgleichung (Gleichung $\ref{eq_of_motion}$) erhalten wir bei
Annahme der Anfangsbedingungen $\phi_0 = 0$ und $\omega_0 = 0$

\begin{equation}
  \phi(t) = \frac{M}{2I} t^2
\end{equation}

Im ersten Plot haben wir exemplarisch für die Messung 57, bei der der Reibungseffekt
besonders prägnat hervortritt, $\phi$ gegen $t^2$ aufgetragen.

Hier ist kein linearer Zusammenhang erkennbar, so dass es sinnvoll erscheint,
zuerst ein Reibungsmodell aufzustellen und dann die Trägheitsmomente
auszurechnen, weil ansonsten die Reibung das Trägheitsmoment völlig verfälscht.

Wir vermuten hier verschiedene Reibungskräfte bzw. Reibungsdrehmomente $M_R$.

Am meisten ins Gewicht fallen dürfte die Luftreibung der Speichen des Rades. Im
Falle einer laminaren Strömung könnten wir von einer
geschwindigkeitsproportionalen Reibung $M_R \propto \dot{\phi}$ ausgehen.
Aufgrund der geringen Reynolds-Zahl der Luft ist dies hier jedoch nicht
anwendbar. Vielmehr muss es sich um turbulente Strömungen handeln so dass ein
nichtlinearer Strömungswiderstanskoeffizient auftritt, damit wäre das
Reibungsdrehmoment dann $M_R(\dot{\phi})$.

Die Radnabe haben wir nicht genauer untersucht, aber bei Fahrrad-Rädern werden
üblicherweise in der Nabe Wälzlager verbaut, so dass man von einer
geschwindigkeitsabhängigen Rollreibung $M_R(\dot{\phi})$ und einer geringen
Haftreibung am Anfang ausgehen könnte, die insbesondere bei langsamen Drehungen
und hohen auf das Lager wirkenenden Kräften (also bei Verwendung der
Zusatzmassen) ins Gewicht fällt.

Es treten weitere eher als gering anzusehende Reibungskräfte auf, die
Luftreibung des fallenden Gewichtes, diverse Spannungen im Rad und in den
Speichen, Luftreibung der Zusatzmassen, etc.

Da die größte Komponente (die Luftreibung der Speichen) nicht-linear ist,
nähern wir die Reibung durch eine Taylor-Entwicklung in erster Ordnung $
M(\dot{\phi}) = A + B \dot{\phi}$
Da wir von einer Anfangsgeschwindigkeit $\omega_0=0$ ausgehen und ohne Bewegung
auch keine Reibung auftreten kann (Energieerhaltung), setzen wir $A=0$ und
erhalten als Näherung

\begin{equation}
  M_R(t) = \mu \dot{\phi}
\end{equation}

Dabei ist $\mu$ eine zu bestimmende Reibungskonstante.

Dazu stellen wir die Bewegungsgleichung im Lagrange-Formalismus auf und
berücksichtigen Reibungskräfte.

Sei $x$ die negative Höhe der angehängten Masse. $x=0$ entspricht der Maximalen
Höhe, mit zunehmender Dauer fällt die Masse und $x$ steigt. Der Zusammenhang
zum Winkel ist dann $x = r \phi$, wobei $r$ der Radius ist, auf den die Schnur
gewickelt wird.

Es ergibt sich dann für die kinetischen Energie als Summe der kinetischen
Energie des Rades und der angehängten Masse
\begin{equation}
  T = \frac{1}{2} m \dot{x}^2 + \frac{1}{2} I \dot{\phi}^2 = \frac{1}{2} \del{m r^2 + I} \dot{\phi}^2
\end{equation}

Das Potential ist ein Gravitationspotential, woraus sich die Potentielle Energie ergibt
\begin{equation}
  V = -mgx = -mgr \phi
\end{equation}

Durch $Q = -\frac{\partial V}{\partial \phi}$ ergibt sich die generalisierte
Zwangskraft (hier also ein Drehmoment)
\begin{equation}
  Q = mgr
\end{equation}

Statt eine Dissipationsfunktion zu nutzen können wir an dieser Stelle die
Zwangskraft nun direkt als Drehmoment angeben. Wir wählen eine
geschwindigkeitsabhängiges Reibungsdrehmoment
\begin{equation}
  M_R = \mu \dot{\phi} = Q_R
\end{equation}

Über die Lagrange-Gleichung 1. Art
\begin{equation}
  \frac{d}{dt} \frac{\partial T}{\partial \dot{\phi}} - \frac{\partial T}{\partial \phi} = Q
\end{equation}

Erhalten wir die Bewegungsgleichung
\begin{equation}
  (mr^2+I) \ddot{\phi} = mgr - \mu \dot{\phi} \Leftrightarrow
  \frac{mr^2+I}{mgr} \ddot{\phi} = 1 - \frac{\mu}{mgr} \dot{\phi}
\end{equation}

Wir definieren
\begin{equation*}
  A = \frac{mr^2+I}{mgr} \quad B = \frac{\mu}{mgr} \quad \omega = \dot{\phi}
\end{equation*}

und erhalten eine handliche Differentialgleichung
\begin{equation}
  A \frac{d \omega}{dt} + B \omega = 1
\end{equation}

die durch scharfes Hinsehen (oder Trennung der Variablen) gelöst werden kann als
\begin{equation}
  \omega(t) = C \exp\del{- \frac{B}{A} t} + \frac{1}{B}
\end{equation}

Die Anfangsbedingung $\omega(0) = 0$ (Anfangsgeschwindigkeit 0) liefert $C =
-\frac{1}{B}$, somit
\begin{equation}
  \omega(t) = \frac{1}{B} \del{1 - \exp\del{-\frac{B}{A} t}}
\end{equation}

Nochmalige Integration liefert
\begin{equation}
  \phi(t) = \frac{1}{B} \del{t + \frac{A}{B} \exp\del{-\frac{B}{A} t}} + C
\end{equation}

Mit der Anfangsbedingung $\phi(0) = 0$ erhält man $C = -\frac{A}{B^2}$, somit
\begin{equation}
  \phi(t) = \frac{A}{B^2} \del{ \exp\del{-\frac{B}{A} t} - 1} + \frac{t}{B}
\end{equation}

Durch Rücksubstitution erhalten wir die Lösung
\begin{equation}
  \phi(t)
  %= \frac{(mr^2+I)(mgr)}{\mu^2} \del{
  %  \exp{-\frac{\mu}{mr^2 + I} t} - 1
  %} + \frac{t}{\mu} \del{mgr}
  = \del{mgr}
  \del{
    \frac{mr^2+I}{\mu^2}
    \del{
      \exp\del{-\frac{\mu}{mr^2 + I} t} - 1
  } + \frac{t}{\mu}}
\end{equation}

Wir definieren nun unsere Fitparameter als
\begin{equation}
  k = \frac{\mu}{mr^2 + I} \quad
  \omega = \frac{mgr}{\mu}
\end{equation}

Und können damit gegen
\begin{equation}
  \phi(t) = \frac{\omega}{k} \del{e^{-kt} - 1} + \omega t
\end{equation}
fitten.

Man sieht hier übrigens direkt, dass für den reibungsfreien Grenzfall $\mu \to
0$ wieder ein Zusammenhang $\phi \propto t^2$ herauskommt. % FIXME: Hier
Taylor-Entwicklung einfügen.


\newpage
\begin{landscape}
  % gnuplot ./plot/plot-57-quad.gnuplot
  \input{plot-57-quad.tex}
\end{landscape}
\section*{Aufgabe 1}
Wir haben nun alle Daten gegen die obige Gleichung gefittet (Plots siehe weiter
hinten). Dabei haben wir auf eine Linearisierung der Darstellung verzichtet,
denn die Gleichung ist auch logarithmisch auf beiden Seiten von beiden
Parametern abhängig, so dass eine linearisierte Darstellung sinnlos ist. Da
außerdem zwei freie Parameter in der Gleichung vorkommen ist sowieso nicht
klar, wie die Grenzgeraden gewählt werden sollen. Stattdessen haben wir dem
Fehler durch die Methode der kleinsten Quadrate ermittelt, wie dies in der
Experimentalphysik allgemein üblich ist.

\subsection*{Fehlerrechnung}
Durch Umstellen erhalten wir das Trägheitsmoment
\begin{equation}
  I = \frac{mgr}{\omega k} - mr^2
\end{equation}

und die Reibungskonstante
\begin{equation}
  \mu = \frac{mgr}{\omega}
\end{equation}

Der Fehler des Trägheitsmoments ist unter Vernachlässigung des Fehlers der sehr
genau bekannten Gravitationskonstante % Zu wenig Platz für g.
\begin{align*}
  \Delta I
  &= \sqrt{
  \del{\frac{gr}{\omega k} - r^2}^2 \Delta m^2 +
  \del{\frac{mg}{\omega k} - 2mr}^2 \Delta r^2 +
  \del{\frac{mgr}{\omega^2k} \Delta \omega}^2 +
  \del{\frac{mgr}{\omega k^2} \Delta k}^2
  }
\end{align*}

Der Fehler der Reibungskonstante ist (ebenfalls unter Vernachlässigung der
Gravitationskonstante)
\begin{equation}
  \Delta \mu = \sqrt{
    \del{\frac{gr}{\omega} \Delta m}^2 +
    \del{\frac{mg}{\omega} \Delta r}^2 +
    \del{\frac{mgr}{\omega^2} \Delta \omega}^2
  }
\end{equation}

\subsection*{Berechnete Werte}

Somit erhalten wir folgende Wertetabelle

\begin{tabular}{l|r|r|r|r|r}
Messung & $A$ & $B$ & $C$ & $I$ & $\mu$ \\
1 & $\num{9.3+-0.31}$ & $\num{0.198+-0.0044}$ & $\num{1.86+-0.023}$, & $\num{0.83+-0.018}$ & $\num{0.165+-0.0021}$ \\
3 & $\num{14.2+-0.98}$ & $\num{0.162+-0.008}$ & $\num{2.31+-0.058}$, & $\num{0.82+-0.039}$ & $\num{0.133+-0.0033}$ \\
4 & $\num{200+-19.0}$ & $\num{0.039+-0.0027}$ & $\num{6.0+-0.34}$, & $\num{1.3+-0.16}$ & $\num{0.051+-0.003}$ \\
5 & $\num{120+-4.6}$ & $\num{0.045+-0.001}$ & $\num{5.43+-0.088}$, & $\num{1.25+-0.043}$ & $\num{0.0567+-0.00094}$ \\
6 & $\num{300+-11.0}$ & $\num{0.0283+-0.00062}$ & $\num{7.6+-0.13}$, & $\num{1.42+-0.059}$ & $\num{0.0406+-0.00075}$ \\
8 & $\num{1000+-280.0}$ & $\num{0.033+-0.0087}$ & $\num{20+-4.4}$, & $\num{0+-1.2}$ & $\num{0.04+-0.01}$ \\
9 & $\num{60+-3.7}$ & $\num{0.115+-0.0042}$ & $\num{6.8+-0.18}$, & $\num{0.94+-0.07}$ & $\num{0.11+-0.003}$ \\
10 & $\num{400+-54.0}$ & $\num{0.041+-0.0034}$ & $\num{10+-1.1}$, & $\num{1.2+-0.31}$ & $\num{0.051+-0.0037}$ \\
12 & $\num{600+-38.0}$ & $\num{0.0301+-0.00095}$ & $\num{19.4+-0.53}$, & $\num{1.3+-0.14}$ & $\num{0.038+-0.0012}$ \\
13 & $\num{50+-4.9}$ & $\num{0.169+-0.0093}$ & $\num{8.4+-0.37}$, & $\num{1.0+-0.13}$ & $\num{0.175+-0.0077}$ \\
14 & $\num{20+-3.5}$ & $\num{0.34+-0.048}$ & $\num{4.9+-0.46}$, & $\num{0.8+-0.21}$ & $\num{0.3+-0.028}$ \\
15 & $\num{500+-43.0}$ & $\num{0.051+-0.0024}$ & $\num{24+-1.0}$, & $\num{1.2+-0.23}$ & $\num{0.062+-0.0028}$ \\
16 & $\num{100+-19.0}$ & $\num{0.104+-0.0083}$ & $\num{13.5+-0.88}$, & $\num{1.0+-0.25}$ & $\num{0.109+-0.0072}$ \\
17 & $\num{600+-55.0}$ & $\num{0.044+-0.002}$ & $\num{30+-1.1}$, & $\num{1.2+-0.23}$ & $\num{0.052+-0.0023}$ \\
19 & $\num{100+-25.0}$ & $\num{0.12+-0.013}$ & $\num{10+-1.4}$, & $\num{1.1+-0.39}$ & $\num{0.14+-0.013}$ \\
20 & $\num{100+-27.0}$ & $\num{0.14+-0.022}$ & $\num{10+-1.7}$, & $\num{1.1+-0.51}$ & $\num{0.16+-0.021}$ \\
21 & $\num{100+-21.0}$ & $\num{0.113+-0.009}$ & $\num{20+-1.1}$, & $\num{1.0+-0.28}$ & $\num{0.124+-0.0083}$ \\
22 & $\num{1000+-200.0}$ & $\num{0.043+-0.0049}$ & $\num{40+-4.1}$, & $\num{1.2+-0.74}$ & $\num{0.053+-0.0059}$ \\
24 & $\num{1000+-190.0}$ & $\num{0.033+-0.0023}$ & $\num{50+-3.0}$, & $\num{1.2+-0.48}$ & $\num{0.042+-0.0029}$ \\
31 & $\num{80+-7.8}$ & $\num{0.095+-0.005}$ & $\num{7.7+-0.34}$, & $\num{3.6+-0.24}$ & $\num{0.35+-0.015}$ \\
34 & $\num{1000+-170.0}$ & $\num{0.031+-0.004}$ & $\num{20+-2.5}$, & $\num{0+-1.1}$ & $\num{0.13+-0.015}$ \\
40 & $\num{100+-20.0}$ & $\num{0.062+-0.0046}$ & $\num{9.0+-0.56}$, & $\num{3.6+-0.36}$ & $\num{0.23+-0.014}$ \\
41 & $\num{1000+-130.0}$ & $\num{0.019+-0.001}$ & $\num{30+-1.2}$, & $\num{4.1+-0.48}$ & $\num{0.081+-0.004}$ \\
46 & $\num{1000+-190.0}$ & $\num{0.019+-0.002}$ & $\num{20+-1.7}$, & $\num{4.3+-0.86}$ & $\num{0.082+-0.008}$ \\
47 & $\num{1100+-91.0}$ & $\num{0.0181+-0.00082}$ & $\num{19.4+-0.78}$, & $\num{4.2+-0.36}$ & $\num{0.076+-0.0032}$ \\
49 & $\num{20+-1.7}$ & $\num{0.102+-0.0043}$ & $\num{2.44+-0.074}$, & $\num{2.97+-0.085}$ & $\num{0.306+-0.0093}$ \\
52 & $\num{1000+-250.0}$ & $\num{0.013+-0.0016}$ & $\num{10+-1.5}$, & $\num{4.6+-0.93}$ & $\num{0.058+-0.007}$ \\
53 & $\num{500+-25.0}$ & $\num{0.0182+-0.00047}$ & $\num{9.7+-0.21}$, & $\num{4.2+-0.14}$ & $\num{0.077+-0.0018}$ \\
54 & $\num{1100+-59.0}$ & $\num{0.0122+-0.00034}$ & $\num{13.6+-0.34}$, & $\num{4.5+-0.2}$ & $\num{0.055+-0.0015}$ \\
55 & $\num{30+-2.2}$ & $\num{0.051+-0.0021}$ & $\num{1.78+-0.045}$, & $\num{3.39+-0.065}$ & $\num{0.173+-0.0044}$ \\
57 & $\num{21.5+-0.69}$ & $\num{0.069+-0.0015}$ & $\num{1.48+-0.019}$, & $\num{2.98+-0.028}$ & $\num{0.208+-0.0026}$ \\
60 & $\num{2000+-530.0}$ & $\num{0.0052+-0.00072}$ & $\num{10+-1.3}$, & $\num{6+-1.0}$ & $\num{0.031+-0.004}$ \\
\end{tabular}


\subsection*{Interpretation und Ergebnis}
Die stark schwankenden und untereinander unverträglichen Werte für $\mu$ lassen
auf ein inkorrektes Reibungsmodell schließen. Die Reibung scheint nicht linear
von der Geschwindigkeit abzuhängen, sondern einer unbekannten Gesetzmäßigkeit
zu folgen, möglicherweise einem Haft- und Gleitreibungsanteil. Insbesondere am
Plott zu Messung 60 sieht man den großen Haftreibungsanteil, der gerade zu
Anfang vom Modell abweicht.

Mit den gegebenen Daten ein exaktes Reibungsmodell aufzustellen würde in pure
Spekulation ausarten. Auch wären die auftretenden Differentialgleichungen
wahrscheinlich nicht elementar lösbar.

Aufgrund des inkorrekten Reibungsmodells wundert es auch nicht, dass die Fehler
für $I$ inkonsistent sind. Dennoch ist unser Modell eine viel bessere
Approximation als die Bewegungsgleichungen für den reibungsfreien Fall. Wenn
wir also die Fehler ignorieren und die ermittelten Werte für $I$ als
Stichproben betrachten, können wir über Mittelwert und Standardverteilung
trotzdem Werte abschätzen. Der Wert des Trägheitsmoments ohne Zusatzmassen
lautet demnach

\begin{equation}
  I_0 = \SI{0.180+-0.036}{kg m^2}
\end{equation}

und mit Zusatzmassen
\begin{equation}
  I = \SI{0.58+-0.13}{kg m^2}
\end{equation}

\subsection*{Vergleich mit dem Steinerschen Satz}
Wir berechnen zuerst das Trägheitsmoment der Zusatzmassen. Wenn wir diese
näherungsweise als Vollzylinder annehmen ist das

\begin{equation}
  I = \frac{1}{2} m r^2 = \frac{1}{8} mD^2 \Rightarrow
  \Delta I = \sqrt{\del{\frac{D}{8} \Delta m}^2 + \del{\frac{md}{4} \Delta D}^2}
\end{equation}

Wir haben die Zylinder mit den Massen $M_1$ und $M_3$ benztzt. Der gemessene
Durchmesser ist für alle Zylinder jeweils $D = \SI{55.0+-0.1}{mm}$ (der Fehler
ergibt sich aus der Messung mit der Schiebelehre). Für die Waage legen wir
einen Ablesefehler von $\SI{0.2}{g}$ ($\SI{0.1}{g}$ für Werte die exakt bei
ganzen Gramm abgelesen werden) und eine Genauigkeit von $\SI{0.1}{g}$ zugrunde
und erhalten die Fehler. Die Massen sind also $M_1 = \SI{981.4+-0.23}{g}$ und $M_3 =
\SI{981.0+-0.14}{g}$.

Somit ergeben sich die Trägheitsmomente
\begin{align*}
  I_1 &= \SI{0.3711+-0.0017}{g m^2} \\
  I_3 &= \SI{0.3709+-0.0010}{g m^2}
\end{align*}

Den Abstand zur Drehachse haben wir gemessen, indem wir zuerst mit der
Schiebelehre den Durchmesser der mittleren Außenkante der Radnabe gemessen
haben ($d_1 = \SI{77.7+-0.1}{mm}$), dann mit dem Lineal bis zur Außenkante des
Zusatzgewichts ($d_2 = \SI{277.5+-0.5}{mm}$). Zusätzlich müssen wir noch den
Radius des Zusatzgewichts addieren um den Abstand zur Drehachse zu erhalten:

\begin{equation}
  d = \SI{382.70\pm0.52}{mm}
\end{equation}

Über den Steinerschen Satz
\begin{equation}
  I = I_1 + M_1 d^2 + I_3 + M_3 d^2 \Rightarrow
  \Delta I = \sqrt{\Delta I_1^2 + \Delta I_3^2 + \del{d^2 \Delta M_1}^2 +
  \del{d^2 \Delta M_3}^2 + \del{2d (M_1+M_2) \Delta d}^2}
\end{equation}
erhalten wir das zusätzliche Trägheitsmoment
\begin{equation}
  % in g*m^2
  % 0.3711+0.3709+(981.4+981.0)*0.3827^2
  % sqrt(0.0017^2+0.0010^2+(0.3827^2*0.23)^2+(0.3827^2*0.14)^2+
  % (2*0.3827*(981.40+981.00)*0.00052)^2)
  I' = \SI{0.28815+-0.00079}{kg m^2}
\end{equation}

Aus den experimentellen Daten erhalten wir hingegen
\begin{equation}
  I' = I - I_0 = \SI{0.180+-0.036}{kg m^2} - \SI{0.58+-0.13}{kg m^2} = \SI{0.46+-0.13}{kg m^2}
\end{equation}

Die beiden Werte sind verträglich.

\newpage
\begin{landscape}
  % gnuplot ./plot/plot-41-exp.gnuplot
  \input{plot-41-exp.tex}
\end{landscape}

\newpage
\begin{landscape}
  % gnuplot ./plot/plot-34-exp.gnuplot
  \input{plot-34-exp.tex}
\end{landscape}

\newpage
\begin{landscape}
  % gnuplot ./plot/plot-24-exp.gnuplot
  \input{plot-24-exp.tex}
\end{landscape}

\newpage
\begin{landscape}
  % gnuplot ./plot/plot-20-exp.gnuplot
  \input{plot-20-exp.tex}
\end{landscape}

\newpage
\begin{landscape}
  % gnuplot ./plot/plot-8-exp.gnuplot
  \input{plot-8-exp.tex}
\end{landscape}

\newpage
\begin{landscape}
  % gnuplot ./plot/plot-1-exp.gnuplot
  \input{plot-1-exp.tex}
\end{landscape}

\newpage
\begin{landscape}
  % gnuplot ./plot/plot-13-exp.gnuplot
  \input{plot-13-exp.tex}
\end{landscape}

\newpage
\begin{landscape}
  % gnuplot ./plot/plot-16-exp.gnuplot
  \input{plot-16-exp.tex}
\end{landscape}

\newpage
\begin{landscape}
  % gnuplot ./plot/plot-14-exp.gnuplot
  \input{plot-14-exp.tex}
\end{landscape}

\newpage
\begin{landscape}
  % gnuplot ./plot/plot-57-exp.gnuplot
  \input{plot-57-exp.tex}
\end{landscape}

\newpage
\begin{landscape}
  % gnuplot ./plot/plot-12-exp.gnuplot
  \input{plot-12-exp.tex}
\end{landscape}

\newpage
\begin{landscape}
  % gnuplot ./plot/plot-19-exp.gnuplot
  \input{plot-19-exp.tex}
\end{landscape}

\newpage
\begin{landscape}
  % gnuplot ./plot/plot-10-exp.gnuplot
  \input{plot-10-exp.tex}
\end{landscape}

\newpage
\begin{landscape}
  % gnuplot ./plot/plot-47-exp.gnuplot
  \input{plot-47-exp.tex}
\end{landscape}

\newpage
\begin{landscape}
  % gnuplot ./plot/plot-22-exp.gnuplot
  \input{plot-22-exp.tex}
\end{landscape}

\newpage
\begin{landscape}
  % gnuplot ./plot/plot-55-exp.gnuplot
  \input{plot-55-exp.tex}
\end{landscape}

\newpage
\begin{landscape}
  % gnuplot ./plot/plot-4-exp.gnuplot
  \input{plot-4-exp.tex}
\end{landscape}

\newpage
\begin{landscape}
  % gnuplot ./plot/plot-46-exp.gnuplot
  \input{plot-46-exp.tex}
\end{landscape}

\newpage
\begin{landscape}
  % gnuplot ./plot/plot-5-exp.gnuplot
  \input{plot-5-exp.tex}
\end{landscape}

\newpage
\begin{landscape}
  % gnuplot ./plot/plot-21-exp.gnuplot
  \input{plot-21-exp.tex}
\end{landscape}

\newpage
\begin{landscape}
  % gnuplot ./plot/plot-31-exp.gnuplot
  \input{plot-31-exp.tex}
\end{landscape}

\newpage
\begin{landscape}
  % gnuplot ./plot/plot-60-exp.gnuplot
  \input{plot-60-exp.tex}
\end{landscape}

\newpage
\begin{landscape}
  % gnuplot ./plot/plot-40-exp.gnuplot
  \input{plot-40-exp.tex}
\end{landscape}

\newpage
\begin{landscape}
  % gnuplot ./plot/plot-6-exp.gnuplot
  \input{plot-6-exp.tex}
\end{landscape}

\newpage
\begin{landscape}
  % gnuplot ./plot/plot-49-exp.gnuplot
  \input{plot-49-exp.tex}
\end{landscape}

\newpage
\begin{landscape}
  % gnuplot ./plot/plot-52-exp.gnuplot
  \input{plot-52-exp.tex}
\end{landscape}

\newpage
\begin{landscape}
  % gnuplot ./plot/plot-17-exp.gnuplot
  \input{plot-17-exp.tex}
\end{landscape}

\newpage
\begin{landscape}
  % gnuplot ./plot/plot-9-exp.gnuplot
  \input{plot-9-exp.tex}
\end{landscape}

\newpage
\begin{landscape}
  % gnuplot ./plot/plot-3-exp.gnuplot
  \input{plot-3-exp.tex}
\end{landscape}

\newpage
\begin{landscape}
  % gnuplot ./plot/plot-54-exp.gnuplot
  \input{plot-54-exp.tex}
\end{landscape}

\newpage
\begin{landscape}
  % gnuplot ./plot/plot-53-exp.gnuplot
  \input{plot-53-exp.tex}
\end{landscape}

\newpage
\begin{landscape}
  % gnuplot ./plot/plot-15-exp.gnuplot
  \input{plot-15-exp.tex}
\end{landscape}


\end{document}
