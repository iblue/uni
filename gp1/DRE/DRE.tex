\documentclass[a4paper,german,12pt,smallheadings]{scrartcl}
\usepackage[T1]{fontenc}
\usepackage[utf8]{inputenc}
\usepackage{babel}
\usepackage{geometry}
\usepackage[fleqn]{amsmath}
\usepackage{amssymb}
\usepackage{float}
\usepackage{enumerate}
\usepackage{listings} % Source code
\usepackage{lscape} % landscape
\usepackage{commath} % http://tex.stackexchange.com/questions/14821/whats-the-proper-way-to-typeset-a-differential-operator
\usepackage{cancel}
\usepackage[fleqn]{mathtools}
% Number only referenced equations
%\mathtoolsset{showonlyrefs}

%\usepackage{wrapfig}
\usepackage{siunitx}
\sisetup{separate-uncertainty=true,locale=DE}

% New command for color underlining
\usepackage{xcolor}

\newsavebox\MBox
\newcommand\colul[2][red]{{\sbox\MBox{$#2$}%
  \rlap{\usebox\MBox}\color{#1}\rule[-1.2\dp\MBox]{\wd\MBox}{0.5pt}}}

\restylefloat{table}
\geometry{a4paper, top=15mm, left=20mm, right=10mm, bottom=20mm, headsep=10mm, footskip=12mm}
\linespread{1.5}
\setlength\parindent{0pt}
\DeclareMathOperator{\Tr}{Tr}
\DeclareMathOperator{\Var}{Var}
\begin{document}
\allowdisplaybreaks % Seitenumbrüche in Formeln erlauben
\begin{center}
\bfseries % Fettdruck einschalten
\sffamily % Serifenlose Schrift
\vspace{-40pt}
Physikalisches Grundpraktikum 1, Sommersemester 2014

Markus Fenske \texttt{<iblue@zedat.fu-berlin.de>}

Paul Rahmann \texttt{<paulrahmann@zedat.fu-berlin.de>}

Drehbewegungen, Tutor: Christian Hindermann
\vspace{-10pt}
\end{center}

\section*{Physikalische Grundlagen}
Das Trägheitsmoment wird durch $F = \dot{p} = m \dot{v}$ motiviert. Wenn auf eine Masse $m$
im Abstand $r$ von der Drehachse eine Kraft wirkt, folgt

\begin{equation}
  \vec{r} \times \vec{F}
  = \vec{r} \times m \dot{\vec{v}}
\end{equation}

mit $v = r \omega$ und $r$ konstant und weil die Drehachse senkrecht auf Kraft
und Ortsvektor steht

\begin{equation}
  = \dot{\vec{\omega}} r^2 m
\end{equation}

Durch Überführung von $m$ in ein Massenelement $\dif m$ folgt

\begin{equation}
  \underbrace{\vec{r} \times \vec{F}}_{=: \vec{M}} = \dot{\vec{\omega}} \underbrace{\int \dif m \; r^2}_{=: I}
\end{equation}

Wir verwenden ab hier nur die Beträge, da alle Vektoren jeweils senkrecht
aufeinander stehen. Die Richtungen können dann aus der Rechte-Hand-Regel
ermittelt werden.

Die Bewegungsgleichung für die Drehbewegung eines starren Körpers lautet also gemäß der Herleitung
\begin{equation}
  M = I \dot{\omega} = I \ddot{\phi}
\end{equation}

Analog zur Newtonschen Bewegungsgleichung für lineare Bewegungen unter Einfluss
einer konstanten Kraft ist die Lösung dieser Differentialgleichung unter
Annahme eines konstanten Drehmoments $M$
\begin{equation}
  \phi(t) = \frac{M}{2I}t^2 + \omega_0 t + \phi_0
\end{equation}

Dabei ist $M$ das konstante Drehmoment, $I$ das Trägheitsmoment, $\omega_0$ die
Anfangswinkelgeschwindigkeit und $\phi_0$ der Anfangswinkel. Als Konsequenz
ergibt sich unter Einfluss einer konstanten Kraft also eine lineare Zunahme der
Winkelgeschwindigkeit
\begin{equation}
  \omega(t) = \frac{M}{I} t + \omega_0
\end{equation}

Das Drehmoment im Schwerpunktkoordinatensystem ist

\begin{equation}
  I = \int \dif m \; r^2
\end{equation}

Sei die Drehachse oBdA $\omega \hat{z}$. Verschieben wir die Drehachse um
$(a_x, a_y)$ erhalten wir

\begin{align*}
  I_a &= \int \dif m \; \del{(x - a_x)^2 + (y - a_y)^2} \\
      &= \int \dif m \; \del{x^2 - 2xa_x + a_x^2 + y^2 - 2ya_y + a_y^2} \\
      &= \int \dif m \; \del{x^2 + y^2}
       + 2 a_x \int \dif m \; x
       + 2 a_y \int \dif m \; y
       + \del{a_x^2 + a_y^2} \int \dif m
\end{align*}

Der erste Term ist das Drehmoment um den Schwerpunkt, die nächsten beiden
Integrale sind die $x$- und die $y$-Koordinate des Schwerpunktes im
Schwerpunktsystem. Da in diesem der Schwerpunkt im Ursprung liegt, verschwinden
sie. Das nächste Integral ist einfach die Gesamtmasse. Somit

\begin{equation}
  I_a = I + ma^2
\end{equation}

Dies ist der Steinersche Satz.

Für die Aufgabenstellung ist außerdem die Berechnung des Trägheitsmoments eines
starren Zylinders gefordert. Ein Vollzylinder wird dem experimentellen Aufbau
nicht gerecht, denn die Masse konzentriert sie unseren Erachtens nach auf den
Reifen, nicht auf die Speichen. Gehen wir also von einer Massenverteilung $\rho
= \frac{m}{2 \pi R} \delta(r-R)\delta(z)$ aus, erhalten wir

\begin{equation}
  I = \int \dif V \; r^2 \rho = m R^2
\end{equation}

\section*{Aufgaben}
%Untersuchung gleichmäßig beschleunigter Drehbewegungen
%\begin{itemize}
%  \item Messung von Weg-Zeit-Abhängigkeiten
%  \item Messung von Drehmoment-Zeit-Abhängigkeiten
%  \item Messung der Reibungsverluste
%\end{itemize}
%für unterschiedliche Trägheitsmomente (mit und ohne Zusatzmassen).
\begin{enumerate}[1.]
  \item Qualitative und quantitative Überprüfung des Bewegungsgesetzes. Messung
    der Zeit in Abhängigkeit des Drehwinkels (bei festem Drehmoment) und in
    Abhängigkeit des Drehmoments (bei festem Drehwinkel). Bestimmung der
    Trägheitsmomente (mit und ohne Zusatzmassen) des Rades aus den Messungen
    und Vergleich mit dem berechneten Wert aus dem \textit{Steinerschen Satz}.
  \item
    Diskussion von Reibungseinflüssen und Reibungsmodellen (Abhängigkeit von
    Reibungskräften bzw. Reibungsmomenten von den verschiedenen
    Bewegungsparametern) aus den Ergebnissen der Messungen.
\end{enumerate}
\end{document}
