\documentclass[a4paper,german,12pt,smallheadings]{scrartcl}
\usepackage[T1]{fontenc}
\usepackage[utf8]{inputenc}
\usepackage{babel}
\usepackage{tikz}
\usepackage{pgfplots}
\usepackage{geometry}
\usepackage[fleqn]{amsmath}
\usepackage{amssymb}
\usepackage{float}
\usepackage{enumerate}
\usepackage{commath} % http://tex.stackexchange.com/questions/14821/whats-the-proper-way-to-typeset-a-differential-operator
\usepackage{cancel}

% Number only referenced equations
\usepackage[fleqn]{mathtools}
\mathtoolsset{showonlyrefs}

%\usepackage{wrapfig}
\usepackage[thinspace,thinqspace,squaren,textstyle]{SIunits}

% New command for color underlining
\usepackage{xcolor}

\newsavebox\MBox
\newcommand\colul[2][red]{{\sbox\MBox{$#2$}%
  \rlap{\usebox\MBox}\color{#1}\rule[-1.2\dp\MBox]{\wd\MBox}{0.5pt}}}

\restylefloat{table}
\geometry{a4paper, top=15mm, left=10mm, right=20mm, bottom=20mm, headsep=10mm, footskip=12mm}
\linespread{1.5}
\setlength\parindent{0pt}
\DeclareMathOperator{\Tr}{Tr}
\DeclareMathOperator{\Var}{Var}
\begin{document}
\allowdisplaybreaks % Seitenumbrüche in Formeln erlauben
\begin{center}
\bfseries % Fettdruck einschalten
\sffamily % Serifenlose Schrift
\vspace{-40pt}
Festkörperphysik, Sommersemester 2014, Aufgabenblatt 1

Markus Fenske, Tutor: Marko Wietstruk
\vspace{-10pt}
\end{center}
\section*{Aufgabe 2: Gleichgewichtsabstand der van-der-Waals-Kraft}

\begin{enumerate}[a)]
  \item
    Die Taylorreihe hier ist

    \begin{equation}
      H_1(x_1, x_2) = H_1(0,0)
        +             \frac{\partial H_1}{\partial x_1}                (0,0) x_1
        +             \frac{\partial H_1}{\partial x_2}                (0,0) x_2
        + \frac{1}{2} \frac{\partial^2 H_1}{\partial x_1^2}            (0,0) x_1^2
        + \frac{1}{2} \frac{\partial^2 H_1}{\partial x_2^2}            (0,0) x_2^2
        + \frac{\partial^2 H_1}{\partial x_1 \partial x_2} (0,0) x_1x_2
        + \dots
    \end{equation}

    Die einzelnen Komponenten sind
    \begin{align}
      &H_1(0,0) = 0 \\
      &\frac{\partial H_1}{\partial x_1}(0,0) x_1
        = \eval{\del{\frac{-e^2}{(R+x_2-x_1)^2} + \frac{e^2}{(R-x_1)^2}}}_{x_1=0, x_2=0} x_1 = 0 \\
      &\frac{\partial H_1}{\partial x_2}(0,0) x_2
        = \eval{\del{\frac{-e^2}{(R+x_2-x_1)^2} + \frac{e^2}{(R-x_2)^2}}}_{x_1=0, x_2=0} x_2 = 0 \\
      &\frac{\partial^2 H_1}{\partial x_1^2}(0,0) x_1^2
        = \eval{\del{\frac{2e^2}{(R+x_2-x_1)^3} - \frac{2e^2}{(R-x_1)^3}}}_{x_1=0, x_2=0} x_1^2 = 0 \\
      &\frac{\partial^2 H_1}{\partial x_2^2}(0,0) x_2^2
        = \eval{\del{\frac{2e^2}{(R+x_2-x_1)^3} - \frac{2e^2}{(R-x_2)^3}}}_{x_1=0, x_2=0} x_2^2 = 0 \\
      &\frac{\partial^2 H_1}{\partial x_1 \partial_2}(0,0) x_1x_2
        = \eval{-\frac{2e^2}{(R+x_2-x_1)^3}}_{x_1=0, x_2=0} x_1x_2 = -\frac{2e^2}{R^3} x_1 x_2 \\
    \end{align}

    Somit ergibt sich für die Taylor-Reihe in der ersten von null verschiedenen Ordnung:
    \begin{align}
      H_1(x_1, x_2) \approx  -\frac{2e^2}{R^3} x_1 x_2
    \end{align}
\end{enumerate}
\end{document}
