\documentclass[a4paper,german,12pt,smallheadings]{scrartcl}
\usepackage[T1]{fontenc}
\usepackage[utf8]{inputenc}
\usepackage{babel}
\usepackage{tikz}
\usepackage{pgfplots}
\usepackage{geometry}
\usepackage[fleqn]{amsmath}
\usepackage{amssymb}
\usepackage{float}
\usepackage{enumerate}
\usepackage{commath} % http://tex.stackexchange.com/questions/14821/whats-the-proper-way-to-typeset-a-differential-operator
\usepackage{cancel}

% Number only referenced equations
\usepackage[fleqn]{mathtools}
\mathtoolsset{showonlyrefs}

%\usepackage{wrapfig}
\usepackage[thinspace,thinqspace,squaren,textstyle]{SIunits}

% New command for color underlining
\usepackage{xcolor}

\newsavebox\MBox
\newcommand\colul[2][red]{{\sbox\MBox{$#2$}%
  \rlap{\usebox\MBox}\color{#1}\rule[-1.2\dp\MBox]{\wd\MBox}{0.5pt}}}

\restylefloat{table}
\geometry{a4paper, top=15mm, left=10mm, right=20mm, bottom=20mm, headsep=10mm, footskip=12mm}
\linespread{1.5}
\setlength\parindent{0pt}
\DeclareMathOperator{\Tr}{Tr}
\DeclareMathOperator{\Var}{Var}
\begin{document}
\allowdisplaybreaks % Seitenumbrüche in Formeln erlauben
\begin{center}
\bfseries % Fettdruck einschalten
\sffamily % Serifenlose Schrift
\vspace{-40pt}
Festkörperphysik, Sommersemester 2014, Aufgabenblatt 1

Markus Fenske, Luis Herrmann, Tutor: Marko Wietstruk
\vspace{-10pt}
\end{center}
\section*{Aufgabe 1: Van-der-Waals-Kräfte: Klassische Betrachtung}
\begin{align}
  &W = -\frac{1}{2} \vec{p}' \vec{E}(\vec{r}), \vec{p}' = \alpha \vec{E} \\
  \Rightarrow \quad &W = - \frac{1}{2} \alpha \vec{E}(\vec{r}) \cdot \vec{E}(\vec{r}) \\
                    &\;\;\;\, = - \frac{\alpha}{2} E(r)^2 \\
  &\;\;\;\,
  = \frac{\alpha}{32 \pi^2 \epsilon_0^2} \del{
    \vec{p}{r^3} - 3 \frac{\vec{p} \cdot \vec{r}}{r^5} \vec{r}
  }^2 \\
  &\;\;\;\,
  = \frac{\alpha}{32 \pi^2 \epsilon_0^2} \del{
    \frac{p^2}{r^6} + 9 \frac{(\vec{p} \cdot \vec{r})^2}{r^{10}} r^2
    - 6 \frac{\del{\vec{p} \cdot \vec{r}} \del{\vec{p} \cdot \vec{r}}}{r^8}
  } \\
  &\;\;\;\,
  = \frac{\alpha}{32 \pi^2 \epsilon_0^2} \del{
    \frac{p^2}{r^6} - \frac{6p^2r^2 \cos^2 \beta}{r^8} + \frac{9 p^2 r^2 \cos^2 \beta}{r^8}
  } \\
  &\;\;\;\,
  = \frac{\alpha}{32 \pi^2 \epsilon_0^2} \del{
    \frac{p^2}{r^6} + \frac{3p^2r^2 \cos^2 \beta}{r^8}
  } \\
  &\;\;\;\,
  = \frac{\alpha}{32 \pi^2 \epsilon_0^2} \del{
    \frac{p^2}{r^6} + \frac{3p^2 \cos^2 \beta}{r^6}
  } \\
  &\;\;\;\,
  = \frac{\alpha}{32 \pi^2 \epsilon_0^2} \frac{p^2}{r^6} \del{
    1 + 3 \cos^2 \beta
  } \\
\end{align}
wobei $\beta$ der Winkel zwischen $\vec{r}$ und $\vec{p}$ sei.

Winkelabhängig ergeben sich die Extrema
\begin{align}
  \text{Minimum: } & \alpha = \frac{\pi}{2} \; \text{ wenn } \vec{r} \perp \vec{p} \\
  \text{Maximum: } & \alpha = \pi \;\; \text{ wenn } \vec{r} \parallel \vec{p} \\
\end{align}

Mit den Werten
\begin{equation}
  p = 10^{-29} \; \ampere \second \meter, \alpha = 10^{-40} \ampere \second \meter^2 \volt^{-1}, r = 10^{-9} \meter
\end{equation}

erhält man
\begin{equation}
  V(r)' = 10^{-40} \ampere \second \meter^2 \volt^{-1} \cdot \frac{10^{-58} \ampere^2 \second^2 \meter^2}{10^{-54} \meter^6}
  \cdot \frac{1}{32 \pi^2 \del{8{,}854 \cdot 10^{-12}}^2 \ampere^2 \second^2 \volt^{-2} \meter^{-2}}
  \approx 4{,}039 \; \coulomb \volt = 4{,}039 \; \joule
\end{equation}

\section*{Aufgabe 2: Gleichgewichtsabstand der van-der-Waals-Kraft}
\begin{enumerate}[a)]
  \item
    Die Taylorreihe hier ist

    \begin{equation}
      H_1(x_1, x_2) = H_1(0,0)
        +             \frac{\partial H_1}{\partial x_1}                (0,0) x_1
        +             \frac{\partial H_1}{\partial x_2}                (0,0) x_2
        + \frac{1}{2} \frac{\partial^2 H_1}{\partial x_1^2}            (0,0) x_1^2
        + \frac{1}{2} \frac{\partial^2 H_1}{\partial x_2^2}            (0,0) x_2^2
        + \frac{\partial^2 H_1}{\partial x_1 \partial x_2} (0,0) x_1x_2
        + \dots
    \end{equation}

    Die einzelnen Komponenten sind
    \begin{align}
      &H_1(0,0) = 0 \\
      &\frac{\partial H_1}{\partial x_1}(0,0) x_1
        = \eval{\del{\frac{-e^2}{(R+x_2-x_1)^2} + \frac{e^2}{(R-x_1)^2}}}_{x_1=0, x_2=0} x_1 = 0 \\
      &\frac{\partial H_1}{\partial x_2}(0,0) x_2
        = \eval{\del{\frac{-e^2}{(R+x_2-x_1)^2} + \frac{e^2}{(R-x_2)^2}}}_{x_1=0, x_2=0} x_2 = 0 \\
      &\frac{\partial^2 H_1}{\partial x_1^2}(0,0) x_1^2
        = \eval{\del{\frac{2e^2}{(R+x_2-x_1)^3} - \frac{2e^2}{(R-x_1)^3}}}_{x_1=0, x_2=0} x_1^2 = 0 \\
      &\frac{\partial^2 H_1}{\partial x_2^2}(0,0) x_2^2
        = \eval{\del{\frac{2e^2}{(R+x_2-x_1)^3} - \frac{2e^2}{(R-x_2)^3}}}_{x_1=0, x_2=0} x_2^2 = 0 \\
      &\frac{\partial^2 H_1}{\partial x_1 \partial_2}(0,0) x_1x_2
        = \eval{-\frac{2e^2}{(R+x_2-x_1)^3}}_{x_1=0, x_2=0} x_1x_2 = -\frac{2e^2}{R^3} x_1 x_2 \\
    \end{align}

    Somit ergibt sich für die Taylor-Reihe in der ersten von null verschiedenen Ordnung:
    \begin{align}
      H_1(x_1, x_2) \approx  -\frac{2e^2}{R^3} x_1 x_2
    \end{align}
  \item
    Das Leonard-Jones-Potential ist gegeben durch
    \begin{equation}
      U(R) = 4 \epsilon \del{\del{\frac{\sigma}{R}}^{12} - \del{\frac{\sigma}{R}}^6}
    \end{equation}

    Gesucht ist das Minimum des Potentials. Dies ist bei
    \begin{align}
      &\frac{\partial U}{\partial R} \overset{!}{=} 0 \\
      \Leftrightarrow \quad &
      4 \epsilon \del{-12\frac{\sigma^{12}}{R^{13}} + \frac{6\sigma^6}{R^7}} = 0 \\
      \Leftrightarrow \quad &
      12\frac{\sigma^{12}}{R^{13}} = \frac{6\sigma^6}{R^7} \\
      \Leftrightarrow \quad &
      2 \sigma^6 = R^6 \\
      \Rightarrow \quad & R = \sqrt[6]{2} \sigma
    \end{align}

    Wobei $\sigma$ der Abstand $r$ derart ist, dass $U(\sigma) = 0$.

    Folie 3, Blatt 1 entnimmt man

    \begin{tabular}{l|c|c|c}
      & $R$ & $\sigma$ & $R/\sigma$ \\
      \hline
      Ne & 3{,}31 & 2{,}74 & 1{,}14 \\
      Ar & 3{,}76 & 3{,}40 & 1{,}11 \\
      Kr & 4{,}01 & 3{,}65 & 1{,}10 \\
      Xe & 4{,}35 & 3{,}98 & 1{,}09
    \end{tabular}

    Der berechnete Wert $R/\sigma = \sqrt[6]{2} \approx 1{,}12$ weicht
    nur gering davon ab. Die Abweichung für Neon ist +1{,}16 \%, während die
    Abweichung für Krypton -2{,}85 \% beträgt.
\end{enumerate}

\section*{Aufgabe 3: Ionenkristall in zwei Dimensionen}

Zu berechnen ist die Madelung-Konstante
\begin{equation}
  \alpha_R = \sum_{j \neq i} \frac{\pm 1}{P_{ij}} = R \cdot \sum_{j \neq i} \frac{\pm 1}{r_{ij}}
\end{equation}

\begin{equation}
  \alpha =  \sum_{j \neq i} \frac{\pm 1}{r_{ij}}
\end{equation}

Ansatz:

\begin{equation}
  \alpha = r  \sum_{j \neq i} \frac{1}{r_{ij}} + \frac{r}{\sqrt{2}} \sum_{k \neq i} \frac{1}{r_{ik}}
\end{equation}

mit $\del{\frac{r}{2}}^2 = R^2 \Rightarrow r = R \sqrt{2}$.

Damit die Schalen jeweils neutral sind wählen wir konzentrische quadratische
Schalen entlang der Diagonalen.

% FIXME: Skizze mit PGF/Tikz?

Jede Schale hat eine Seitenlänge von $n \cdot \sqrt{2}R$

Dementsprechend hat die $n$-te Schale $4n$ Ionen.

Pro Schale tauchen $n$ unterschiedliche Radien auf

\begin{equation}
  \sigma = \sqrt{(mR)^2 + ((n-m)R)^2} = R \sqrt{m^2 + (n-m)^2}, m \in \mathrm{N}, m \le n
\end{equation}

In jeder Schale nehmen 4 Ionen diesen Radius ein. Somit erhält man für die Summe in der $n$-ten Schale

\begin{equation}
  \sum_{m=1}^n \frac{4 (-1)^n}{R \sqrt{m^2 + (n-m)^2}}
\end{equation}

Über alle $l$ Schalen summiert:

\begin{align}
  \alpha &= R \sum_{n=1}^l \sum_{m=1}^n \frac{4 (-1)^n}{R \sqrt{m^2 + (n-m)^2}} \\
         &= 4 \sum_{n=1}^l \sum_{m=1}^n \frac{(-1)^n}{R \sqrt{m^2 + (n-m)^2}} \\
\end{align}

Für $l=1000$ erhalten wir $\alpha_+ \approx 0{,}8774$, für $l=1001: \alpha_- \approx -4{,}1084$.



\section*{Aufgabe 4: NaCl- CsCl-Struktur}

Sei $r_A \ge r_b$ und $r_A$ und $r_B$ bekannt. Für ein primitives Gitter
beträgt die Koordinationszahl $K = 6$. Für ein kubisch-zentrisches Gitter ist
$K=8$.

Wir betrachten den Fall, in dem sich die $B^-$-Ionen berühren. Es ein steht
eine Oktaederlücke (wegen $K = 6$)

Sei $a = 2 r_A$ die Kantenlänge, dann ist der Umkugelradius $R = \frac{a
\sqrt{2}}{2}$, somit $R = \sqrt{2} r_A$.

Der Umkugelradius ist gerade der Abstand von $A^+$ und $B^-$, also
\begin{equation}
  r_A + r_B \le \sqrt{2} r_A \Leftrightarrow r_B \le r_A \del{\sqrt{2} - 1}
\end{equation}

Wird also $r_B$ größer als $r_A \del{\sqrt{2} - 1}$, kann keine Oktaederlücke
mehr zustande kommen.

Da es aber nur zwei Lückentypen gibt (Tetraederlücke für $K = 4$, Oktaederlücke
für $K = 6$), sodass die ``Kugeln'' vom Radius $r_A$ unter gegenseitigem
Kontakt gestapelt werden können und eine dichte Kugelpackung bilden.

Für $r_B > r_A \del{\sqrt{2} - 1}$ stehen die $A^-$ also nicht mehr in direktem
Kontakt untereinander. Stattdessen ergibt sich eine kubisch-zentrische
Struktur, derart, dass $A^-$ und $B^+$ in direktem Kontakt zueinander stehen.

Sei $a$ die Kantenlänge des Würfels. Dann ist $r_A + r_B = \frac{a
\sqrt{3}}{2}$. Dabei soll $2 r_A < a$ sein, also

\begin{equation}
  r_A + r_B = \frac{a \sqrt{3}}{2} > r_A \sqrt{3} \Rightarrow r_B > r_A \del{\sqrt{3} - 1}
\end{equation}

Im Fall $\del{ \sqrt{2} - 1 } < \frac{r_B}{r_A}  < \del{\sqrt{3} - 1}$ befindet
sich der Kristall also im Übergang zwischen beiden Zuständen.

$\frac{r_B}{r_A}$ kann offensichtlich nicht beide Bedingungen gleichzeitig
erfüllen, also findet der Übergang in den primitiven Zustund gerade statt, wenn

\begin{equation}
  \frac{r_B}{r_A} = \sqrt{2} - 1
\end{equation}

Den Fall $2 r_A = a$ haben wir für das kubisch-zentrische Gitte mit
eingeschlossen, da dichteste Kugelpackungen energetisch günstiger infolge der
Anzieungskräfte sind und das primitive Gitter diesen Fall abdeckt.

Berücksichtigen wir noch nicht-dichte Kugelpackungen im Fall des primitiven
Gitters, so ist

\begin{align*}
  &a \ge 2 r_A, R = \frac{a \sqrt{2}}{2} \ge r_a \sqrt{2} \\
  \Rightarrow & r_A + r_B \le \frac{a \sqrt{2}}{2} \ge r_a \sqrt{2}
\end{align*}

Abhängig von der Wahl von $r_A$ und $r_B$ gilt also:

\begin{equation}
  r_A + r_B \le r_A \sqrt{2} \quad \text{oder} \quad r_A + r_B \ge r_A \sqrt{2}
\end{equation}

Also wird das primitive Gitter auch für die Zwischenzustände angenommen. Also ist der Übergang bei
\begin{equation}
  \frac{r_A}{r_B} = \sqrt{3} - 1
\end{equation}

\end{document}
