\documentclass[a4paper,german,12pt]{article}
\usepackage[T1]{fontenc}
\usepackage[utf8]{inputenc}
\usepackage{babel}
\usepackage{tikz}
\usepackage{geometry}
\usepackage{amsmath}
\geometry{a4paper, top=25mm, left=20mm, right=40mm, bottom=20mm, headsep=10mm, footskip=12mm}
\linespread{1.5}
\setlength\parindent{0pt}
\begin{document}

Mathematik für Physiker I, Wintersemester 2012/2013, 1. Übungsblatt

Markus Fenske und Florian Neumeyer, Tutor: Stephan Schwartz

\section*{Aufgabe 1.1}
\subsection*{Teil a)}

Gegeben ist $(x+iy)^2 = a+bi$. Daraus erhält man (weil $i^2 = -1$) durch
binomische Formel $x^2+2xyi-y^2 = a+bi$. Dann Auftrennen in Real- und
Imaginärteil. $a = x^2-y^2$ und $b = 2xy$. Wir lösen die zweite Gleichung nach
$x$ auf: $x = \frac{b}{2y}$ und setzen in die erste Gleichung ein: $a =
\left(\frac{b}{2y}\right)^2 - y^2 = \frac{b^2}{4y^2} - y^2$. Diese Gleichung
gilt es nun nach $y$ aufzulösen.  Durch Multiplikation mit $4y^2$ erhalten wir
$4ay^2 = b^2 - 4y^4$, durch Umstellen $4y^4 - 4ay^2 - b^2 = 0$, durch Division
durch 4 dann $y^4 - ay^2 - \frac{b^2}{4} = 0$. Um die Gleichung nun mittels
PQ-Formel zu lösen substituieren wir $e = y^2$, behalten aber im Hinterkopf,
dass $y$ reell ist, damit dann also $e$ positiv und die negative Lösung
wegfällt. Die Lösung ist also $e = \frac{a}{2} + \sqrt{\frac{a^2}{4} +
\frac{b^2}{4}}$. Nun kann man noch $\frac{1}{4}$ aus der Wurzel ziehen, so dass
alles auf einem Bruchstrich steht: $e = \frac{a + \sqrt{a^2 + b^2}}{2}$. Durch
Rücksubstitution und Wurzel ziehen gilt nun $y = \sqrt{\frac{a + \sqrt{a^2 +
b^2}}{2}}$.

Nun kann man in $x = \frac{b}{2y}$ einsetzen um auch den Realteil der Wurzel zu
erhalten. Insgesamt erhält man dann für Real- und Imaginärteil von $z = x+iy$:

$$\operatorname{Re}(z) = \sqrt{\frac{a + \sqrt{a^2 + b^2}}{2}}$$
$$\operatorname{Im}(z) = \frac{b}{2\sqrt{\frac{a + \sqrt{a^2 + b^2}}{2}}}$$

\subsection*{Teil b)}

Berechnung der Quadratwurzel $z = \sqrt{8+6i}$.

\begin{align}
\operatorname{Re}(z) &= \sqrt{\frac{8 + \sqrt{8^2 + 6^2}}{2}} \\
&= \sqrt{\frac{8 + \sqrt{64 + 36}}{2}} \\
&= \sqrt{\frac{8 + \sqrt{100}}{2}} \\
&= \sqrt{\frac{8 + 10}{2}} \\
&= \sqrt{\frac{18}{2}} \\
&= \sqrt{9} \\
&= 3 \\
\end{align}

Für den Imaginärteil können wir nutzen, dass der Imaginärteil vom Realteil abhängt.

\begin{align}
\operatorname{Im}(z) &= \frac{6}{2 \cdot 3} \\
&= 1
\end{align}

Die Quadratwurzel aus $8+6i$ ist also $3 + 1i$.

\section*{Aufgabe 1.2}

Wir betrachten komplexen Zahlen hier geometrisch und in Polarform. Sofern $z_0$
im Ursprung liegt muss $z_2$, um mit $z_0$ und $z_1$ ein gleichseitiges Dreieck
zu bilden die um 60 Grad um den Ursprung gedrehte Abbildung von $z_1$ sein.

Sollte $z_0$ nicht im Ursprung liegen, wird das Referenzsystem durch eine
Verschiebungstransformation so geändert, dass $z_0$ im Ursprung liegt, die
Drehung durchgeführt und die Transformation rückgängig gemacht.

Konkret wird also zuerst um $-z_0$ verschoben, dann durch Multiplikation mit
der noch zu findenden komplexen Zahl $w$ die Zahl $z_1$ um 60 Grad gedreht und
anschließend wieder um $z_0$ verschoben. Das Ergebnis ist $z_2$. Die Gleichung
sieht also so aus:

$$z_2 = (z_1 - z_0) \cdot w + z_0$$

Da $w$ eine Drehung ohne Streckung ausführen soll, muss $|w| = 1$. Somit ist $w
= e^{i\phi}$ mit $\phi = \frac{\pi}{3}$ (Bogenmaß von 60 Grad). Somit $w =
\cos(\frac{\pi}{3}) + i\sin(\frac{\pi}{3}) = \frac{1}{2} +
\frac{\sqrt{3}{2}}i$

Die Formel lautet also:

$$z_2 = \left(\frac{1}{2} + \frac{\sqrt{3}}{2}i\right) \left(z_1 - z_0\right)  + z_0$$

\section*{Aufgabe 1.3}
\subsection*{Teil a)}

Wir suchen $$\left|\frac{z-1}{z+1}\right| = 1, z \in \mathbf{C}$$

\begin{align}
\left|\frac{z-1}{z+1}\right| &= 1 \\
\frac{|z-1|}{|z+1|} &= 1
\end{align}

Wir substituieren $z = x+yi$ wobei $x,y \in \mathbf{R}$, dann ist der Betrag
$|z| = \sqrt{x^2+y^2}$

\begin{align}
\frac{\sqrt{(x-1)^2+y^2}}{\sqrt{(x+1)^2+y^2}} &= 1 \\
\frac{(x-1)^2+y^2}{(x+1)^2+y^2} &= 1^2 \\
(x-1)^2+y^2 &= (x+1)^2+y^2 \\
(x-1)^2 &= (x+1)^2 \\
x^2-2x+1 &= x^2+2x+1 \\
-2x &= 2x
\end{align}

Dafür existriert nur eine Lösung: $x = 0$. Somit hat $\frac{z-1}{z+1}$ nur dann den
Betrag 1, wenn $\operatorname{Re}(z) = 0$. Geometrisch gedacht verlaufen
alle Lösungen also direkt auf der imaginären Aches. Das liegt daran, dass hier
die von $z_1 = z-1$ und $z_2 = z+1$ gleich sein müssen. Bei Darstellung der
Zahlen als Vektoren $(\operatorname{Re}(z_1), \operatorname{Im}(z_1))$ und
$(\operatorname{Re}(z_2), \operatorname{Im}(z_2))$ muss dazu $z_1$ die
Spiegelung von $z_2$ an der imaginären Achse sein. Das bedeutet dann, dass $z$
genau auf der imaginären Achse liegt.
Verschiebt sich $z$ nach rechts (also in Richtung $\operatorname{Re}(z) > 0$,
wird $z+1$ länger, während $z-1$ kürzer wird, so dass der Betrag nicht mehr 1
ist. Sobald $z-1$ die imaginäre Aches überschreitet ($\operatorname{Re}(z-1) >
0$), wird $z-1$ wieder länger, allerdings ist in diesem Fall offensichtlich
dass $|z+1| > |z-1|$. Die Verschiebung in die andere Richtung verläuft aus
Symmetrie gründen genauso.

\subsection*{Teil b)}

Die geometrische Begründung befindet sich bereits in Teil a. Es ist klar, dass
für den Fall $\operatorname{Re}(z) > 0$ immer $|z-1| < |z+1|$ gilt. Somit ist
$\left|\frac{z-1}{z+1}\right| < 1$.

\subsection*{Teil c)}

Da der Betrag wie in Teil b ermittelt immer kleiner als 1 ist, liegen alle
Bildpunkte innerhalb der offenen Einheitskreisscheibe $\{w \in \mathbf{C} : |w|
< 1\}$. Zu klären bleibt nur die Frage, ob auch wirklich alle Bildpunkte benutzt
werden, oder ob sich das Bild weiter eingrenzen lässt.

Es lässt sich wahrscheinlich zeigen, dass $w$ mit steigendem Realteil von $z$
immer weiter in den Mittelpunkt des Einheitskreises rückt. Da der Kreis
rotationssymmetrisch ist, sollten der gesamte Einheitskreis ausgefüllt werden.

Für eine saubere Herleitung und Begründung fehlte uns aber leider die Zeit.

\end{document}

