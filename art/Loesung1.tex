\documentclass[a4paper,german,12pt,smallheadings]{scrartcl}
\usepackage[T1]{fontenc}
\usepackage[utf8]{inputenc}
\usepackage{babel}
\usepackage{geometry}
\usepackage[fleqn]{amsmath}
\usepackage{amssymb}
\usepackage{float}
\usepackage{enumerate}
\usepackage{commath} % http://tex.stackexchange.com/questions/14821/whats-the-proper-way-to-typeset-a-differential-operator
\usepackage{cancel}

\usepackage[fleqn]{mathtools}
% Number only referenced equations
%\mathtoolsset{showonlyrefs}
%\usepackage{wrapfig}
\usepackage[thinspace,thinqspace,squaren,textstyle]{SIunits}

% New command for color underlining
\usepackage{xcolor}

\newsavebox\MBox
\newcommand\colul[2][red]{{\sbox\MBox{$#2$}%
  \rlap{\usebox\MBox}\color{#1}\rule[-1.2\dp\MBox]{\wd\MBox}{0.5pt}}}

\restylefloat{table}
\geometry{a4paper, top=15mm, left=10mm, right=20mm, bottom=20mm, headsep=10mm, footskip=12mm}
\linespread{1.5}
\setlength\parindent{0pt}
\DeclareMathOperator{\Tr}{Tr}
\DeclareMathOperator{\Var}{Var}
\begin{document}
\allowdisplaybreaks % Seitenumbrüche in Formeln erlauben
\begin{center}
\bfseries % Fettdruck einschalten
\sffamily % Serifenlose Schrift
\vspace{-40pt}
Allgemeine Relativitätstheorie, Sommersemester 2014, Aufgabenblatt 1

Markus Fenske, Luis Herrmann, Jonathan Plato
\vspace{-10pt}
\end{center}

\section*{Aufgabe 1: Symmetrieeigenschaften von Tensoren I}

\begin{enumerate}[i)]
  \item
    Aus $T_{\alpha [\beta \gamma]} = 0$ folgt
    \begin{equation}
      T_{\alpha \beta \gamma} = T_{\alpha \gamma \beta}
      \label{eq:1}
    \end{equation}

    Aus $T_{(\alpha \beta) \gamma} = 0$ folgt
    \begin{equation}
      T_{\alpha \beta \gamma} = -T_{\beta \alpha \gamma}
      \label{eq:2}
    \end{equation}

    Gleichsetzen von (\ref{eq:1}) und (\ref{eq:2}):
    % Schöner wärs mit amscd:
    % http://tex.stackexchange.com/q/125508/50173
    \begin{align*}
      &T_{\alpha \gamma \beta}          &= -&T_{\beta \alpha \gamma}          \\
      &\downarrow \text{(\ref{eq:2})}   &   &\downarrow \text{(\ref{eq:1})}   \\
     -&T_{\gamma \alpha \beta}          &= -&T_{\beta \gamma \alpha}          \\
      &\downarrow \text{(\ref{eq:1})}   &   &\downarrow \text{(\ref{eq:2})}   \\
     -&T_{\gamma \beta \alpha}          &=  &T_{\gamma \beta \alpha}          \\
    \end{align*}

    Durch Umbenennung der Indicies: $T_{\alpha \beta \gamma} = -T_{\alpha \beta \gamma} \;\Rightarrow\; T_{\alpha \beta \gamma} = 0$
\end{enumerate}


\end{document}
