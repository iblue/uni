\documentclass[a4paper,german,12pt,smallheadings]{scrartcl}
\usepackage[T1]{fontenc}
\usepackage[utf8]{inputenc}
\usepackage{babel}
\usepackage{tikz}
\usepackage{geometry}
\usepackage{amsmath}
\usepackage{amssymb}
\usepackage{float}
%\usepackage{wrapfig}
\usepackage{pdflscape}
\pagenumbering{gobble}
\usepackage[thinspace,thinqspace,squaren,textstyle]{SIunits}
\restylefloat{table}
\geometry{a4paper, top=15mm, left=20mm, right=40mm, bottom=20mm, headsep=10mm, footskip=12mm}
\linespread{1.5}
\setlength\parindent{0pt}
\begin{document}
\begin{center}
\bfseries % Fettdruck einschalten
\sffamily % Serifenlose Schrift
\vspace{-40pt}
Analysis I, Sommersemester 2013, 5. Übungsblatt \\
Luis Herrmann und Markus Fenske, Tutor: Adam Schienle
\vspace{-10pt}
\end{center}

\section*{Aufgabe 5.1}
Gegeben ist

\begin{equation*}
\sum_{n=1}^\infty nx^n 
\end{equation*}

Erweitern mit $1-x$ liefert:
\begin{align*}
  &=\frac{1-x}{1-x} \sum\limits_{n=1}^{\infty} nx^n \\
  &=\frac{1}{1-x} \sum\limits_{n=1}^{\infty} nx^n\cdot(1-x) \\
  &=\frac{1}{1-x} \sum_{n=1}^\infty n(x^n - x^{n+1})
\end{align*}

Untersuchen und Umsortieren der Summe:
\begin{align*}
  &\sum_{n=1}^\infty nx^n - nx^{n+1} \\
  &= \left. n(x^n - x^{n+1}) \right|_{n=1} + \left. n(x^n - x^{n+1}) \right|_{n=2} + \left. n(x^n - x^{n+1}) \right|_{n=3} + \dots \\
  &= 1(x - x^2) + 2(x^2 - x^3) + 3(x^3 - x^4) + 4(x^4 - x^5) + \dots \\
  &= x \underbrace{- x^2 + 2x^2}_{=+x^2} \underbrace{- 2x^3 + 3x^3}_{=+x^3} \underbrace{- 3x^4 + 4x^4}_{=+x^3} - 4x^5 + \dots \underbrace{- \lim\limits_{n \to \infty} nx^{n+1}}_{\text{``Schlußterm''} \to 0}\\
  &= x + x^2 + x^3 + x^4 + \dots \\
  &= \sum_{n=1}^\infty x^n
\end{align*}

Am Schluss bleibt beim Umsortieren ein Term übrig. Dass dieser ``Schlussterm''
gegen Null geht und somit wegfallen darf, möchten wir noch kurz erläutern.
``Schlussterm'' in Anführungszeichen, weil es bei einer unendlichen Reihe per
definitionem eigentlich keinen letzten Term gibt.

\begin{equation*}
  \lim\limits_{n \to \infty} nx^{n+1}=0
\end{equation*}

Dies kann man sich auch ohne formellen Beweis überlegen, da wir wissen, dass
$x^n$ für $|x|<1$ gegen 0 konvergiert. (Auf dem letzten Übungszettel bewiesen).
Zwar divergiert $n$ bestimmt gegen $\infty$, aber durch das exponentielle
Wachstum konvergiert $x^n$ schneller gegen den Grenzwert 0 als $n$ gegen
$\infty$.

Den Reihenterm können wir über die Summenformel für die geometrische Reihe ausdrücken, wenn $|x| < 1$.
\begin{align*}
  &\sum_{n=0}^\infty x^n = \frac{1}{1-x} \qquad\Rightarrow\qquad 1 + \sum_{n=1}^\infty x^n= \frac{1}{1-x}\\ 
  \Leftrightarrow\quad & \sum_{n=1}^\infty x^n =\frac{1}{1-x} - 1 = \frac{1-\left(1-x\right)}{1-x}= \frac{x}{1-x}
\end{align*}


Überführen wir ausgehend von diesen Überlegungen unsere Terme in die Ausgangsgleichung, erhalten wir:
\begin{align*}
  &\sum_{n=1}^\infty nx^n = \frac{x}{1-x} \cdot \frac{1}{1-x} = \frac{x}{(1-x)^2}
\end{align*}

Was zu zeigen war.

\section*{Aufgabe 5.2}
\subsection*{Teil a}

Gesucht ist der Konvergenzradius von $\sum\limits_{n=0}^{\infty} i^nz^n$.\\
Die Reihe konvergiert, falls das Wurzelkriterium erfüllt ist:
\begin{align*}
  & \limsup\limits_{n \to \infty} \sqrt[n]{|i^nz^n|}<1\\
  \Leftrightarrow\quad&\limsup\limits_{n \to \infty} \sqrt[n]{|i|^n|z|^n}<1
\end{align*}

Wegen $|i|=1$ und $|z|= \text{const.}$:

\begin{equation*}
\Leftrightarrow |z|<1
\end{equation*}

Also ist der Konvergenzradius gerade $R=1$.

\subsection*{Teil b}

Gesucht ist der Konvergenzradius von $\sum\limits_{n=0}^{\infty} \frac{(z-2i)^n}{n\cdot3^n}$.

Die Reihe konvergiert, falls das Wurzelkriterium erfüllt ist:

\begin{align*}
  & \limsup_{n \to \infty} \sqrt[n]{\left|\frac{(z-2i)^n}{n\cdot3^n}\right|}<1\\
  \Leftrightarrow\quad& \frac{\limsup\limits_{n \to \infty} \sqrt[n]{|z-2i|^n}}{\limsup\limits_{n \to \infty} \sqrt[n]{3^n} \cdot \underbrace{\lim_{n \to \infty} \sqrt[n]{n}}_{\to 1}} < 1\\
  \Leftrightarrow\quad& \frac{|z-2i|}{3} < 1\\
  \Leftrightarrow\quad& |z-2i| < 3\\
\end{align*}

Die Reihe konvergiert also für $|z-2i| < 3$.

\subsection*{Teil c}
Gesucht ist der Konvergenzradius von $\sum\limits_{n=0}^{\infty} \frac{z^{2n}}{2^n}$.

Die Reihe konvergiert, falls das Wurzelkriterium erfüllt ist:

\begin{align*}
  & \limsup\limits_{n \to \infty} \sqrt[n]{|\frac{z^{2n}}{2^n}|}<1\\
  \Leftrightarrow\quad& \limsup\limits_{n \to \infty} \sqrt[n]{\frac{{|z^2|}^n}{|2|^n}}<1\\
  \Leftrightarrow\quad& \frac{|z|^2}{2}<1\\
  \Leftrightarrow\quad& |z|<\sqrt{2}
  \end{align*}

Der Konvergenzradius ist also gerade $R=\sqrt{2}$.

\section*{Aufgabe 5.3}
\subsection*{Teil a}

Damit die gegebene Reihe absolut konvergiert, muss gemäß Wurzelkriterium gegeben sein:

\begin{align*}
  &\limsup_{n \to \infty} \sqrt[n]{|a_nb_n(uv)^n|} < 1 \\
  &\Leftrightarrow \limsup_{n \to \infty} \sqrt[n]{|a_nu^n|} \sqrt[n]{|b_nv^n|} < 1 \\
  &\Leftrightarrow \limsup_{n \to \infty} \sqrt[n]{|a_nu^n|} \sqrt[n]{|b_nv^n|} \leq \underbrace{\left(\limsup_{n \to \infty} \sqrt[n]{|a_nu^n|}\right)}_{<1} \cdot \underbrace{\left(\limsup_{n \to \infty} \sqrt[n]{|b_nv^n|}\right)}_{<1} < 1 \\
\end{align*}

Diese beiden Grenzwerte existieren und sind nach Voraussetzung beide $<1$, da unsere Reihen ansonsten das Wurzelkriterium nicht erfüllen. Damit ist der Beweis erbracht.\\
\\
Anmerkung: Dass $\limsup_{n \to \infty} a_nb_n \leq \limsup_{n \to \infty} a_n \cdot \limsup_{n \to \infty} b_n$ zeigen wir in der (b).

\subsection*{Teil b}

Gegeben sind:
\begin{equation*}
\sum\limits_{n=0}^{\infty} a_n z^n \; \text{mit} \; r_a>0\quad ; \quad \sum\limits_{n=0}^{\infty} b_n z^n \; \text{mit} \; r_b>0 \quad \text{und} \quad  \sum\limits_{n=0}^{\infty} a_n b_n z^n
\end{equation*}
Zu zeigen:
\begin{equation*}
r_c \geq r_a r_b
\end{equation*}
Dies zeigen wir ausgehend vom Wurzelkriterium.\\
Angenommen, $\sum\limits_{n=0}^{\infty} a_n b_n z^n$ konvergiert absolut, muss gelten:
\begin{align*}
  & \limsup\limits_{n \to \infty} \sqrt[n]{|a_n b_n z^n|} < 1 \\
  & \limsup\limits_{n \to \infty} \sqrt[n]{|a_n b_n| \; |z|^n} < 1\\
  & |z| < \frac{1}{\limsup\limits_{n \to \infty} \sqrt[n]{|a_n b_n|}}
\end{align*}
Der Konvergenzradius ist also genau der Ausdruck auf der rechten Seite der Gleichung.\\
\\
Nebenüberlegung: In einem beliebigen Intervall $I$ lässt sich zu jeder Folge ein Supremum finden:
\begin{align*}
  & \sqrt[n]{\underset{n \in I}{|a_n|}} \leq \sup \sqrt[n]{|a_n|}\\
  & \sqrt[n]{\underset{n \in I}{|b_n|}} \leq \sup \sqrt[n]{|b_n|}
\end{align*}

Dann wird auch gelten:
\begin{align*} 
  & \sqrt[n]{\underset{n \in I}{|a_n|}} \cdot \sqrt[n]{\underset{n \in I}{|b_n|}} \leq \sup \sqrt[n]{|a_n|} \sup \sqrt[n]{|b_n|}
\end{align*}
Also ist der Aufdruck auf der rechten Seite obere Schranke der Folge $\sqrt[n]{\underset{n \in I}{|a_n|} \cdot \underset{n \in I}{|b_n|}}$. Das Supremum ist per Definition kleiner/gleich jeder oberen Schranke, also ist:
\begin{equation*}
\sup \sqrt[n]{\underset{n \in I}{|a_n|} \cdot \underset{n \in I}{|b_n|}} \leq \sup \sqrt[n]{|a_n|} \sup \sqrt[n]{|b_n|}
\end{equation*}

Diese Betrachtung gilt für das Supremum eines beliebiges Intervall. Betrachtet man ein beliebig großes Intervall, wird logischerweise für den Grenzwert der Suprema der Folgen, dem Limes Superior, gelten: 
\begin{equation*}
\limsup\limits_{n \to \infty} \sqrt[n]{|a_n b_n|} \leq \limsup\limits_{n \to \infty} \sqrt[n]{|a_n|} \cdot \limsup\limits_{n \to \infty} \sqrt[n]{|b_n|}
\end{equation*}
Und für die Kehrwerte:
\begin{equation*}
\underbrace{\frac{1}{\limsup\limits_{n \to \infty} \sqrt[n]{|a_n b_n|}}}_{r_c} \geq \frac{1}{\limsup\limits_{n \to \infty} \sqrt[n]{|a_n|} \cdot \limsup\limits_{n \to \infty} \sqrt[n]{|b_n|}}= \underbrace{\frac{1}{\limsup\limits_{n \to \infty} \sqrt[n]{|a_n|}}}_{r_a} \cdot \underbrace{\frac{1}{\limsup\limits_{n \to \infty} \sqrt[n]{|b_n|}}}_{r_b}
\end{equation*}

\subsection*{Teil c}

Seien $\sum\limits_{n=0}^{\infty} a_nz^n$ und $\sum\limits_{n=0}^{\infty} b_n z^n$ gegeben mit $a_n=\sin(n)$ und $b_n=\cos(n)$.\\
\\
Wir können sofort den Limes Superior von $a_n$ und $b_n$ und damit den Konvergenzradius beider Reihen bestimmen:
\begin{equation*}
r_a=\frac{1}{\limsup\limits_{n \to \infty} \sin(n)}=\frac{1}{1}=1 \quad \text{und} \quad r_b=\frac{1}{\limsup\limits_{n \to \infty} \cos(n)}=\frac{1}{1}=1
\end{equation*}
Hingegen gilt für den Konvergenzradius von $\sum\limits_{n=0}^{\infty} a_n b_n z^n$:
\begin{equation*}
r_c=\frac{1}{\limsup\limits_{n \to \infty} \sin(n)}=\frac{1}{\frac{1}{2}}=2>1=r_a \cdot r_b
\end{equation*}

\subsection*{Teil d}

Hierfür nehmen wir die konvergente Reihe:
\begin{equation*}
\sum\limits_{n=0}^{\infty} a_nu^n:=\sum\limits_{n=0}^{\infty} \frac{(-1)^n}{\sqrt{n}} \quad ; \quad \text{sei} \quad a_n=\frac{1}{\sqrt{n}} \quad \text{und} \quad u=-1
\end{equation*}
Die Konvergenz beweisen wir über das Leibniz-Kriterium für alternierende Folgen. Monotonie von $a_n$ ist gegeben:
\begin{align*}
  & \frac{1}{|n|}<\frac{1}{n+1} \quad \forall n\\
  & \Leftrightarrow \sqrt{\frac{1}{n}}<\sqrt{\frac{1}{n+1}} \quad \forall n\\
  & \Leftrightarrow \frac{1}{\sqrt{n}}< \frac{1}{\sqrt{n+1}} \quad \forall n
\end{align*}
Des weiteren ist $a_n$ Nullfolge:
\begin{equation}
\lim\limits_{n \to \infty} \frac{1}{\sqrt{n}}=\lim\limits_{n \to \infty} \sqrt{\frac{1}{n}}=\sqrt{\lim\limits_{n \to \infty} \frac{1}{n}}=0
\end{equation}
Damit schlägt das Leibniz-Kriterium zu und unsere Reihe ist konvergent.\\
Absolute Konvergenz ist nicht gegeben,denn:
\begin{equation*}
\sum\limits_{n=0}^{\infty} |\frac{(-1)^n}{\sqrt{n}}| = \sum\limits_{n=0}^{\infty} \frac{1}{\sqrt{n}}=\sum\limits_{n=0}^{\infty} \sqrt{\frac{1}{n}}
\end{equation*}
Die harmonische Reihe divergiert bestimmt. Das Ziehen der Quadratwurzel aller Summanden ändert nichts an der Divergenz der Reihe.
Als zweite Reihe nehmen wir bequemerweise wieder die Gleiche (es gibt keinen Grund zur Annahme, dass dies nicht legitim wäre).\\
\\
Dann erhalten wir also die Produktreihe:
\begin{equation*}
\sum\limits_{n=0}^{\infty} \frac{(-1)^n}{\sqrt{n}} \cdot \frac{(-1)^n}{\sqrt{n}}=\sum\limits_{n=0}^{\infty} \frac{1}{n}
\end{equation*}
Dies ist die harmonische Reihe, von der wir bereits wissen, dass sie divergiert. Damit ist das Beispiel erbracht.

\section*{Aufgabe 5.4}

Sei $p(x)$ ein Polynom beliebig hohen Grades $N$. Die Exponentialreihe enthält
dann immer einen Term $\frac{x^{N+1}}{(N+1)!}$, der schneller wächst als das
Polynom. Denn $\sum_{i=0}^n \alpha x^i < \beta x^{n+1}$ für genügend große $x$,
oder anders ausgedrückt: Das Polynom vom höheren Grad wächst schneller.

Das können wir einfach beweisen. Sei $p_1(x)=ax^n$ und $p_2(x)=\sum\limits_{i=0}^{\infty} b_i x^i$.

Zu zeigen: Es gibt ein x, sodass:
\begin{align*}
  &p_1(x)>p_2(x)\\
  & \Leftrightarrow ax^n>\sum\limits_{i=0}^{\infty} b_i x^i\\
  & \Leftrightarrow ax^n>b_0+b_1x^1+b_2x^2+ \cdots\\
  & \Leftrightarrow ax^n-b_0-b_1x^1-b_2x^2- \cdots >0
\end{align*}
Sei $c:=\max\{b_0,b_1,b_2,\cdots\}$, dann ist $\sum\limits_{i=0}^{\infty} b_i x^i \leq \sum\limits_{i=0}^{\infty} c x^i$ und es reicht aus, zu zeigen:
\begin{align*}
  & \Leftrightarrow ax^n-b_0-b_1x^1-b_2x^2- \cdots \geq ax^n-c-cx^1-cx^2- \cdots>0\\
  & \Rightarrow ax^n-c\underbrace{\left(1+x^1+x^2+\cdots \right)}_{n Summanden}>0
\end{align*}
Für $x \geq 1$ ist $x^{n-1} \geq x^m \; \forall m \geq n-1$, sodass er reicht, zu zeigen:
\begin{align*}
  & \Leftrightarrow ax^n-c\underbrace{\left(1+x^1+x^2+\cdots \right)}_{n_Summanden} \geq ax^n-c\underbrace{\left(x^{n-1}+x^{n-1}+x^{n-1}+\cdots \right)}_{n_Summanden} = ax^n-cnx^{n-1} > 0\\
  & \Rightarrow x^{n-1}\left(ax-cn\right)>0 \quad \Rightarrow \quad x^{n-1}\left(ax-cn\right)>ax-cn>0
\end{align*}
Offensichtlich gibt es ein x, welches diese Bedingung erfüllt. Das heißt, es gibt ein x, ab dem ein Polynom n-ten Grades größer wird als eines (n-1)-ten Grades.\\
Obwohl die Potenzreihe von $\exp x$ nicht ein Polynom ist, wird sich auch hier ein x finden lassen, sodass die Funktion größer wird als Polynom beliebigen Grades (s.o.) \\
\\
Für die Sinusreihe gilt unser Argumentationsweg nicht, weil sie immer auch einen
höhergradigen subtraktive Term enthält, der das Wachstum wieder zunichte macht.
Außerdem ist bekannt, dass $\sin x \le 1$ für reelle $x$.

\end{document}
