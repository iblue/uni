\documentclass[a4paper,german,12pt,smallheadings]{scrartcl}
\usepackage[T1]{fontenc}
\usepackage[utf8]{inputenc}
\usepackage{babel}
\usepackage{tikz}
\usepackage{geometry}
\usepackage{amsmath}
\usepackage{amssymb}
\usepackage{float}
%\usepackage{wrapfig}
\usepackage{pdflscape}
\pagenumbering{gobble}
\usepackage[thinspace,thinqspace,squaren,textstyle]{SIunits}
\restylefloat{table}
\geometry{a4paper, top=15mm, left=20mm, right=40mm, bottom=20mm, headsep=10mm, footskip=12mm}
\linespread{1.5}
\setlength\parindent{0pt}
\begin{document}
\begin{center}
\bfseries % Fettdruck einschalten
\sffamily % Serifenlose Schrift
\vspace{-40pt}
Analysis I, Sommersemester 2013, 5. Übungsblatt \\
Luis Herrmann und Markus Fenske, Tutor: Adam Schienle
\vspace{-10pt}
\end{center}

\section{Aufgabe 5.1}
\textbf{Achtung: Falsch}
Der Quotient zweier benachbarter Folgeglieder ist hier:

\begin{align*}
  q = \frac{(n+1)x^{n+1}}{nx^n} = \frac{(n+1)x}{n}
\end{align*}

Mit der Summenformel für die geometrische Reihe gilt dann:

\begin{align*}
   &\sum_{n=1}^\infty n x^n \\
  =&\frac{1}{1 - \frac{(n+1)x}{n}} \\
  =&\frac{1}{\frac{n}{n} - \frac{(n+1)x}{n}} \\
  =&\frac{n}{n - (n+1)x} \\
\end{align*}
\section{Aufgabe 5.2}
\subsection*{Teil a}
Nach Wurzelkriterium ($|i^n| = |i|^n = 1$):

\begin{align*}
  \frac{1}{R} &= \lim_{n \to \infty} \sqrt[n]{|i^nz^n|} \\
              &= \lim_{n \to \infty} \sqrt[n]{|z|^n} \\
              &= |z| \\
  \Leftrightarrow\quad R &= \frac{1}{|z|}
\end{align*}

Der Reihe konvergiert also für $|z| < 1$.

\subsection*{Teil b}
Nach Wurzelkriterium:

\begin{align*}
  \frac{1}{R} &= \lim_{n \to \infty} \sqrt[n]{\left|\frac{(z-2i)^n}{n\cdot3^n}\right|} \\
              &= \frac{\lim_{n \to \infty} \sqrt[n]{|z-2i|^n}}{\lim_{n \to \infty} \sqrt[n]{3^n} \cdot \underbrace{\lim_{n \to \infty} \sqrt[n]{n}}_{\to 1}} \\
              &= \frac{|z-2i|}{3} \\
  \Leftrightarrow\quad R &= \frac{3}{|z-2i|}
\end{align*}

Die Reihe konvergiert also für $|z-2i| < 3$.

\subsection*{Teil c}
Nach Wurzelkriterium:
\begin{align*}
  \frac{1}{R} &= \lim_{n \to \infty} \sqrt[n]{\left|\frac{z^{2n}}{2^n}\right|} \\
              &= \frac{\lim_{n \to \infty} \sqrt[n]{|z|^{2n}}}{\lim_{n \to \infty} \sqrt[n]{2^n}} \\
              &= \frac{|z|^2}{2} \\
  \Leftrightarrow\quad R &= \frac{2}{|z|^2}
\end{align*}

Die Reihe konvergiert also für $|z|^2 < 2 \Leftrightarrow |z| < \sqrt{2}$.


\section{Aufgabe 5.4}

Sei $p(x)$ ein Polynom beliebig hohen Grades $N$. Die Exponentialreihe enthält
dann immer einen Term $\frac{x^{N+1}}{(N+1)!}$, der schneller wächst als das
Polynom. Denn $\sum_{i=0}^n \alpha x^i < \beta x^{n+1}$ für genügend große $x$,
oder anders ausgedrückt: Das Polynom vom höheren Grad wächst schneller.

Für die Sinusreihe gilt das Argument nicht, weil sie immer auch einen
höhergradigen subtraktive Term enthält, der das Wachstum wieder zunichte macht.
Außerdem ist bekannt, dass $\sin x \le 1$ für reele $x$.

\end{document}
