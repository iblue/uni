\documentclass[a4paper,german,12pt,smallheadings]{scrartcl}
\usepackage[T1]{fontenc}
\usepackage[utf8]{inputenc}
\usepackage{babel}
\usepackage{tikz}
\usepackage{geometry}
\usepackage{amsmath}
\usepackage{amssymb}
\usepackage{float}
\usepackage[thinspace,thinqspace,squaren,textstyle]{SIunits}
\restylefloat{table}
\geometry{a4paper, top=15mm, left=20mm, right=40mm, bottom=20mm, headsep=10mm, footskip=12mm}
\linespread{1.5}
\setlength\parindent{0pt}
\begin{document}
\begin{center}
\bfseries % Fettdruck einschalten
\sffamily % Serifenlose Schrift
\vspace{-40pt}
Analytische Mechanik, Sommersemester 2013, 5. Blatt \\
Luis Herrmann und Markus Fenske, Tutor: Clemens Meyer zu Rheda
\vspace{-10pt}
\end{center}
\section*{Aufgabe 1: Vektoren}
\subsection*{Teil a: Beweis der Lagrange-Identität}

Wir rechnen hier mit Pseudotensoren, weil uns die Lösung eleganter
erschien, anstatt mit der ``Bac-Cab-Regel'' und Spatprodukt zu hantieren.
Außerdem haben wir dann keinen Stilbruch mit Aufgabe 1b. Es gelte Einstein'sche
Summenkonvention.

\begin{align*}
  &(\vec{a} \times \vec{b}) \cdot (\vec{c} \times \vec{d}) =\\
  &\delta_{ab}(\vec{a} \times \vec{b})_a (\vec{c} \times \vec{d})_b =\\
  &\delta_{ab}(\epsilon_{aij} a_i b_j) (\epsilon_{bkl}c_k d_l) =\\
  &\delta_{ab}\epsilon_{aij}\epsilon_{blk} a_i b_jc_kd_l =\\
  &\epsilon_{aij}\epsilon_{alk} a_i b_jc_kd_l =\\
  &(\delta_{ik}\delta_{jl}-\delta_{il}\delta_{jl}) a_i b_j c_k d_l =\\
  &(\delta_{ik}a_ic_k \delta_{jl}b_jd_l)-(\delta_{il}a_id_l\delta_{jk}b_jc_k) =\\
  &(\vec{a} \cdot \vec{c})(\vec{b} \cdot \vec{d})-(\vec{a}\cdot\vec{d})(\vec{b} \cdot \vec{c})
\end{align*}

\subsection*{Teil b: Nabla-Produktregel}

In Vektorschreibweise muss man beachten, dass $\nabla$ ein Differentialoperator
und gleichzeitig ein Vektor ist. Er wirkt nur auf Dinge, die rechts von ihm
stehen. Gleichzeitig kann man sich in Vektorschreibweise (im Gegensatz zu
Pseudotensoren) nicht aussuchen, an welche Stelle man den Vektor schreibt. Eine
Möglichkeit, das Problem trotzdem in Vektorschreibweise anzugehen, wäre
gewesen, Vektoren zu unterstreichen, auf die $\nabla$ wirkt. Dies erschien uns
zu unschön. Deswegen also Pseudotensoren.

Sei $\nabla_i = \frac{\partial}{\partial x_i} = \partial_i$. Das sieht hübscher
aus und erinnert uns außerdem daran, dass $\nabla$, wie geschrieben, auch ein
Operator ist.

Es gelte Einstein'sche Summenkonvention. Summiert wird nur in Produkten, nicht in Summen.

\begin{align*}
  &\left[(\vec{a}\cdot\nabla)\vec{b} + (\vec{b}\cdot\nabla)\vec{a} + \vec{a} \times (\nabla \times b) + \vec{b} \times (\nabla \times \vec{a})\right]_i = \\
  &(\delta_{jk}a_j\partial_k)b_i + (\delta_{jk}b_j\partial_k)a_i + \epsilon_{ijk}a_j(\epsilon_{klm}\partial_lb_m) + \epsilon_{ijk} b_j (\epsilon_{klm} \partial_l a_m) =\\
  &a_j\partial_jb_i + b_j\partial_ja_i + \epsilon_{ijk}a_j(\epsilon_{klm}\partial_lb_m) + \epsilon_{ijk} b_j (\epsilon_{klm} \partial_l a_m) =\\
  &a_j\partial_jb_i + b_j\partial_ja_i + \epsilon_{ijk}\epsilon_{klm} (a_j\partial_lb_m + b_j\partial_l a_m) =\\
\end{align*}

Es gilt $\epsilon_{ijk} = -\epsilon_{ikj} = \epsilon_{kij}$. Und deswegen
$\epsilon_{ijk}\epsilon_{klm} = \epsilon_{kij}\epsilon_{klm} =
\delta_{il}\delta_{jm} - \delta_{im}\delta_{jl}$. Das setzen wir ein:

\begin{align*}
  &a_j\partial_jb_i + b_j\partial_ja_i + (\delta_{il}\delta_{jm} - \delta_{im}\delta_{jl}) (a_j\partial_lb_m + b_j\partial_l a_m) =\\
  &a_j\partial_jb_i + b_j\partial_ja_i + \delta_{il}\delta_{jm}(a_j\partial_lb_m + b_j\partial_l a_m) - \delta_{im}\delta_{jl}(a_j\partial_lb_m + b_j\partial_l a_m) =\\
  &a_j\partial_jb_i + b_j\partial_ja_i + \delta_{il}a_j\partial_lb_j + \delta_{il}b_j\partial_l a_j - \delta_{im}a_j\partial_jb_m - \delta_{im}b_j\partial_j a_m =\\
  &a_j\partial_ib_j + b_j\partial_i a_j =\\
  &\partial_ia_jb_j =\\
  &\partial_i(\delta_jka_jb_k) =\\
  &\nabla_i(\vec{a} \cdot \vec{b})
\end{align*}



\end{document}
