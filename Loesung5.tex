\documentclass[a4paper,german,12pt,smallheadings]{scrartcl}
\usepackage[T1]{fontenc}
\usepackage[utf8]{inputenc}
\usepackage{babel}
\usepackage{tikz}
\usepackage{geometry}
\usepackage{amsmath}
\usepackage{amssymb}
\usepackage{float}
%\usepackage{wrapfig}
\usepackage[thinspace,thinqspace,squaren,textstyle]{SIunits}
\restylefloat{table}
\geometry{a4paper, top=15mm, left=20mm, right=40mm, bottom=20mm, headsep=10mm, footskip=12mm}
\linespread{1.5}
\setlength\parindent{0pt}
\begin{document}
\begin{center}
\bfseries % Fettdruck einschalten
\sffamily % Serifenlose Schrift
\vspace{-40pt}
Elektrodynamik und Optik, Sommersemester 2013, 5. Blatt \\
Markus Fenske, Tutor: Dr. Marko Wietstruk
\vspace{-10pt}
\end{center}
\section*{Aufgabe 1}
\subsection*{Teil a}

Zur Anwendung der Kirchhoffschen Regel teilen ich den Stromkreis in einen
linken Kreis ($I_1$) und einen rechten Kreis ($I_2$). Der linke Kreis fließe
gegen den Uhrzeigersinn, der rechte Kreis fließe im Uhrzeigersinn.  Gestartet
wird jeweils vom Knotenpunkt $b$.

Somit ergeben sich folgende Gleichungen:

\begin{align*}
  \underbrace{-3I_1+5}_{\text{links}}\underbrace{-3I_2+5}_{\text{rechts}}+7-2I_1 &= 0 \\
  \underbrace{-3I_1+5}_{\text{links}}\underbrace{-3I_2+5}_{\text{rechts}}+1I_2 &= 0
\end{align*}

Die Lösung dieses Gleichungssystems lautet:

\begin{align*}
  I_1 = \frac{38}{11}\;\ampere\approx3{,}5\;\ampere,\quad I_2 = -\frac{1}{11}\;\ampere\approx-0{,}1\;\ampere
\end{align*}

Durch den Widerstand $R_1$ fließt der Strom $I_2$ von oben nach unten, mit dem
Betrag wie oben angegeben. Durch den Widerstand $R_2$ fließt der Strom $I_1$
von links nach rechts, Betrag wie oben angegeben. Durch den Widerstand $R_3$
fließen die Ströme $I_1$ und $I_2$ in die selbe Richtung, nämlich von unten
nach oben. Es wird also addiert. Der Strom durch $R_3$ ist dann
$\frac{37}{11}\;\ampere$.

\subsection*{Teil b}

Die Spannungsdifferenz zwischen Punkt $a$ und Punkt $b$ ist ($I$ sei der Strom
durch $R_3$ (siehe oben)):

\begin{align*}
  R_3 \cdot I - 5\;\volt = \frac{111}{11}\;\volt - \frac{55}{11} = \frac{56}{11}\;\volt \approx 5{,}1\;\volt
\end{align*}

\subsection*{Teil c}
Die Spannungsquelle $U_l$ muss den Strom $I_1$ liefern, die Leistung ist somit:

\begin{align*}
  P_l = U_l \cdot I_1 = \frac{266}{11} \;\watt \approx 24{,}2\;\watt
\end{align*}

Die Spannungsquelle $U_r$ liefert den Strom $I_1 + I_2$:

\begin{align*}
  P_r = U_r \cdot (I_1 + I_2) = \frac{185}{11} \;\watt \approx 16{,}8\;\watt
\end{align*}

\section*{Aufgabe 2}
Die Spannung am Kondensator steigt gemäß folgender Gleichung:

\begin{align*}
  U(t) = U_0 \cdot (1 - e^{-\frac{t}{RC}})
\end{align*}

Mit den gegebenen Werten $U_0 = 60\;\volt$, $t = 55\;\milli\second$, $R = 270\;\kilo\ohm$, $C = 47\;\nano\farad$ ergibt sich am Ende des Ladevorgangs die Spannung
\begin{align*}
  U_1 \approx 59{,}21 \;\volt
\end{align*}

Wenn der Kondensator nun entladen wird, ergibt sich ein anfänglicher
Entladestrom von $I_0 = \frac{U_1}{R_2}$. Es gilt dann also für den
Entladestrom nach $t_2$

\begin{align*}
  I = \frac{U_1}{R_2} \cdot e^{-\frac{t_2}{R_2C}}
\end{align*}

Durch Einsetzen der entsprechenden Werte ergibt sich $I \approx 256\;\micro\ampere$.

\section*{Aufgabe 4}
Das Gesetz von Amperè besagt, dass für eine Fläche, durch die ein Strom $I$
fließt, durch den (notwendigerweise geschlossenen) Rand dieser Fläche ein
Magnetfeld fließt:

\begin{align*}
  \oint_{\mathcal{S}} \vec{B} \cdot \vec{ds} = \mu_0 \cdot I
\end{align*}

Die Fläche ist in diesem Fall eine Kreisfläche, der Rand ein geschlossener
Kreisbogen. Die Windungen der Spule gehen um den Kreisbogen herum. Das
bedeutet, dass der Strom auf der Innenseite der Spule durch die Kreisfläche
hindurchfließt, aber auf der anderen Außenseite, außerhalb der Kreisfläche
wieder hinaus. Durch die Kreisfläche fließt also insgesamt der Strom $I=NI_0$,
wobei $N$ die Anzahl der Windungen sei. Dann ist:

\begin{align*}
  \oint_{\mathcal{S}} \vec{B} \cdot \vec{ds} = \mu_0 \cdot NI_0
\end{align*}

Da das Problem rotationssysmmetrisch ist, ändert das Magnetfeld $\vec{B}$ seine
Richtung, aber nicht seinen Betrag. Das Feld läuft außerdem in Richtung des
Kreisbogens. Wir können also schreiben:

\begin{align*}
  B \oint_{\mathcal{S}} ds = \mu_0 \cdot NI_0
\end{align*}

Der Umfang des Kreises ist $2\pi r$. Damit ist das Ringintegral ausgeführt:

\begin{align*}
  B 2 \pi r = \mu_0 \cdot NI_0
\end{align*}

Umstellen ergibt den von der Aufgabenstellung geforderten Zusammenhang:
\begin{align*}
  B = \frac{\mu_0}{2 \pi} \frac{NI_0}{r}
\end{align*}

\end{document}
