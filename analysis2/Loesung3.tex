\documentclass[a4paper,german,12pt,smallheadings]{scrartcl}
\usepackage[T1]{fontenc}
\usepackage[utf8]{inputenc}
\usepackage{babel}
\usepackage{tikz}
\usepackage{geometry}
\usepackage{amsmath}
\usepackage{amssymb}
\usepackage{float}
\usepackage{enumerate}
%\usepackage{wrapfig}
\usepackage[thinspace,thinqspace,squaren,textstyle]{SIunits}
\restylefloat{table}
\renewcommand{\thefootnote}{\fnsymbol{footnote}}
\geometry{a4paper, top=15mm, left=20mm, right=40mm, bottom=20mm, headsep=10mm, footskip=12mm}
\linespread{1.5}
\setlength\parindent{0pt}
\begin{document}
\begin{center}
\bfseries % Fettdruck einschalten
\sffamily % Serifenlose Schrift
\vspace{-40pt}
Analysis II, Wintersemester 2013/2013, 3. Übungsblatt

Markus Fenske, Luis Herrmann, Tutor: Sebastian Bierke
\vspace{-10pt}
\end{center}

\section*{Aufgabe 3.1}
\begin{enumerate}[a)]
  \item
    \textbf{falsch}

    Die Menge ist nicht offen, weil sie ganz offensichtlich nur aus Randpunkten
    besteht. Es reicht, sich einen kleinen Schritt (beliebig kleines
    $\epsilon$) senkrecht zur Ebene zu bewegen um die Ebene zu verlassen.
  \item
    \textbf{wahr}

    Die Menge ist abgeschlossen, weil sie nur aus Randpunkten besteht (siehe
    oben).
  \item
    \textbf{wahr}

    Die Menge ist offensichtlich zusammenhängend, weil konvex (siehe unten) und
    nicht leer.

  \item
    \textbf{falsch}

    Die Menge ist unbeschränkt (ich kann zu jedem gegebenem beliebig großen
    Abstand $d$ vom Mittelpunkt einen Punkt finden, der in der Menge liegt).

  \item
    \textbf{falsch}

    Die Menge kann nicht kompakt sein, weil sie unbeschränkt ist.
  \item
    \textbf{wahr}

    Die Menge ist konvex, weil bekanntermaßen die Verbindungsgerade zwischen
    zwei beliebigen Punkten in der Ebene komplett in der Ebene liegt.
\end{enumerate}

\section*{Aufgabe 3.2}

\begin{enumerate}[(1)]
  \item
    Wenn $A$ offen ist, kann ich sie durch Grenzübergang verlassen. Sei $a
    \notin A$ ein beliebiger Randpunkt von $A$. Sei $a_n \in A$ eine Folge, die
    für diesen Grenzübergang benutzt werden kann. Dann gilt:

    \begin{equation*}
      \lim_{n \to \infty} a_n = a
    \end{equation*}

    Für $b$ gilt das gleiche:
    \begin{equation*}
      \lim_{n \to \infty} b_n = b
    \end{equation*}

    Also ist:
    \begin{equation*}
      \lim_{n \to \infty} (a_n + b_n) = \lim_{n \to \infty} a_n + \lim_{n \to \infty} b_n = (a + b) \notin A + B
    \end{equation*}

    Also ist $A+B$ offen.
  \item
    Wenn $A$ zusammenhängen ist, kann ich durch zwei Punkte $a_1, a_2 \in A$
    mindestens eine Kurve $k_a(t), t \in [0,1], k_a(0) = a_1, k_a(1) = a_2$
    finden, so dass alle Punkte dieser Kurve in $A$ liegen: $k_a(t) \in A$.

    Für $B$ analog mit analogen Bezeichnungen

    \begin{align*}
      k_a(t) + k_b(t) \in A+B
    \end{align*}

    Seien $C = A+B$. Seien $c_1, c_2 \in C$. Dann kann ich die beiden Punkte schreiben als:

    \begin{align*}
      c_1 = a_1 + b_1 \\
      c_2 = a_2 + b_2
    \end{align*}
  \item

  \item
    Zur Bonusaufgabe:

    Sei die Menge $A := \{(x,y): y = \frac{1}{x}, x \ge 1\}$. Wir
    haben in Aufgabe $2.9$ bewiesen, dass die Graphen stetiger Funktionen $f:
    \mathbb{R} \to \mathbb{R}$ im $\mathbb{R}^2$ abgeschlossene Mengen sind. Da der
    Graph im Punkt $(1,1)$ abgeschnitten ist und dieser Punkt ein Randpunkt ist,
    handelt es sich bei $A$ um eine abgeschlossene Menge.

    Sei die Menge $B$ die $x$-Achse. Diese kann auch als Graph der Funktion $f(x) =
    0$ interpretiert werden und ist daher ebenfalls abgeschlossen.

    Sei $t \in \mathbb{R}, t > 1$. Dann ist $a = (t, \frac{1}{t}) \in A$. Außerdem
    $b = (-t, 0) \in B$. Dann muss $a+b = (0, \frac{1}{t}) \in A+B$ sein.

    Dann gilt
    \begin{equation*}
      \lim_{t \to \infty} (a+b) = (0,0) \notin A+B
    \end{equation*}

    Ich kann die Menge also durch Grenzübergang verlassen, also ist $A+B$ nicht
    abgeschlossen. Daraus folgt: $A + B$ muss nicht abgeschlossen sein, wenn $A$
    und $B$ beide abgeschlossen sind.
\end{enumerate}


\end{document}
