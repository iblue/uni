\documentclass[a4paper,german,12pt,smallheadings]{scrartcl}
\usepackage[T1]{fontenc}
\usepackage[utf8]{inputenc}
\usepackage{babel}
\usepackage{tikz}
\usepackage{geometry}
\usepackage{amsmath}
\usepackage{amssymb}
\usepackage{float}
\usepackage{enumerate}
%\usepackage{wrapfig}
\usepackage[thinspace,thinqspace,squaren,textstyle]{SIunits}
\restylefloat{table}
\renewcommand{\thefootnote}{\fnsymbol{footnote}}
\geometry{a4paper, top=15mm, left=20mm, right=40mm, bottom=20mm, headsep=10mm, footskip=12mm}
\linespread{1.5}
\setlength\parindent{0pt}
\begin{document}
\begin{center}
\bfseries % Fettdruck einschalten
\sffamily % Serifenlose Schrift
\vspace{-40pt}
Analysis II, Wintersemester 2013/2013, 3. Übungsblatt

Markus Fenske, Luis Herrmann, Tutor: Sebastian Bierke
\vspace{-10pt}
\end{center}

\section*{Aufgabe 2.1}
\begin{enumerate}[a)]
  \item
    \textbf{falsch}

    Die Menge ist nicht offen, weil sie ganz offensichtlich nur aus Randpunkten
    besteht. Es reicht, sich einen kleinen Schritt (beliebig kleines
    $\epsilon$) senkrecht zur Ebene zu bewegen um die Ebene zu verlassen.
  \item
    \textbf{wahr}

    Die Menge ist abgeschlossen, weil sie nur aus Randpunkten besteht (siehe
    oben).
  \item
    \textbf{wahr}

    Die Menge ist offensichtlich zusammenhängend, weil konvex (siehe unten) und
    nicht leer.

  \item
    \textbf{falsch}

    Die Menge ist unbeschränkt (ich kann zu jedem gegebenem beliebig großen
    Abstand $d$ vom Mittelpunkt einen Punkt finden, der in der Menge liegt).

  \item
    \textbf{falsch}

    Die Menge kann nicht kompakt sein, weil sie unbeschränkt ist.
  \item
    \textbf{wahr}

    Die Menge ist konvex, weil bekanntermaßen die Verbindungsgerade zwischen
    zwei beliebigen Punkten in der Ebene komplett in der Ebene liegt.
\end{enumerate}


\end{document}
