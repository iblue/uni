\documentclass[a4paper,german,12pt,smallheadings]{scrartcl}
\usepackage[T1]{fontenc}
\usepackage[utf8]{inputenc}
\usepackage{babel}
\usepackage{tikz}
\usepackage{geometry}
\usepackage{amsmath}
\usepackage{amssymb}
\usepackage{float}
\usepackage{enumerate}
%\usepackage{wrapfig}
\usepackage[thinspace,thinqspace,squaren,textstyle]{SIunits}
\restylefloat{table}
\renewcommand{\thefootnote}{\fnsymbol{footnote}}
\geometry{a4paper, top=15mm, left=20mm, right=40mm, bottom=20mm, headsep=10mm, footskip=12mm}
\linespread{1.5}
\setlength\parindent{0pt}
\begin{document}
\begin{center}
\bfseries % Fettdruck einschalten
\sffamily % Serifenlose Schrift
\vspace{-40pt}
Analysis II, Wintersemester 2013/2013, 4. Übungsblatt

Markus Fenske, Luis Herrmann, Tutor: Sebastian Bierke
\vspace{-10pt}
\end{center}

\section*{Aufgabe 4.1}
\begin{enumerate}[(1)]
  \item
    Wir arbeiten hier im euklidischen Raum, also ist die Norm definiert über den
    Satz des Pythagoras in $n$ Dimensionen. Wir wenden die Kettenregel an, nutzen
    die Definition des Skalarproduktes, das Ausklammern in Summen, usw.

    \begin{align*}
      &||\vec{f}(t)||^2 = \sum_{i=1}^{n}f_i(t)^2\\
      \Rightarrow\quad &\frac{d}{dt}\left(||\vec{f}(t)||^2\right)=\frac{d}{dt}\sum_{i=1}^{n}f_i(t)^2\\
      \Leftrightarrow \quad & 2||\vec{f}(t)||\frac{d}{dt}||\vec{f}(t)||=\sum_{i=1}^{n}\frac{d}{dt}(f_i(t)^2)\\
      \Leftrightarrow \quad & 2||\vec{f}(t)||\frac{d}{dt}||\vec{f}(t)||=\sum_{i=1}^{n}2f_i(t)f_i'(t)\\
      \Leftrightarrow \quad & ||\vec{f}(t)||\frac{d}{dt}||\vec{f}(t)||=\underbrace{\sum_{i=1}^{n}f_i(t)f_i'(t)}_{\vec{f}(t)\bullet\vec{f'}(t)}
    \end{align*}
    Unter der Bedingung $||\vec{f}(t)||\neq 0 \Rightarrow \vec{f}(t)\neq 0$:
    \begin{equation*}
    \frac{d}{dt}||\vec{f}(t)||=\frac{\vec{f}(t)\bullet\vec{f'}(t)}{||\vec{f}(t)||}
    \end{equation*}
    
  \item
    Seien alle Komponenten des $k$-dimensionalen Vektorfeldes gleich (das spart Arbeit). Die
    Komponenten sollen $f_1(t), f_2(t), \dots, f_n(t), \dots, f_k(t)$ heißen. Da sich der
    Grenzwert in den Vektor ziehen lässt, reduziert sich das Problem darauf,
    Funktionen derart zu finden, dass $f_n(t_0) = 0$, aber $f_n'(t_0)$
    definiert bzw.  $f_n'(t_0)$ undefiniert.

    Beispiel für Ersteres: $\vec{f}(t) = \vec{0}$. Es gilt für jede Komponente:

    \begin{equation*}
      f_n'(t) = \lim_{h \to 0} \frac{f(t + h) - f(t)}{h} = \lim_{h \to 0} \frac{0}{h} = 0
    \end{equation*}

    Also ist $\vec{f}'(t) = \vec{0}$, insbesondere auch an der Stelle $t = t_0$.

    Beispiel für Zweiteres: $f_n(t) = \sqrt{t_0-t}$. Es gilt $f_n(t_0) =
    \sqrt{0} = 0$ und damit $\vec{f}(t_0) = \vec{0}$. Die Wurzelfunktion ist
    bekanntlich im Nullpunkt nicht differenzierbar:

    \begin{equation*}
      \lim_{h \to 0} f_n'(t_0) = \lim_{h \to 0} \frac{f_n(t_0 + h) - f_n(t_0)}{h} = 
      \lim_{h \to 0} \frac{\sqrt{h} - 0}{h} = \lim_{h \to 0} \frac{1}{\sqrt{h}}
    \end{equation*}

    Dieser Grenzwert divergiert, also existiert $\vec{f}'(t_0)$ nicht.
\end{enumerate}

\section*{Aufgabe 4.2}

Diese Kurve ist offenbar symmetrisch (Begründung: siehe Fußnote auf dem
Aufgabenblatt) und beim Quadrantenübergang nicht differenzierbar. Aufgrund der
Symmetrie ist die Länge in allen 4 Quadranten gleich. Es gilt also

\begin{align*}
  L &= 4 \int\limits_{\overline{\gamma}}  ds \qquad \text{(wobei }\overline{\gamma}\text{ die Kurve im 1. Quadranten ist)} \\
    &= 4 \int_0^{\frac{\pi}{2}} \sqrt{\left(\frac{dx}{dt}\right)^2 + \left(\frac{dy}{dt}\right)^2} dt
\end{align*}

Mit der gegebenen Parametrisierung $x=a \cos^3 t$, $y = a \sin^3 t$:

\begin{align*}
  &= 4 \int_0^{\frac{\pi}{2}}   \sqrt{\left(-3a \sin(t) \cos^2(t) \right)^2 + \left(3a \sin^2(t) \cos(t)\right)^2} \; dt \\
  &= 12a \int_0^{\frac{\pi}{2}} \sqrt{\sin^2(t) \cos^4(t) + \sin^4(t) \cos^2(t)} \; dt \\
  &= 12a \int_0^{\frac{\pi}{2}} \sqrt{(\sin^2(t) + \cos^2(t)) \sin^2(t)\cos^2(t)} \; dt \\
  &= 12a \int_0^{\frac{\pi}{2}} \sqrt{\sin^2(t)\cos^2(t)} \; dt \\
  &= 6a \int_0^{\frac{\pi}{2}}  \sqrt{\sin^2(2t)} \; dt \\
  &= 6a \int_0^{\frac{\pi}{2}}  \sin(2t) \; dt \qquad \text{(weil $\sin(2t) > 0$ für $0 \le t \le \frac{\pi}{2}$)} \\
  &= 3a \int_0^{\pi}            \sin(t) \; dt \\
  &= 6a
\end{align*}

Sei $\overline{\nu}(t)$ die Kurventangente zu $\overline{\gamma}(t)$ im
(Zeit)punkt $t_0$, dann ist diese definiert durch

\begin{equation*}
  \overline{\nu}(t) = \overline{\gamma}(t_0) + (t - t_0) \dot{\overline{\gamma}}(t_0)
\end{equation*}

Mit Einsetzen der Ableitungen die wir weiter oben bereits bestimmt haben,
erhalten wir folgende Gleichung, bei der wir die in der Vorlesung übliche
Unterscheidung zwischen Punkten und Vektoren beibehalten.

\begin{equation*}
  \overline{\nu}(t) = (a \cos^3 t_0, a \sin^3 t_0) + (t-t_0) \begin{pmatrix} -3a \cos^2(t_0) \sin(t_0) \\ 3a \sin^2(t_0) \cos(t_0) \end{pmatrix}
\end{equation*}

Um die Länge der Strecke zwischen den Schnittpunkten mit den Koordinatenachsen
zu berechnen, lösen wir zuerst, bei welchen Zeitpunkt $t$ der Schnitt mit der
$x$- und der $y$-Achse stattfindet.

$x$-Achse:
\begin{align*}
  &0 \overset{!}{=} a \cos^3(t_0) + (t-t_0)(-3a \cos^2(t_0) \sin(t_0)) \\
  \Leftrightarrow\qquad &0 = \cos^3(t_0) + (t-t_0)(-3 \cos^2(t_0) \sin(t_0)) \qquad \text{(erlaubt wegen $a>0$)}\\
  \Leftrightarrow\qquad &0 = \cos(t_0) + (t-t_0)(-3 \sin(t_0)) \qquad \text{(erlaubt wegen $t_0 > 0$)}\\
  \Leftrightarrow\qquad &0 = \frac{1}{3} \cot(t_0) - (t - t_0) \\
  \Leftrightarrow\qquad &t_x = t_0 - \frac{1}{3} \cot(t_0)
\end{align*}

$y$-Achse:
\begin{align*}
  &0 \overset{!}{=} a \sin^3(t_0) + (t-t_0)(3a \sin^2(t_0) \cos(t_0)) \\
  \Leftrightarrow\qquad &t_y = t_0 - \frac{1}{3} \tan(t_0)
\end{align*}

Die Länge der Tangente berechnet sich dann als
\begin{align*}
  L_T &= \int_{t_x}^{t_y} ||\dot{\overline{\nu}}(t)|| dt \\
      &= \int_{t_x}^{t_y} \sqrt{\dot{\overline{\nu}}_1(t)^2 + \dot{\overline{\nu}}_2(t)^2} dt \\
      &= \int_{t_x}^{t_y} 3a \sqrt{\cos^4(t_0)\sin^2(t_0) + \sin^4(t_0) \cos^2(t_0)} dt \\
      &= \int_{t_x}^{t_y} 3a \sqrt{(\cos^2(t_0)\sin^2(t_0) (\cos^2(t_0) + \sin^2(t_0))} dt \\
      &= \int_{t_x}^{t_y} 3a \sqrt{\cos^2(t_0)\sin^2(t_0)} dt \\
      &= 3a \cos(t_0)\sin(t_0) \int_{t_x}^{t_y} dt \\
      &= 3a \cos(t_0)\sin(t_0) (t_y - t_x) \\
      &= 3a \cos(t_0)\sin(t_0) (\frac{1}{3} \cot t_0 - \frac{1}{3} \tan t_0) \\
      &= a \cos(t_0) \sin(t_0) (\cot t_0 - \tan t_0) \\
      &= a \cos^2(t_0) - a \sin^2(t_0) \\
      &= a (\cos^2(t_0) - \sin^2(t_0)) \\
      &= a \cos(2 t_0)
\end{align*}

\section*{Aufgabe 4.3}
\begin{enumerate}[(1)]
  \item Keine Abgabe gefordert
  \item
    \begin{equation*}
      L = \int_a^b || \dot{\overline{\gamma}}(t) || dt
    \end{equation*}

    Mit
    \begin{equation*}
      \dot{\overline{\gamma}}(t) = \begin{pmatrix}
        e^{ct} (c \cos t - \sin t) \\
        e^{ct} (c \sin t - \cos t)
      \end{pmatrix}
    \end{equation*}

    folgt

    \begin{align*}
      L &= \int_a^b \sqrt{e^{2ct} (c \cos t - \sin t)^2 + e^{2ct} (c \sin t + \cos t)}^2 \; dt \\
        &= \int_a^b e^{ct} \sqrt{c^2 \cos^2 t - 2c\cos t \sin t + \sin^2 t + c^2 \sin^2t + 2c \cos t \sin t + \cos^2 t} \; dt \\
        &= \int_a^b e^{ct} \sqrt{1+c^2} \; dt \\
        &= \sqrt{1+c^2} \int_a^b e^{ct} \; dt \\
        &= \sqrt{1+c^2} \frac{e^{bt} - e^{at}}{c} \\
        &= \frac{e^c \sqrt{1+c^2}}{c} (e^b - e^a)
    \end{align*}

    Das ist die Länge der Kurve.

    Für den Grenzübergang $a \to -\infty$ ist:

    \begin{equation*}
      \lim_{a \to -\infty} \frac{e^c\sqrt{1+c^2}}{c} (e^b - e^a) = \frac{e^c\sqrt{1+c^2}}{c} e^b - \frac{e^c\sqrt{1+c^2}}{c} \underbrace{\lim_{a \to -\infty} e^a}_{=0} = \frac{e^{c+b} \sqrt{1+c^2}}{c}
    \end{equation*}

  \item
    Der Kreis ist definiert als
    \begin{equation*}
      \overline{K}_{\overline{0}}(r) = \{(x,y) \in \mathbb{R}^2: d((x,y), \overline{0}) = r \}
    \end{equation*}

    Zu zeigen ist also:
    \begin{equation*}
      \forall r \; \exists \overline{x} \in \overline{K}_{\overline{0}}(r): \overline{x} = \overline{\gamma}(t), t \in \mathbb{R}
    \end{equation*}

    Sei ein beliebiger Radius $r$ vorgegeben. Nach Behauptung exisitiert $\overline{x}, t$ derart, dass
    \begin{equation*}
      \overline{x} = \overline{\gamma}(t)
    \end{equation*}

    Es erscheint sinnvoll, die Punktmenge $\overline{K}$ als Kurve $\overline{\nu}(t)$ zu definieren mit der Parametrisierung
    \begin{equation}
      (x,y) = (r \cos t, r \sin t)
    \end{equation}

    Zu zeigen ist also
    \begin{align*}
      r \cos s &= e^{ct} \cos t \\
      r \sin s &= e^{ct} \sin t
    \end{align*}

    Das ist erfüllt für
    \begin{equation*}
      r = e^{ct} \Rightarrow t = \frac{\ln r}{c} \qquad \text{und} \qquad s=t
    \end{equation*}

    $\ln r$ ist für alle $r > 0$ definiert, also findet man zu jedem Radius $r$
    einen Zeitpunkt $t$, zu dem die Kurve den Kreis schneidet. Da die
    Gleichung(en) nur eine Lösung für $t$ haben, wird der Kreis nur exakt
    einmal geschnitten.

    Zum Schluss ermitteln wir den Kosinus des Schnittwinkels. Dieser ist gegeben durch
    \begin{equation*}
      \cos \phi = \frac{\dot{\overline{\nu}}(s) \bullet \dot{\overline{\gamma}}(t)}{||\dot{\overline{\nu}}(s)|| \cdot ||\dot{\overline{\gamma}}}
    \end{equation*}

    Wir haben gefunden, dass $s=t$. Die Ableitung von $\overline{\gamma}(t)$
    haben wir berechnet, die Ableitung von $\overline{\nu}(t)$ ist

    \begin{equation*}
      \dot{\overline{\nu}}(t) = \begin{pmatrix}
        -r \sin t \\
        r \cos t 
      \end{pmatrix}
    \end{equation*}

    Daraus ergibt sich die Norm
    \begin{equation*}
      ||\dot{\overline{\nu}}(t)|| = \sqrt{r^2 \sin^2 t + r^2 \cos^2 t} = r
    \end{equation*}

    Und für $\overline{\nu}$ haben wir weiter oben im Längenintegral bereits gezeigt, dass
    \begin{equation*}
      ||\dot{\overline{\nu}}(t)|| = e^{ct} \sqrt{1+c^2}
    \end{equation*}

    Damit ergibt sich
    \begin{align*}
      \cos \phi &= \frac{e^{ct} (c \cos t - \sin t)(-r \sin t) + e^{ct} (c \sin t + \cos t)(r \cos t)}{re^{ct}\sqrt{1+c^2}} \\
                &= \frac{re^{ct} (-c\cos t \sin t + \sin^2 t + c \cos t \sin t + \cos^2 t)}{re^{ct} \sqrt{1+c^2}} \\
                &= \frac{1}{\sqrt{1+c^2}}
    \end{align*}
\end{enumerate}

\section*{Aufgabe 4.4}
Gesucht ist die Menge aller Punkte, die in beiden Mengen gleichzeitig enthalten
ist, also die Menge aller Punkte $(x,y,z)$ die folgende Gleichungen erfüllen.

\begin{align*}
  x^2+y^2+z^2 = 4 \\
  z \ge 0 \\
  (x-1)^2 + y^2 = 1
\end{align*}

Wir lösen zuerst die Gleichungen nach $x^2+y^2$

\begin{align*}
  &x^2 + y^2 = 4 - z^2 \\
  &(x-1)^2 + y^2 = 1 \Leftrightarrow x^2 - 2x + 1 + y^2 = 1 \Leftrightarrow x^2+y^2 = 2x
\end{align*}

Durch Gleichsetzen erhalten wir
\begin{equation*}
  2x = 4 - z^2 \Leftrightarrow x = 2 - \frac{z^2}{2}
\end{equation*}

Durch Einsetzen in die umgeformte Kugelgleichung
\begin{align*}
  &\left(z - \frac{z^2}{2}\right)^2 + y^2 = 4 - z^2 \\
  \Leftrightarrow\qquad &y^2 = 4 - z^2 - \left(2 - \frac{z^2}{2}\right)^2 \\
  \Leftrightarrow\qquad &y^2 = 4 - z^2 - \left(4 - 2\cdot2\cdot\frac{z^2}{2} + \frac{z^2}{4}\right) \\
  \Leftrightarrow\qquad &y^2 = - z^2 + 2z^2 - \frac{z^2}{4} \\
  \Leftrightarrow\qquad &y = \pm \sqrt{z^2 - \frac{z^4}{4}} \\
\end{align*}

Die Schnittkurve ist also die Menge aller Punkte, die folgende Gleichung erfüllen

\begin{equation*}
  \overline{\gamma}(t) = (x(z(t)), y(z(t)), z(t)) = \left(2 - \frac{z(t)^2}{2}, \pm \sqrt{z(t)^2 - \frac{z(t)^4}{4}}, z(t)\right)
\end{equation*}

Wir benötigen nur noch eine geeingete Parametrisierung von $z(t)$, derart, dass
ein Minimum und ein Maximum von $z(t)$ festgelegt ist, denn die Schnittkurve
kann nicht unendlich lang sein. Sie muss einen Anfangs und einen Endpunkt
haben.

Der maximale Wert, den $z$ einnehmen kann, ist entsprechend des Kugelradius
$z_\text{max}=2$. An dieser Stelle ist nach Kugelgleichung $x^2+y^2=0$. Dies
ist nur erfüllt für $x=y=0$, was auch vom Zylinder efüllt wird ($(0-1)^2 + 0^2
= 1$).

Der minimale Wert, den $z$ einnehmen kann ist $z_\text{min} = 0$. Der Wert wird
von der Kugel erfüllt, dort ist $x^2 + y^2 = 4 \Leftrightarrow y^2 = 4 - x^2$.
Eingesetzt in die Zylindergleichung ergibt sich $x^2 + (4-x^2) = 2x
\Leftrightarrow x = 2$. Also stimmt auch dieser Wert.

Also lässt sich die Kurve parametrisieren durch die Zusammenstückung der beiden
Halbkurven $\gamma_1$ und $\gamma_2$ $z(t) = t, t \in [0,2]$. Diese durchlaufen
einmal die positive und einmal die negative $y$-Seite.

\begin{align*}
  \overline{\gamma}_1(t) &= \left(2 - \frac{t^2}{2}, -\sqrt{t^2 - \frac{t^4}{4}}, t\right), t \in [0,2] \\
  \overline{\gamma}_2(t) &= \left(2 - \frac{t^2}{2}, \sqrt{t^2 - \frac{t^4}{4}}, t\right), t \in [0,2] \\
\end{align*}

Um nun noch Stetigkeit zu erhalten, definieren wir kurzerhand

\begin{equation*}
  \overline{\gamma}(t) =
  \begin{cases}
    \left(2 - \frac{(2-t)^2}{2}, -\sqrt{(2-t)^2 - \frac{(t-2)^4}{4}}, (2-t)\right) &\mbox{für } t \in [0,2]  \\
    \left(2 - \frac{(t-2)^2}{2}, \sqrt{(t-2)^2 - \frac{(t-2)^4}{4}}, (t-2)\right) &\mbox{für } t \in [2,4]  \\
  \end{cases}
\end{equation*}

Alternativ lässt sich auch eine geschlossene Kurve angeben, die alle Punkte beliebig oft durchläuft. Man fordert eine Parametrisierung der Form:
\begin{align*}
z(t+\tau)=z(t) \quad \text{für gegebenes $\tau$}\\
z_{max}=2\\
z_{min}=0
\end{align*}

Es bietet sich folgender Ansatz an:
\begin{equation*}
z(t)=(z_{max}-z_{min})\sin^2(t)+z_{min}=2\sin^2(t)
\end{equation*}

Damit wird:
\begin{equation*}
\overline{\gamma}(t)=\left(2\sin^2(t)\sqrt{1-\sin^4(t)},2-\sin^4(t),2\sin^2(t)\right)
\end{equation*}


\end{document}
