\documentclass[a4paper,german,12pt,smallheadings]{scrartcl}
\usepackage[T1]{fontenc}
\usepackage[utf8]{inputenc}
\usepackage{babel}
\usepackage{tikz}
\usepackage{geometry}
\usepackage{amsmath}
\usepackage{amssymb}
\usepackage{float}
\usepackage{enumerate}
%\usepackage{wrapfig}
\usepackage[thinspace,thinqspace,squaren,textstyle]{SIunits}
\restylefloat{table}
\renewcommand{\thefootnote}{\fnsymbol{footnote}}
\geometry{a4paper, top=15mm, left=20mm, right=40mm, bottom=20mm, headsep=10mm, footskip=12mm}
\linespread{1.5}
\setlength\parindent{0pt}
\begin{document}
\begin{center}
\bfseries % Fettdruck einschalten
\sffamily % Serifenlose Schrift
\vspace{-40pt}
Analysis II, Wintersemester 2013/2013, 3. Übungsblatt

Markus Fenske, Luis Herrmann, Tutor: Sebastian Bierke
\vspace{-10pt}
\end{center}

\section*{Aufgabe 3.1}
\begin{enumerate}[a)]
  \item
    \textbf{falsch}

    Es reicht, sich einen kleinen Schritt (beliebig kleines
    $\epsilon$) senkrecht zur Ebene zu bewegen um die Ebene zu verlassen.
    Die Menge besteht also ganz offensichtlich nur aus Randpunkten
    besteht und ist daher nicht offen.
  \item
    \textbf{wahr}

    Die Menge ist abgeschlossen, weil sie nur aus Randpunkten besteht (siehe
    oben).
  \item
    \textbf{wahr}

    Die Menge ist offensichtlich zusammenhängend, weil konvex (siehe unten) und
    nicht leer.

  \item
    \textbf{falsch}

    Die Menge ist unbeschränkt (da sie nicht in eine Kugel endlicher Größe passt).
  \item
    \textbf{falsch}

    Die Menge kann nicht kompakt sein, weil sie unbeschränkt ist.
  \item
    \textbf{wahr}

    Die Menge ist konvex, weil bekanntermaßen die Verbindungsgerade zwischen
    zwei beliebigen Punkten in der Ebene komplett in der Ebene liegt.
\end{enumerate}

\section*{Aufgabe 3.2}

\begin{enumerate}[(1)]
  \item
    Wenn $A$ offen ist, kann ich sie durch Grenzübergang verlassen. Seien $a
\notin A$ ein beliebiger Randpunkt von $A$ (der nicht in der Menge liegt,
    weil die Menge offen ist). Es existiert auf jeden Fall eine Folge $a_n \in
A \forall n$, für die gilt:

    \begin{equation*}
      \lim_{n \to \infty} a_n = a
    \end{equation*}

	Fall 1: \\
	Der offensichtliche Fall: Sei B offen. Analog zu $A$ gilt für eine Folge $b_n$ mit den gleichen Eigenschaften wie oben
    und $b \notin B$:
    \begin{equation*}
      \lim_{n \to \infty} b_n = b
    \end{equation*}

    Also ist:
    \begin{equation*}
      \lim_{n \to \infty} (a_n + b_n) = \lim_{n \to \infty} a_n + \lim_{n \to \infty} b_n = a + b, \quad a\notin A,b\notin B \Rightarrow (a+b)\notin A + B
    \end{equation*}

    Also enthält $A+B$ keinen ihrer Randpunkte und ist daher offen.\\
    
    Fall 2:\\
    Sei $B$ geschlossen und $b$ ein Randpunkt von $B$, dann findet man eine Folge $b_n$ die gegen $b$ konvergiert:
    \begin{equation*}
    \lim(b_n)=b, \; b\in B
    \end{equation*}
    
    Dann gibt es konvergente Punktfolgen aus $A+B$, für die:
    \begin{equation*}
    \lim_{n \to \infty} (a_n + b_n) = \lim_{n \to \infty} a_n + \lim_{n \to \infty} b_n = a + b, \quad a\notin A,b\in B \Rightarrow (a+b)\notin A + B
    \end{equation*}
	Der Grenzwert $(a+b)$ ist Randpunkt, aber kein Element der Menge $A+B$, denn dafür müsste auch $a\in A$ sein. ?

  \item
    Wenn $A$ zusammenhängend ist findet man für beliebige Puntke $a_1, a_2 \in A$
    mindestens eine stetige Kurve $k_a(t)$ mit $ t \in [0,1]$, sodass:
    \begin{equation*}
     k_a(0) = a_1 \quad k_a(1) = a_2 \quad \quad k_a(t) \in A \forall t\in[0,1]
     \end{equation*},
	Für $B$ analog mit entsprechenden Bezeichnungen.\\

    Sei $C = A+B$ und $c_1, c_2$ beliebige Elemente aus $C$.
    Dann kann man für diese schreiben:
    \begin{equation*}
    c_1 = a_1 + b_1 \quad \quad c_2 = a_2 + b_2
    \end{equation*}
    Nach Voraussetzung erfüllen die Kurven:
    \begin{equation*}
    c_1 = k_a(0) + k_b(0) \quad \quad c_2=k_a(1)+k_b(1)
    \end{equation*} 
   	
   	Die Kurve $\gamma(t)=k_a(t)+k_b(t)$ verbindet also beliebige Punkte $c_1,c_2\in C$. Als Summe stetiger Kurven ist sie selbst stetig. Zu prüfen bleibt also nur noch, ob:
   	\begin{equation*}
   	\gamma(t)\in C \quad \forall t\in[0,1]
   	\end{equation*}
   	
   	Das ist trivial und leicht zu zeigen. Für beliebiges t:
   	\begin{equation*}
   	\gamma(t)=\underbrace{k_a(t)}_{\in A\forall t\in[0,1]}+\underbrace{k_b(t)}_{\in B\forall t\in[0,1]}
   	\end{equation*}
	
	...erfüllt $\gamma(t) \forall t\in[0,1]$ die Mengendefinition von $C$, bzw $A+B$.
    Also ist $A+B$ zusammenhängend.
  \item
    Wenn $A$/$B$ kompakt ist, heißt das, dass jede Folge $a_m$/$b_m$ von Punkten aus
    $A$/$B$ eine Teilfolge enthält, die gegen einen Grenzwert aus $A$ konvergiert.\\
    Sei $a_{m_k}$ konvergente Teilfolge von $a_m$, $b_{mk}$ konvergente Teilfolge von $b_m$, mit:
    \begin{equation*}
    \lim(a_{mk})=a, \; a\in A \quad \quad \lim(b_{mk})=b, \; b\in B
    \end{equation*}
    
    Betrachte eine Punktfolge $c_m\in C\forall n$. Wir können diese auch schreiben als:
    \begin{equation*}
    c_m=a_m+b_m
    \end{equation*}
    
    Um die Kompaktheit von $C$ zu prüfen, zeigen wir, für alle $c_{m}$ existiert eine in $C$ konvergente Teilfolge $c_{mk}$:
    \begin{equation*}
    \lim(c_{mk})=\lim(a_{mk}+b_{mk})=\lim({a_{mk}})+\lim({b_{mk}})=a + b,\quad a\in A,\; b\in B \Rightarrow a+b \in A + B
    \end{equation*}

    Die obigen Schritte sind aufgrund der Stetigkeit der Vektoraddition legitim und es wir haben die Kompaktheit von $C$, bzw. $A+B$ gezeigt.
  \item
    Sei die Menge:
    \begin{equation*}
    A := \{(x,y): y = \frac{1}{x}, x \ge 1\}
    \end{equation*}
    
    Wir
    haben in Aufgabe $2.9$ bewiesen, dass die Graphen beschränkter, stetiger Funktionen $f:
\mathbb{R} \to \mathbb{R}$ im $\mathbb{R}^2$ abgeschlossene Mengen sind. Da der
    Graph im Punkt $(1,1)$ abgeschnitten ist und dieser Punkt ein Randpunkt ist,
    handelt es sich bei $A$ um eine abgeschlossene Menge.
    Sei ferner:
    \begin{equation*}
    B:=\{(x,y): y=0 \forall x\in \mathbb(R)\}
    \end{equation*}
    Das ist der Graph der stetigen, beschränkten Funktion $f(x) =
0$ auf $\mathbb{R}$. Auch dieser ist abgeschlossen.

    Betrachte die Punktfolgen $a_n = (n, \frac{1}{n}) \in A$ und $b_n = (-n, 0) \in B$ für $n > 1$. Dann definieren wir hierauf eine Punktfolge $(a_n+b_n)\in A+B$.\\
    Falls $A+B$ abgeschlossen ist, konvergieren alle Punktfolgen aus $A+B$ wieder in der Menge. Unsere definierte Punktfolge konvergiert:
    \begin{equation*}
    \lim\limits_{n\to\infty}(a_n+b_n)=\lim\limits_{n\to\infty}\left(n-n,\frac{1}{n}\right)=\lim\limits_{n\to\infty}\left(0,\frac{1}{n}\right)=(0,0)
    \end{equation*}
	
	Aber $(0,0)$ ist offensichtlich nicht Element der Menge $A+B$, denn es gibt keine Punkte $a\in A,b\in B$, aus deren Summe sich der Punkt ergeben könnte, da $b_y>0 \forall b\in B$. Also ist die Menge nicht abgeschlossen.
\end{enumerate}

\section*{Aufgabe 3.3}

\begin{enumerate}[(1)]
  \item
    Sei $D = [0,23]$, sei $E = ]23,42]$ (das sind Teilmengen des
    $\mathbb{R}^1$). Sei $f(x) = 23$, Sei $g(x) = 42$. Dann ist es
    offensichtlich, dass $h(x)$ nicht stetig ist.

  \item
    Wenn sich $D$ und $E$ nicht schneiden, ist offensichtlich, dass $h$ stetig
    ist, denn die Funktion ist auf keinem Randpunkt definiert. Es gibt immer
    eine Definitionslücke zwischen $f$ und $g$.

    Wenn sich $D$ und $E$ schneiden, gibt es eine Schnittmenge $\empty \neq S = D \cap E$.
    Für $x \in S$ gilt laut Aufgabenstellung $f(x) = g(x)$.

    Die Stetigkeit außerhalb der Schnittmenge steht außer Frage, da $f$ und
    $g$ stetig sind.

    Die Stetigkeit innerhalb der Schnittmenge steht ebenfalls außer Frage (aus
    obigen Gründen).

    Interessant sind also nur die Randpunkte der Schnittmenge. Sei $d \in D$
    Randpunkt von $S$ ($S$ ist offen, deswegen $d \neq S$). An diesem Punkt hat
    $f$ den Wert $f(d)$. Sei $s_n$ eine Folge in $S$ mit dem Grenzwert $d$.
    Dann gilt aufgrund der Stetigkeit von $f$.

    \begin{equation*}
      \lim_{n \to \infty} f(s_n) = f(\lim s_n) = f(d)
    \end{equation*}

    Und weil $f$ und $g$ im Bereich von $S$ übereinsteimmen:

    \begin{equation*}
      \lim_{n \to \infty} f(s_n) = g(d)
    \end{equation*}

    Also geht es an dieser Stelle nahtlos weiter, die Funktion $h$ ist stetig.
\end{enumerate}

\section*{Aufgabe 3.4}

Eine Grenzwertbetrachtung mit Limes erscheint legitim, da die Metrik des Funktionsraumes $(C,d)$ mit $C=C[0,2\pi]$ als eine sinnvolle Abstandsfunktion zwischen Funktionen definiert ist:
\begin{equation*}
d(f,g)=\sqrt{\int_{0}^{2\pi}\left(f(t)-g(t)\right)^2dt}
\end{equation*}

Wenn $D$ stetig wäre, müsste für jede Grenzfunktion $g$ gelten:
\begin{equation*}
  \lim\limits_{f\to g} D(f) = D\left(\lim\limits_{f\to g} f\right)=D(g)
\end{equation*}

Dem ist aber nicht so. Betrachten wir die gegebene Funktionsfolge $f_n(x) = \frac{\sin nx}{n}$. Nehmen wir einen beliebigen Grenzwert der Folge, z.B.: 
\begin{equation*}
\lim\limits_{n\to\infty}(f_n)=\lim\limits_{n\to\infty}\frac{\overbrace{\sin(nx)}^{\le 1\forall n,x}}{n}=0
\end{equation*}

Dann ist also der Limes zu betrachten:
\begin{equation*}
 \lim\limits_{f_n\to 0} D(f_n) =  \lim\limits_{n\to \infty} D\left(\frac{\sin(nx)}{n}\right)= \lim\limits_{n\to \infty} \cos(nx)
\end{equation*}

Dieser Ausdruck divergiert und es gilt sicherlich:
\begin{equation*}
  \lim\limits_{f_n\to 0} D(f) \neq D\left(\lim\limits_{f_n\to 0}f\right)=D(0)=0
\end{equation*}

Also ist $D$ nicht stetig.

\end{document}