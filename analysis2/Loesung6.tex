\documentclass[a4paper,german,12pt,smallheadings]{scrartcl}
\usepackage[T1]{fontenc}
\usepackage[utf8]{inputenc}
\usepackage{babel}
\usepackage{tikz}
\usepackage{geometry}
\usepackage{amsmath}
\usepackage{amssymb}
\usepackage{float}
\usepackage{enumerate}
\usepackage{cancel}
\usepackage{pgfplots}
\usepackage{commath}
\pgfplotsset{compat=1.7}
\usepgfplotslibrary{polar}
%\usepackage{wrapfig}
\usepackage[thinspace,thinqspace,squaren,textstyle]{SIunits}
\restylefloat{table}
\renewcommand{\thefootnote}{\fnsymbol{footnote}}
\geometry{a4paper, top=15mm, left=20mm, right=40mm, bottom=20mm, headsep=10mm, footskip=12mm}
\linespread{1.5}
\setlength\parindent{0pt}
\begin{document}
\begin{center}
\bfseries % Fettdruck einschalten
\sffamily % Serifenlose Schrift
\vspace{-40pt}
Analysis II, Wintersemester 2013/2013, 6. Übungsblatt

Markus Fenske, Luis Herrmann, Tutor: Sebastian Bierke
\vspace{-10pt}
\end{center}
\allowdisplaybreaks % Seitenumbrüche in Formeln erlauben
\section*{Aufgabe 6.1}
Für die Richtungsableitung gilt in diesem Fall einfach
\begin{align*}
  \vec{e} \cdot \vec{\operatorname{grad}} f &= e_x \cdot \pd{f}{x} + e_y \cdot \pd{f}{y} \\
                                            &= \frac{1}{\sqrt{5}} \del{2x \arctan\del{xy} + x^2 \frac{y}{1+x^2y^2}}
                                               - \frac{2}{\sqrt{5}} \del{x^2 \frac{x}{1+x^2y^2} - 2}
\end{align*}

Durch Einsetzen von $x=1, y=1$:
\begin{equation*}
  = \frac{\frac{\pi}{2} + \frac{1}{2}}{\sqrt{5}} + \frac{3}{\sqrt{5}} = \del{\pi+7}\frac{\sqrt{5}}{10}
\end{equation*}

\section*{Aufgabe 6.2}
Sei $d$ die Dimension dieses (euklidischen, kartesischen) Raumes. Seien
$\vec{e}_i$ die Einheitsvektoren.

\begin{align*}
  \vec{\operatorname{grad}} \frac{1}{r^k} &= \vec{\operatorname{grad}} \frac{1}{\sqrt{\sum_{i=1}^d x_i^2}^k} \\
                                          &= \sum_{i=1}^d \vec{e}_i \pd{}{x_i} \del{\sum_{i=1}^d x_i^2}^{-\frac{k}{2}} \\
                                          &= \sum_{i=1}^d \vec{e}_i \frac{-k}{\cancel{2}} \del{\sum_{i=1}^d x_i^2}^{-\frac{k}{2}-1} \cdot \cancel{2} x_i \\
                                          &= \sum_{i=1}^d \vec{e}_i \frac{-kx_i}{\sqrt{\sum_{i=1}^d x_i^2}^{k-1}}
\end{align*}

Es ist also die $i$-te Komponente des Gradientenvektorfeldes genau
\begin{equation*}
  = \frac{-k x_i}{r^{k-1}}
\end{equation*}

\end{document}
