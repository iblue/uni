\documentclass[a4paper,german,12pt,smallheadings]{scrartcl}
\usepackage[T1]{fontenc}
\usepackage[utf8]{inputenc}
\usepackage{babel}
\usepackage{tikz}
\usepackage{geometry}
\usepackage{amsmath}
\usepackage{amssymb}
\usepackage{float}
\usepackage{enumerate}
\usepackage{cancel}
\usepackage{pgfplots}
\usepackage{commath}
\pgfplotsset{compat=1.7}
\usepgfplotslibrary{polar}
%\usepackage{wrapfig}
\usepackage[thinspace,thinqspace,squaren,textstyle]{SIunits}
\restylefloat{table}
\renewcommand{\thefootnote}{\fnsymbol{footnote}}
\geometry{a4paper, top=15mm, left=20mm, right=40mm, bottom=20mm, headsep=10mm, footskip=12mm}
\linespread{1.5}
\setlength\parindent{0pt}
\begin{document}
\begin{center}
\bfseries % Fettdruck einschalten
\sffamily % Serifenlose Schrift
\vspace{-40pt}
Analysis II, Wintersemester 2013/2013, 6. Übungsblatt

Markus Fenske, Luis Herrmann, Tutor: Sebastian Bierke
\vspace{-10pt}
\end{center}
\allowdisplaybreaks % Seitenumbrüche in Formeln erlauben
\section*{Aufgabe 6.1}
Für die Richtungsableitung gilt in diesem Fall einfach
\begin{align*}
  \vec{e} \cdot \vec{\operatorname{grad}} f &= e_x \cdot \pd{f}{x} + e_y \cdot \pd{f}{y} \\
                                            &= \frac{1}{\sqrt{5}} \del{2x \arctan\del{xy} + x^2 \frac{y}{1+x^2y^2}}
                                               - \frac{2}{\sqrt{5}} \del{x^2 \frac{x}{1+x^2y^2} - 2}
\end{align*}

Durch Einsetzen von $x=1, y=1$:
\begin{equation*}
  = \frac{\frac{\pi}{2} + \frac{1}{2}}{\sqrt{5}} + \frac{3}{\sqrt{5}} = \frac{\pi+7}{2\sqrt{5}}
\end{equation*}
Im Nachhinein rechtfertigt sich unsere Herangehensweise, denn die angewendete Formel für die Richtungsableitungen ist bei totaler Differenzierbarkeit anwendbar. $f$ ist aber total differenzierbar, denn die partiellen Ableitungen existieren auf dem ganzen Definitionsbereich und sind als Kompositionen stetiger Funktionen stetig.

\section*{Aufgabe 6.2}
Sei $d$ die Dimension dieses (euklidischen, kartesischen) Raumes und $\vec{e}_i$ die Einheitsvektoren. Dann erhalten wir für den Gradienten der betrachteten Funktion $\frac{1}{r^k}$ den folgenden Ausdruck:
\begin{align*}
  \vec{\operatorname{grad}} \frac{1}{r^k} &= \vec{\operatorname{grad}} \frac{1}{\sqrt{\sum_{i=1}^d x_i^2}^k} \\
                                          &= \sum_{i=1}^d \vec{e}_i \pd{}{x_i} \del{\sum_{i=1}^d x_i^2}^{-\frac{k}{2}} \\
                                          &= \sum_{i=1}^d \vec{e}_i \frac{-k}{\cancel{2}} \del{\sum_{i=1}^d x_i^2}^{-\frac{k}{2}-1} \cdot \cancel{2} x_i \\
                                          &= \sum_{i=1}^d \vec{e}_i \frac{-kx_i}{\sqrt{\sum_{i=1}^d x_i^2}^{k+2}}
\end{align*}

Die $i$-te Komponente des Gradientenvektorfeldes ist also genau $\frac{-k x_i}{r^{k+2}}$.
Über alle $i$ summiert ergibt sich also:
\begin{equation*}
\vec{\operatorname{grad}} \frac{1}{r^k} = \sum_{i=1}^{d}\vec{e_i} \frac{-k x_i}{r^{k+2}}=\frac{-k}{r^{k+2}}\sum_{i=1}^{d}\vec{e_i}x_i=\frac{-k}{r^{k+2}}\overline{x}
\end{equation*}

\section*{Aufgabe 6.3}

\begin{enumerate}[(1)]

\item
Berechne $\frac{\partial f}{\partial x}$ und $\frac{\partial f}{\partial y}$ unter besonderer Rücksicht auf $(0,0)$.
\begin{align*}
\partial_xf(\overline{a}): \quad \frac{\partial f}{\partial x}&=\lim\limits_{h\to 0}\frac{f(x+h,y)-f(x,y)}{h}\\
&=\lim\limits_{h\to 0}\frac{\sqrt[3]{(x+h)^3+y^3}-\sqrt[3]{x^3+y^3}}{h}
\end{align*}
Für alle $x,y\neq 0$ können wir bedenkenlos (unter Anwendung der Kettenregel) davon ausgehen, dass die partielle Ableitung definiert ist als:
\begin{align*}
\frac{\partial f}{\partial x}(\overline{a})&=\frac{\partial \sqrt[3]{x^3+y^3}}{\partial x}=3x^2\cdot\left(x^3+y^3\right)^{-\frac{2}{3}}\cdot \frac{1}{3}\\
&=\frac{x^2}{(x^3+y^3)^{\frac{2}{3}}}\\
&=\left(\frac{x}{\sqrt[3]{x^3+y^3}}\right)^2
\end{align*}

Punkte $(x,0)$ erfordern dagegen eine genauere Untersuchung:
\begin{align*}
\frac{\partial f}{\partial x}(x,0)&=\lim\limits_{h\to 0}\frac{\sqrt[3]{(x+h)^3+0^3}-\sqrt[3]{x^3+0^3}}{h}\\
&=\lim\limits_{h\to 0}\frac{\sqrt[3]{(x+h)^3}-\sqrt[3]{x^3}}{h}\\
&=\lim\limits_{h\to 0}\frac{x+h-x}{h}=1
\end{align*}
Das stimmt mit der oben hergeleiteten Ableitungsfunktion überein, denn:
\begin{align*}
\lim\limits_{y\to 0}\left(\frac{x}{\sqrt[^3]{x^3+y^3}}\right)^2=\left(\frac{x}{\sqrt[3]{x^3}}\right)^2=1
\end{align*}
Eine Ausnahme bildet der Punkt $(0,0)$, denn hier ist die oben hergeleitete Ableitungsfunktion nicht definiert.\\

Für die partielle Ableitung nach y sind die Überlegungen und Rechnungen genau analog. Ich beschränke mich daher darauf, festzustellen:
\begin{align*}
\frac{\partial f}{\partial y}(\overline{a})&=\frac{\partial \sqrt[3]{x^3+y^3}}{\partial x}=3x^2\cdot\left(x^3+y^3\right)^{-\frac{2}{3}}\cdot \frac{1}{3}\\
&=\frac{y^2}{(x^3+y^3)^{\frac{2}{3}}}\\
&=\left(\frac{y}{\sqrt[3]{x^3+y^3}}\right)^2
\end{align*}

\item Eine Funktion heißt $C^1$-Funktion, wenn sie in jedem Punkt ihres Definitionsbereiches alle partiellen Ableitungen besitzt und diese als Funktionen stetig sind.\\
\\
Wie wir in Aufgabe (1) gesehen haben, sind beide Partielle Ableitungsfunktionen an einem Punkt des Definitionsbereiches von $f$, nämlich $(0,0)$, nicht definiert. $f$ ist also nicht $C^1$-Funktion.

\item Die Richtungsableitung an der Stelle $(0,0)$ für $\vec{e}=\begin{pmatrix}
a\\b
\end{pmatrix}$ ist gegeben durch:
\begin{align*}
\frac{\partial f}{\partial \vec{e}}(0,0)&=\lim\limits_{h\to 0}\frac{f(\overline{a}+h\vec{e})-f(\overline{a})}{h}\\
&=\lim\limits_{h\to 0}\frac{\sqrt[3]{(0+a)^3+(0+b)^3}}{h}\\
&=\lim\limits_{h\to 0}\sqrt[3]{\frac{a^3+b^3}{h^3}}=\sqrt[3]{\lim\limits_{h\to 0}\frac{a^3+b^3}{h^3}}\\
\end{align*}
Damit der Ausdruck konvergiert, müssen wir fordern:
\begin{equation*}
a^3+b^3=0
\end{equation*}
In der Aufgabenstellung ist gefordert, dass:
\begin{equation*}
a^2+b^2=1 \quad \Leftrightarrow \quad b^2=1-a^2 \quad \Leftrightarrow \quad b=\pm\sqrt{1-a^2}
\end{equation*}
Wir müssen also fordern:
\begin{align*}
& a^3+(\sqrt{1-a^2})^3=0\\
\Leftrightarrow \quad & (\sqrt{1-a^2})^3=-a^3\\
\Leftrightarrow \quad & (-a)^3=(\sqrt{1-a^2})^3\\
\Leftrightarrow \quad & -a=\sqrt{1-a^2}\\
\Leftrightarrow \quad & a^2=1-a^2\\
\Leftrightarrow \quad & a=\pm\frac{1}{\sqrt{2}} \quad \Rightarrow \quad b=\pm\sqrt{1-\frac{1}{\sqrt{2}^2}}=\pm\frac{1}{\sqrt{2}}
\end{align*}

Es folgt also, dass die Richtungsableitung unter den gegeben Bedingungen nur für die Richtungsvektoren konvergiert:
\begin{equation*}
\vec{e}=\frac{1}{\sqrt{2}}\begin{pmatrix}
\pm1\\\mp1
\end{pmatrix}
\end{equation*}

\item Gemäß Beobachtung nach 3.3.6 im Skript folgt aus der totalen Differenzierbarkeit eines Skalarfelds die Existenz aller Richtungsableitungen. Wie wir in (c) gesehen haben, ist das nicht der Fall. Folglich kann per Umkehrschluss das Skalarfeld nicht total differenzierbar sein.

\end{enumerate}

\section*{Aufgabe 6.4}
Der für Skalarfelder normalerweise gültige Mittelwertsatz kann nicht einfach auf Vektorfelder übertragen werden.\\
Der Grund ist folgender: Zum Nachweis des Mittelwertsatzes für Skalarfelder hatten wir eine skalare Hilfsfunktion $\phi(t)$ eingeführt:
\begin{equation*}
\phi(t)=f\left(\gamma_1(t),...,\gamma_n(t)\right)
\end{equation*}
... und anschließend den Mittelwertsatz der Differentialrechnung aus dem letzten Semester angewandt. Für ein Vektorfeld $\vec{f}$ ist unser Argument aber nicht mehr gültig.\\

Wir können aber einen alternativen Ausdruck über den Hauptsatz der Differentialrechnung herleiten. Dazu überlegen wir uns zunächst, dass der Hauptsatz auch für Vektorfelder gilt. Für eine Funktion $f(x)$:
\begin{equation*}
f(b)-f(a)=\int_{a}^{b}f'(x)dx
\end{equation*}
Wobei mit $f'(x)$ gerade die Ableitungsfunktion von $f(x)$ gemeint ist. Betrachten wir $n$ solcher Gleichungen:
\begin{equation*}
f_i(b)-f_i(a)=\int_{a}^{b}f_i'(x)dx
\end{equation*}
Dann kann man sich überlegen, dass die $f_i(x)$ den Komponenten eines Vektors entsprechen:
\begin{align*}
&\begin{pmatrix}
f_1(b)-f_1(a)\\\vdots\\ f_n(b)-f_n(a)\end{pmatrix}
=\begin{pmatrix}
\int_{a}^{b}f_1'(x)dx\\\vdots\\\int_{a}^{b}f_n'(x)dx
\end{pmatrix}\\
\Leftrightarrow \quad &\begin{pmatrix}
f_1(b)\\\vdots\\f_n(b)
\end{pmatrix}-\begin{pmatrix}
f_1(a)\\\vdots\\f_n(a)
\end{pmatrix}=\int_{a}^{b}\begin{pmatrix}
f_1'(x)dx\\\vdots\\f_n'(x)
\end{pmatrix}
dx\\
\Leftrightarrow \quad &\vec{f}(b)-\vec{f}(a)=\int_{a}^{b}\vec{f'}(x)dx\\
\end{align*}
Das gilt für $C^1$-Vektorfelder genauso, wenn sich eine geeignete Parametrisierung $t$ finden lässt. Durch die Voraussetzung von $C^1$ existiert $\vec{f'}(x)$, welcher gerade die Jacobi-Matrix ist, welche aus den Partiellen Ableitungen der Funktion auf dem Definitionsbereich von $f$ besteht; durch die Parametrisierung kommt noch ein Vektor dazu, welcher wieder dafür sorgt, dass auf der rechten Seite der Gleichung ein Vektor steht, wie wir gleich sehen werden.

Es gilt:
\begin{equation*}
\vec{f}\left(\overline{a}+\vec{v}\right)-\vec{f}(a)=\int_{\overline{a}}^{\overline{a}+\vec{v}}  \vec{f'}(\overline{x})ds
\end{equation*}

$\vec{f'}(\overline{x})$ meint die totale Ortsableitung der Vektorfunktion. Diese ist aber gerade gegeben durch die entsprechende Jacobi-Matrix:
\begin{equation*}
J_{\vec{f}}(\overline{x})=\begin{pmatrix}
\frac{\partial f_1}{\partial x_1}(\overline{x}) & \dots & \frac{\partial f_1}{\partial x_n}(\overline{x})\\ \vdots & \ddots & \vdots\\ \frac{\partial f_n}{\partial x_1}(\overline{x}) &
\dots & \frac{\partial f_n}{\partial x_n}(\overline{x})
\end{pmatrix}
\end{equation*}

Wir schreiben also:
\begin{equation*}
\vec{f}\left(\overline{a}+\vec{v}\right)-\vec{f}(a)=\int_\gamma J_{\vec{f}}(\overline{x})ds
\end{equation*}

Das Integral lässt sich als Kurvenintegral über $\overline{\gamma}(t)$ auffassen, womit die gerade Verbindungsstrecke zwischen den Punkten $\overline{a}$ und $\overline{a}+\vec{v}$ gemeint ist:
\begin{equation*}
\overline{\gamma}(t)=\overline{a}+\vec{v}t \quad \quad \overline{\gamma}(0)=\overline{a}; \;\; \overline{\gamma}(1)=\overline{a}+\vec{v}
\end{equation*}
Das führt uns auf:
\begin{align*}
\vec{f}(\overline{a}+\vec{v})-\vec{f}(\overline{a})&=\int_{0}^{1}J_{\vec{f}}(\overline{\gamma}(t))\cdot \dot{\overline{\gamma}}(t)dt\\
&=\int_{0}^{1}J_{\vec{f}}(\overline{a}t+\vec{v})\vec{v}dt\\
&=\left(\int_{0}^{1}J_{\vec{f}}\left(\overline{a}t+\vec{v}\right)dt\right)\vec{v}
\end{align*}
...da $\vec{v}$ nicht von $t$ abhängt, kann dieser aus dem Integral rausgezogen werden und wir kommen auf das gewünschte Ergebnis.

\end{document}