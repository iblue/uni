\documentclass[a4paper,german,12pt,smallheadings]{scrartcl}
\usepackage[T1]{fontenc}
\usepackage[utf8]{inputenc}
\usepackage{babel}
\usepackage{tikz}
\usepackage{geometry}
\usepackage{amsmath}
\usepackage{amssymb}
\usepackage{float}
\usepackage{enumerate}
%\usepackage{wrapfig}
\usepackage[thinspace,thinqspace,squaren,textstyle]{SIunits}
\restylefloat{table}
\renewcommand{\thefootnote}{\fnsymbol{footnote}}
\geometry{a4paper, top=15mm, left=20mm, right=40mm, bottom=20mm, headsep=10mm, footskip=12mm}
\linespread{1.5}
\setlength\parindent{0pt}
\begin{document}
\begin{center}
\bfseries % Fettdruck einschalten
\sffamily % Serifenlose Schrift
\vspace{-40pt}
Analysis II, Wintersemester 2013/2013, 1. Übungsblatt

Markus Fenske, Luis Herrmann, Tutor: undefined
\vspace{-10pt}
\end{center}

\section*{Aufgabe 1.1}

\textbf{Voraberklärung:} Die lockeren Schreibweisen $\int_{a}^{\infty} \dots$ und
$\left[\dots\right]_a^\infty$ sind hier und in allen folgenden Übungsblättern
wie folgt definiert, in der Annahme, dass diese Grenzwerte existieren.
\begin{equation*}
  \int_{a}^{\infty} \dots \; := \lim\limits_{b \to \infty}\int_{a}^{b} \dots \qquad \qquad 
  \text{und}
  \qquad \qquad \left[\dots\right]_a^\infty := \lim\limits_{b \to \infty}\left[\dots\right]_a^b
\end{equation*}

\rule{\textwidth}{0.4pt}
\vspace{4mm}


Wir folgen dem Hinweis aus der Aufgabenstellung und zeigen zuerst für $k=0$:

\begin{equation*}
  \int_0^\infty e^{-nx} \; dx = \left[ -\frac{1}{n} e^{-nx} \right]_0^\infty = \underbrace{\left(\lim_{x \to \infty} - \frac{1}{n} e^{-nx} \right)}_{=0} + \frac{e^0}{n} = \frac{1}{n}
\end{equation*}

Anschließend für $k > 0$:

\begin{equation*}
  \int_0^\infty x^k \cdot e^{-nx} \; dx = \text{?}
\end{equation*}

Wir benutzen partielle Integration (Kurzschreibweise)

\begin{equation*}
  \int_a^b f' \cdot g = \left[f \cdot g\right]_a^b - \int_a^b f \cdot g'
\end{equation*}

mit

\begin{align*}
  f'(x) &= e^{-nx} & g(x) &= x^k \\
        &\Downarrow & &\Downarrow\\
  f(x) &= -\frac{1}{n} e^{-nx} & g'(x) &= kx^{k-1}
\end{align*}

und erhalten:

\begin{align*}
  \int_0^\infty x^k \cdot e^{-nx} \; dx  &= \overbrace{\left[-\frac{1}{n} e^{-nx} \cdot x^k\right]_0^\infty}^{=0-0\footnotemark[1]} - \int_0^\infty -\frac{1}{n}e^{-nx} \cdot kx^{k-1} \; dx \\
                                         &= \int_0^\infty \frac{1}{n}e^{-nx} \cdot kx^{k-1} \; dx \\
                                         &= \frac{k}{n} \int_0^\infty e^{-nx} \cdot x^{k-1} \; dx
\end{align*}

\footnotetext[1]{
  Falls das nicht offensichtlich sein sollte, gilt das wegen
  \begin{equation*}
  \left[-\frac{1}{n} e^{-nx} \cdot x^k\right]_0^\infty = \lim\limits_{x \to \infty}\left(-\frac{1}{e^{nx}}\right) - \frac{0^k}{e^{n\cdot 0}}=0 + 0
  \end{equation*}
}

Nach $k$-facher Rekursion erhalten wir demnach:

\begin{equation*}
  \int_0^\infty x^k \cdot e^{-nx} \; dx = \underbrace{\frac{k}{n} \cdot \frac{k-1}{n} \cdot \frac{k-2}{n} \cdot \dots \cdot \frac{k-k}{n}}_{k\text{ Faktoren}} \cdot \frac{1}{n} = \frac{k!}{n^{k+1}}
\end{equation*}

Was zu beweisen war.


\section*{Aufgabe 1.2}

\begin{enumerate}[(1)]
\item Stimmt.
\item Es gilt
  \begin{equation*}
    \sum_{n=1}^\infty \frac{x}{e^{nx}} = x \cdot \left(\sum_{n=1}^\infty \left(e^{-x}\right)^n\right) = x \cdot \left(\left(\sum_{n=0}^\infty \left(e^{-x}\right)^n\right) - 1\right)
  \end{equation*}

  Dies enthält die Geometrische Reihe mit $q=e^{-x}$, sie konvergiert
  für $|q|<1$.  Da wir auf einem Interval $[a,b]$ mit $a > 0$ operieren und
  $e^{-x} < 1$ für $x>0$, ist $|q| < 1$, also konvergiert die Reihe.

  \begin{equation*}
      = x \cdot \left( \frac{1}{1-e^{-x}} - 1 \right) = x \frac{1 - (1-e^{-x})}{1-e^{-x}} = x \frac{e^{-x}}{e^{-x} (e^x - 1)} = \frac{x}{e^x-1}
  \end{equation*}

  Für gleichmäßige Konvergenz muss für alle $x \in [a,b], 0 < a < b < \infty$ gegeben sein:

  \begin{align*}
    & \quad \lim_{m \to \infty} \left| g_m(x) - g(x) \right| = 0 \\
    \Leftrightarrow & \quad \lim_{m \to \infty} \left|\left(\sum_{n=1}^{m}\frac{x}{e^{nx}}\right) - x \left(\frac{1}{1-e^{-x}} - 1\right)\right| = 0 \\
    \Leftrightarrow & \quad \lim_{m \to \infty} x \left|\left(\sum_{n=1}^{m} \left(e^{-x}\right)^n \right) - \left(\frac{1}{1-e^{-x}} - 1\right)\right| = 0 \\
    \Leftrightarrow & \quad \lim_{m \to \infty} x \left|\left(\frac{1-e^{-x(m+1)}}{1-e^{-x}} - 1 \right)- \left(\frac{1}{1-e^{-x}} - 1\right)\right| = 0 \\
    \Leftrightarrow & \quad \lim_{m \to \infty} x \left|\left(\frac{1-e^{-x(m+1)}}{1-e^{-x}} \right)- \left(\frac{1}{1-e^{-x}} \right)\right| = 0 \\
    \Leftrightarrow & \quad \lim_{m \to \infty} x \left|\left(\frac{-e^{-x(m+1)}}{1-e^{-x}} \right)\right| = 0 \\
  \end{align*}

  Da $x > 0$ bleibt der Nenner stets größer als $0$. Es muss also gelten

  \begin{align*}
    \lim_{n \to \infty} x e^{-xn} = 0 \\
  \end{align*}

  Was für $x > 0$ immer wahr ist. Somit konvergiert die Reihe gleichmäßig.

  \item 
  Besteht aus 3 Teilschritten.
  Der erste Teiilschritt 
  \begin{equation*}
    \sum_{n=1}^{\infty}\frac{1}{n^2}=\sum_{n=1}^{\infty}\int_{0}^{\infty}\frac{xdx}{e^{nx}}
  \end{equation*}
  ist einfaches Einsetzen von Teil (1).

  Der zweite Teilschritt

  \begin{equation*}
    \dots %=\lim\limits_{m \to \infty}\left(\sum_{n=1}^{m}\int_{0}^{\infty}\frac{xdx}{e^{nx}}\right)
  \end{equation*}

  ist die Anwendung des Satzes der majorisierten Konvergenz. Dieser besagt, dass

  \begin{equation*}
    \lim_{n \to \infty} \int_0^\infty f_n(x) \; dx = \int_0^\infty f(x) \; dx
  \end{equation*}

  erlaubt ist, wenn
  \begin{enumerate}[a)]
    \item Alle Partialsummenfunktionen $f_n$ stetig sind.
    \item Die Folge auf jedem kompakten Teilintervall von $[0,\infty[$ gleichmäßig gegen $f$ konvergiert.
    \item Eine Funktion $g$ existiert, für die gilt $|f_n(x)| \le g(x)$, $\int_0^\infty g(x)$ existiert
  \end{enumerate}

  Für $f_n(x) = \sum_{n=1}^\infty xe^{-nx}$ und $f(x) = x/(e^x - 1)$ gilt:
  \begin{enumerate}[a)]
    \item Alle $f_n$ sind als Verknüpfung stetiger Funktionen stetig.
    \item Haben wir in (2) bewiesen
    \item Es gilt $|f_n(x)| \le f(x)$ (haben wir in (2) auch bewiesen) und $\int_0^\infty f(x)$ existiert.
  \end{enumerate}

  Deswegen gilt
  \begin{equation*}
    \sum_{n=1}^{\infty}\int_{0}^{\infty}\frac{xdx}{e^{nx}}
    =\lim\limits_{m \to \infty}\left(\sum_{n=1}^{m}\int_{0}^{\infty}\frac{xdx}{e^{nx}}\right)
    =\lim\limits_{m \to \infty} \int_{0}^{\infty} \left(\sum_{n=1}^{m}\frac{x}{e^{nx}}\right) \; dx
    =\lim\limits_{m \to \infty} \int_{0}^{\infty} f_m(x) \; dx
  \end{equation*}

  Nach Satz der majorisierten Konvergenz (deren Anwendbarkeit wir gerade begründet haben) ist
  \begin{equation*}
    \lim\limits_{m \to \infty} \int_{0}^{\infty} f_m(x) \; dx = \int_0^\infty f(x) \; dx = \int_0^\infty \frac{x \; dx}{e^x-1}
  \end{equation*}

  Was zu zeigen war.
\end{enumerate}
\section*{Aufgabe 1.3}
\begin{enumerate}[(1)]
\item Gegenbeispiel. Sei

\begin{equation*}
  f_n(x) = \begin{cases} 
    0           & \mbox{wenn} \; x = 0 \\
    \frac{1}{x + \frac{1}{n}} & \mbox{wenn} \; x \neq 0
  \end{cases}
\end{equation*}


Man sieht die Grenzfunktion sofort als

\begin{equation*}
  f(x) = \begin{cases} 
    0           & \mbox{wenn} \; x = 0 \\
    \frac{1}{x} & \mbox{wenn} \; x \neq 0
  \end{cases}
\end{equation*}

Die Grenzfunktion $f$ ist ganz klar unbeschränkt ($\lim_{x \to 0} f(x) = \infty$), während alle $f_n(x)$ beschränkt sind ($\lim_{x \to 0} f_n = n$).

\item
  Seien alle $f_n$ für alle $x_0$ im Definitionsbereich beschränkt:

  \begin{equation*}
    \lim_{x \to x_0} |f_n(x)| \neq \infty
  \end{equation*}

  Konvergiere außerdem $f_n$ gleichmäßig gegen $f$:

  \begin{align*}
    &\quad \lim_{x \to x_0} \lim_{n \to \infty} |f_n(x) - f(x)| = 0 \\
    \Leftrightarrow&\quad \lim_{x \to x_0} \lim_{n \to \infty} |f_n(x)| = \lim_{x \to x_0} |f(x)| \\
  \end{align*}

  Daraus folgt
  \begin{equation*}
    \lim_{x \to x_0} |f(x)| \neq \infty
  \end{equation*}

  Also muss $f$ beschränkt sein.
\end{enumerate}
\section{Aufgabe 1.4}
Es ist sinnvoll, $f_n(x)$ gleich auf gleichmäßige Konvergenz zu prüfen, da Beobachtung auf 9.1.2 im Skript zufolge aus gleichmäßiger Konvergenz stets punktweise Konvergenz gegen die selbe Grenzfunktion folgt.\\

Sei ein beliebig kleines $\epsilon>0$ vorgegeben, so muss es ein $N \in \mathbb{N}$ geben, sodass:

\begin{equation*}
|f_n(x)-f(x)|<\epsilon \quad \forall n \geq N, \quad \forall x
\end{equation*}

Wir prüfen zunächst, ob eine Grenzfunktion zu:
\begin{equation*}
f_n(x):= \frac{x}{1+nx^2}
\end{equation*}

...existiert. Ohne Mühe (und ohne formalen Beweis) erkennen wir:
\begin{equation*}
\lim\limits_{n \to \infty} \left(\frac{x}{1+nx^2}\right)=0 \quad \forall x
\end{equation*}

...denn $f_n(x)$ verhält sich für hinreichend große $n$ in etwa wie $\frac{1}{n}$. Die Grenzfunktion ist folglich die Nullfunktion. Zu finden ist ein N, sodass:

\begin{equation*}
|f_n(x)-0|=|f_n(x)|<\epsilon \quad \forall n \geq N, \;\; \forall x
\end{equation*}

Alle Funktionen der Folge $f_n(x)$ konvergieren:

\begin{equation*}
\lim\limits_{x \to \infty} \left(\frac{x}{1+nx^2}\right)=0
\end{equation*}

...denn für hinreichend große $x$ verhält sich eine beliebige Funktion der Folge wie $\frac{1}{x} \rightarrow 0$.

Die Ableitung einer beliebigen Funktion der Folge ist:

\begin{equation*}
f_n'(x)=\frac{(1+nx^2)- x\cdot 2nx}{\left(1+nx^2\right)^2}=\frac{1-nx^2}{\left(1+nx^2\right)^2}\\
\end{equation*}

Lokale Extrema erhält man somit für:
\begin{align*}
f_n'(x)=0\\
\Leftrightarrow \quad \frac{1-nx^2}{\left(1+nx^2\right)^2}=0\\
\Leftrightarrow \quad 1-nx^2=0\\
\Leftrightarrow \quad x=\overset{+}{-}\sqrt{\frac{1}{n}}
\end{align*}

Einsetzen in $f_n(x)$:
\begin{equation*}
f_n(x)=\frac{\sqrt{\frac{1}{n}}}{1+n\cdot\frac{1}{n}}=\frac{1}{2\sqrt{n}}
\end{equation*}

Geben wir also ein $\epsilon>0$ vor, so finden wir ein hinreichend großes $n$, sodass:

\begin{equation*}
\frac{1}{2\sqrt{n}} < \epsilon
\end{equation*}

Und nennen dieses $N$. Offensichtlich ist dann:

\begin{equation*}
|f_n(x)|\leq\left|\frac{1}{2\sqrt{n}}\right| < \epsilon \quad \forall n\geq N, \;\;\forall x
\end{equation*}

Damit ist der Beweis für gleichmäßige Konvergenz erbracht und es liegt auch punktweise Konvergenz vor.\\
\\

Als nächstes betrachten wir die Ableitungsfolge $f_n'(x)$. Für ein beliebiges Glied der Folge gilt:

\begin{equation*}
f_n'(x)=\frac{1-nx^2}{\left(1+nx^2\right)^2}
\end{equation*}

Die Ableitung der Grenzfunktion $f(x)=0$ lautet:
\begin{equation*}
f'(x)=0
\end{equation*}

Zu prüfen ist, für welche $x$ erfüllt wird:

\begin{equation*}
\lim\limits_{n \to \infty}f_n'(x)=f'(x)=0
\end{equation*}

Dazu nehmen wir eine Fallunterscheidung vor:

\subsection*{$|x|\ge0:$}
Ohne großen Aufwand sehen wir, dass bei fixiertem x und hinreichend großem n sich $f_n'(x)$ im wesentlichen wie $\left(\frac{-1}{n}\right)\rightarrow 0$ verhält.\\
Ohne formalen Beweis folgern wir:

\begin{equation*}
\lim\limits_{n \to \infty}f_n'(x)=f_n(x)
\end{equation*}

\subsection*{$x=0:$}
\begin{equation*}
f_n'(x)=\left(\frac{1-0}{\left(1+0^2\right)^2}\right)=1 \quad \forall n\in\mathbb{N}
\end{equation*}

$f_n'(x)$ ist stationäre Folge und konvergiert als solche:

\begin{equation*}
\lim\limits_{n \to \infty} f_n'(x)=1 \neq f'(x)
\end{equation*}

Wir finden also, dass $f_n'(x)$ auf den offenen Teilintervallen $I_1=]-\infty,0[$ und $I_2=]0,\infty[$ konvergiert. Zu prüfen bleibt, ob innerhalb aller kompakten Teilintervalle von diesen lediglich punktweise oder doch gleichmäßige Konvergenz vorliegt.\\
Läge gleichmäßige Konvergenz vor, müssten ein $N$ finden, sodass bei gegebenem $\epsilon>0$:

\begin{equation*}
|f_n'(x)-f'(x)|=|f_n'(x)-0|=|f_n'(x)|<\epsilon \quad \forall n\ge N, \;\; \forall x\in I_{1/2}
\end{equation*}

Dazu gehen wie vor wie beim letzten Mal und ermitteln Extrema.\\
Unter Anwendung der Quotientenregel ermitteln wir:
\begin{align*}
f_n''(x)=\frac{-2nx\left(1+nx^2\right)^2-\left(1-nx^2\right)\cdot 2nx\left(1+nx^2\right)\cdot 2}{\left(1+nx^2\right)^4}\\
\Leftrightarrow \quad f_n''(x)=\frac{-2nx\left(1+nx^2\right)\left(\left(1+nx^2\right)+2\left(1-nx^2\right)\right)}{\left(1+nx^2\right)^4}\\
\Leftrightarrow \quad f_n''(x)=\frac{2nx\left(nx^2-3\right)}{\left(1+nx^2\right)^3}
\end{align*}

Bedingung für Extrema:
\begin{align*}
f_n''(x)=\frac{2nx\left(nx^2-3\right)}{\left(1+nx^2\right)^3}=0\\
\Rightarrow \quad 2nx\left(nx^2-3\right)=0\\
\Leftrightarrow \quad (1)\;x=0\notin I_{1/2} \quad \quad (2)\;nx^2-3=0\\
\Rightarrow \quad x=\overset{+}{-}\sqrt{\frac{3}{n}} 
\end{align*}

Einsetzen in $fn'(x)$:
\begin{equation*}
fn'(x)=\frac{1-3}{\left(1+3\right)^2}=-\frac{1}{8}
\end{equation*}

Unabhängig von der Wahl von n erhalten wir jeweils auf $I_1$ und $I_2$ als globale Extrema von $fn'(x)=-\frac{1}{8}$. Es gilt:

\begin{equation*}
|fn'(x)|\leq\left|-\frac{1}{8}\right| \quad \forall x\in I_{1/2} \forall n
\end{equation*}

Das heißt aber: Geben wir ein $\epsilon>0$ vor, welches $\epsilon<-\frac{1}{8}$ erfüllt, so finden wir kein einziges n, sodass:

\begin{equation*}
|fn'(x)|\le\epsilon \quad \forall x \in I_{1/2}
\end{equation*}

Damit kann $f_n'(x)$ auf den entsprechenden Intervallen unmöglich gleichmäßig gegen $f'(x)$ konvergieren.

\end{document}
