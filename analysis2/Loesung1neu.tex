\documentclass[a4paper,german,12pt]{scrartcl}
\usepackage[T1]{fontenc}
\usepackage[utf8]{inputenc}
\usepackage{babel}
\usepackage{tikz}
\usepackage{geometry}
\usepackage{amsmath}
\usepackage{amssymb}
\usepackage{float}
\usepackage{enumerate}
%\usepackage{wrapfig}
\usepackage[thinspace,thinqspace,squaren,textstyle]{SIunits}
\restylefloat{table}
\geometry{a4paper, top=15mm, left=20mm, right=40mm, bottom=20mm, headsep=10mm, footskip=12mm}
\linespread{1.5}
\setlength\parindent{0pt}
\begin{document}
\begin{center}
\bfseries % Fettdruck einschalten
\sffamily % Serifenlose Schrift
\vspace{-40pt}
Analysis II, Wintersemester 2013/2013, 1. Übungsblatt

Markus Fenske, Tutor: undefined
\vspace{-10pt}
\end{center}

\section*{Aufgabe 1.1}

Vorab soll geklärt werden: Mit $\int_{a}^{\infty}$ und $\left[\cdots\right]_a^\infty$ sind stets:
\begin{equation*}
\lim\limits_{b \to \infty}\int_{a}^{b} \quad \quad \lim\limits_{b \to \infty}\left[\cdots\right]_a^b
\end{equation*}

gemeint, in der Annahme, dass diese Grenzwerte existieren.
Wir folgen dem Hinweis aus der Aufgabenstellung und zeigen zuerst für $k=0$:

\begin{equation*}
  \int_0^\infty e^{-nx} \; dx = \left[ -\frac{1}{n} e^{-nx} \right]_0^\infty = \underbrace{\left(\lim_{x \to \infty} - \frac{1}{n} e^{-nx} \right)}_{=0} + \frac{e^0}{n} = \frac{1}{n}
\end{equation*}

Anschließend für $k > 0$:

\begin{equation*}
  \int_0^\infty x^k \cdot e^{-nx} \; dx = \text{?}
\end{equation*}

Wir benutzen partielle Integration (Kurzschreibweise)

\begin{equation*}
  \int_a^b f' \cdot g = \left[f \cdot g\right]_a^b - \int_a^b f \cdot g'
\end{equation*}

mit

\begin{align*}
  f'(x) &= e^{-nx} & g(x) &= x^k \\
        &\Downarrow & &\Downarrow\\
  f(x) &= -\frac{1}{n} e^{-nx} & g'(x) &= kx^{k-1}
\end{align*}

und erhalten:

\begin{align*}
  \int_0^\infty x^k \cdot e^{-nx} \; dx &= \overbrace{\left[-\frac{1}{n} e^{-nx} \cdot x^k\right]_0^\infty}^{=0-0\; (1)} - \int_0^\infty -\frac{1}{n}e^{-nx} \cdot kx^{k-1} \; dx \\
                                         &= \int_0^\infty \frac{1}{n}e^{-nx} \cdot kx^{k-1} \; dx \\
                                         &= \frac{k}{n} \int_0^\infty e^{-nx} \cdot x^{k-1} \; dx
\end{align*}

(1) Falls dieser Schritt zu schnell ging hier nochmal die Nebenrechnung:
\begin{equation*}
\left[-\frac{1}{n} e^{-nx} \cdot x^k\right]_0^\infty = \lim\limits_{x \to \infty}\left(-\frac{1}{e^{nx}}\right) + \frac{0^k}{e^{n\cdot 0}}=0 + 0
\end{equation*}

Nach k-facher Rekursion erhalten wir demnach:

\begin{equation*}
  \int_0^\infty x^k \cdot e^{-nx} \; dx = \underbrace{\frac{k}{n} \cdot \frac{k-1}{n} \cdot \frac{k-2}{n} \cdot \dots \cdot \frac{1}{n}}_{k\text{ Faktoren}} \cdot  \int_{0}^{\infty} \frac{x^0dx}{e^{nx}}= \frac{k!}{n^k}\cdot\left[-\frac{1}{ne^{nx}}\right]_0^\infty=\frac{k!}{n^{k+1}}
\end{equation*}

Was zu beweisen war.


\section*{Aufgabe 1.2}

Gemäß vorheriger Aufgabe ist:

\begin{equation*}
\int_{0}^{\infty}\frac{x^kdx}{e^{nx}}=\frac{k!}{n^{k+1}}
\end{equation*}

Für k=1:
\begin{equation*}
\int_{0}^{\infty}\frac{xdx}{e^{nx}}=\frac{1}{n^2}
\end{equation*}

Zu zeigen ist:

\begin{equation*}
\sum_{n=1}^{\infty}=\sum_{n=1}^{\infty}x\cdot\left(\frac{1}{e^x}\right)^n=\frac{x}{e^x-1}
\end{equation*}

Offensichtlich ist:
\begin{equation*}
    \sum_{n=1}^\infty \frac{x}{e^{nx}} = x \cdot \left(\sum_{n=1}^\infty \left(e^{-x}\right)^n\right) = x \cdot \left(\left(\sum_{n=0}^\infty \left(e^{-x}\right)^n\right) - 1\right)
\end{equation*}

Wir erkennen die Geometrische Reihe mit $q=\frac{1}{e^x}$. Wie wir wissen, konvergiert die Geometrische Reihe für $|q|<1$. Für Konvergenz müssen wir also fordern, dass:

\begin{equation*}
\frac{1}{e^x}<1
\end{equation*}

Und erhalten dann:

\begin{equation*}
    = x \cdot \left( \frac{1}{1-e^{-x}} - 1 \right) = x \frac{1 - (1-e^{-x})}{1-e^{-x}} = x \frac{e^{-x}}{e^{-x} (e^x - 1)} = \frac{x}{e^x-1}
\end{equation*}

Dies gilt jedoch nur unter oben genannter Voraussetzung. Offensichtlich gilt:
\begin{align*}
\frac{1}{e^x}\geq 1 \quad \forall x\leq 0\\
\frac{1}{e^x}\le 1 \quad \forall x\ge 0\\
\end{align*}

Wir müssen fordern: $x\in I$ mit $I=]a,b[$ und $0<a<b<\infty$\\
Zuletzt wollen wir zeigen, dass die über diese Reihe definierte Partialsummen-Funktionsfolge $g_m(x)$ auf I gleichmäßig konvergiert. Wir definieren:

\begin{equation*}
g_m(x):=\sum_{n=1}^{m}\frac{x}{e^{nx}} \quad x\in I
\end{equation*}

Um die gleichmäßige Konvergenz zu beweisen schreiben wir wieder ein beliebiges $\epsilon>0$ vor und suchen ein M, sodass:

\begin{equation}
|g_m(x)-g(x)|<\epsilon \quad \forall m\geq M, \;\; \forall x\in I
\end{equation}

...erfüllt wird. Wie wir wissen, ist:

\begin{equation*}
\lim\limits_{m \to \infty}\left(\sum_{n=1}^{m}\frac{x}{e^{nx}}\right)=\sum_{n=1}^{\infty}\frac{x}{e^{nx}}=\frac{x}{e^x-1} \quad x\in I
\end{equation*}

Wir müssen also zeigen:
\begin{align*}
\left|\sum_{n=1}^{m}\frac{x}{e^{nx}}-\sum_{n=1}^{\infty}\frac{x}{e^{nx}}\right| < \epsilon \quad \forall m\geq M, \;\; \forall x \in I\\
\Leftrightarrow \quad \left|-\sum_{n=m}^{\infty}\frac{x}{e^{nx}}\right| < \epsilon \quad \forall m\geq M, \;\; \forall x \in I
\end{align*}

Die Grenzfunktion $g(x)=\frac{x}{e^x-1}$ kann auf $I$ nur endliche Werte zwischen $0$ und $1$ annehmen, da diese die Grenzwerte auf dem offenen Intervall $[0,\infty[$ sind (von dem I ein Teilintervall ist) und $g(x)$ streng monoton fällt.\\

Beweisen wir zunächst die Monotonie (streng fallend). Dazu muss $g'(x)<0 \; \forall x$ gelten. $g'(x)$ ist:
\begin{equation*}
g'(x)=\frac{(e^x-1)-xe^x}{(e^x-1)^2}=-\frac{e^x(x-1)+1}{(e^x-1)^2} \forall x\in I
\end{equation*}

Tatsächlich wird die Bedingung für strenge Monotonie erfüllt:
\begin{align*}
(1)\;\; (e^x-1)^2 > 0 \quad \forall x\in\mathbb{R}\; \text{mit}\; I\subset\mathbb{R}\\
(2)\;\; e^x(x-1)+1 > 0 \quad \forall x \in I\\
\Rightarrow \quad -\frac{e^x(x-1)+1}{(e^x-1)^2}<0
\end{align*}

Als nächstes betrachten wir die Grenzwerte:
\begin{align*}
\lim\limits_{x \to \infty}\left(\frac{x}{e^x-1}\right)=0\\
\lim\limits_{x \to 0}\left(\frac{x}{e^x-1}\right)=?\\
\end{align*}
Um den letzten Grenzwert herauszufinden nehmen wir eine Substitution vor:
\begin{align*}
t=e^x-1\\
\Leftrightarrow e^x=1+\frac{1}{t}\\
\Leftrightarrow \quad x=\ln(1+\frac{1}{t})\\
\end{align*}
Wegen $\ln(1)=0$ gilt offensichtlich:
\begin{equation*}
x\rightarrow0 \Rightarrow t\rightarrow \infty
\end{equation*}

Und somit:
\begin{align*}
\lim\limits_{x \to 0}\left(\frac{x}{e^x-1}\right)=\lim\limits_{t \to \infty}\left(\frac{\ln(1+\frac{1}{t})}{\left(1+\frac{1}{t}\right)-1}\right)=\lim\limits_{t\to\infty}\left(t\ln(1+\frac{1}{t})\right)\\
=\lim\limits_{t\to\infty}\left(\ln\left(\left(1+\frac{1}{t}\right)^t\right)\right)=\ln\left(\lim\limits_{t \to \infty}\left(1+\frac{1}{t}\right)^t\right)=ln(e)=1
\end{align*}
Der drittletzte Schritt ist legitim, da $ln(x)$ die Umkehrfunktion einer stetigen Funktion ($e^x$) und somit selbst stetig ist.\\
Wenn $g(x)$ auf $I$ stets Werte zwischen 0 und 1 annimmt, dann muss jedes Glied der Folge $g_m(x)$ das ebenfalls tun, denn $g_m(x) > 0 \; \forall x\in I \; \forall m$ und $g_m(x) < g(x) \forall x\in I \; \forall m$. Das geht unmittelbar hervor aus:

\begin{align*}
\sum_{n=1}^{M}\frac{x}{e^{nx}} \ge \sum_{n=1}^{m}\frac{x}{e^{nx}} \;\; \text{für $M>m$}\\
\Rightarrow \quad \frac{x}{e^x-1} \ge \sum_{n=1}^{m}\frac{x}{e^{nx}} \;\; \forall m \in \mathbb{N}
\end{align*}

Daraus folgt aber auch, dass die Glieder der Folge mit zunehmendem $m$ für alle $x\in I$ größer werden, also muss die Differenz zwischen Grenzfunktion und Gliedern der Folge für alle $x \in I$ kleiner werden.\\
Wählen wir also ein hinreichend großes $M$, so wird die Rest-Funktionsreihe auf dem ganzen Intervall für dieses und alle nachfolgenden $m$ einen bestimmten Wert unterschreiten. Dann gilt aber gerade:

\begin{equation*}
\left|-\sum_{n=m}^{\infty}\frac{x}{e^{nx}}\right| < \epsilon \quad \forall m\geq M, \;\; \forall x \in I
\end{equation*}

\subsection*{Teilaufgabe 3}

Unter Verwendung von (1) können wir ohne weiteres feststellen:
\begin{equation*}
\sum_{n=1}^{\infty}\frac{1}{n^2}=\sum_{n=1}^{\infty}\int_{0}^{\infty}\frac{xdx}{e^{nx}}=\lim\limits_{m \to \infty}\left(\sum_{n=1}^{m}\int_{0}^{\infty}\frac{xdx}{e^{nx}}\right)
\end{equation*}

Nenne $f_n(x)=\frac{x}{e^{nx}}:$
\begin{equation*}
\Leftrightarrow \quad \sum_{n=1}^{\infty}\frac{1}{n^2}=\lim\limits_{m \to \infty}\left(\sum_{n=1}^{m}\int_{0}^{\infty}f_n(x)dx\right)
\end{equation*}

Wir definieren eine Partialsummen-Funktionsfolge $g_m(x)$:
\begin{equation*}
g_m(x):=\sum_{n=1}^{m}f_n(x)dx\\
\end{equation*}

Falls wir zeigen können, dass die so definierte Folge gleichmäßig konvergiert und dass ferner:
\begin{equation*}
\lim\limits_{n \to \infty}\int_{0}^{\infty}g_m(x)dx=\int_{0}^{\infty}g(x)dx
\end{equation*}

...dann dürfen wir Summe und Integration vertauschen und schließen:
\begin{align*}
\Leftrightarrow \quad \sum_{n=1}^{\infty}\frac{1}{n^2}=\lim\limits_{m \to \infty}\left(\sum_{n=1}^{m}\int_{0}^{\infty}f_n(x)dx\right)=\lim\limits_{m \to \infty}\left(\int_{0}^{\infty}\sum_{n=1}^{m}f_n(x)dx\right)\\
=\lim\limits_{m \to \infty}\left(\int_{0}^{\infty}g_m(x)dx\right)=\int_{n=1}^{\infty}g(x)dx
\end{align*}

Dazu benutzen wir den Satz über die majorisierte Konvergenz. Nehmen wir an, alle Glieder der Folge $g_m(x)$ sind stetig und besagte Folge konvergiert auf jedem kompakten Teilintervall von $[0,\infty]$ gleichmäßig gegen ein Funktion $g(x)$. Dann gilt die zu beweisende Gleichung, wenn wir eine Funktion $h(x)$ finden, welche die Bedingungen erfüllt:

\begin{align*}
(1)\quad h(x)\geq |g_m(x)| \quad \forall x\\
(2)\quad \int_{0}^{\infty}h(x) \text \quad \text{existiert}
\end{align*}
Prüfen wir zunächst, ob die Voraussetzungen erfüllt sind. Die Stetigkeit aller $g_m(x)$ ist gegeben, denn:
\begin{equation*}
g_m(x):=\sum_{n=1}^{m}\frac{x}{e^{nx}}
\end{equation*}

...ist eine Komposition stetiger Funktionen und somit selbst stetig (siehe letztes Semester).\\
Die gleichmäßige Konvergenz ist ebenfalls gegeben, wie wir in Teil (2) der Aufgabe gezeigt haben. Es gilt:
\begin{equation*}
\lim\limits_{m \to \infty} g_m(x)=\sum_{n=1}^{\infty}\frac{x}{e^{nx}}=x\sum_{n=1}^{\infty}\left(\frac{1}{e^x}\right)^ndx=\frac{x}{e^x-1}=g(x)
\end{equation*}

Wir können nun noch eine wichtige Aussage treffen, die uns die Rechnung erleichtern wird, nämlich dass die Folge $g_m(x)$ von unten gegen $g(x)$ konvergiert. Denn:

\begin{equation*}
\sum_{n=1}^{M}\frac{x}{e^{nx}} \ge \sum_{n=1}^{m}\frac{x}{e^{nx}} \;\; \text{für $M>m$}
\end{equation*}

Damit gilt unweigerlich:
\begin{equation*}
\frac{x}{e^x-1} \ge \sum_{n=1}^{m}\frac{x}{e^{nx}} \;\; \forall m\\
\Rightarrow \quad g(x)>g_m(x) \;\; \forall m
\end{equation*}

Um eine Funktion $h(x)$ zu finden, welche der Bedingung:
\begin{equation*}
h(x) \geq |g_m(x)| \;\; \forall m
\end{equation*}

...gerecht wird, genügt es, eine solche zu finden, die:
\begin{equation*}
h(x)\geq g(x)=\left|\frac{x}{e^x-1}\right| \;\; \forall x \in I
\end{equation*}

...erfüllt. Da aber $g(x)$ auf einem beliebigen, geschlossenen Teilintervall von I nur endliche Werte liefert und demnach auch das Integral über ein geschlossenes Teilintervall endlich sein muss, genügt es zu zeigen:

\begin{equation*}
h(x) \geq |g(x)| \;\; \forall x>X, \; x,X\in I
\end{equation*}

Ein naheliegender Ansatz ist:
\begin{equation*}
h(x):=\frac{x}{e^x}
\end{equation*}

Wir prüfen:
\begin{align*}
\frac{x}{e^x} \geq \frac{1}{e^x-1} \quad \forall x\geq X, \; x,X\in I\\
\frac{x(e^x-1)}{e^x} \geq 1 \quad \forall x\geq X, \; x,X\in I\\
x-\frac{x}{e^x} \geq 1 \quad \forall x\geq X, \; x,X\in I\\
x-\underbrace{\frac{x}{e^x}}_{\rightarrow 0} \geq 1 \quad \forall x\geq X, \; x,X\in I
\end{align*}

...für hinreichend große $x$. Wählen wir $X$ also groß genug, so wird die obige Bedingung erfüllt.\\
Damit bleibt nur noch zu prüfen, ob das uneingentliche Integral von $h(x)$ existiert. Das ist aber gemäß der Formel aus Aufgabe (1):

\begin{equation*}
\int_{0}^{\infty}\frac{xdx}{e^x}=1
\end{equation*}

Damit sind alle Voraussetzungen erfüllt und wir können schließen:
\begin{equation*}
\sum_{n=1}^{\infty}\frac{1}{n^2}=\sum_{n=1}^{\infty}\int_{0}^{\infty}\frac{xdx}{e^{nx}}=x\sum_{n=1}^{\infty}\left(\frac{1}{e^x}\right)^ndx=\frac{x}{e^x-1}
\end{equation*}


\section*{Aufgabe 1.3}

Im den Argumentationsfluss nicht zu unterbrechen, beginnen wir mit dem Beweis für Aussage 2.\\
Nehmen wir an, $f$ ist nicht beschränkt. Das heißt, für mindestens ein $x=x_0$ auf dem geschlossenen Intervall $I=[0,1]$ muss gelten:

\begin{equation*}
\lim\limits_{x \to x_0} f(x)=\infty
\end{equation*}

 Wir wollen beweisen, dass $f$ gleichmäßig konvergiert. Dazu müssen wir zeigen:

\begin{equation*}
\forall \epsilon \ge 0 \exists N: \forall n \geq N |f_n(x)-f(x)| \le \epsilon \forall x
\end{equation*}

Verbal ausgedrückt: Wenn wir ein endlich großes $\epsilon>0$ vorgeben, finden wir ein $N$ sodass:

\begin{equation*}
|f_n(x)-f(x)| \le \epsilon 
\end{equation*}

... erfüllt wird, und zwar für alle $x$. Da $\epsilon$ endlich ist, müssen wir fordern, dass auch $|f_n(x)-f(x)|$ für alle $x$ endlich ist. Da aber mindestens für $x_0$ gelten muss, dass:

\begin{equation*}
\lim\limits_{x \to x_0} f(x)=\infty
\end{equation*}

Wird auch gelten:

\begin{equation*}
\lim\limits_{x \to x_0} |f_n(x)-f(x)|=|f_n(x)-\lim\limits_{x \to x_0} f(x)|=\infty
\end{equation*}

Womit die Bedingung für gleichmäßige Konvergenz unmöglich für alle $x \in I$ erfüllt wird.\\

Wir müssen also folgern, dass wenn $f_n(x) \rightrightarrows f(x)$ und $f_n(x)$ beschränkt ist auch $f(x)$ beschränkt sein muss.\\

Anders sieht es für punktweise Konvergenz aus. Wir nehmen erneut an, dass $f$ nicht beschränkt ist und demnach für mindestens ein $x=x_0$ aus dem Definitionsbereich $f_n(x) \rightarrow \infty$. 
Um punktweise Konvergenz von $f_n$ für ein bestimmtes x zu beweisen, müssen wir zeigen:

\begin{equation*}
\forall \epsilon \le 0 \exists N: |f_n(x)-f(x)| \le \epsilon \forall n \geq N
\end{equation*}

Ausformuliert: Geben wir ein beliebiges $\epsilon$ vor und betrachten ein bestimmtes $x$, so müssen wir ein $N$ finden können, ab dem $|f_n(x)-f(x)|$ das $\epsilon$ unterbietet. Dies ist jeweils für alle $x \in I$ demonstrieren.\\

Offensichtlich kann jeweils für alle $x \neq x_0$ ein hinreichend großes $N \in \mathbb{N}$ gefunden werden, sodass:

\begin{equation*}
|f_n(x)-f(x)| \le \epsilon \quad \forall n \ge N
\end{equation*}

Für $x=x_0$ ist diese Formulierung zum Beweis der Konvergenz jedoch problematisch, da $f(x)$ hier divergiert. Allerdings können wir uns mit der alternativen Grenzwertdefinition behelfen, wonach $f_n$ punktweise gegen $f:I \rightarrow \mathbb{R}$ konvergiert, wenn:

\begin{equation*}
\lim\limits_{n \to \infty}f_n(x)=f(x) \quad \forall x \in I
\end{equation*} 

Also insbesondere auch für $x=x_0$. Dann muss für $x_0$ gelten:

\begin{equation*}
\lim\limits_{n \to \infty}f_n(x_0) = \infty
\end{equation*}

Das ist eine Bedingung die sehr wohl unter unserer Voraussetzung erfüllt werden kann, dass $f_n(x)$ auf dem geschlossenen Intervall $I=[0,1]$ für alle $n$ beschränkt sein und somit nur endlich Werte annehmen darf.\\

Also muss bei punktweiser Konvergenz einer beschränkten Folge $f_n(x) \rightarrow f(x)$ auf einem geschlossenen Intervall $I$ (hier $I=[0,1]$) die Grenzwertfunktion nicht notwendigerweise beschränkt sein.\\

Beispiel: Sei

\begin{equation*}
  f_n(x) = \begin{cases}
    0 & \mbox{wenn} \; x = 0 \\
    \frac{1}{x} & \mbox{wenn} \; x \neq 0
  \end{cases}
\end{equation*}

Es ist klar, dass die Funktion konvergiert, denn sie enthält überhaupt kein
$n$. Trotzdem ist $f$ unbeschränkt, denn

\begin{equation*}
  \lim_{x \to 0} \frac{1}{x} = \infty
\end{equation*}

\section{Aufgabe 1.4}


Es ist sinnvoll, $f_n(x)$ gleich auf gleichmäßige Konvergenz zu prüfen, da Beobachtung auf 9.1.2 im Skript zufolge aus gleichmäßiger Konvergenz stets punktweise Konvergenz gegen die selbe Grenzfunktion folgt.\\

Sei ein beliebig kleines $\epsilon>0$ vorgegeben, so muss es ein $N \in \mathbb{N}$ geben, sodass:

\begin{equation*}
|f_n(x)-f(x)|<\epsilon \quad \forall n \geq N, \quad \forall x
\end{equation*}

Wir prüfen zunächst, ob eine Grenzfunktion zu:
\begin{equation*}
f_n(x):= \frac{x}{1+nx^2}
\end{equation*}

...existiert. Ohne Mühe (und ohne formalen Beweis) erkennen wir:
\begin{equation*}
\lim\limits_{n \to \infty} \left(\frac{x}{1+nx^2}\right)=0 \quad \forall x
\end{equation*}

...denn $f_n(x)$ verhält sich für hinreichend große $n$ in etwa wie $\frac{1}{n}$. Die Grenzfunktion ist folglich die Nullfunktion. Zu finden ist ein N, sodass:

\begin{equation*}
|f_n(x)-0|=|f_n(x)|<\epsilon \quad \forall n \geq N, \;\; \forall x
\end{equation*}

Alle Funktionen der Folge $f_n(x)$ konvergieren:

\begin{equation*}
\lim\limits_{x \to \infty} \left(\frac{x}{1+nx^2}\right)=0
\end{equation*}

...denn für hinreichend große $x$ verhält sich eine beliebige Funktion der Folge wie $\frac{1}{x} \rightarrow 0$.

Die Ableitung einer beliebigen Funktion der Folge ist:

\begin{equation*}
f_n'(x)=\frac{(1+nx^2)- x\cdot 2nx}{\left(1+nx^2\right)^2}=\frac{1-nx^2}{\left(1+nx^2\right)^2}\\
\end{equation*}

Lokale Extrema erhält man somit für:
\begin{align*}
f_n'(x)=0\\
\Leftrightarrow \quad \frac{1-nx^2}{\left(1+nx^2\right)^2}=0\\
\Leftrightarrow \quad 1-nx^2=0\\
\Leftrightarrow \quad x=\overset{+}{-}\sqrt{\frac{1}{n}}
\end{align*}

Einsetzen in $f_n(x)$:
\begin{equation*}
f_n(x)=\frac{\sqrt{\frac{1}{n}}}{1+n\cdot\frac{1}{n}}=\frac{1}{2\sqrt{n}}
\end{equation*}

Geben wir also ein $\epsilon>0$ vor, so finden wir ein hinreichend großes $n$, sodass:

\begin{equation*}
\frac{1}{2\sqrt{n}} < \epsilon
\end{equation*}

Und nennen dieses $N$. Offensichtlich ist dann:

\begin{equation*}
|f_n(x)|\leq\left|\frac{1}{2\sqrt{n}}\right| < \epsilon \quad \forall n\geq N, \;\;\forall x
\end{equation*}

Damit ist der Beweis für gleichmäßige Konvergenz erbracht und es liegt auch punktweise Konvergenz vor.\\
\\

Als nächstes betrachten wir die Ableitungsfolge $f_n'(x)$. Für ein beliebiges Glied der Folge gilt:

\begin{equation*}
f_n'(x)=\frac{1-nx^2}{\left(1+nx^2\right)^2}
\end{equation*}

Die Ableitung der Grenzfunktion $f(x)=0$ lautet:
\begin{equation*}
f'(x)=0
\end{equation*}

Zu prüfen ist, für welche $x$ erfüllt wird:

\begin{equation*}
\lim\limits_{n \to \infty}f_n'(x)=f'(x)=0
\end{equation*}

Dazu nehmen wir eine Fallunterscheidung vor:

\subsection*{$|x|\ge0:$}
Ohne großen Aufwand sehen wir, dass bei fixiertem x und hinreichend großem n sich $f_n'(x)$ im wesentlichen wie $\left(\frac{-1}{n}\right)\rightarrow 0$ verhält.\\
Ohne formalen Beweis folgern wir:

\begin{equation*}
\lim\limits_{n \to \infty}f_n'(x)=f_n(x)
\end{equation*}

\subsection*{$x=0:$}
\begin{equation*}
f_n'(x)=\left(\frac{1-0}{\left(1+0^2\right)^2}\right)=1 \quad \forall n\in\mathbb{N}
\end{equation*}

$f_n'(x)$ ist stationäre Folge und konvergiert als solche:

\begin{equation*}
\lim\limits_{n \to \infty} f_n'(x)=1 \neq f'(x)
\end{equation*}

Wir finden also, dass $f_n'(x)$ auf den offenen Teilintervallen $I_1=]-\infty,0[$ und $I_2=]0,\infty[$ konvergiert. Zu prüfen bleibt, ob innerhalb aller kompakten Teilintervalle von diesen lediglich punktweise oder doch gleichmäßige Konvergenz vorliegt.\\
Läge gleichmäßige Konvergenz vor, müssten ein $N$ finden, sodass bei gegebenem $\epsilon>0$:

\begin{equation*}
|f_n'(x)-f'(x)|=|f_n'(x)-0|=|f_n'(x)|<\epsilon \quad \forall n\ge N, \;\; \forall x\in I_{1/2}
\end{equation*}

Dazu gehen wie vor wie beim letzten Mal und ermitteln Extrema.\\
Unter Anwendung der Quotientenregel ermitteln wir:
\begin{align*}
f_n''(x)=\frac{-2nx\left(1+nx^2\right)^2-\left(1-nx^2\right)\cdot 2nx\left(1+nx^2\right)\cdot 2}{\left(1+nx^2\right)^4}\\
\Leftrightarrow \quad f_n''(x)=\frac{-2nx\left(1+nx^2\right)\left(\left(1+nx^2\right)+2\left(1-nx^2\right)\right)}{\left(1+nx^2\right)^4}\\
\Leftrightarrow \quad f_n''(x)=\frac{2nx\left(nx^2-3\right)}{\left(1+nx^2\right)^3}
\end{align*}

Bedingung für Extrema:
\begin{align*}
f_n''(x)=\frac{2nx\left(nx^2-3\right)}{\left(1+nx^2\right)^3}=0\\
\Rightarrow \quad 2nx\left(nx^2-3\right)=0\\
\Leftrightarrow \quad (1)\;x=0\notin I_{1/2} \quad \quad (2)\;nx^2-3=0\\
\Rightarrow \quad x=\overset{+}{-}\sqrt{\frac{3}{n}} 
\end{align*}

Einsetzen in $fn'(x)$:
\begin{equation*}
fn'(x)=\frac{1-3}{\left(1+3\right)^2}=-\frac{1}{8}
\end{equation*}

Unabhängig von der Wahl von n erhalten wir jeweils auf $I_1$ und $I_2$ als globale Extrema von $fn'(x)=-\frac{1}{8}$. Es gilt:

\begin{equation*}
|fn'(x)|\leq\left|-\frac{1}{8}\right| \quad \forall x\in I_{1/2} \forall n
\end{equation*}

Das heißt aber: Geben wir ein $\epsilon>0$ vor, welches $\epsilon<-\frac{1}{8}$ erfüllt, so finden wir kein einziges n, sodass:

\begin{equation*}
|fn'(x)|\le\epsilon \quad \forall x \in I_{1/2}
\end{equation*}

Damit kann $f_n'(x)$ auf den entsprechenden Intervallen unmöglich gleichmäßig gegen $f'(x)$ konvergieren.

\end{document}