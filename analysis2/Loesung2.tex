\documentclass[a4paper,german,12pt,smallheadings]{scrartcl}
\usepackage[T1]{fontenc}
\usepackage[utf8]{inputenc}
\usepackage{babel}
\usepackage{tikz}
\usepackage{geometry}
\usepackage{amsmath}
\usepackage{amssymb}
\usepackage{float}
\usepackage{enumerate}
%\usepackage{wrapfig}
\usepackage[thinspace,thinqspace,squaren,textstyle]{SIunits}
\restylefloat{table}
\renewcommand{\thefootnote}{\fnsymbol{footnote}}
\geometry{a4paper, top=15mm, left=20mm, right=40mm, bottom=20mm, headsep=10mm, footskip=12mm}
\linespread{1.5}
\setlength\parindent{0pt}
\begin{document}
\begin{center}
\bfseries % Fettdruck einschalten
\sffamily % Serifenlose Schrift
\vspace{-40pt}
Analysis II, Wintersemester 2013/2013, 2. Übungsblatt

Markus Fenske, Luis Herrmann, Tutor: Sebastian Bierke
\vspace{-10pt}
\end{center}

\section*{Aufgabe 2.1}
Die Punkte, die am weitesten vom Mittelpunkt des Würfels entfernt sind, sind
seine Ecken (das kann man sich ohne Beweis überlegen). Der größte
abgeschlossene Würfel, der noch in die abgeschlossene Kugel passt, ist also der
Würfel, dessen Ecken auf der Oberfläche der Kugel liegen. Da Mittelpunkt der
Kugel und Mittelpunkt des Würfels zusammenfallen, ist das Problem symmetrisch,
es muss also nur eine Ecke betrachtet werden.

Sei $\overline{e}$ eine beliebige Ecke. Der größte Würfel erfüllt also die Beziehung:
\begin{equation}
  d(\overline{e},\overline{0}) = 1
\end{equation}

Mit der euklidischen Metrik ergibt dies:
\begin{equation*}
  \sqrt{\sum_{i=1}^n w_n^2} = \sqrt{n} w_n = 1
\end{equation*}

Das bedeutet, die maximale halbe Kantenlänge ist:
\begin{equation*}
  w_n = \frac{1}{\sqrt{n}}
\end{equation*}

Der Grenzwert des Würfelvolumens ist dann:
\begin{equation*}
  \lim_{n \to \infty} (2w_n)^n = \lim_{n \to \infty} \left(\frac{2}{\sqrt{n}}\right)^n
\end{equation*}

Da ab $n > 4$ gilt, dass $\sqrt{n} > 2$, muss diese Folge eine
Nullfolge sein. Das bedeutet, dass das Würfelvolumen gegen 0 geht.

\begin{equation*}
  \lim_{n \to \infty} (2w_n)^n = 0
\end{equation*}

\section*{Aufgabe 2.2}

Wir schätzen ab:

\begin{equation*}
  0 <
  \left| \frac{xy}{\sqrt{|x| + y^2}}          \right| =
  \left| \frac{x}{\sqrt{\frac{|x|}{y^2} + 1}} \right| <
  \left| \frac{x}{\sqrt{\frac{|x|}{y^2}}}     \right| =
  \left| \frac{xy}{\sqrt{|x|}}                \right| =
  \left| y\sqrt{|x|}                          \right|
\end{equation*}

Da $\lim_{(x,y) \to (0,0)} y\sqrt{|x|} = 0$, ist nach Sandwich-Theorem

\begin{equation*}
  \lim_{(x,y) \to (0,0)} \frac{xy}{\sqrt{|x| + y^2}} = 0
\end{equation*}

\section*{Aufgabe 2.3}

\begin{enumerate}[(1)]
  \item
    \begin{equation*}
      \lim_{x \to 0} \left( \lim_{y \to 0} \frac{x-y+x^2+y^2}{x+y} \right) =
      \lim_{x \to 0} \frac{x+x^2}{x}  =
      \lim_{x \to 0} 1+x =
      1
    \end{equation*}
  \item
    \begin{equation*}
      \lim_{y \to 0} \left( \lim_{x \to 0} \frac{x-y+x^2+y^2}{x+y} \right) =
      \lim_{y \to 0} \frac{y^2-y}{y} =
      \lim_{y \to 0} y-1 =
      -1
    \end{equation*}
  \item
    Da zwei unterschiedliche Grenzwerte herauskommen, existiert der Grenzwert
    nicht (Skript 1.2.4: ``Alle Wege müssen zum Grenzwert führen'')
\end{enumerate}


\end{document}
