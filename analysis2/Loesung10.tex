\documentclass[a4paper,german,12pt,smallheadings]{scrartcl}
\usepackage[T1]{fontenc}
\usepackage[utf8]{inputenc}
\usepackage{babel}
\usepackage{tikz}
\usepackage{geometry}
\usepackage{amsmath}
\usepackage{amssymb}
\usepackage{float}
\usepackage{enumerate}
\usepackage{cancel}
\usepackage{pgfplots}
\usepackage{commath}
\pgfplotsset{compat=1.7}
\usepgfplotslibrary{polar}
%\usepackage{wrapfig}
\usepackage[thinspace,thinqspace,squaren,textstyle]{SIunits}
\restylefloat{table}
\renewcommand{\thefootnote}{\fnsymbol{footnote}}
\geometry{a4paper, top=15mm, left=20mm, right=40mm, bottom=20mm, headsep=10mm, footskip=12mm}
\linespread{1.5}
\setlength\parindent{0pt}
\begin{document}
\begin{center}
\bfseries % Fettdruck einschalten
\sffamily % Serifenlose Schrift
\vspace{-40pt}
Analysis II, Wintersemester 2013/2014, 10. Übungsblatt

Markus Fenske, Luis Herrmann, Tutor: Sebastian Bierke
\vspace{-10pt}
\end{center}
\allowdisplaybreaks % Seitenumbrüche in Formeln erlauben
\section*{Aufgabe 10.1}
\begin{align*}
  & \int\limits_0^2 \dif x \int\limits_1^2 \dif y \; y \sin\del{\pi x y} \overset{\text{\textsc{Fubini}}}{=}
    \int\limits_1^2 \dif y \int\limits_0^2 \dif x \; y \sin\del{\pi x y} =
    \int\limits_1^2 \dif y \; y \int\limits_0^2 \dif x \; \sin\del{\pi x y} \\ =
  & \int\limits_1^2 \dif y \; y \sbr{- \frac{\cos\del{\pi x y}}{\pi y}}_0^2 =
    \int\limits_1^2 \dif y \; y \del{\frac{1}{\pi y} - \frac{\cos\del{2 \pi y}}{\pi y}} =
    \int\limits_1^2 \dif y \; \del{\frac{1}{\pi} - \frac{\cos\del{2 \pi y}}{\pi}} \\ =
  & \int\limits_1^2 \dif y \; \frac{1}{\pi} - \int\limits_1^2 \dif y \; \frac{\cos\del{2 \pi y}}{\pi}
    \frac{1}{\pi} - \frac{1}{\pi} \underbrace{\sbr{\frac{\sin\del{2 \pi y}}{2 \pi}}_1^2}_{=0} = \frac{1}{\pi}
\end{align*}

\section*{Aufgabe 10.2}
Wegen $x > 0$ folgt aus $4x = y^2$, dass $x = 2 \sqrt{x}$. Somit ist die Masse

\begin{align*}
  M
  = \int\limits_0^4 \dif x \int\limits_0^{2 \sqrt{x}} \dif y \; \mu
  = \int\limits_0^4 \dif x \; 2 \sqrt{x} \mu
  = 2 \mu \sbr{\frac{2}{3} x^\frac{3}{2}}_0^4
  = \frac{32}{3} \mu
\end{align*}

Und die Koordinate des Schwerpunkts ist
\begin{align*}
  x_c
  = \frac{1}{M} \int\limits_0^4 \dif x \int\limits_0^{2\sqrt{x}} \dif y \; x \mu
  = \frac{3}{32} \int\limits_0^4 \dif x \; x \cdot 2 \sqrt{x}
  = \frac{3}{16} \int\limits_0^4 x^\frac{3}{2}
  = \frac{3}{16} \sbr{\frac{2}{5} x^\frac{5}{2}}_0^4
  = \frac{12}{5}
\end{align*}

\begin{align*}
  y_c
  = \frac{1}{M} \int\limits_0^4 \dif x \int\limits_0^{2 \sqrt{x}} \dif y \; y \mu
  = \frac{3}{32} \int\limits_0^4 \sbr{\frac{1}{2} y^2}_0^{2\sqrt{x}}
  = \frac{3}{32} \int\limits_0^4 \frac{1}{2} 4 x
  = \frac{3}{16} \int\limits_0^4 x
  = \frac{3}{16} \sbr{\frac{1}{2} x^2}_0^4
  = \frac{3}{2}
\end{align*}

\end{document}
