\documentclass[a4paper,german,12pt,smallheadings]{scrartcl}
\usepackage[T1]{fontenc}
\usepackage[utf8]{inputenc}
\usepackage{babel}
\usepackage{tikz}
\usepackage{geometry}
\usepackage{amsmath}
\usepackage{amssymb}
\usepackage{float}
\usepackage{enumerate}
\usepackage{cancel}
\usepackage{pgfplots}
\usepackage{commath}
\pgfplotsset{compat=1.7}
\usepgfplotslibrary{polar}
%\usepackage{wrapfig}
\usepackage[thinspace,thinqspace,squaren,textstyle]{SIunits}
\restylefloat{table}
\renewcommand{\thefootnote}{\fnsymbol{footnote}}
\geometry{a4paper, top=15mm, left=20mm, right=40mm, bottom=20mm, headsep=10mm, footskip=12mm}
\linespread{1.5}
\setlength\parindent{0pt}
\begin{document}
\begin{center}
\bfseries % Fettdruck einschalten
\sffamily % Serifenlose Schrift
\vspace{-40pt}
Analysis II, Wintersemester 2013/2013, 7. Übungsblatt

Markus Fenske, Luis Herrmann, Tutor: Sebastian Bierke
\vspace{-10pt}
\end{center}
\allowdisplaybreaks % Seitenumbrüche in Formeln erlauben
\section*{Aufgabe 6.1}
\begin{enumerate}[(1)]
  \item
    \begin{align*}
                            &x^2 + y^2 + z^2 = 3xyz \\
      \Leftrightarrow\quad  &x^2 + y^2 + z^2 - 3xyz = 0 \\
      \Rightarrow\quad      &F(x,y,z) = x^2 + y^2 + z^2 - 3xyz \\
    \end{align*}

    Wir prüfen $\pd{F}{z}\del{1,1,1} \neq 0$:
    \begin{align*}
                          &\pd{F}{z} = 2z - 3xy \\
      \Rightarrow\quad    &\pd{F}{z}(1,1,1) = 2-3 = -1
    \end{align*}

    Also gilt nach dem Satz über implizite Funktionen:
    \begin{align*}
      \pd{f}{x}(1,1) = \frac{\pd{F}{x}(1,1,f(1))}{\pd{F}{z}(1,1,f(1))}
    \end{align*}

    Nach dem Satz über implizite Funktionen gilt:
    \begin{align*}
      \pd{f}{x}(x,y) = \frac{\pd{F}{x}\del{x,y,f(x,y)}}{\pd{F}{z}\del{x,y,f(x,y)}}
    \end{align*}

    Dabei muss $f(x,y)$ bei $(x,y,z) = (1,1,1)$ erfüllen $f(1,1) = 1$. Wir rechnen also nach:

    \begin{align*}
      \pd{f}{x}\del{x,y,f(x,y)} = \pd{}{x} \del{x^2 + y^2 + f(x,y)^2 - 3xyf(x,y)} = 2x
    \end{align*}

    Durch Ableiten ergibt sich:
    \begin{align*}
                           &\pd{F}{x}\del{x,y,f(x,y)} = 0 \\
      \Leftrightarrow\quad &\pd{}{x}(x^2 + y^2 + f(x,y)^2 - 3xyf(x,y)) = 0 \\
      \Leftrightarrow\quad &2x +2f(x,y) \pd{}{x} f(x,y) - 3y f(x,y) - 3xy \pd{}{x} f(x,y) = 0 \\
      \Leftrightarrow\quad &2f(x,y)\pd{}{x}f(x,y) - 3xy \pd{}{x}f(x,y) = 3yf(x,y) - 2x \\
      \Leftrightarrow\quad &\pd{}{x}f(x,y) \del{2f(x,y) - 3xy} = 3yf(x,y) - 2x \\
      \Leftrightarrow\quad &\pd{}{x}f(x,y) = \frac{3yf(x,y) - 2x}{2f(x,y) - 3xy}
    \end{align*}

    Nun setzen wir ein: $x=1, y=1, f(x,y) = 1$:

    \begin{align*}
      \pd{}{x} f(1,1) = \frac{3-2}{2-3} = -1
    \end{align*}

    Wir verfahren analog für $g$. Sei $y = g(x,z)$. Prüfe:
    \begin{align*}
      \pd{F}{y}(x,y,z) = \pd{}{x}(x^2 + y^2+z^2-3xyz) = 2y-3xz
    \end{align*}

    Im Punkt $(1,1,1)$:
    \begin{align*}
      \pd{F}{y}(1,1,1) = 2-3 = -1 \neq 0
    \end{align*}

    Damit berechnen wir:
    \begin{align*}
      \pd{g}{x}(x,z) &= \frac{\pd{F}{x}(x, g(x,z), z)}{\pd{F}{y}(x,y,z)} \\
                     &= \frac{\pd{}{x}(x^2+y^2+z^2 - 3xyz}{\pd{}{y} (x^2+y^2+z^2 - 3xyz} \\
                     &= \frac{-(2x-3yz)}{2y - 3xz}
    \end{align*}

    Für $(x,y,z) = (1,1,1)$:
    \begin{align*}
      \pd{g}{x}(1,1) = \frac{-(2-3)}{2-3} = -1
    \end{align*}
\end{enumerate}




\end{document}
