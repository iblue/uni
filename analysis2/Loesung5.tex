\documentclass[a4paper,german,12pt,smallheadings]{scrartcl}
\usepackage[T1]{fontenc}
\usepackage[utf8]{inputenc}
\usepackage{babel}
\usepackage{tikz}
\usepackage{geometry}
\usepackage{amsmath}
\usepackage{amssymb}
\usepackage{float}
\usepackage{enumerate}
\usepackage{cancel}
\usepackage{pgfplots}
\pgfplotsset{compat=1.7}
\usepgfplotslibrary{polar}
%\usepackage{wrapfig}
\usepackage[thinspace,thinqspace,squaren,textstyle]{SIunits}
\restylefloat{table}
\renewcommand{\thefootnote}{\fnsymbol{footnote}}
\geometry{a4paper, top=15mm, left=20mm, right=40mm, bottom=20mm, headsep=10mm, footskip=12mm}
\linespread{1.5}
\setlength\parindent{0pt}
\begin{document}
\begin{center}
\bfseries % Fettdruck einschalten
\sffamily % Serifenlose Schrift
\vspace{-40pt}
Analysis II, Wintersemester 2013/2013, 5. Übungsblatt

Markus Fenske, Luis Herrmann, Tutor: Sebastian Bierke
\vspace{-10pt}
\end{center}
\allowdisplaybreaks % Seitenumbrüche in Formeln erlauben
\section*{Aufgabe 5.1}
Der Halbkreisbogen (den wir hier $\overline{\gamma}(\phi)$ nennen wollen) ist
parametrisiert durch
\begin{equation*}
  x = R \sin \phi,\quad y = R \cos \phi, \quad \phi \in [0, \pi]
\end{equation*}

Das führt auf die Ableitungen
\begin{equation*}
  \frac{d}{d \phi} x= \cos \phi,\quad \frac{d}{d \phi} y = -\sin \phi
\end{equation*}

Die Länge ist folglich
\begin{align*}
  L &= \int_{\overline{\gamma}} ds \\ 
    &= \int_0^\pi \sqrt{\left(\frac{dx}{d\phi}\right)^2 + \left(\frac{dy}{d \phi}\right)^2} \; d\phi \\
    &= \int_0^\pi \sqrt{R^2 \cos(\phi)^2 + R^2 \sin(\phi)^2} \; d\phi \\
    &= \int_0^\pi \sqrt{R^2 \underbrace{(\cos(\phi)^2 + \sin(\phi)^2)}_{=1}} \; d\phi \\
    &= \int_0^\pi |R| \; d\phi \\
    &= R \int_0^\pi \; d\phi \qquad \text{(weil $R$ immmer $>0$)} \\
    &= \pi R
\end{align*}

Der Schwerpunkt auf der $x$-Achse ist dann
\begin{align*}
  x_s &= \frac{1}{L} \int_{\overline{\gamma}} x \; ds \\
      &= \frac{1}{\pi R} \int_0^\pi x \; ds \\
      &= \frac{1}{\pi \cancel{R}} \int_0^\pi \cancel{R} \cos(\phi) \sqrt{R^2 \cos(\phi)^2 + R^2 \sin(\phi)^2} \; d\phi \\
      &= \frac{1}{\pi} \int_0^\pi \cos(\phi) \sqrt{R^2 \underbrace{(\cos(\phi)^2 + \sin(\phi)^2)}_{=1}} \; d\phi \\
      &= \frac{1}{\pi} \int_0^\pi \cos(\phi) |R| \; d\phi \\
      &= \frac{R}{\pi} \int_0^\pi \cos \phi  \; d\phi \qquad \text{(weil $R$ immer $>0$)}\\
      &= \frac{R}{\pi} \left[ \sin \phi\right]_0^\pi \\
      &= \frac{2R}{\pi}
\end{align*}

Die Oberfläche einer Sphäre berechnet sich dann gemäß der \textsc{Guldin}schen
Regel über die Mantelfläche der Kurve $\overline{\gamma}$. Diese wollen wir
hier $M$ nennen.

\begin{equation*}
  M = 2 \pi L x_s = 2 \pi \cdot \cancel{\pi} R \cdot \frac{2R}{\cancel{\pi}} = 4 \pi R^2
\end{equation*}

\section*{Aufgabe 5.2}
\begin{enumerate}[a)]
  \item
    Folgende Skizze zeigt die (nicht ganz so \textit{grobe}) Form und die
    Schnittpunkte mit den Achsen.

    \begin{figure*}[h]
      \begin{tikzpicture}
        \begin{axis}[
          ymax=0.5,          % Den ganzen Kram etwas skalieren, damit die
          ymin=-2.5,         % Achsenbeschriftungen und die Markierungen
          xmax=1.8,          % noch richtig drauf sind.
          xmin=-1.8,         %
          samples=120,          % Kurve ohne Kanten
          axis lines=middle,    % Achsen durch den Ursprung
          xlabel=$x$,           % X-Achse labeln
          ylabel=$y$,           % D'oh!
          xtick={-1, 0,1},           % Wir markieren an den Stellen x=-1 und x=1
          xticklabels={$-a$, 0,$a$}, % dass dort -a und a ist.
          ytick={0,-2},              % Wir markieren an der Stelle y=-2
          yticklabels={0,$-2a$},     % dass dort -2a ist.
          every tick/.style={line width=1pt, black} % Markierungen groß und fett, damit sichtbar
        ]
          \addplot[variable=\t, domain=0:2*pi] ({(1-sin(deg(t)))*cos(deg(t))}, {(1-sin(deg(t)))*sin(deg(t))});
          \end{axis}
      \end{tikzpicture}
    \end{figure*}
  \item
    Gegeben ist die Parametrisierung
    \begin{equation*}
      x = (1 - \sin \phi)\cos \phi,\quad y = (1-\sin \phi)\sin \phi, \quad \phi \in [-\frac{\pi}{2},\frac{\pi}{2}]
    \end{equation*}

    Für die Ableitungen folgt durch die Kettenregel
    \begin{align*}
      \frac{dx}{d\phi} &= -\cos(\phi)\cos(\phi) + (1-\sin(\phi))(-\sin(\phi)) \\
                       &= -\cos^2(\phi) - (1-\sin(\phi))\sin(\phi) \\
                       &= -\cos^2(\phi) - (\sin(\phi) + \sin^2(\phi)) \\
                       &= -\cos^2(\phi) - \sin(\phi) - \sin^2(\phi) \\
      \frac{dy}{d\phi} &= -\cos(\phi)\sin(\phi) + (1-\sin(\phi))\cos(\phi) \\
                       &= -\cos(\phi)\sin(\phi) + \cos(\phi) - \sin(\phi)\cos(\phi) \\
                       &= -2\cos(\phi)\sin(\phi) + \cos(\phi)
    \end{align*}

    Für die Ableitungsquadrate damit:
    \begin{align*}
      \left(\frac{dx}{d\phi}\right)^2
      &= (-\cos^2(\phi) - \sin(\phi) - \sin^2(\phi))^2 \\
      &= (-\cos^2(\phi) - (\sin(\phi)+\sin^2(\phi))^2 \\
      &= \cos^4(\phi) + 2(-\cos^2(\phi))(-(\sin(\phi)-\sin^2(\phi))+(\sin(\phi)-\sin^2(\phi))^2 \\
      &= \cos^4(\phi) + 2\cos^2(\phi)\sin(\phi)-2\cos^2(\phi)\sin^2(\phi) + \sin^2(\phi)-2\sin^3(\phi)+\sin^4(\phi) \\
      \left(\frac{dy}{d\phi}\right)^2
      &= (\cos(\phi)-2\cos(\phi)\sin(\phi))^2 \\
      &= \cos^2(\phi)-4\cos^2(\phi)\sin(\phi)+4\cos^2(\phi)\sin^2(\phi)
    \end{align*}

    Die Summe davon ist
    \begin{align*}
      \left(\frac{dx}{d\phi}\right)^2 + \left(\frac{dy}{d\phi}\right)^2
      &= &&\cos^4(\phi) + 2\cos^2(\phi)\sin(\phi)-2\cos^2(\phi)\sin^2(\phi) + \cancel{\sin^2(\phi)}-2\sin^3(\phi)+\sin^4(\phi) \\
      &&&  + \cancel{\cos^2(\phi)}-4\cos^2(\phi)\sin(\phi)+4\cos^2(\phi)\sin^2(\phi) \\
      &= &&\cos^4(\phi) + \cancel{2\cos^2(\phi)\sin(\phi)}-2\cos^2(\phi)\sin^2(\phi) + 2\sin^3(\phi)+\sin^4(\phi) \\
      &&&  \cancel{-4\cos^2(\phi)\sin(\phi)}+4\cos^2(\phi)\sin^2(\phi)+1 \\
      &= &&\cos^4(\phi) - 2\cos^2(\phi)\sin(\phi)\cancel{-2\cos^2(\phi)\sin^2(\phi)} + 2\sin^3(\phi)+\sin^4(\phi) \\
      &&&  +\cancel{4\cos^2(\phi)\sin^2(\phi)}+1 \\
      &= &&\cancel{\cos^4(\phi)} - 2\cos^2(\phi)\sin(\phi) + 2\sin^3(\phi) \cancel{+\sin^4(\phi)} \cancel{+2\cos^2(\phi)\sin^2(\phi)}+1 \\
      &= &&\cancel{(\cos^2(\phi)+\sin^2(\phi))^2} - 2\cos^2(\phi)\sin(\phi) + 2\sin^3(\phi) +1 \\
      &= &&\cancel{1} - 2\cos^2(\phi)\sin(\phi) + 2\sin^3(\phi) \cancel{+1} \\
      &= &&2 - 2\cos^2(\phi)\sin(\phi) + 2\sin^3(\phi) \\
      &= &&2 - 2\cancel{(\cos^2(\phi) + \sin^2(\phi))}\sin(\phi) \\
      &= &&2 - 2\sin(\phi) \\
      &= &&2\cdot(1- \sin(\phi))
    \end{align*}

    Nach diesen simplen Umformungen können wir nun die Länge berechnen.
    \begin{align*}
      L = \int ds &= \int\limits_{-\pi/2}^{\pi/2} \sqrt{\left(\frac{dx}{d\phi}\right)^2 + \left(\frac{dy}{d\phi}\right)^2} \; d\phi \\
              &= \int\limits_{-\pi/2}^{\pi/2} \sqrt{2\cdot(1- \sin(\phi))} \; d\phi \\
              &= \sqrt{2} \int\limits_{-\pi/2}^{\pi/2} \sqrt{1- \sin \phi} \; d\phi \\
              &= \sqrt{2} \int\limits_{-\pi/2}^{\pi/2} \sqrt{\left(\cos\frac{\phi}{2} - \sin\frac{\phi}{2}\right)^2} \; d\phi  \qquad \text{(unter Benutzung der Formel vom Aufgabenblatt)}\\
              &= \sqrt{2} \int\limits_{-\pi/2}^{\pi/2} \left|\cos\frac{\phi}{2} - \sin\frac{\phi}{2}\right| \; d\phi\\
              &= \left|\sqrt{2} \int\limits_{-\pi/2}^{\pi/2} \cos\frac{\phi}{2} - \sin\frac{\phi}{2}\right| \; d\phi \\
              &= \left|\sqrt{2} \int\limits_{-\pi/2}^{\pi/2} \cos\frac{\phi}{2} \; d\phi - \sqrt{2}\int\limits_{-\pi/2}^{\pi/2} \sin\frac{\phi}{2} \; d\phi \right| \\
              &= \sqrt{2} \left|\int\limits_{-\pi/2}^{\pi/2} \cos\frac{\phi}{2} \; d\phi - \int\limits_{-\pi/2}^{\pi/2} \sin\frac{\phi}{2} \; d\phi \right| \\
              &= \sqrt{2} \left|\left[2\sin \frac{\phi}{2}\right]_{-\pi/2}^{\pi/2} + \left[2\cos \frac{\phi}{2}\right]_{-\pi/2}^{\pi/2} \right| \\
              &= \sqrt{2} \left|2 \sqrt{2} + 0\right| \\
              &= 4
    \end{align*}
  \item
    Der Schwerpunkt auf der $x$-Achse ist
    \begin{align*}
      x_s &= \frac{1}{L} \int x \; ds \\
          &= \frac{1}{4} \int\limits_{-\pi/2}^{\pi/2} (1-\sin \phi)\cos(\phi)\sqrt{2(1- \sin \phi)} \; d\phi \\
          &= \frac{\sqrt{2}}{4} \int\limits_{-\pi/2}^{\pi/2} \cos(\phi)(1- \sin \phi)^{3/2} \; d\phi \\
    \end{align*}

    Wir substitutieren mit
    \begin{align*}
      &u(\phi) = 1 - \sin \phi \\
      \Rightarrow\quad& du = - \cos \phi \; d\phi \qquad \Leftrightarrow \qquad d\phi = -\frac{1}{\cos \phi} du
    \end{align*}

    Und erhalten weiter
    \begin{align*}
      x_s &= \frac{\sqrt{2}}{4} \int\limits_{u(-\pi/2}^{u(\pi/2)} \cancel{\cos(\phi)} u^{3/2} \left(-\cancel{\frac{1}{\cos \phi}}\right) \; d\phi \\
          &= -\frac{\sqrt{2}}{4} \int\limits_{u(-\pi/2)}^{u(\pi/2)} u^{3/2} \; d\phi \\
          &= -\frac{\sqrt{2}}{4} \frac{2}{5} \left[u^{5/2}\right]_{u(-\pi/2)}^{u(\pi/2)} \\
          &= -\frac{\sqrt{2}}{10}\left[(1-\sin \phi)\right]_{-\pi/2}^{\pi/2} \\
          &= -\frac{\sqrt{2}}{10} \left(\left(1-\sin \frac{\pi}{2}\right)^{5/2} - \left(1-\sin \frac{-\pi}{2}\right)^{5/2}\right) \\
          &= -\frac{\sqrt{2}}{10} \left(\cancel{\left(1-1\right)^{5/2}} - \left(1+1\right)^{5/2}\right) \\
          &= \frac{\sqrt{2}}{10} 2^{5/2} \\
          &= \frac{\sqrt{2}}{10} \left(2\cdot2\cdot\sqrt{2}\right) \\
          &= \frac{8}{10} \\
          &= \frac{4}{5} \\
    \end{align*}
  \item
    Die Oberfläche $A$ des Rotationskörpers berechnet sich gemäß der
    \textsc{Guldin}schen Regel über
    \begin{align*}
      A = 2 \pi L x_s = 2 \pi \cdot 4 \cdot \frac{4}{5} = \frac{32 \pi}{5}
    \end{align*}
  \item
    \textbf{Bonusaufgabe:}
    Wir nehmen an, der Schwerpunkt genau der selben halben Herzkurve sei
    gemeint, weil nichts anderes gegeben ist. Obwohl es sinnvoller wäre, den
    Schwerpunkt für die Kurve oberhalb oder unterhalb der $x$-Achse zu
    berechnen (bei einer Rotation um dieselbige). Da aber nichts dergleichen
    gegeben ist (wir müssten ja wissen, ob Ober- oder Unterseite), nehmen wir
    den Schwerpunkt für genau die selbe Kurve.

    Der Schwerpunkt liegt bei
    \begin{align*}
      y_s &= \frac{1}{L} \int y \; ds \\
          &= \frac{1}{4} \int_{-\pi/2}^{\pi/2} \sin(t)(1-\sin t)^{3/2} dt
    \end{align*}

    Wir substituieren:
    \begin{align*}
      &x = \sin(t) \\
      \Rightarrow\quad&dx = \cos(t) dt \\
      \Leftrightarrow\quad& dt = \frac{dx}{\cos(t)} = \frac{dx}{\sqrt{1-\sin(t)^2}} = \frac{dx}{\sqrt{1-x^2}}
    \end{align*}

    Und erhalten:
    \begin{align*}
      y_s &= \frac{1}{4} \int_{-1}^{1} \frac{x(1-x)^{3/2}}{\sqrt{1-x^2}} dx \\
          &= \frac{1}{4} \int_{-1}^{1} \frac{x(1-x)^{3/2}}{\sqrt{1-x}\sqrt{1+x}} dx \\
          &= \frac{1}{4} \int_{-1}^{1} \frac{x(1-x)}{\sqrt{1+x}} dx
    \end{align*}

    Wir substituieren erneut:
    \begin{align*}
      y &= 1+x, \qquad dy = dx
    \end{align*}

    Und erhalten nun
    \begin{align*}
      y_s &= \frac{1}{4} \int_{0}^{2} \frac{(y-1)(2-y)}{\sqrt{y}} dy \\
          &= \frac{1}{4} \int_{0}^{2} \frac{-y^2+3y-2}{\sqrt{y}} dy \\
          &= \frac{1}{4} \left(-\int_0^2 \frac{y^2}{\sqrt{y}} dy + 3 \int_0^2 \frac{y}{\sqrt{y}} dy - 2 \int_0^2 \frac{1}{\sqrt{y}} dy\right) \\
          &= \frac{1}{4} \left(-\int_0^2 y^{3/2} dy + 3 \int_0^2 y^{1/2} dy - 2 \int_0^2 y^{-1/2} dy\right) \\
          &= \frac{1}{4} \left(-\left[\frac{2}{5} y^{5/2}\right]_0^2 + 3 \left[\frac{2}{3} y^{3/2}\right]_0^2 - 2 \left[2\sqrt{y}\right]_0^2\right) \\
          &= \frac{1}{4} \left(-\frac{2}{5}4\sqrt{2} + 3 \frac{2}{3} 2 \sqrt{2} - 2 \cdot 2\sqrt{2}\right) \\
          &= - \frac{2\sqrt{2}}{5}
    \end{align*}



\end{enumerate}

\section*{Aufgabe 5.3}
\begin{enumerate}[(1)]
  \item
    Zuerst brauchen wir eine Kurvendarstellung für die Gerade von Punkt $(a,b)$
    zum Punkt $(c,d)$, um dann das Vektorfeld entlang dieser Kurve zu
    integrieren. Diese Kurve sei $\overline{\gamma}$. Diese ist natürlich
    definiert als
    \begin{equation*}
      x = a + (a-c)t, \qquad y = b + (b-d)t, \qquad t \in [0,1]
    \end{equation*}

    Mit den Ableitungen
    \begin{equation*}
      \frac{dx}{dt} = (a-c), \qquad \frac{dy}{dt} = (b-d)
    \end{equation*}

    Damit können wir nun das Feld entlang der Kurve integrieren
    \begin{align*}
      \int_{\overline{\gamma}} -y dx + x dy &= \int_0^1 \left(-y \frac{dx}{dt} + x \frac{dy}{dt}\right) \; dt \\
                                            &= \int_0^1 \left(-(b + (b-d)t)(a-c) + (a+(a-c)t)(b-d)\right) \; dt \\
                                            &= \int_0^1 b(c-a) + (b-d)(c-a)t + a(b-d)+(a-c)(b-d)t \; dt \\
                                            &= b(c-a)+a(b-d) +\int_0^1 (b-c)\underbrace{((c-a) + (a-c))}_{=0}t \; dt \\
                                            &= b(c-a)+a(b-d) = bc\cancel{-ab}+\cancel{ab}-ad = bc-ad
    \end{align*}
  \item
    Wie im Tutorium angesprochen, verstehen wir die Aufgabe so: Die Kurve
    besteht aus drei Teilen: Zuerst vom Nullpunkt nach $(1,0)$ (diese Kurve
    nennen wir $\overline{\xi}$, dann auf der Kurve $(\cosh t, \sinh t)$
    entlang mit $t \in [0, u]$ (diese Kurve nennen wir $\overline{\gamma}$ und
    dann dort aus zurück zum Nullpunkt (diese Kurve nennen wir
    $\overline{\eta}$)

    Die Integrale für die zwei Geraden verschwinden direkt, denn wir haben in
    Teil (1) berechnet, dass für Geraden gilt:

    \begin{equation*}
      \int_{(a,b)}^{(c,d)} \vec{f} \; d\vec{s} = bc-ad
    \end{equation*}

    Beide Geraden starten oder enden im Nullpunkt (heißt $(a,b) = (0,0)$ oder $(c,d) =
    (0,0)$). Für beide Fälle verschwindet das Integral. Somit
    \begin{equation*}
      \int_{\overline{\xi}} \vec{f} \; d\vec{s} = \int_{\overline{\eta}} \vec{f} \; d\vec{s} = 0
    \end{equation*}

    Zu Berechnen bleibt also noch das Integral für $\overline{\gamma}$. Diese ist parametrisiert durch
    \begin{equation*}
      x = \cosh(t),\qquad y = \sinh(t), \qquad t \in [0,u]
    \end{equation*}

    Mit den Ableitungen
    \begin{equation*}
      \frac{dx}{dt} = \sinh(t), \qquad \frac{dy}{dt} = \cosh(t)
    \end{equation*}

    ergibt das
    \begin{align*}
      \int_{\overline{\gamma}} \vec{f} \; d\vec{s} &=  \int_{\overline{\gamma}} -y dx + x dy \\ 
                                                   &= \int_0^u \left(-y \frac{dx}{dt} + x \frac{dy}{dt}\right) \; dt \\
                                                   &= \int_0^u \left(-\sinh(t) \sinh(t) + \cosh(t) \cosh(t)\right) \; dt \\
                                                   &= \int_0^u \underbrace{\left(\cosh^2(t) - \sinh^2(t)\right)}_{=1} \; dt \\
                                                   &= \int_0^u dt \\
                                                   &= u
    \end{align*}

    Also gilt für das Integral über das gesamte ``Dreieck''
    \begin{equation*}
      \int_{\overline{\gamma}\overline{\xi}\overline{\eta}} \vec{f} \; d\vec{s} = 0 + u + 0 = u
    \end{equation*}
\end{enumerate}

\section*{Aufgabe 5.4}
\begin{align*}
  J_f &= \left( \frac{\partial f}{\partial x}, \frac{\partial f}{\partial y} \right) \\
      &= \left( \frac{\partial}{\partial x} xy+\frac{x}{y}, \frac{\partial}{\partial y} xy+\frac{x}{y}\right) \\
      &= \left( y+\frac{1}{y}, x-\frac{1}{y^2}x\right) \\
  J_f(1,-1) &= \left( (-1)+\frac{1}{-1}, 1-\frac{1}{(-1)^2}\right) \\
  J_f(1,-1) &= \left( -2, 0\right) \\
\end{align*}


\end{document}
