\documentclass[a4paper,german,12pt,smallheadings]{scrartcl}
\usepackage[T1]{fontenc}
\usepackage[utf8]{inputenc}
\usepackage{babel}
\usepackage{tikz}
\usepackage{geometry}
\usepackage{amsmath}
\usepackage{amssymb}
\usepackage{float}
\usepackage{enumerate}
\usepackage{cancel}
\usepackage{pgfplots}
\usepackage{commath}
\pgfplotsset{compat=1.7}
\usepgfplotslibrary{polar}
%\usepackage{wrapfig}
\usepackage[thinspace,thinqspace,squaren,textstyle]{SIunits}
\restylefloat{table}
\renewcommand{\thefootnote}{\fnsymbol{footnote}}
\geometry{a4paper, top=15mm, left=20mm, right=40mm, bottom=20mm, headsep=10mm, footskip=12mm}
\linespread{1.5}
\setlength\parindent{0pt}
\begin{document}
\begin{center}
\bfseries % Fettdruck einschalten
\sffamily % Serifenlose Schrift
\vspace{-40pt}
Analysis II, Wintersemester 2013/2014, 11. Übungsblatt

Markus Fenske, Luis Herrmann, Tutor: Sebastian Bierke
\vspace{-10pt}
\end{center}
\allowdisplaybreaks % Seitenumbrüche in Formeln erlauben
\section*{Aufgabe 11.1}
\begin{align*}
  Z_1 &= \cbr{(x,y,z): x^2+y^2 \le r^2} \\
  Z_2 &= \cbr{(x,y,z): y^2+z^2 \le r^2}
\end{align*}

Zu berechnen ist
\begin{equation*}
  \iiint\limits_{Z_1 \cap Z_2} 1 \dif\del{x,y,z}
\end{equation*}

Dabei sind die Integrationsgrenzen wie folgt zu wählen. Da $y^2 \le r^2 - x^2
\Rightarrow y^2 \le r^2$ ist $\envert{y_\text{max}} = r$. Und $\envert{y_\text{min}} = 0$, nämlich genau dann wenn
\begin{equation*}
  x = r
\end{equation*}

Entsprechend
\begin{align*}
  x^2 \le r^2 - y^2 \Rightarrow \envert{x} &\le \sqrt{r^2 - y^2} \\
  z^2 \le r^2 - y^2 \Rightarrow \envert{z} &\le \sqrt{r^2 - y^2} \\
\end{align*}

\begin{align*}
  \Rightarrow \qquad &x: \sbr{- \sqrt{r^2 - y^2}, + \sqrt{r^2 - y^2}} \\
              &z: \sbr{- \sqrt{r^2 - y^2}, + \sqrt{r^2 - y^2}} \\
\end{align*}

\begin{align*}
  \int\limits_{-r}^r
  \int\limits_{-\sqrt{r^2 - y^2}}^{\sqrt{r^2 - y^2}}
  \int\limits_{-\sqrt{r^2 - y^2}}^{\sqrt{r^2 - y^2}}
  \dif x \dif z \dif y
  &=
  \int\limits_{-r}^r
  \int\limits_{-\sqrt{r^2 - y^2}}^{\sqrt{r^2 - y^2}}
  2 \sqrt{r^2 - y^2}
  \dif z \dif y \\
  &=
  \int\limits_{-r}^r
  4 \del{r^2 - y^2}
  \dif y \\
  &= \sbr{4r^2y - \frac{y^3}{3}}_{-r}^{r} \\
  &= 4 \del{\del{r^3 - \frac{r^3}{3}} - \del{-r^3 + \frac{r^3}{3}}} \\
  &= 4 \del{\frac{2}{3}r^2 + \frac{2}{3}r^3} \\
  &= \frac{16}{3} r^3
\end{align*}

\section*{Aufgabe 11.2}

Das Simplex $\operatorname{Sim}\del{n, a}$ entsteht aus dem Einheitssimplex
$\operatorname{Sim}\del{n,1}$ durch Skalierung. Dabei muss dieses wenn $a$ die
Kantenlänge ist, offensichtlich mit $a^n$ skalieren.

\begin{equation*}
  \envert{\operatorname{Sim}\del{n, a}} = a^n \envert{\operatorname{Sim}\del{n, 1}}
\end{equation*}

Es genügt also $\envert{\operatorname{Sim}\del{n, 1}}$ zu ermitteln.

Alle Koordinaten variieren zwischen $0$ und $1$.

Der Schnitt von $\operatorname{Sim}\del{n,1}$ in Höhe $t$ ist:

\begin{equation*}
  \cbr{\overline{x} \in \mathbb{R}^{n-1}: x_1 + \dots + x_{n-1} + t \le 1} =
  \cbr{\overline{x} \in \mathbb{R}^{n-1}: \sum_{i=1}^{n-1} x_i\le 1 - t}
\end{equation*}

\begin{equation*}
  \operatorname{Sim}\del{n,1}^t = \operatorname{Sim}\del{n-1, 1}^t (1-t)
\end{equation*}

Entsprechend folgt

\begin{align*}
  \envert{\operatorname{Sim}\del{n,1}} &= \int_0^1 \envert{\operatorname{Sim}\del{n-1,1}} (1-t) \dif t \\
                                       &= \int_0^1 \del{{\int \dots \int}_{\operatorname{Sim}\del{n-1,1} (1-t)} 1 \dif x} \dif t \\
    &= \envert{\operatorname{Sim}\del{n-1, 1}} \int_0^1 \del{1-t} \dif t \\
    &= \frac{1}{2} \envert{\operatorname{Sim}\del{n-1, 1}}
\end{align*}

Dann ergibt sich rasch die gesuchte Formel.

\begin{align*}
  n = 1: \qquad &\envert{\operatorname{Sim}(1,1)} = 1 \\
  n = 2: \qquad &\envert{\operatorname{Sim}(2,1)} = \frac{1}{2} \envert{\operatorname{Sim}(1,1)} = \frac{1}{2} \\
  n = 3: \qquad &\envert{\operatorname{Sim}(3,1)} = \frac{1}{2} \envert{\operatorname{Sim}(2,1)} = \frac{1}{4} \\
  \vdots \qquad \quad \;\;  & \qquad \qquad \quad \, \vdots \\
  n = n: \qquad &\envert{\operatorname{Sim}(n,1)} = \frac{1}{2^{n-1}}
\end{align*}

% FIXME: Beweis über vollständige Induktion trivial. Muss ich das wirklich abtippen?

Also gilt
\begin{equation*}
  \envert{\operatorname{Sim}(n,a)} = a^n \envert{\operatorname{Sim}(n,1)} = \frac{a^n}{2^{n-1}}
\end{equation*}

Daraus folgt für $n \to \infty$:

\begin{align*}
  \lim\limits_{n \to \infty} \envert{\operatorname{Sim}(n,a)} &= \lim\limits_{n \to \infty} \frac{a^n}{2^{n-1}} \\
                                                              &= a \lim\limits_{n \to \infty} \del{\frac{a}{2}}^{n-1} \\
                                                              &= a \lim\limits_{n \to \infty} \del{\frac{a}{2}}^{n}
\end{align*}

Für $a < 2$ ist der Grenzwert offensichtlich existent und verschwindet. % Ich mag die Formulierung.
Für $a = 2$ ist der innere Grenzwert $1$, also geht der ganze Term gegen $a$.
Für $a > 2$ divergert die Folge gegen $+ \infty$.

Also ist
\begin{align*}
  \lim\limits_{n \to \infty} \envert{\operatorname{Sim}(n,a)} =
  \begin{cases} 
    0 &\mbox{für } a < 2 \\
    2 & \mbox{für } a = 2 \\
    \infty & \mbox{für } a > 2 \\
    \end{cases}
\end{align*}

\section*{Aufgabe 11.3}

In Zylinderkoordinaten $(r, \phi, z)$ gilt für den Zylinder
$K_{\overline{0}}(R)$ und die Kugel $Z(a)$ (wir entschuldigen uns für die
Vertauschung der Namen und Variablen und korrigieren das am Ende):

\begin{align*}
  K_{\overline{0}}(R) &= \cbr{(r, \phi, z): r^2 + z^2 \le R} \\
  Z(a) &= \cbr{(r, \phi, z): r \le a} \\
\end{align*}

Entsprechend ist
\begin{align*}
  S = K_{\overline{0}} \backslash Z(a) = \cbr{(r, \phi, z): r^2+z^2 \le R^2 \text{ oder } a \le r \le R} \\
  \Rightarrow z^2 \le R^2 - r^2
\end{align*}

Dann ist klar: $z^2: \cbr{R^2 - R^2, R^2 - a} = [0,R^2 -a^2] \Leftrightarrow z: \sbr{-\sqrt{R^2-r^2}, \sqrt{R^2- r^2}}$.

Sei $g: (x,y,z) \to (r, \phi, z)$. Dann ist das gesuchte Volumenintegral

\begin{align*}
  \iiint\limits_{S} 1 \dif(x,y,z) &= \iiint\limits_{S} \det J_g(r, \phi, z) \dif(r, \phi, z)
\end{align*}

Mit $x = r \cos \phi, y = r \sin \phi, z = z$ ist die Jacobi-Matrix:

\begin{align*}
  J_g = \begin{pmatrix}
    \cos \phi & -r \sin \phi & 0 \\
    \sin \phi & r \cos \phi & 0 \\
    0 & 0 & 1
  \end{pmatrix}
\end{align*}

Damit ist $\det J_g$ nach Saruss
\begin{align*}
  \det J_g = r cos^2 \phi - \del{-r \sin^2 \phi} = r
\end{align*}

Das führt dann auf
\begin{align*}
  \iiint\limits_{S} r \dif(r, \phi, z)
\end{align*}

mit der Parametrisierung
\begin{align*}
  &\phi: [0, 2\pi] \\
  &r: [a, R] \\
  &z: [-\sqrt{R^2 - r^2}, \sqrt{R^2 - r^2}]
\end{align*}

wird das zu

\begin{align*}
  \int\limits_{0}^{2 \pi}
  \int\limits_{a}^{R}
  \int\limits_{-\sqrt{R^2 - r^2}}^{\sqrt{R^2 - r^2}}
  r
  \dif z
  \dif r
  \dif \phi
  &=
  \int\limits_{0}^{2 \pi}
  \int\limits_{a}^{R}
  2r \sqrt{R^2 - r^2}
  \dif r
  \dif \phi
  \intertext{
    Substitution $u = R^2 - r^2 \Rightarrow \dif r = \frac{\dif u}{-2 r}$
  }
  &=
  \int\limits_{0}^{2 \pi}
  \int\limits_{R^2-a^2}^{0}
  \frac{2r}{-2r} \sqrt{u}
  \dif u
  \dif \phi \\
  &=
  \int\limits_{0}^{2 \pi}
  \frac{2}{3}\del{R^2 -a^2}^\frac{3}{2}
  \dif \phi \\
  &=
  \frac{4 \pi}{3}\del{R^2 -a^2}^\frac{3}{2}
\end{align*}


\end{document}
