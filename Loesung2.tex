\documentclass[a4paper,german,12pt,smallheadings]{scrartcl}
\usepackage[T1]{fontenc}
\usepackage[utf8]{inputenc}
\usepackage{babel}
\usepackage{tikz}
\usepackage{geometry}
\usepackage{amsmath}
\usepackage{amssymb}
\usepackage{float}
%\usepackage{wrapfig}
\usepackage[thinspace,thinqspace,squaren,textstyle]{SIunits}
\restylefloat{table}
\geometry{a4paper, top=15mm, left=20mm, right=40mm, bottom=20mm, headsep=10mm, footskip=12mm}
\linespread{1.5}
\setlength\parindent{0pt}
\begin{document}
\begin{center}
\bfseries % Fettdruck einschalten
\sffamily % Serifenlose Schrift
\vspace{-40pt}
Analysis I, Sommersemester 2013, 2. Übungsblatt

Luis Herrmann und Markus Fenske, Tutor: Adam Schienle
\vspace{-10pt}
\end{center}

\section*{Aufgabe 2.1}
\subsection*{Teil a}
Wenn $f, g$ beschränkt sind, gilt $|f(x)| \le K_1 \forall x$ und $|g(x)| \le K_2
\forall x$. Addieren der beiden Ungleichungen:

\begin{align*}
  |f(x)| + |g(x)| \le K_1 + K_2 \forall x \\
\end{align*}
Wegen der Dreieckesungleichung ($|a + b| \le |a| + |b|$) gilt dann:
\begin{align*}
  |f(x) + g(x)| \le K_1 + K_2 \forall x \\
\end{align*}

Somit ist $f + g$ beschränkt.
\subsection*{Teil b}
\textbf{FIXME} Fehlt % FIXME

\subsection*{Teil c}
Sei $f: x \mapsto x^2$ und $g: x \mapsto x$:
\begin{align*}
  \sup_{x \in I} f(x) &= f(1) = 1 \\
  \sup_{x \in I} g(x) &= g(1) = 1 \\
  \sup_{x \in I} f(x) + g(x) &= 2
\end{align*}

\subsection*{Teil d}
Sei $f: x \mapsto -x$ und $g: x \mapsto x$:

\begin{align*}
  \sup_{x \in I} f(x) &= f(0) = 0 \\
  \sup_{x \in I} g(x) &= g(1) = 1 \\
  \sup_{x \in I} f(x) + g(x) &= 0
\end{align*}

\section*{Aufgabe 2.2}

Bekannt ist die allgemeine binomische Formel:

\begin{equation*}
  (a+b)^n = \sum_{k=0}^n {n \choose k} a^{n-k} b^k
\end{equation*}

Mit $a=b=1$:
\begin{align*}
 \sum_{k=0}^n {n \choose k} 1^{n-k} 1^k &= (1+1)^n \\
  \sum_{k=0}^n {n \choose k} &= 2^n
\end{align*}

Mit $a=1, b=-1$:
\begin{align*}
  \sum_{k=0}^n {n \choose k} 1^{n-k} (-1)^k &= (1-1)^n \\
  \sum_{k=0}^n {n \choose k} (-1)^k &= 0
\end{align*}


\section*{Aufgabe 2.3}
\textbf{FIXME} Fehlt % FIXME

\section*{Aufgabe 2.4}
Wir wissen, dass $\left(\sqrt{n^2}\right) = (n)$ bestimmt divergiert. Wenn also
$n!$ schneller wächst als $n^2$, muss die Folge zwangsläufig bestimmt
divergieren.

Dies ist für $n = 4$ der Fall. Per Induktion kann man zeigen, dass dies für
jedes $n \ge 4$ gilt.

\begin{align*}
  (n+1)! &\gt (n+1)^2
\end{align*}

Da $(n+1) \gt 0$, kann dividiert werden:
\begin{align*}
  n! &\gt (n+1)
\end{align*}

Da wir wissen, dass $n! \gt n^2$, ist zu zeigen, dass $n^2 \gt n+1$ für alle
$n \gt 4$. Ebenfalls per Induktion. Für $n = 4$ ist die Gleichung erfüllt:

\begin{align*}
  n^2 &\gt n+1 \\
  4^2 &\gt 4+1 \\
  16 &\gt 5
\end{align*}

Für $n+1$ muss dann gelten:
\begin{align*}
  (n+1)^2 \gt (n+1)+1 \\
  (n^2+2n+1) \gt (n+2) \\
  (n^2+2n) \gt (n+1) \\
\end{align}


\end{document}
