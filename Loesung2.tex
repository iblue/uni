\documentclass[a4paper,german,12pt,smallheadings]{scrartcl}
\usepackage[T1]{fontenc}
\usepackage[utf8]{inputenc}
\usepackage{babel}
\usepackage{tikz}
\usepackage{geometry}
\usepackage{amsmath}
\usepackage{amssymb}
\usepackage{float}
%\usepackage{wrapfig}
\usepackage[thinspace,thinqspace,squaren,textstyle]{SIunits}
\restylefloat{table}
\geometry{a4paper, top=15mm, left=20mm, right=40mm, bottom=20mm, headsep=10mm, footskip=12mm}
\linespread{1.5}
\setlength\parindent{0pt}
\begin{document}
\begin{center}
\bfseries % Fettdruck einschalten
\sffamily % Serifenlose Schrift
\vspace{-40pt}
Elektrodynamik und Optik, Sommersemester 2013, 3. Blatt \\
Markus Fenske, Tutor: Dr. Marko Wietstruk
\vspace{-10pt}
\end{center}
\section*{Aufgabe 1}
\subsection*{Teil a}

Homogene Ladungsverteilungen auf einer Kugel verhalten sich von außerhalb der
Kugel betrachtet wie eine Punktladung im Zentrum. Dementsprechend:

\begin{align*}
  &E = \frac{q}{4 \pi \epsilon_0r^2} \\
  \Leftrightarrow \quad&q = \frac{E 4 \pi \epsilon_0}{r^2}
\end{align*}

Einsetzen der Werte:

\begin{align*}
  q &= \frac{3 \cdot 10^6 \cdot 4 \cdot \pi \cdot 8{,}86 \cdot 10^{-12}}{0{,}05^2} \coulomb \\
    & \approx 0{,}13 \;\coulomb
\end{align*}

\subsection*{Teil b}

Homogene Ladungsverteilungen auf einer Kugel verhalten sich von außerhalb der
Kugel betrachtet wie eine Punktladung im Zentrum. FÜr Potential zwischen einem
Punkt in unendlichem Abstand und einem Punkt im Abstand zu einer Punktladung gilt:

\begin{align*}
  \Phi &= \frac{E}{r} = \frac{3 \cdot 10^6}{0{,}05} \volt = 60 \;\mega\volt
\end{align*}

\subsection*{Teil c}

In einem kubischen Atomgitter befindet sich ein Atom pro Gittereinheit. Die Atomdichte ist
diesem Fall also $\frac{1}{0{,}25^3 \;\nano\meter^3} = \frac{1}{1{,}56 \cdot 10^{-29} \;\meter^3}$.

Für das Volumen der Hohlkugel gilt:

\begin{align*}
  V &= \frac{4}{3} \pi (r_a^3 - r_i^3), \quad r_a = 50 \;\milli\meter, r_i = 47\;\milli\meter \\
    &= 8{,}87 \cdot 10^{-5}\; \meter^3 \\
\end{align*}

Die Anzahl der Atome ist dann

\begin{align*}
  n &= V \cdot \frac{n}{V} = 8{,}87 \cdot 10^{-5} \cdot \frac{1}{1{,}56 \cdot 10^{-29} \;\meter^3} \\
    &= 5{,}69 \cdot 10^{24}
\end{align*}

Die Anzahl der Elektronen in der Ladung ist

\begin{align*}
  n_e &= \frac{0{,}133 \;\coulomb}{1{,}6 \cdot 10^{-19} \coulomb} \\
      &= 8{,}3 \cdot 10^{17}
\end{align*}

Das bedeutet, dass bei der gegebenen Ladung etwa $10^{7}$-mal mehr Atome im
Hohlkopf enthalten sind, als Elektronen auf der Oberfläche.


In der obersten Schicht ist ein Atom auf einer Oberfläche von jeweils $(0{,}25
\;\nano\meter)^2 = 6{,}25 \cdot 10^{-20} \meter^2$. Die Oberfläche des
Hohlkopfes ist

\begin{align*}
  A = 4 \pi r^2 = 4 \pi \cdot 0{,}05^2 \meter^2 \approx 3{,}1 \cdot 10^{-2} \meter^2
\end{align*}

Die Anzahl der Atome auf der Oberfläche ist dann

\begin{align*}
  n &= A \cdot \frac{n}{A} \\
    &= 3{,}1 \cdot 10^{-2} \meter^2 \cdot \frac{1}{6{,}25 \cdot 10^{-20} \;\meter^2} \\
    &\approx 5{,}0 \cdot 10^{17}
\end{align*}

Die Anzahl der Elektronen auf der Oberfläche liegt trotzdem noch um den Faktor
$1{,}6$ über diesem Wert.

\section*{Aufgabe 2}
Wenn man das elektrische Feld als homogen annimmt, ist die Feldstärke $E = 25
\kilo\volt\per\meter = 25 \newton\per\coulomb$. Auf ein Elektron wirkt dann
eine Kraft von $F = eE = 4 \cdot 10^{-15} \newton$ in Richtung einer der beiden
Platten.

Die x-Komponente des Ortes des Elektrons ist dementsprechend 

\begin{align*}
  x &= \frac{1}{2}at^2 = \frac{1}{2} \frac{F}{m_e} t^2 \\
    &= \frac{1}{2} \frac{4 \cdot 10^{-15} \newton}{9{,}1 \cdot 10^{-31} \kilo\gram} \cdot t^2
\end{align*}

Das Elektron trifft auf die Platte, sofern es um mehr als $1 \centi\meter$ in
eine der beiden Richtungen abgelenkt wird. Die Zeit, die das Feld dazu
einwirken muss ist

\begin{align*}
  t &= \sqrt{2\cdot 0{,}01 \meter \frac{9{,}1 \cdot 10^{-31} \kilo\gram}{4 \cdot 10^{-15} \newton}} \\
    &\approx 2{,}1 \cdot 10^{-9} \second
\end{align*}

Das Elektron muss die Plattenlänge von $10 \centi\meter$ also in weniger als dieser Zeit durchqueren. Dazu muss es dazu folgende Geschwindigkeit haben.

\begin{align*}
  v = \frac{0{,}1 \meter}{2{,}1 \cdot 10^{-9} \second} = 4{,}8 \cdot 10^{7} \meter\per\second
\end{align*}

Um diese Geschwindigkeit zu erreichen, muss $E_{\operatorname{kin}} = E_{\operatorname{el}}$:

\begin{align*}
  &\frac{1}{2} m_e v^2 = qU \\
  \Leftrightarrow \quad & U = \frac{1}{2} \frac{m_e}{q} v^2
\end{align*}

Daraus ergibt sich eine Beschleunigungsspannung $U_0 \approx 6{,}5 \;\kilo\volt$.

\section*{Aufgabe 3}
\subsection*{Teil a}
Das Feld in einem Plattenkondensator ist näherungsweise homogen. Werden die
Platten auf einen größeren Abstand bei gleichzeitigem Ladungserhalt gebracht,
steigt das Volumen das vom Feld ausgefüllt wird. Somit muss das Potential wegen
der Homogenität des Feldes steigen.

Sollte diese Begründung nicht nachvollziehbar sein, kann ich auch alternativ
begründen, dass beim Entfernen der ungleichnamig geladenen Platten Arbeit
aufgewendet werden muss. Da die Ladung erhalten bleibt, kann sich diese Arbeit
nur in einer Potentialdifferenz niederschlagen.

\subsection*{Teil b}
Wenn die Spannungsquelle verbunden bleibt, kann Ladung abfließen. Die Spannung
bleibt gleich. Die Feldstärke verringert sich, die Kapazität steigt.


\section*{Aufgabe 4}

\begin{align*}
  C_{\operatorname{gesamt}} = \frac{1}{\frac{1}{C_1} + \frac{1}{C_2 + C_3}} = 0{,}1\overline{3} \;\micro\farad
\end{align*}

\section*{Aufgabe 5}
\subsection*{Teil a}
Für die Bestimmung des elektrischen Feldes betrachten wir Innen- und Außenkugel
separat. Der Beitrag der Außenkugel ist $E=0$, denn das Innere einer leitenden
Kugel ist feldfrei, auch im inneren der Innenkugel herrscht deswegen kein Feld.
Die Innenkugel erzeugt ein Feld, das dem einer Punktladung entspricht.
Dementsprechend ist:

\begin{align*}
  E(r) = \left\{\begin{array}{ll} \frac{Q}{4 \pi \epsilon_0 r^2} & a < r < b \\
         0 & r < a \lor r > b\end{array}\right.
\end{align*}

Das Potential bestimmt sich durch $\Phi(r) = -\int_\infty^r E(r')\, dr'$. Da
außerhalb des Kondensators kein Feld herrscht, reicht es, von der äußeren
Kugelschale zu integrieren.

\begin{align*}
  \Phi(r) &= -\int_{b}^r E(r')\, dr' \\
          &= -\int_{b}^r \frac{Q}{4 \pi \epsilon_0 r'^2}\, dr' \\
          &= \frac{Q}{4\pi \epsilon_0} \left(\frac{1}{r}-\frac{1}{b}\right)
\end{align*}

Damit gilt für die Spannung zwischen den Kugelschalen:

\begin{align*}
  U = \frac{Q}{4\pi \epsilon_0} \left(\frac{1}{a}-\frac{1}{b}\right)
\end{align*}

Dies liefert dann, wie gefordert, die Kapazität:

\begin{align*}
  C &= \frac{Q}{U} =  4\pi \epsilon_0 \frac{1}{\left(\frac{1}{a}-\frac{1}{b}\right)} \\
    &=  4\pi \epsilon_0 \frac{1}{\left(\frac{b}{ab}-\frac{a}{ab}\right)} \\
    &=  4\pi \epsilon_0 \frac{1}{\frac{b-a}{ab}} \\
    &=  4\pi \epsilon_0 \frac{ab}{b-a}
\end{align*}

\subsection*{Teil b}
Vom letzten Blatt ist das elektrische Feld eines Zylinders bekannt. Mit der
selben Argumentation wie in Teil a ergibt sich dann für das elektrische Feld:

\begin{align*}
  E(r) = \left\{\begin{array}{ll} \frac{Q}{2 \pi \epsilon_0 lr} & a < r < b \\
         0 & r < a \lor r > b\end{array}\right.
\end{align*}

Damit gilt für das Potential (Argumente wie vorher)
\begin{align*}
  \Phi(r) &= -\int_{b}^r E(r')\, dr' \\
          &= -\int_{b}^r \frac{Q}{2 \pi \epsilon_0 r'}\, dr' \\
          &= -\frac{Q}{2 \pi \epsilon_0} \ln \frac{b}{r}
\end{align*}

Das Potential zwischen den Zylindern ist:

\begin{align*}
  U = -\frac{Q}{2 \pi \epsilon_0} \ln \frac{b}{a}
\end{align*}

Mit $C = \frac{Q}{U}$ ergibt sich die Kapazität:

\begin{align*}
  C &= \frac{Q}{\frac{Q}{2 \pi \epsilon_0} \ln \frac{b}{a}} \\
    &= \frac{2 \pi \epsilon_0}{\ln \frac{b}{a}}
\end{align*}


\end{document}
