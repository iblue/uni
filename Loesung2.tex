\documentclass[a4paper,german,12pt,smallheadings]{scrartcl}
\usepackage[T1]{fontenc}
\usepackage[utf8]{inputenc}
\usepackage{babel}
\usepackage{tikz}
\usepackage{geometry}
\usepackage{amsmath}
\usepackage{amssymb}
\usepackage{float}
%\usepackage{wrapfig}
\usepackage[thinspace,thinqspace,squaren,textstyle]{SIunits}
\restylefloat{table}
\geometry{a4paper, top=15mm, left=20mm, right=40mm, bottom=20mm, headsep=10mm, footskip=12mm}
\linespread{1.5}
\setlength\parindent{0pt}
\begin{document}
\begin{center}
\bfseries % Fettdruck einschalten
\sffamily % Serifenlose Schrift
\vspace{-40pt}
Analytische Mechanik, Sommersemester 2013, 2. Blatt

Luis Herrmann und Markus Fenske, Tutor: Clemens Meyer zu Rheda
\vspace{-10pt}
\end{center}
\section*{Aufgabe 1}
\subsection*{Teil a}
Damit $L'(q, \dot{q}, t) = L(q, \dot{q}, t) + \frac{d}{dt} F(q(t))$ auf die
selben Bewegungsgleichungen führt, muss

\begin{align*}
  \left(\frac{d}{dt} \frac{\partial}{\partial \dot{q}} - \frac{\partial}{\partial q}\right) \left(L - L'\right) &= 0 \\
  \left(\frac{d}{dt} \frac{\partial}{\partial \dot{q}} - \frac{\partial}{\partial q}\right) \left(\frac{d}{dt} F\right) &= 0 \\
\end{align*}

Dies führt zu
\begin{align*}
  \frac{d}{dt} \frac{\partial}{\partial \dot{q}}\frac{d}{dt} F &= \frac{\partial}{\partial q}\frac{d}{dt} F \\
  \frac{d}{dt} \frac{\partial}{\partial \dot{q}}\frac{\partial F}{\partial q} \frac{dq}{dt} &= \frac{\partial}{\partial q}\frac{d}{dt} F\\
  \frac{d}{dt} \frac{\partial}{\partial \dot{q}}\frac{\partial F}{\partial q} \dot{q} &= \frac{\partial}{\partial q}\frac{d}{dt} F\\
  \frac{d}{dt} \frac{\partial}{\partial \dot{q}} \dot{q} \frac{\partial F}{\partial q} &= \frac{\partial}{\partial q}\frac{d}{dt} F\\
  \frac{d}{dt} \frac{\partial F}{\partial q} &= \frac{\partial}{\partial q}\frac{d}{dt} F
\end{align*}

Die Ableitungen sollten vertauschbar sein, zur Sicherheit beweisen wir von der rechten Seite her.

\begin{align*}
  \frac{\partial}{\partial q}\frac{d}{dt} F &= \frac{\partial \dot{q}}{\partial q} \frac{\partial F}{\partial q} \\
                                            &= \frac{\partial}{\partial q} \frac{dq}{dt} \frac{\partial F}{\partial q} \\
                                            &= \frac{d}{dt} \frac{\partial F}{\partial q}
\end{align*}

Somit führen $L$ und $L'$ auf die selben Bewegungsgleichungen.

\subsection*{Teil b}
$L_1$ und $L_2$ führen auf die selben Bewegungsgleichungen, denn $L_2 - L_1 =
q^3\dot{q} = \frac{d}{dt} \frac{q^4}{4}$. Damit ergibt sich aus obigem Beweis
die Äquivalenz der Bewegungsgleichungen.

$L_2$ und $L_3$ führen auf die selben Bewegungsgleichungen denn eine
Lagrange-Funktion kann mit beliebigem $c \neq 0$ multipliziert werden und führt
auf die selben Bewegungsgleichungen. Beweis:

\begin{align*}
  &\left(\frac{d}{dt} \frac{\partial}{\partial \dot{q}} - \frac{\partial}{\partial q}\right) c L = 0 \\
  \Leftrightarrow & \left(\frac{d}{dt} \frac{\partial}{\partial \dot{q}} - \frac{\partial}{\partial q}\right) L = \frac{0}{c} = 0
\end{align*}

$L_4$ führt zu einer anderen Bewegungsgleichung als $L_1$, $L_2$ und $L_3$,
denn $q^3$ kann nicht als Zeitableitung eines $F(q(t))$ geschrieben werden,
ohne direkt von $t$ abzuhängen.

$L_5$ führt widerrum zu einer anderen Bewegungsgleichung, als alle
vorhergehenden, denn ohne die vorherige Bewegungsgleichung explizit
auszurechnen, ist klar, dass aus $L_4$ bei der Anwendung des
Lagrange-Formalismus keine gemischten Terme der Form ($c \cdot q^i \cdot
\dot{q}^j, \quad i,j > 0$) entstehen können, während dies bei $L_5$ und $L_6$
der Fall sein muss.

$L_6$ ist die $\frac{3}{2}$te Potenz von $L_5$. Die Vermutung, dass dies auf
die selben Bewegungsgleichungen führt, überprüfe ich durch Ausrechnen:

Für $L_5$ ergibt sich:

\begin{align*}
  \frac{d}{dt} \frac{\partial}{\partial \dot{q}} q^2\dot{q}^4 &= \frac{\partial}{\partial q} q^2\dot{q}^4 \\
  \frac{d}{dt} (q^24\dot{q}^3) &= 2q\dot{q}^4 \\
  2\dot{q}q4\dot{q}^3 + q^24 \cdot 3 \ddot{q}\dot{q}^2 &= 2q\dot{q}^4 \\
  8q\dot{q}^4+ 12q^2\dot{q}^2\ddot{q} &= 2q\dot{q}^4 \\
  4q\dot{q}^4+ 6q^2\dot{q}^2\ddot{q} &= q\dot{q}^4 \\
  4\dot{q}^4+ 6q\dot{q}^2\ddot{q} &= \dot{q}^4 \\
  3\dot{q}^2+ 6q\ddot{q} &= 0 \\
  \dot{q}^2+ 2q\ddot{q} &= 0 \\
\end{align*}

Für $L_6$:

\begin{align*}
  \frac{d}{dt} \frac{\partial}{\partial \dot{q}} q^3\dot{q}^6 &= \frac{\partial}{\partial q} q^3\dot{q}^6 \\
  \frac{d}{dt} (q^36\dot{q}^5) &= 3q^2\dot{q}^6 \\
  6q^3 \cdot 5\dot{q}^4\ddot{q} + 6\dot{q}^5\cdot3q^2\dot{q} &= 3q^2\dot{q}^6 \\
  30q^3\dot{q}^4\ddot{q} + 18q^2\dot{q}^6 &= 3q^2\dot{q}^6 \\
  10q^3\dot{q}^4\ddot{q} + 6q^2\dot{q}^6 &= q^2\dot{q}^6 \\
  10q\dot{q}^4\ddot{q} + 6\dot{q}^6 &= \dot{q}^6 \\
  10q\ddot{q} + 6\dot{q}^2 &= \dot{q}^2 \\
  \dot{q}^2 + 2q\ddot{q} &= 0
\end{align*}

Für $L_5$ und $L_6$ erhält man also die selben Bewegungsgleichungen.

Zusammenfassend: $L_1 \equiv L_2 \equiv L_3 \not \equiv L_4 \not \equiv L_5 \equiv L_6$.


\end{document}
