\documentclass[a4paper,german,12pt,smallheadings]{scrartcl}
\usepackage[T1]{fontenc}
\usepackage[utf8]{inputenc}
\usepackage{babel}
\usepackage{tikz}
\usepackage{geometry}
\usepackage{amsmath}
\usepackage{amssymb}
\usepackage{float}
%\usepackage{wrapfig}
\usepackage[thinspace,thinqspace,squaren,textstyle]{SIunits}
\restylefloat{table}
\geometry{a4paper, top=15mm, left=20mm, right=40mm, bottom=20mm, headsep=10mm, footskip=12mm}
\linespread{1.5}
\setlength\parindent{0pt}
\begin{document}
\begin{center}
\bfseries % Fettdruck einschalten
\sffamily % Serifenlose Schrift
\vspace{-40pt}
Analytische Mechanik, Sommersemester 2013, 2. Blatt \\
Luis Herrmann und Markus Fenske, Tutor: Clemens Meyer zu Rheda
\vspace{-10pt}
\end{center}
\section*{Aufgabe 1}
\subsection*{Teil a}
Damit $L'(q, \dot{q}, t) = L(q, \dot{q}, t) + \frac{d}{dt} F(q(t))$ auf die
selben Bewegungsgleichungen führt, muss

\begin{align*}
  \left(\frac{d}{dt} \frac{\partial}{\partial \dot{q}} - \frac{\partial}{\partial q}\right) \left(L - L'\right) &= 0 \\
  \left(\frac{d}{dt} \frac{\partial}{\partial \dot{q}} - \frac{\partial}{\partial q}\right) \left(\frac{d}{dt} F\right) &= 0 \\
\end{align*}

Dies führt zu
\begin{align*}
  \frac{d}{dt} \frac{\partial}{\partial \dot{q}}\frac{d}{dt} F &= \frac{\partial}{\partial q}\frac{d}{dt} F \\
  \frac{d}{dt} \frac{\partial}{\partial \dot{q}}\frac{\partial F}{\partial q} \frac{dq}{dt} &= \frac{\partial}{\partial q}\frac{d}{dt} F\\
  \frac{d}{dt} \frac{\partial}{\partial \dot{q}}\frac{\partial F}{\partial q} \dot{q} &= \frac{\partial}{\partial q}\frac{d}{dt} F\\
  \frac{d}{dt} \frac{\partial}{\partial \dot{q}} \dot{q} \frac{\partial F}{\partial q} &= \frac{\partial}{\partial q}\frac{d}{dt} F\\
  \frac{d}{dt} \frac{\partial F}{\partial q} &= \frac{\partial}{\partial q}\frac{d}{dt} F
\end{align*}

Die Ableitungen sollten vertauschbar sein, zur Sicherheit beweisen wir von der
rechten Seite her.

\begin{align*}
  \frac{\partial}{\partial q}\frac{d}{dt} F &= \frac{\partial \dot{q}}{\partial q} \frac{\partial F}{\partial q} \\
                                            &= \frac{\partial}{\partial q} \frac{dq}{dt} \frac{\partial F}{\partial q} \\
                                            &= \frac{d}{dt} \frac{\partial F}{\partial q}
\end{align*}

Somit führen $L$ und $L'$ auf die selben Bewegungsgleichungen.

\subsection*{Teil b}
$L_1$ und $L_2$ führen auf die selben Bewegungsgleichungen, denn $L_2 - L_1 =
q^3\dot{q} = \frac{d}{dt} \frac{q^4}{4}$. Damit ergibt sich aus obigem Beweis
die Äquivalenz der Bewegungsgleichungen.

$L_2$ und $L_3$ führen auf die selben Bewegungsgleichungen denn eine
Lagrange-Funktion kann mit beliebigem $c \neq 0$ multipliziert werden und führt
auf die selben Bewegungsgleichungen. Beweis:

\begin{align*}
  &\left(\frac{d}{dt} \frac{\partial}{\partial \dot{q}} - \frac{\partial}{\partial q}\right) c L = 0 \\
  \Leftrightarrow & \left(\frac{d}{dt} \frac{\partial}{\partial \dot{q}} - \frac{\partial}{\partial q}\right) L = \frac{0}{c} = 0
\end{align*}

$L_4$ führt zu einer anderen Bewegungsgleichung als $L_1$, $L_2$ und $L_3$,
denn $q^3$ kann nicht als Zeitableitung eines $F(q(t))$ geschrieben werden,
ohne direkt von $t$ abzuhängen.

$L_5$ führt widerrum zu einer anderen Bewegungsgleichung, als alle
vorhergehenden, denn ohne die vorherige Bewegungsgleichung von $L_4$ explizit
auszurechnen, ist klar, dass aus $L_4$ bei der Anwendung des
Lagrange-Formalismus keine gemischten Terme der Form $cq^i\dot{q}^j$
($c{,}i{,}j \neq 0$) entstehen können, während dies bei $L_5$ und $L_6$ der
Fall sein muss.

$L_6$ ist die $\frac{3}{2}$te Potenz von $L_5$. Die Vermutung, dass dies auf
die selben Bewegungsgleichungen führt, überprüfe ich durch Ausrechnen:

Für $L_5$ ergibt sich:

\begin{align*}
  \frac{d}{dt} \frac{\partial}{\partial \dot{q}} q^2\dot{q}^4 &= \frac{\partial}{\partial q} q^2\dot{q}^4 \\
  \frac{d}{dt} (q^24\dot{q}^3) &= 2q\dot{q}^4 \\
  2\dot{q}q4\dot{q}^3 + q^24 \cdot 3 \ddot{q}\dot{q}^2 &= 2q\dot{q}^4 \\
  8q\dot{q}^4+ 12q^2\dot{q}^2\ddot{q} &= 2q\dot{q}^4 \\
  4q\dot{q}^4+ 6q^2\dot{q}^2\ddot{q} &= q\dot{q}^4 \\
  4\dot{q}^4+ 6q\dot{q}^2\ddot{q} &= \dot{q}^4 \\
  3\dot{q}^2+ 6q\ddot{q} &= 0 \\
  \dot{q}^2+ 2q\ddot{q} &= 0 \\
\end{align*}

Für $L_6$:

\begin{align*}
  \frac{d}{dt} \frac{\partial}{\partial \dot{q}} q^3\dot{q}^6 &= \frac{\partial}{\partial q} q^3\dot{q}^6 \\
  \frac{d}{dt} (q^36\dot{q}^5) &= 3q^2\dot{q}^6 \\
  6q^3 \cdot 5\dot{q}^4\ddot{q} + 6\dot{q}^5\cdot3q^2\dot{q} &= 3q^2\dot{q}^6 \\
  30q^3\dot{q}^4\ddot{q} + 18q^2\dot{q}^6 &= 3q^2\dot{q}^6 \\
  10q^3\dot{q}^4\ddot{q} + 6q^2\dot{q}^6 &= q^2\dot{q}^6 \\
  10q\dot{q}^4\ddot{q} + 6\dot{q}^6 &= \dot{q}^6 \\
  10q\ddot{q} + 6\dot{q}^2 &= \dot{q}^2 \\
  \dot{q}^2 + 2q\ddot{q} &= 0
\end{align*}

Für $L_5$ und $L_6$ erhält man also die selben Bewegungsgleichungen.

Zusammenfassend: $L_1 \widehat{=} L_2 \widehat{=} L_3 \widehat{\neq} L_4 \widehat{\neq} L_5 \widehat{=} L_6$.

\section*{Aufgabe 2}
\subsection*{Teil a}

\begin{equation*}
  \frac{\partial \dot{q}_i}{\partial \dot{s}_j} =
  \frac{\partial}{\partial \dot{s}_j} \frac{dq_i}{dt} =
  \frac{\partial}{\partial \dot{s}_j} \left(\frac{\partial q_i}{\partial t} + \sum_{k=1}^n \frac{\partial q_i}{\partial s_k} \frac{ds_k}{dt}\right) =
  \frac{\partial}{\partial \dot{s}_j} \left(\frac{\partial q_i}{\partial t} + \sum_{k=1}^n \frac{\partial q_i}{\partial s_k} \dot{s_k}\right) =
  \frac{\partial q_i}{\partial s_j}
\end{equation*}

Der Übergang vom vorletzten auf den letzten Gleichungsterm erfolgt, weil durch
$\frac{\partial}{\partial \dot{s_j}}$ nur derjenige Term aus der Klammer
ausgewählt wird, der $\dot{s_j}$ enthält, die restlichen Terme verschwinden.

Somit sollte auch $\frac{\partial q_i}{\partial \dot{s_j}} = 0$ klar sein, denn
$q_i = q(s_1, \dots, s_N, t)$ hängt von keinem $\dot{s_j}$ ab, die partielle
Ableitung muss also verschwinden.

\subsection*{Teil b}

\begin{align*}
  \frac{\partial \tilde{L}}{\partial \dot{s_j}} &= \frac{dL}{d\dot{s}_j} \\
                                                &= \sum_{i=1}^N \frac{\partial L}{\partial q_i}       \frac{\partial q_i}{\partial \dot{s_j}} +
                                                   \sum_{i=1}^N \frac{\partial L}{\partial \dot{q}_i} \frac{\partial \dot{q_i}}{\partial \dot{s_j}} +
                                                   \frac{\partial L}{\partial t} \frac{\partial t}{\partial \dot{s_j}}
\end{align*}

Aus Teil a ist bekannt, dass $\frac{\partial q_i}{\partial \dot{s_j}} = 0$ und
$\frac{\partial \dot{q_i}}{\partial \dot{s_j}} = \frac{\partial q_i}{\partial
s_j}$. Dass $\frac{\partial t}{\partial \dot{s_j}} = 0$, ist klar, denn $t$ ist
keine Funktion von $\dot{s_j}$. Insgesamt also:


\begin{align*}
  \frac{\partial \tilde{L}}{\partial \dot{s}_j} = \sum_{i=1}^N \frac{\partial L}{\partial \dot{q}_i} \frac{\partial q_i}{\partial s_j}
\end{align*}

\subsection*{Teil c}

\begin{align*}
  \frac{d}{dt} \left(\frac{\partial \tilde{L}}{\partial \dot{s_j}}\right) - \frac{\partial \tilde{L}}{\partial s_j} = 0
\end{align*}

Einsetzen von Teil b:

\begin{align*}
\frac{d}{dt} \left(
  \sum_{i=1}^N \frac{\partial L}{\partial \dot{q}_i} \frac{\partial q_i}{\partial s_j}
\right) - \frac{\partial \tilde{L}}{\partial s_j} = 0
\end{align*}

Betrachtung von $\frac{\partial \tilde{L}}{\partial s_j}$ liefert:

\begin{align*}
  \frac{\partial \tilde{L}}{\partial s_j} &= \frac{dL}{ds_j} = \sum_{i=1}^N \left(\frac{\partial L}{\partial q_i} \frac{\partial q_i}{\partial s_j}\right) + \frac{\partial L}{\partial t}\frac{\partial t}{\partial s_j} \\
  &= \sum_{i=1}^N \frac{\partial L}{\partial q_i} \frac{\partial q_i}{\partial s_j}
\end{align*}

Durch Einsetzen in die Anfangsgleichung erhält man:

\begin{align*}
\frac{d}{dt} \left(
  \sum_{i=1}^N \frac{\partial L}{\partial \dot{q}_i} \frac{\partial q_i}{\partial s_j}
\right) - \left(
  \sum_{i=1}^N \frac{\partial L}{\partial q_i} \frac{\partial q_i}{\partial s_j}
\right) = 0
\end{align*}

Wenn $L$ eine Lagrange-Funktion ist, gilt

\begin{align*}
  &\frac{d}{dt}\frac{\partial L}{\partial \dot{q}_i} - \frac{\partial L}{\partial q_i} = 0 \\
  \Leftrightarrow &\frac{d}{dt}\frac{\partial L}{\partial \dot{q}_i} = \frac{\partial L}{\partial q_i}
\end{align*}

Wenn man das $\frac{d}{dt}$ in die Summe zieht und den linken Teil einsetzt, liefert das:

\begin{align*}
&\left(
  \sum_{i=1}^N \frac{\partial L}{\partial q_i} \frac{\partial q_i}{\partial s_j}
\right) - \left(
  \sum_{i=1}^N \frac{\partial L}{\partial q_i} \frac{\partial q_i}{\partial s_j}
\right) = 0 \\
\Leftrightarrow &0 = 0
\end{align*}

Somit ist $\tilde{L}$ eine Lagrange-Funktion, wenn $L$ eine Lagrange-Funktion
ist. Die Lagrange-Funktion ist also unter der gegebenen Transformation
invariant.

\section*{Aufgabe 3}
\subsection*{Teil a}

Die Zwangsbedingung ist in der Aufgabe als $z = \alpha (x^2+y^2)$ gegeben. Da
das Problem eine Rotationssymmetrie aufweist, würde ich Zylinderkoordinaten
verwenden. Dies führt auf die neue Zwangsbedingung $z = \alpha r^2$. Meine
unabhängigen Koordinaten wären dann $r$ und $\phi$.

\subsection*{Teil b}
Als Koordinatensystem würde ich ein kartesisches Koordinatensystem wählen.
Dabei ist $\vec{a} = (x_a, y_a, z_a)$ mein Aufhängepunkt, für den die
Zwangsbedingungen $z_a = 0$, $y_a = 0$ und $x_a = A \cdot \sin (\omega t +
\alpha)$ gelten.
Die Pendelmasse befindet sich an $\vec{b} = (x_b, y_b, z_b)$. Offensichtlich
ist $z_b = 0$. Aufgrund des festen Abstandes zum Punkt $\vec{a}$ (nennen wir
ihn $r$) gilt außerdem

\begin{align*}
  r^2 &= |\vec{b} - \vec{a}|^2 \\
    &= (x_b-x_a)^2 + (y_b)^2
\end{align*}

Als unabhängige Koordinate bleibt also $y_b$ übrig.


\subsection*{Teil c}
Da die Beschleunigung und der Anstiegswinkel konstant sind, erscheint es nicht
sinnvoll, das Koordinatensystem irgendwie zu transformieren. Ich würde ein
2-dimensionales kartesisches Koordinatensystem $(x, y)$ benutzen.

Die Zwangsbedingung lautet $y = (x + \frac{1}{2}at^2) \cdot \cos \alpha$, als
freie Koordinate bleibt $x$ übrig.

\section*{Aufgabe 4}

\subsection*{Bewegungsgleichungen unter Benutzung der Newtonschen Mechanik}

Entlang des Drahtes wirkt zum Einen die Zentripetalkraft:

\begin{align*}
  F_z = m \omega^2 r = m \dot{phi}^2 r
\end{align*}

Außerdem der tangentiale Anteil der Gewichtskraft:

\begin{align*}
  F_a = F_g \cdot \tan \alpha
\end{align*}

Dabei ist $\alpha$ der Steigungswinkel des Drahtes. Da wir die Funktion kennen,
die die Form des Drahtes beschreibt, gilt

\begin{align*}
  \tan \alpha = \frac{d}{dr} ar^4 = 4ar^3
\end{align*}

Wir erhalten:

\begin{align*}
  F_a = 4mgar^3
\end{align*}

Die effektive Kraft $F$ ergibt sich also aus der Summe:

\begin{align*}
  F = F_a + F_z
\end{align*}

Aus dem zweiten Newtonschen Gesetz ergibt sich dann:

\begin{align*}
  ma &= 4mgar^3 + m \omega^2 r \\
  m\ddot{r} &= 4mgar^3 + m \omega^2 r \\
  \ddot{r} &= 4gar^3 + \omega^2 r
\end{align*}

\subsection*{Bewegungsgleichungen unter Benutzung der Lagrange-Mechanik}

Zur Bearbeitung des Problems bieten sich Zylinderkoordinaten an, wobei der
Nullpunkt des parabelförmigen Drahtes im Ursprung liegt.

Die Zwangsbedingungen sind dann dementsprechend $z=ar^4$ (wie gegeben) und
$\dot{\phi} = \omega$.

Durch das Gravitationspotential ergibt sich

\begin{align*}
  V = mgz = mgar^4
\end{align*}

Und für die kinetische Energie

\begin{align*}
  T &= \frac{1}{2}mv^2 \\
    &= \frac{1}{2}m(\dot{r}^2 + r^2\dot{\phi}^2 + \dot{z}^2) \\
    &= \frac{1}{2}m(\dot{r}^2 + r^2\omega^2 + 4ar^3\dot{r})
\end{align*}

Damit ist die Lagrangefunktion insgesamt:

\begin{align*}
  L = T - V = \frac{1}{2}m(\dot{r}^2 + r^2\omega^2 + 4ar^3\dot{r}) - mgar^4
\end{align*}

Ableitungen:
\begin{align*}
  \frac{d}{dt} \frac{\partial L}{\partial \dot{r}} = \frac{d}{dt} m\dot{r} + 2ar^3 = m\ddot{r} + 6ar^2\dot{r} \\
  \frac{\partial L}{\partial r} = mr\omega^2+6ar^2\dot{r} - 4mgar^3
\end{align*}

Unter Benutzung der Lagrangegleichung ergibt sich dann ($m \neq 0$):

\begin{align*}
  m\ddot{r} + 6ar^2\dot{r} &=  mr\omega^2 + 6ar^2\dot{r} - 4mgar^3 \\
  m\ddot{r} &= mr\omega^2 - 4mgar^3 \\
  \ddot{r} &= r\omega^2 - 4gar^3
\end{align*}

Damit sind die Bewegungsgleichungen insgesamt:
\begin{align*}
  \ddot{r} &= r\omega^2 - 4gar^3 \\
  \dot{phi} &= \omega \\
  z &= ar^4
\end{align*}

\end{document}
