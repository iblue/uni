\documentclass[a4paper,german,12pt,smallheadings]{scrartcl}
\usepackage[T1]{fontenc}
\usepackage[utf8]{inputenc}
\usepackage{babel}
\usepackage{tikz}
\usepackage{geometry}
\usepackage{amsmath}
\usepackage{amssymb}
\usepackage{float}
%\usepackage{wrapfig}
\usepackage[thinspace,thinqspace,squaren,textstyle]{SIunits}
\restylefloat{table}
\geometry{a4paper, top=15mm, left=20mm, right=40mm, bottom=20mm, headsep=10mm, footskip=12mm}
\linespread{1.5}
\setlength\parindent{0pt}
\begin{document}
\begin{center}
\bfseries % Fettdruck einschalten
\sffamily % Serifenlose Schrift
\vspace{-40pt}
Analysis I, Sommersemester 2013, 2. Übungsblatt

Luis Herrmann und Markus Fenske, Tutor: Adam Schienle
\vspace{-10pt}
\end{center}

\section*{Aufgabe 2.1}
\subsection*{Teil a}
Wenn $f, g$ beschränkt sind, gilt $|f(x)| \le K_1 \forall x$ und $|g(x)| \le K_2
\forall x$. Addieren der beiden Ungleichungen:

\begin{align*}
  |f(x)| + |g(x)| \le K_1 + K_2 \forall x \\
\end{align*}
Wegen der Dreieckesungleichung ($|a + b| \le |a| + |b|$) gilt dann:
\begin{align*}
  |f(x) + g(x)| \le K_1 + K_2 \forall x \\
\end{align*}

Somit ist $f + g$ beschränkt.
\subsection*{Teil b}
Betrachte $(f+g)(x) = f(x)+g(x)$:

$(f+g)(x)$ bildet ein beliebiges $x \in I$ ab wie folgt:

\begin{equation*}
  (f+g)(x) = f(x) + g(x)
\end{equation*}

Annahme: Da $\sup f \ge f(x) \forall x \in I$:

\begin{align*}
  f(x) \le \sup{x \in I} f
\end{align*}

Da $\sup g \ge g(x) \forall x \in I$:

\begin{align*}
  g(x) \le \sup_{x \in I}\ g
\end{align*}

Dann trifft auch die Aussage zu:

\begin{equation*}
  f(x) + g(x) \le \sup{x \in I} f(x) + \sup_{x \in I} g(x) \forall x \in I
\end{equation*}

Da das Intervall I geschlossen ist, ist das Supremum von $(f+g)$ selbst Element der Bildmenge:

\begin{align*}
  \sup_{x \in I} (f+g) \in (f+g)[I]
\end{align*}

Das heißt:

\begin{align*}
  \exists x \in I: f(x) + g(x) = \sup_{x \in I} (f+g)
\end{align*}

Und mit $f(x)+g(x) \le \sup_{x \in I} f(x) + \sup_{x \in I} g(x) \forall x \in I$ folgt:

\begin{align*}
  \sup_{x \in I} (f+g)(x) \le \sup_{x \in I} f(x) + \sup_{x \in I} g(x)
\end{align*}

Was zu zeigen war.

\subsection*{Teil c}
Sei $f: x \mapsto x^2$ und $g: x \mapsto x$:
\begin{align*}
  \sup_{x \in I} f(x) &= f(1) = 1 \\
  \sup_{x \in I} g(x) &= g(1) = 1 \\
  \sup_{x \in I} f(x) + g(x) &= 2
\end{align*}

\subsection*{Teil d}
Sei $f: x \mapsto -x$ und $g: x \mapsto x$:

\begin{align*}
  \sup_{x \in I} f(x) &= f(0) = 0 \\
  \sup_{x \in I} g(x) &= g(1) = 1 \\
  \sup_{x \in I} f(x) + g(x) &= 0
\end{align*}

\section*{Aufgabe 2.2}

Bekannt ist die allgemeine binomische Formel:

\begin{equation*}
  (a+b)^n = \sum_{k=0}^n {n \choose k} a^{n-k} b^k
\end{equation*}

Mit $a=b=1$:
\begin{align*}
 \sum_{k=0}^n {n \choose k} 1^{n-k} 1^k &= (1+1)^n \\
  \sum_{k=0}^n {n \choose k} &= 2^n
\end{align*}

Mit $a=1, b=-1$:
\begin{align*}
  \sum_{k=0}^n {n \choose k} 1^{n-k} (-1)^k &= (1-1)^n \\
  \sum_{k=0}^n {n \choose k} (-1)^k &= 0
\end{align*}


\section*{Aufgabe 2.3}
\subsection*{Teil a}
Wenn $3 \le k \le n$, zu zeigende Relation:
\begin{align*}
  {k \choose p} n^{k-p} \le n^k \forall p \ge 1
\end{align*}

Wir können diese auch ausdrücken als:

\begin{align*}
  \Leftrightarrow {k \choose p} \le \frac{n^k}{n^{k-p}}
\end{align*}

...solange $n^{k-p} > 0$, was erfüllt ist, da $n > 0$.

\begin{align*}
  \Leftrightarrow {k \choose p} &\le n^p \\
  \Leftrightarrow \frac{k!}{p!(k-p)!} &\le n^p
\end{align*}

$k \choose p$ wird maximal für $p=\frac{k}{2}$ bei geraden $p$, bzw. für $p=\frac{k+1}{2}$ und $p=\frac{k-1}{2}$ bei ungeradem $p$.

Betrachte gerades $p$. Zu zeigen:

\begin{align*}
  \frac{k!}{\left(\frac{k}{2}\right)! \left(k-\frac{k}{2}\right)!} &\le n^\frac{k}{2} \\
  \Leftrightarrow \frac{k!}{\left(\frac{k}{2}\right)! \left(\frac{k}{2}\right)!} &\le n^\frac{k}{2} \\
  \Leftrightarrow \frac{k \cdot (k-1) \cdot \dots \cdot \left(\frac{k}{2}+1\right)}{ \left(\frac{k}{2}\right) \cdot \left(\frac{k}{2}-1\right) \cdot \dots \cdot 1} &\le n^\frac{k}{2} \\
  \Leftrightarrow \frac{k}{\frac{k}{2}} \cdot \frac{k-1}{\frac{k}{2} - 1} \cdot \dots \cdot \left(\frac{k}{2} + 1\right) &\le n^\frac{k}{2} \\
\end{align*}
Auf der linken Seite existieren nun $\frac{k}{2}$ Faktoren. Aber auch auf der rechten Seite der Ungleichung haben wir $\frac{k}{2}$ Faktoren ($n$).

Wenn es uns nun gelingt zu zeigen:

\begin{align*}
  \frac{k}{\frac{k}{2}}, \frac{k-1}{\frac{k}{2} - 1}, \dots, \frac{k+1}{2} \le n
\end{align*}

\dots ist unsere Relation bewiesen. Das ist einfach, denn

\begin{align*}
  n \ge k > (k-1) > \left(\frac{k}{2} + 1\right) \forall k \ge 3
\end{align*}

Und für die Terme in den Nennern gilt:

\begin{align*}
  \frac{k}{2} > \left(\frac{k}{2} - 1\right) > \dots > 1
\end{align*}

Und somit gilt erst recht:

\begin{align*}
  \frac{k}{\frac{k}{2}}, \frac{k-1}{\frac{k}{2} - 1}, \dots, \frac{k+1}{2} \le n
\end{align*}

Analog für ungerade $p$ zu zeigen:

\begin{align*}
  \frac{k!}{\left(\frac{k-1}{2}\right)! \cdot \left(k - \frac{k-1}{2}\right)!} &\le n^{\frac{k}{2} - 1} \\
  \Leftrightarrow \frac{k!}{\left(\frac{k-1}{2}\right)! \cdot \left(\frac{k+1}{2}\right)!} &\le n^{\frac{k}{2} - 1} \\
  \Leftrightarrow \frac{k \cdot (k-1) \cdot \dots \cdot \left(\frac{k+1}{2} + 1\right)!}{\frac{k-1}{2} \cdot \left(\frac{k-1}{2} -1 \right) \cdot \dots \cdot 1} &\le n^{\frac{k}{2} - 1} \\
  \Leftrightarrow \frac{k}{\frac{k-1}{2} \cdot \dots \cdot \frac{\frac{k+1}{2}+1}{1}} &\le n^{\frac{k}{2} - 1}
\end{align*}


Zeige:

\begin{align*}
  \frac{k}{\frac{k-1}{2}}, \frac{k-1}{\frac{k-1}{2} -1}, \dots, \frac{\frac{k+1}{2}+1}{1} \le n
\end{align*}

Dass

\begin{align*}
  \frac{k+1}{2} + 1 < \dots < k -1 < k \le n
\end{align*}

ist klar. Und auch hier gilt für die Nenner:

\begin{align*}
  \frac{k-1}{2} > \frac{k-1}{2} -1 > \dots \ge 1
\end{align*}

da $k \ge 3$. Und somit erhalten wir, dass die Relation stimmt.

\subsection*{Teil b}

Zu zeigen:

\begin{align*}
  (n+1)^k \le n^k \cdot k
\end{align*}

Bereits gezeigt:

\begin{align*}
  {k \choose p} \cdot n^{k-p} \le n^k
\end{align*}

Drücke $(n+1)^k$ durch den Binomialkoeffizienten aus, bzw. durch den Binomischen Satz:

\begin{align*}
  (n+1)^k = \sum_{p=0}^k {k \choose p} n^{k-p} 1^p
\end{align*}

Da $1^p = 1 \forall p$ können wir schreiben:

\begin{align*}
  (n+1)^k = \sum_{p=0}^k {k \choose p} n^{k-p}
\end{align*}

Und mit der Relation aus Teil a gilt für jeden einzelnen der $k+1$ Summanden:

\begin{align*}
  {k \choose p} n^{k-p} \le n^k
\end{align*}

Und damit:

\begin{align*}
  \frac{\sum_{p=0}^k {k \choose p} n^{k-p}}{k+1} \le n^k
\end{align*}

Insbesondere gilt aber auch:

\begin{align*}
  {k \choose k} n^0 + {k \choose k-1}n^1 &\le n^k \\
  1 + kn \le n^k
\end{align*}

Offensichtlich trifft diese Aussage für $3 \le k \le n$ zu.

Das heißt, wir können unseren Ausdruck auch umschreiben zu:

\begin{align*}
  \frac{\sum_{p=0}^k {k \choose p} n^{k-p}}{k} &\le n^k \\
  \Leftrightarrow \sum_{p=0}^k {k \choose p} n^{k-p} &\le n^k \cdot k\\
  \Leftrightarrow (n+1)^k &\le u^k \cdot k
\end{align*}

\subsection*{Teil c}

Zeige $n^k \le k^n$ per Induktion. Betrachte $k$ als fixiert.

\subsubsection*{Induktionsanfang}

Für $n=1$:

\begin{align*}
  1^k \le k^1 \Leftrightarrow 1 \le k
\end{align*}

Die Aussage trifft zu, da $k$ per Definition natürliche Zahl ist, somit $k > 0$

\subsubsection*{Induktionsvorraussetzung}

\begin{align*}
  n^k \le k^n
\end{align*}

\subsubsection*{Induktionsschritt}

\begin{align*}
  (n+1)^k &\le k^{n+1} \\
  (n+1)^k &\le k^n \cdot k \\
\end{align*}

Nach Induktionsvorraussetzung gilt:

\begin{align*}
  n^k \le k^n \Leftrightarrow n^k \cdot k \le k^n \cdot k \\
  \Rightarrow (n+1)^k \le n^k \cdot k \le k^n \cdot k
\end{align*}

Der vordere Teil ($(n+1)^k \le n^k \cdot k$) trifft zu, wie wir in in Teil b gezeigt haben. Damit ist der Induktionsbeweis erbracht.
\subsection*{Teil d}

Die Tatsache, dass $k \ge 3$ wurde in Teil a, b und c benutzt. Der Induktionsbeweis in Teil c funktioniert nur für $k \ge 3$, da die Induktionsvoraussetzung an sich nur für $k \ge 3$ gilt. Betrachte $k=2$ für unterschiedliche $n$:

\begin{align*}
  n=1&: 1 \le 2 \operatorname{w.A!}\\
  n=2&: 4 \le 4 \operatorname{w.A!}\\
  n=3&: 9 \le 8 \operatorname{f.}\\
  n=4&: 16 \le 16 \operatorname{w.A!}\\
  n=5&: 25 \le 32 \operatorname{w.A!}\\
  n=6&: 36 \le 64 \operatorname{w.A!}
\end{align*}

\section*{Aufgabe 2.4}
Abschätzung: $n^\frac{n}{2} \le n!$

Denn es lässt sich stets ein $n > N$ finden, sodass $|N^{\frac{N}{2}} - N!| \le
|n^\frac{n}{2} - n!|$.

\begin{align*}
  \forall n > N: |N^\frac{N}{2} - N!| \le |n^\frac{n}{2} - n!| \qquad n,N \in \mathbb{N}
\end{align*}

So ist $1^\frac{1}{2} \le 1!$, $2^{\frac{2}{2}} \le 2!$, $3^{3}{2} \le 3!$, \dots usw., wobei

\begin{align*}
  (1! - 1^{\frac{1}{2}}) \le (2! - 2^{\frac{2}{2}}) \le (3! - 3^{\frac{3}{2}}) \le (4! - 4^{\frac{4}{2}}) \le \dots
\end{align*}

Offensichtlich divergiert $\left(n^\frac{n}{2}\right)_{n=0}^{\infty}$ bestimmt, denn

\begin{align*}
  \lim_{n \to \infty} n^\frac{n}{2} = \infty
\end{align*}

Was gilt jedoch für die Folge $\left( \sqrt[n]{n^\frac{n}{2}} \right)_{n=0}^{\infty}$?

\begin{align*}
  \sqrt[n]{n^{\frac{n}{2}}} = n^\frac{1}{2} = \sqrt{n}
\end{align*}

Jedoch gilt $\lim_{n \to \infty} \sqrt{n} = \infty$. Also divergiert auch 
 $\left( \sqrt[n]{n!} \right)_{n=0}^{\infty}$.
\end{document}
