\documentclass[a4paper,german,12pt,smallheadings]{scrartcl}
\usepackage[T1]{fontenc}
\usepackage[utf8]{inputenc}
\usepackage{babel}
\usepackage{tikz}
\usepackage{geometry}
\usepackage{amsmath}
\usepackage{amssymb}
\usepackage{float}
\usepackage[thinspace,thinqspace,squaren,textstyle]{SIunits}
\restylefloat{table}
\geometry{a4paper, top=15mm, left=20mm, right=40mm, bottom=20mm, headsep=10mm, footskip=12mm}
\linespread{1.5}
\setlength\parindent{0pt}
\begin{document}
\begin{center}
\bfseries % Fettdruck einschalten
\sffamily % Serifenlose Schrift
\vspace{-40pt}
Analytische Mechanik, Sommersemester 2013, 7. Blatt \\
Luis Herrmann und Markus Fenske, Tutor: Clemens Meyer zu Rheda
\vspace{-10pt}
\end{center}
\section*{Aufgabe 1: Rotierendes Bezugssystem}
\subsection*{Teil a}

Aus geometrischen Überlegungen wird sofort klar, dass es geeignete $x,y$ geben muss, sodass die Gleichung:
\begin{equation*}
\vec{r_ {rot}}=\vec{ON}+x\vec{NA}+y\vec{NC}
\end{equation*}

...erfüllt wird. Dies verdanken wir der Tatsache, dass $\vec{NA}$, $\vec{NC}$ und $\vec{NB}$ in einer Ebene liegen.\\
Sei $\vec{NB}$ der Verbindungsvektor zwischen den Punkten $N$ und $B$ und sei $B$ der Punkt, welchen der Vektor $\vec{r_{rot}}$ trifft. (siehe Skizze). Dann gilt:
\begin{align*}
\vec{r_{rot}}=\vec{ON}+\vec{NB}\\
\vec{ON}+x\vec{NA}+y\vec{NC}
\end{align*}

Welchen Wert müssen $x,y$ annehmen? Dazu überlegen wir uns zunächst, dass:
\begin{align*}
\vec{NC}=\vec{r}\times\ \hat{n}\\
\Rightarrow NC=|\vec{NC}|=|\vec{r}| \cdot |\hat{n}| \cdot |\sin \alpha|=r\cdot 1 \cdot\sin\alpha\\
\Leftrightarrow NC=r\sin\alpha
\end{align*}

Aus geometrischen Überlegungen folgt:
\begin{align*}
\sin\alpha=\frac{{NA}}{r}\\
\Leftrightarrow {NA}=r\sin\alpha
\end{align*}

Das kommt uns doch bekannt vor...
\begin{equation*}
|\vec{NC}|=NC=r\sin\alpha=NA=|\vec{NC}|
\end{equation*}

Da wir wissen, dass $\vec{NC}\perp\vec{NA}$, können wir uns jetzt mit dem Satz des Pythagoras ganz einfach überlegen, dass $x,y$ folgende Gleichung erfüllen müssen:

\begin{equation*}
NB^2=(x\cdot NA)^2+(y\cdot NC)^2
\end{equation*}

mit $NB=NA$ und $NC=NA$:
\begin{align*}
NA^2=(x\cdot NA)^2 + (y\cdot NA)^2\\
\Leftrightarrow NA^2=NA^2\left(x^2+y^2\right)\\
\Leftrightarrow 1=x^2+y^2
\end{align*}

Die Bedingung wird erfüllt, wenn wir
\begin{equation*}
x=\cos\phi \quad \text{und} \quad y=\sin\phi 
\end{equation*}
setzen. Dann erhalten wir nämlich:

\begin{equation*}
1=\cos\phi^2+\sin\phi^2
\end{equation*}

Offensichtlich trifft dies zu. Zusammenfassend erhalten wir somit:
\begin{align*}
\vec{r_{rot}}=\vec{ON}+cos\phi\vec{NA}+\sin\phi\vec{NC}\\
\vec{r_{rot}}=\left(\hat{n} \times \vec{r}\right)+\cos\phi\left(\vec{r}-\left(\vec{r}-\hat{n}\right) \cdot \hat{n}\right)+\sin\phi \; \hat{n} \times \vec{r}
\end{align*}
...wie gefordert.

\subsection*{Teil b}

Sei $\vec{AB}$ der Verbindungsvektor zwischen $\vec{r_{rot}}$ und $\vec{r}$, sodass:
\begin{equation*}
\vec{r_{rot}}-\vec{r}=\vec{AB}
\end{equation*}

Da wir nun kleine Winkel betrachten werden, schreiben wir suggestiv:
\begin{align*}
& \vec{AB}=\vec{dr}\\
& \Rightarrow \vec{r_{rot}}-\vec{r}=\vec{dr}\\
& \Leftrightarrow \vec{r_{rot}}=\vec{r}+\vec{dr}
\end{align*}

Für kleine Winkel trifft zu:
\begin{align*}
& R\cdot {d\phi}=dr \quad \text{, mit $R=NA$:}\\
& NA\cdot {d\phi}=dr
\end{align*}

Vektoriell:
\begin{align*}
& NA\cdot {d\phi} \; \hat{n_t}=dr \; \hat{n_t}\\
& (NA\cdot {d\phi}) \; \hat{n_t}=\vec{dr}
\end{align*}

Sei $\vec{n_t}$ der normierte Vektor in Richtung von $\vec{dr}$, für den wir bei kleinen Winkeln $d\phi$ approximiert annehmen können, dass $\vec{dr}\perp\vec{NC}$, sowie $\hat{n_t}\parallel\hat{n_c}$ und $\hat{n_t}=\hat{n_c}$;\\
sei $\hat{n_c}$ der normierte Vektor in $\vec{NC}$-Richtung.\\
Wir substituieren und erhalten:

\begin{align*}
& \vec{dr}=\left(NA \cdot d\phi\right) \; \hat{n_c} \quad \text{, wegen $NA=NC$:}\\
& \vec{dr}=\left(NC \cdot d\phi\right) \; \hat{n_c}\\
& \vec{dr}=d\phi \; \vec{NC}
\end{align*}

Eingesetzt in unsere Ausgangsgleichung kriegen wir:
\begin{align*}
& \vec{r_{rot}}=\vec{r}+d\phi \; \vec{NC}\\
& \Leftrightarrow \vec{r_{rot}}=\vec{r}+d\phi \; (\hat{n} \times \vec{r})
\end{align*}

...wie gefordert.

\subsection*{Teil c}

Zu zeigen:
\begin{align*}
& \vec{v_{rot}}=\vec{v_{ruh}}+\vec{\omega} \times \vec{r_{ruh}}+d\phi\hat{n} \times \vec{v_{ruh}}\\
& \vec{a_{rot}}=\vec{a_{rot}}+2\vec{\omega} \times \vec{v_{ruh}}+d\phi \hat{n} \times \vec{r_{ruh}}+\frac{\partial^2 \phi}{\partial t^2}\vec{n} \times \vec{r_{ruh}}
\end{align*}

Hierzu werden wir uns der Formel aus (b) bedienen (mit $\vec{r}\Rightarrow\vec{r_{ruh}}$). Als Vorwissen setzen wir voraus, dass wir im letzten Semester gezeigt haben, dass Kreuzprodukte wie gewöhnliche Produkte mit der Summen- und Produktregel abgeleitet werden können.\\

Sei $\vec{v_{rot}}=\frac{d}{dt} \vec{r_{rot}}$, dann erhalten wir:
\begin{align*}
& \vec{v_{rot}}= \frac{d}{dt} \vec{r_{rot}}\\
& \Leftrightarrow \vec{v_{rot}}=\frac{d}{dt} \left(\vec{r_{ruh}}+d\phi \; \left(\hat{n} \times \vec{r_{ruh}}\right)\right)\\
& \Leftrightarrow \vec{v_{rot}}=\frac{d}{dt} \vec{r_{ruh}}+\frac{d}{dt}d\phi \; \left(\hat{n} \times \vec{r_{ruh}}\right)+d\phi \; \frac{d}{dt}\left(\hat{n} \times \vec{r_{ruh}}\right)\\
\end{align*}

Wir verwenden, dass:
\begin{equation*}
\frac{d}{dt} \vec{r_{ruh}}=\vec{v_{ruh}} \quad \text{und} \quad \frac{d}{dt}d\phi=\frac{d\phi}{dt}=\omega
\end{equation*}

Einsetzen:
\begin{align*}
& \Leftrightarrow \vec{v_{rot}}=\vec{v_{ruh}}+\omega \; \left(\hat{n} \times \vec{r_{ruh}}\right)+d\phi\left({\frac{d}{dt}\hat{n}} \times \vec{r_{ruh}}+\hat{n} \times \underbrace{\frac{d}{dt}\vec{r_{ruh}}}_{\vec{v_{ruh}}}\right)
\end{align*}

Nun gilt:
\begin{align*}
& \frac{d}{dt}\hat{n}=0 \quad \text{, da } \quad n_i=const.\; \forall i\\
& \omega \; \left(\hat{n} \times \vec{r_{ruh}}\right)\overset{(1)}{=}\omega\hat{n} \times \vec{r_{ruh}}=\vec{\omega} \times \vec{r_{ruh}}
\end{align*}

...und wir erhalten:
\begin{equation*}
\vec{v_{rot}}=\vec{v_{ruh}}+\vec{\omega} \times \vec{r_{ruh}}+d\phi\left(\hat{n} \times \vec{v_{ruh}}\right)=\vec{v_{ruh}}+\vec{\omega} \times \vec{r_{ruh}}+d\phi\hat{n} \times \vec{v_{ruh}}
\end{equation*}


(1) Dieser Schritt ist legitim. Allgemeiner Beweis:
\begin{align*}
& c\left(\vec{a} \times \vec{b}\right)=c\left(\left(a_2b_3-a_3b_2\right)\hat{e_1}+\left(a_3b_1-a_1b_3\right)\hat{e_2}+\left(a_1b_2-a_2b_1\right)\hat{e_3}\right)\\
& \Leftrightarrow c\left(\vec{a} \times \vec{b}\right)=\left(c \left(a_2b_3-a_3b_2\right)\hat{e_1}+c \left (a_3b_1-a_1b_3\right)\hat{e_2}+c \left(a_1b_2-a_2b_1\right)\hat{e_3}\right)\\
& \Leftrightarrow c\left(\vec{a} \times \vec{b}\right)=\left(\left(ca_2b_3-ca_3b_2\right)\hat{e_1}+\left (ca_3b_1-ca_1b_3\right)\hat{e_2}+\left(ca_1b_2-ca_2b_1\right)\hat{e_3}\right)\\
& \Leftrightarrow c\left(\vec{a} \times \vec{b}\right)=c\vec{a}\times \vec{b}
\end{align*}
\\

Zweite Ableitung:
\begin{align*}
& \vec{a_{rot}}=\frac{d}{dt}\vec{v_{rot}}=\frac{d}{dt} \left(\vec{v_{ruh}}+\vec{\omega} \times \vec{r_{ruh}}+d\phi\hat{n} \times \vec{v_{ruh}}\right)\\
& \Leftrightarrow \vec{a_{rot}}=\underbrace{\frac{d}{dt}{v_{ruh}}}_{\vec{a_{ruh}}}+\frac{d}{dt}\left(\vec{\omega} \times \vec{r_{ruh}}\right)+\frac{d}{dt}\left(d\phi\hat{n} \times \vec{v_{ruh}}\right)\\
& \Leftrightarrow \vec{a_{rot}}=\vec{a_{ruh}}+\underbrace{\frac{d}{dt}\left(\frac{d\phi}{dt}\left(\hat{n}\times\vec{r_{ruh}}\right)\right)}_{(a)}+\underbrace{\frac{d}{dt}\left(d\phi\left(\hat{n}\times\vec{v_{ruh}}\right)\right)}_{(b)}
\end{align*}

(a):
\begin{align*}
& \frac{d}{dt}\left(\frac{d\phi}{dt}\left(\hat{n}\times\vec{r_{ruh}}\right)\right)=\frac{d^2\phi}{dt^2}\left(\hat{n}\times\vec{r_{ruh}}\right)+\frac{d\phi}{dt}\left(\left(\underbrace{\frac{d}{dt}\hat{n}}_{=0}\times\vec{r_{ruh}}\right)+\hat{n}\times\underbrace{\frac{d}{dt}\vec{r_{ruh}}}_{\vec{v_{ruh}}}\right)\\
& \frac{d}{dt}\left(\frac{d\phi}{dt}\left(\hat{n}\times\vec{r_{ruh}}\right)\right)=\frac{d^2\phi}{dt^2}\left(\hat{n}\times\vec{r_{ruh}}\right)+\frac{d\phi}{dt}\left(\hat{n}\times\vec{v_{ruh}}\right)=\frac{d^2\phi}{dt^2}\left(\hat{n}\times\vec{r_{ruh}}\right)+\vec{\omega}\times\vec{v_{ruh}}
\end{align*}
\\

(b):
\begin{align*}
& \frac{d}{dt}\left(d\phi\left(\hat{n}\times\vec{v_{ruh}}\right)\right)=\frac{d\phi}{dt}\left(\hat{n}\times\vec{v_{ruh}}\right)+d\phi\left(\underbrace{\frac{d}{dt}\hat{n}}_{=0}\times\vec{v_{ruh}}+\hat{n}\times\underbrace{\frac{d}{dt}\vec{v_{ruh}}}_{\vec{a_{ruh}}}\right)\\
& \Leftrightarrow \frac{d}{dt}\left(d\phi\left(\hat{n}\times\vec{v_{ruh}}\right)\right)=\vec{\omega}\times\vec{v_{ruh}}+d\phi\left(\hat{n}\times\vec{a_{ruh}}\right)
\end{align*}

Einsetzen in die Ausgangsgleichung liefert:
\begin{equation*}
\vec{a_{rot}=\vec{a_{ruh}}}+2\vec{\omega}\times\vec{v_{ruh}}+d\phi\left(\hat{n}\times\vec{a_{ruh}}\right)+\frac{d^2\phi}{dt^2}\left(\hat{n}\times\vec{r_{ruh}}\right)
\end{equation*}

\subsection*{Teil d}

\begin{equation*}
\vec{F_{rot}}=m\vec{a_{rot}}=\underbrace{m\vec{a_{ruh}}}_{a}+\underbrace{2m\vec{\omega}\times\vec{v_{ruh}}}_{b}+\underbrace{md\phi\left(\hat{n}\times\vec{a_{ruh}}\right)}_{c}+\underbrace{m\frac{d^2\phi}{dt^2}\left(\hat{n}\times\vec{r_{ruh}}\right)}_{d}
\end{equation*}

Betrachte die Terme einzeln:

a:  Die Kraft, die auf das Teilchen innerhalb des rotierenden Bezugssystem wirkt.
\begin{align*}
& \vec{F_{ruh}}=m\vec{a_{ruh}}\quad \text{Betrag in Richtung $\vec{r_{ruh}}$:}\\
& F_{ruh}=ma_{ruh}
\end{align*}

b: Coreoliskraft
\begin{align*}
& \vec{F_{cor}}=2m\vec{\omega}\times\vec{v_{ruh}}\quad \text{Betrag:}\\
& F_{cor}=2mv_{ruh}\omega\sin\beta
\end{align*}

sei $\beta$ der Winkel zwischen $\vec{n}\parallel\vec{\omega}$ und $\vec{r_{ruh}}\parallel\vec{v_{ruh}}$.\\

c: 
\begin{align*}
& \vec{F_d}=md\phi\left(\hat{n}\times\vec{a_{ruh}}\right) \quad \text{Betrag:}\\
& F_d=md\phi a_{ruh}\sin\beta
\end{align*}
Hierzu fällt uns keine physikalische Interpretation ein.\\

d: Die Kraft, die aus dem Drehmoment des Teilchens hervorgeht.
\begin{align*}
& \vec{F_J}=m\frac{d^2\phi}{dt^2}\left(\hat{n}\times\vec{r_{ruh}}\right)\quad \text{$\frac{d\phi}{dt}$ ist die Winkelgeschwindigkeit $\alpha$}\\
& F_J=m\alpha r_{ruh}\sin\beta
\end{align*}

Trägheitsmoment $J$ einer Punktmasse:
\begin{equation*}
J=mr^2 \quad \Rightarrow \quad J=m\left(r_{ruh}\sin\beta\right)^2\\
\end{equation*}

Eingesetzt:
\begin{align*}
& F_J=\frac{J}{r_{ruh}\sin\beta}\alpha \quad \text{, mit $M=J\alpha$:}\\
& F_J=\frac{M}{r_{ruh}\sin\beta}\\
\end{align*}


\subsection*{Teil e}

Wir wissen:
\begin{equation*}
F_{rot}=-grad \; V_{rot}
\end{equation*}

Da $V_{rot}=V_{rot}(r)$:
\begin{align*}
F_{rot}=-\frac{\partial V_{rot}}{\partial r_{ruh}}
\end{align*}

Auflösen nach $V_{rot}$ durch Trennung der Variablen und Integration (nur bei expliziter Abhängigkeit):
\begin{align*}
& -\partial V_{rot}=\vec{F_{rot}}\partial r_{ruh}\\
& \Leftrightarrow -\int\limits_{0}^{V_{rot}(r_{ruh})}\partial V_{rot}=\int\limits_{0}^{r_{ruh}}\vec{F_{rot}}\partial r_{ruh}\\
& \Leftrightarrow -V_{rot}(r)=\int\limits_{0}^{r_{ruh}}F_{ruh}\partial r_{ruh}+\int\limits_{0}^{r_{ruh}}|2m\vec{w}\times\vec{v_{ruh}}|\;\partial r_{ruh}+\int\limits_{0}^{r_{ruh}}|d\phi\left(\hat{n}\times\vec{a_{ruh}}\right)|\partial r_{ruh}+\int\limits_{0}^{r_{ruh}}|\frac{d^2\phi}{dt^2}\hat{n}\times\vec{r_{ruh}}|\;\partial r_{ruh}\\
& \Leftrightarrow -V_{rot}=-V_{ruh}+|2m\left(\vec{w}\times\vec{v_{ruh}}\right)|r_{ruh}+|d\phi\left(\hat{n}\times\vec{a_{ruh}}\right)|r_{ruh}+|\frac{d^2\phi}{dt^2}\hat{n}\times\hat{r_{ruh}}\frac{r_{ruh}^2}{2}|\\
\end{align*}

\section*{Aufgabe 3: Keplersche Gesetze für harmonisches Potential}
\subsection*{Teil a: Aufstellen und Lösen der Bewegungsgleichungen}
Wir stellen die Lagrange-Funktion in 2-dimensionalen kartesischen Koordinaten
auf:

\begin{align*}
  \mathcal{L} = \frac{m}{2}(\dot{x}^2 + \dot{y}^2) - \frac{k}{2}(x^2 + y^2)
\end{align*}

Durch Anwendung der Euler-Lagrange-Gleichung erhalten wir direkt die
Bewegungsgleichungen:

\begin{align*}
  m\ddot{x} + kx &= 0\\
  m\ddot{y} + ky &= 0
\end{align*}

Mit $\omega_{x} = \frac{k}{m}$ erhalten wir die üblichen Lösungen für den
harmonischen Oszillator:

\begin{align*}
  x(t) = A_x \sin(\omega_0 t + \phi_x) \\
  y(t) = A_y \sin(\omega_0 t + \phi_y)
\end{align*}

Dies kann eine Ellipse sein, wenn $\phi_x = \phi_y + \frac{\pi}{2} + n\pi$.
Denn dann kann der Sinus in der zweiten Gleichung als Kosinus geschrieben
werden und die Ellipsengleichung ist wegen $A_x, A_y > 0$ erfüllt. Für $A_x =
A_y$ kann sich sogar ein Kreis ergeben. Es können aber auch andere Bahnkurven
herauskommen. Beispielsweise die harmonische Bewegung auf einer geraden Linie,
wenn $\phi_x = \phi_y$ oder wenn $A_x = 0$ oder $A_y = 0$. Zusammengefasst
können sich alle Bewegungen Ergeben, die einer Lissajous-Figur mit
Frequenzverhältnis 1:1 entsprechen.

\subsection*{Teil b: Flächensatz}

Der Flächensatz ist eine Folge der Drehimpulserhaltung, die immer gilt, wenn
das Potential konservativ ist. Da das Potential nur von $r$ abhängt, ist es
konservativ, also gilt Drehimpulserhaltung und damit der Flächensatz.

\subsection*{Teil c: Bahngröße und Umlaufzeit}

Nicht bearbeitet.

\section*{Aufgabe 4: Gravitation in höherdimensionalen reellen Räumen}

Die Wahl hypersphährischer Koordinaten ist hier unangemessen und führt zu
abartigem Rechenaufwand. Grundgedanke der Aufgabe ist es, zu zeigen, dass die
Gravitation für $d > 3$ keine stabilen Bahnen mehr aufweist, unter der Annahme,
dass $V(r) = -kr^{2-d}$.

Wir sind der Meinung, dass sich $4$-dimensionale Zylinderkoordinaten besser
eignen. Da der Lagrange-Formalismus von der Koordinatenwahl unabhängig ist,
führt dies zum selben Ergebnis. Wenn es dafür einen kleinen Punktabzug gibt,
soll uns das Recht sein.

\subsection*{Teil a: Aufstellen der Lagrange-Funktion}

Die Zylinderkoordinaten in 4 Dimensionen lauten:

\begin{align*}
  x &= r \cos \phi \\
  y &= r \sin \phi \\
  z &= z \\
  u &= u
\end{align*}

Die Zeitableitungen:

\begin{align*}
  \dot{x} &= -r \dot{\phi} \sin \phi + \dot{r} \cos \phi\\
  \dot{y} &= r \dot{\phi} \cos \phi + \dot{r} \sin \phi\\
  \dot{z} &= \dot{z} \\
  \dot{u} &= \dot{u}
\end{align*}

Die Lagrange-Funktion ist dann:

\begin{align*}
  \mathcal{L} = \frac{m}{2}(r^2\dot{\phi}^2 + \dot{r}^2 + \dot{z}^2 + \dot{u}^2) - V(r)
\end{align*}

\subsection*{Teil b: Erhaltungsgrößen}

Hier sieht man sofort, dass $\phi$ zyklische Koordinate ist. Die entsprechende Erhaltungsgröße $\partial_{\dot{\phi}} L$ (Drehimpuls) ist dann:

\begin{align*}
  \frac{\partial L}{\partial \dot{\phi}} = mr^2 \dot{\phi} = L = \text{const.}
\end{align*}

Außerdem sind $z$ und $u$ zyklisch, was jeweils auf die Impulserhaltungssätze führt:

\begin{align*}
  p_z &= m\dot{z} = \text{const.} \\
  p_u &= m\dot{u} = \text{const.}
\end{align*}

Damit lässt sich die Lagrange-Funktion schreiben als:

\begin{align*}
  \mathcal{L} = \frac{1}{2}(L\dot{\phi} + m\dot{r^2} + p_z\dot{z} + p_u \dot{u}) - V(r)
\end{align*}

\subsection*{Teil c: Gesamtenergie}

Wegen $\mathcal{L} = T - V$ und $E_{\text{ges}} = T + V$ ist die Gesamtenergie
einfach die Lagrange-Funktion, aber mit umgekehrtem Vorzeichen des Potentials.

\begin{align*}
  E_{\text{ges}} = \frac{1}{2}(L\dot{\phi} + m\dot{r^2} + p_z\dot{z} + p_u \dot{u}) + V(r)
\end{align*}

\subsection*{Teil d: Stabile Minima des effektiven Potentials}
Das effektive Potential ergibt sich als Summe der Energiebeiträge aus Potential
und Erhaltungsgrößen. Dementsprechend ist

\begin{align*}
  V_{\text{eff}} = \frac{1}{2}(L\dot{\phi} + p_z\dot{z} + p_u \dot{u}) + V(r)
\end{align*}

Durch Einsetzen des angenommenen Gravitationspotentials $V(r) = -kr^{2-d}$
erhalten wir:

\begin{align*}
  &V_{\text{eff}} = \frac{1}{2}(L\dot{\phi} + p_z\dot{z} + p_u \dot{u}) - \frac{k}{r^2} \\
  \Leftrightarrow\quad&V_{\text{eff}} = \frac{m}{2}(r^2\dot{\phi}^2 + \dot{z}^2 + \dot{u}^2) - \frac{k}{r^2}
\end{align*}

Wir bestimmen nun die Minimalstellen dieser Funktion in Abhängigkeit von $r$
(unter der Annahme $r \neq 0$).

\begin{align*}
  &\frac{d}{dr} V_{\text{eff}} = mr\dot{\phi}^2 + \frac{2k}{r^3} \overset{!}{=} 0 \\
  \Leftrightarrow\quad&\frac{mr^4\dot{\phi}^2}{r^3} + \frac{2k}{r^3}=0\\
  \Leftrightarrow\quad&\frac{mr^4\dot{\phi}^2 + 2k}{r^3}=0
\end{align*}

Dies wird offensichtlich nur dann null, wenn der Zähler verschwindet. Also

\begin{align*}
  &mr^4\dot{\phi}^2 = -2k \\
  \Leftrightarrow\quad &r^4 = \frac{-2k}{m\dot{\phi}^2} \\
\end{align*}

Dies hat keine Lösungen im reelen, also gibt es keine Extremwerte, somit keine stabilen Bahnen.

\subsection*{Teil e: Stabile Minima des effektiven Potentials für $d=2$ und $d=3$}

Für $d = 3$ erhalten wir das effektive Potential, indem wir $u = 0$ setzen und
das Potential der Dimensionszahl anpassen. Ansonsten wie oben.

\begin{align*}
  V_{\text{eff}} = \frac{m}{2}(r^2\dot{\phi}^2 + \dot{z}^2) - \frac{k}{r}
\end{align*}

Analog zu oben:

\begin{align*}
  &\frac{d}{dr} V_{\text{eff}} = mr\dot{\phi}^2 + \frac{k}{r^2} \overset{!}{=} 0 \\
  \Leftrightarrow\quad&\frac{mr^3\dot{\phi}^2}{r^2} + \frac{k}{r^2} \\
  \Leftrightarrow\quad&\frac{mr^3\dot{\phi}^2 + 2k}{r^2}
\end{align*}

Die Lösung ist dann:

\begin{align*}
  &mr^3\dot{\phi}^2 = -2k \\
  \Leftrightarrow\quad r^3 = \frac{-2k}{m\dot{\phi}^2} \\
  \Leftrightarrow\quad r = \sqrt[3]{\frac{-2k}{m\dot{\phi}^2}}
\end{align*}

Um zu bestimmen, ob es sich wirklich um ein Minimum handelt, einsetzen in die zweite $r$-Ableitung:
\begin{align*}
  &\frac{d^2}{dr^2} V_{\text{eff}} = m\dot{\phi}^2 - \frac{2k}{r^3}\\
  &\frac{d^2}{dr^2} V_{\text{eff}}(\sqrt[3]{\frac{-2k}{m\dot{\phi}^2}}) =  m\dot{\phi}^2 - \frac{2k}{-2k} m\dot{\phi}^2 \\
  = &m\dot{\phi}^2 + m\dot{\phi}^2 \\
  = &2m\dot{\phi}^2\\
\end{align*}

Es ist klar, dass $m > 0$. Da $\dot{\phi}$ reell ist, ist $\dot{\phi}^2 > 0$. Damit existiert für den Fall $d=3$ eine stabile Bahn.

Wir prüfen nun $d=2$. Effektives Potential ist dann (wie oben, nur mit $u=z=0$, angepasstes Potential):

\begin{align*}
  V_{\text{eff}} = \frac{m}{2}r^2\dot{\phi}^2 - k
\end{align*}

In $r$ ausgewertet ist dies eine nach oben offene Parabel, von der wir wissen,
dass sie ein Minimum hat. Wir sparen uns also weitere Berechnungen. Die Bahn
ist stabil.



\end{document}
