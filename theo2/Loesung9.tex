\documentclass[a4paper,german,12pt,smallheadings]{scrartcl}
\usepackage[T1]{fontenc}
\usepackage[utf8]{inputenc}
\usepackage{babel}
\usepackage{tikz}
\usepackage{geometry}
\usepackage{amsmath}
\usepackage{amssymb}
\usepackage{float}
\usepackage[thinspace,thinqspace,squaren,textstyle]{SIunits}
\restylefloat{table}
\geometry{a4paper, top=15mm, left=20mm, right=40mm, bottom=20mm, headsep=10mm, footskip=12mm}
\linespread{1.5}
\setlength\parindent{0pt}
\begin{document}
\begin{center}
\bfseries % Fettdruck einschalten
\sffamily % Serifenlose Schrift
\vspace{-40pt}
Analytische Mechanik, Sommersemester 2013, 9. Blatt \\
Luis Herrmann und Markus Fenske, Tutor: Clemens Meyer zu Rheda
\vspace{-10pt}
\end{center}
\section*{Exercise 1}
\subsection*{Part a}

Let $c_{ij} = a_{ik} b_{kj}$ (which is the definition of matrix multiplication)
then

\begin{align*}
  x_i'' &= a_{ik} x'_k \\
         = a_{ik} b_{kj} x_j \\
         = c_{ij} x_j
\end{align*}

\subsection*{Part b}

It's obiously true that

\begin{align*}
  a_{ik} b_{kj} \neq b_{ik} a_{kj}
\end{align*}

if $a_{ik} \neq a_{ki}$


\section*{Exercise 2}

Let

\begin{align*}
  r_i  &= a^{-1}_{ij} r_j'
  r_j' &= a_{jk} r_k \\
\end{align*}

Then

\begin{align*}
   r_i  &= \underbrace_{a^{-1}_{ij} a_{jk}}_{\text{must $= \delta_{ik}$}} r_k \\
   \Leftrightarrow
\end{align*}

Because $r_i$ should equal $r_k$ iff $i = k$, otherwise $a^{-1}_{ij} a_{jk}$
should be zero, we can write

\begin{align*}
  a^{-1}_{ij} a_{jk} = \delta_{ik}
\end{align*}

which is the identity matrix. This works in both directions $r \to r' \ro r$ as
well as $r' \to r \to r'$. Thus $AA^{-1} = A^{-1}A = I$.

An orthogal transformation is defined as 



\end{document}
