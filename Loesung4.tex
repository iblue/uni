\documentclass[a4paper,german,12pt,smallheadings]{scrartcl}
\usepackage[T1]{fontenc}
\usepackage[utf8]{inputenc}
\usepackage{babel}
\usepackage{tikz}
\usepackage{geometry}
\usepackage{amsmath}
\usepackage{float}
\usepackage{wrapfig}
\usepackage[thinspace,thinqspace,squaren,textstyle]{SIunits}
\restylefloat{table}
\geometry{a4paper, top=15mm, left=20mm, right=40mm, bottom=20mm, headsep=10mm, footskip=12mm}
\linespread{1.5}
\setlength\parindent{0pt}
\begin{document}
\begin{center}
\bfseries % Fettdruck einschalten
\sffamily % Serifenlose Schrift
\vspace{-40pt}
Mathematik für Physiker I, Wintersemester 2012/2013, 4. Übungsblatt

Florian Neumeyer und Markus Fenske, Tutor: Stephan Schwartz
\vspace{-10pt}
\end{center}

\section*{4.1 Kommutierende Matrix}

Wir stellen aus $A \cdot B = B \cdot A$ ein Gleichungssystem auf und lösen dieses.

\begin{align*}
A \cdot B &= B \cdot A \\
\begin{pmatrix}
1 & 2 \\
3 & 4
\end{pmatrix}
\cdot
\begin{pmatrix}
b_{11} & b_{12} \\
b_{21} & b_{22}
\end{pmatrix}
&=
\begin{pmatrix}
b_{11} & b_{12} \\
b_{21} & b_{22}
\end{pmatrix}
\cdot
\begin{pmatrix}
1 & 2 \\
3 & 4
\end{pmatrix} \\
\begin{pmatrix}
1b_{11} + 2b_{21} & 1b_{12} + 2b_{22} \\
3b_{11} + 4b_{21} & 3b_{12} + 4b_{22}
\end{pmatrix}
&=
\begin{pmatrix}
1b_{11} + 3b_{12} & 2b_{11} + 4b_{12} \\
1b_{21} + 3b_{22} & 2b_{21} + 4b_{22}
\end{pmatrix}
\end{align*}

Damit ergeben sich folgende Gleichungen:

\begin{align*}
1b_{11} + 2b_{21} - 1b_{11} - 3b_{12} &= 0 \\
1b_{12} + 2b_{22} - 2b_{11} - 4b_{12} &= 0 \\
3b_{11} + 4b_{21} - 1b_{21} - 3b_{22} &= 0 \\
3b_{12} + 4b_{22} - 2b_{21} - 4b_{22} &= 0
\end{align*}

Oder umgeformt:
\begin{equation*}
\left.
\begin{aligned}
 0b_{11} + 3b_{12} + 2b_{21} + 0b_{22} &= 0 \\
-2b_{11} - 3b_{12} + 0b_{21} + 2b_{22} &= 0 \\
 3b_{11} + 0b_{12} + 3b_{21} - 3b_{22} &= 0 \\
 0b_{11} + 3b_{12} - 2b_{21} + 0b_{22} &= 0
\end{aligned}
\right\}
\left(
 \begin{matrix}
  0 & 3 & 2 & 0\\
  -2 & -3 & 0 & 2 \\
  3 & 0 & 3 & -3 \\
  0 & 3 & -2 & 0 \\
 \end{matrix}
 \left|
  \hspace{5pt}
  \begin{matrix}
   0\\
   0\\
   0\\
   0
  \end{matrix}
 \right)
\right.
\end{equation*}

Um das Gleichungssystem zu Lösen, kann man auf den \textsc{GAUSS}schen
Algorithmus verzichten, dazu reicht scharfes Hinsehen.

Wenn man nun die erste und letzte Zeile des Gleichungssystems betrachtet, fällt
auf, dass $b_{12} = 0$ und $b_{21} = 0$ sein müssen. Daraus folgt bei
Betrachtung der beiden mittleren Zeilen, dass $b_{11} = b_{22}$ sein müssen.

Wenn die Matrix $B$ mit der gegebenen Matrix $A$ kommutieren soll, muss
folgende Gleichung gelten, in der $a$ irgendeine Zahl ist.

\begin{equation*}
B
=
\begin{pmatrix}
a & 0 \\
0 & a
\end{pmatrix}
\end{equation*}

\section*{4.2 Spuren}
\subsection*{a. Distributivgesetz der Spurbildung}
Nicht bearbeitet.

\subsection*{b. Kommutativgesetz der Spurbildung}
Ich verwende ab hier \textsc{Einstein}sche Summenkonvention. Die Summenzeichen
werden weggelassen, es wird über gleiche Indices summiert.

Bekannt ist, dass die Multiplikation zweier Matrizen definiert ist als $A \cdot
B = A_{ij} B_{jk}$. Die Spurbildung ist dann $\operatorname{Sp}(A) = A_{ii}$.
Somit ist:

\begin{align*}
\operatorname{Sp}(A \cdot B) &=  A_{ij} B_{ji} \\
&= B_{ji} A_{ij} \\
&= B_{ij} A_{ji} \\
&= \operatorname{Sp}(B \cdot A)
\end{align*}

Die Gültigkeit der zweiten Zeile ergibt sich aus dem Kommutativgesetz das für
Skalare gilt. In der dritten Zeile wurden einfach nur die Namen der Indices
vertauscht, was nichts anderes als eine Umbenennung ansonsten gleichbleibender
Variablen ist.

\subsection*{c. Kommutativgesetz der Spurbildung}
Nicht bearbeitet.

\section*{4.3 Inverse Matrix}

Dazu benutzen wir den \textsc{Gauß}-\textsc{Jordan}-Algorithmus.

\begin{equation*}
\left(
 \begin{matrix}
 1 & 3 & -1 & 4 \\
 2 & 5 & -1 & 3 \\
 0 & 4 & -3 & 1 \\
 -3 & 1 & -5 & -2
 \end{matrix}
 \left|
  \hspace{5pt}
  \begin{matrix}
  1 & 0 & 0 & 0 \\
  0 & 1 & 0 & 0 \\
  0 & 0 & 1 & 0 \\
  0 & 0 & 0 & 1
  \end{matrix}
 \right)
\right.
\end{equation*}

Abziehen der doppelten ersten Zeile und Addition der $3$-fachen ersten Zeile von der vierten Zeile:

\begin{equation*}
\left(
 \begin{matrix}
 1 & 3 & -1 & 4 \\
 0 & -1 & 1 & -5 \\
 0 & 4 & -3 & 1 \\
 0 & 10 & -8 & 10
 \end{matrix}
 \left|
  \hspace{5pt}
  \begin{matrix}
  1 & 0 & 0 & 0 \\
  -2 & 1 & 0 & 0 \\
  0 & 0 & 1 & 0 \\
  3 & 0 & 0 & 1
  \end{matrix}
 \right)
\right.
\end{equation*}

Multiplikation der zweiten Zeile mit $-1$
\begin{equation*}
\left(
 \begin{matrix}
 1 & 3 & -1 & 4 \\
 0 & 1 & -1 & 5 \\
 0 & 4 & -3 & 1 \\
 0 & 10 & -8 & 10
 \end{matrix}
 \left|
  \hspace{5pt}
  \begin{matrix}
  1 & 0 & 0 & 0 \\
  2 & -1 & 0 & 0 \\
  0 & 0 & 1 & 0 \\
  3 & 0 & 0 & 1
  \end{matrix}
 \right)
\right.
\end{equation*}

Subtraktion des $4$ und $10$-fachen der zweiten Zeile von der dritten und vierten Zeile.

\begin{equation*}
\left(
 \begin{matrix}
 1 & 3 & -1 & 4 \\
 0 & 1 & -1 & 5 \\
 0 & 0 & 1 & -19 \\
 0 & 0 & 2 & -40
 \end{matrix}
 \left|
  \hspace{5pt}
  \begin{matrix}
  1 & 0 & 0 & 0 \\
  2 & -1 & 0 & 0 \\
  -8 & 4 & 1 & 0 \\
  -17 & 10 & 0 & 1
  \end{matrix}
 \right)
\right.
\end{equation*}

Subtraktion des doppelten der dritten Zeile von der letzten Zeile.
\begin{equation*}
\left(
 \begin{matrix}
 1 & 3 & -1 & 4 \\
 0 & 1 & -1 & 5 \\
 0 & 0 & 1 & -19 \\
 0 & 0 & 0 & -2
 \end{matrix}
 \left|
  \hspace{5pt}
  \begin{matrix}
  1 & 0 & 0 & 0 \\
  2 & -1 & 0 & 0 \\
  -8 & 4 & 1 & 0 \\
  -1 & 2 & 0 & 1
  \end{matrix}
 \right)
\right.
\end{equation*}

Teilen der letzten Zeile durch $-2$.
\begin{equation*}
\left(
 \begin{matrix}
 1 & 3 & -1 & 4 \\
 0 & 1 & -1 & 5 \\
 0 & 0 & 1 & -19 \\
 0 & 0 & 0 & 1
 \end{matrix}
 \left|
  \hspace{5pt}
  \begin{matrix}
  1 & 0 & 0 & 0 \\
  2 & -1 & 0 & 0 \\
  -8 & 4 & 1 & 0 \\
  \frac{1}{2} & -1 & 0 & -\frac{1}{2}
  \end{matrix}
 \right)
\right.
\end{equation*}

Subtraktion des $-19$-fachen der vierten Zeile von der dritten Zeile.
\begin{equation*}
\left(
 \begin{matrix}
 1 & 3 & -1 & 4 \\
 0 & 1 & -1 & 5 \\
 0 & 0 & 1 & 0 \\
 0 & 0 & 0 & 1
 \end{matrix}
 \left|
  \hspace{5pt}
  \begin{matrix}
  1 & 0 & 0 & 0 \\
  2 & -1 & 0 & 0 \\
  \frac{3}{2} & -15 & 1 & -\frac{19}{2} \\
  \frac{1}{2} & -1 & 0 & -\frac{1}{2}
  \end{matrix}
 \right)
\right.
\end{equation*}

Subtraktion des $5$-fachen der vierten Zeile von der zweiten Zeile.
\begin{equation*}
\left(
 \begin{matrix}
 1 & 3 & -1 & 4 \\
 0 & 1 & -1 & 0 \\
 0 & 0 & 1 & 0 \\
 0 & 0 & 0 & 1
 \end{matrix}
 \left|
  \hspace{5pt}
  \begin{matrix}
  1 & 0 & 0 & 0 \\
  -\frac{1}{2} & 4 & 0 & \frac{5}{2} \\
  \frac{3}{2} & -15 & 1 & -\frac{19}{2} \\
  \frac{1}{2} & -1 & 0 & -\frac{1}{2}
  \end{matrix}
 \right)
\right.
\end{equation*}

Addition der dritten Zeile zur zweiten und aus praktischen Gründen auch gleich zur ersten Zeile:
\begin{equation*}
\left(
 \begin{matrix}
 1 & 3 & 0 & 4 \\
 0 & 1 & 0 & 0 \\
 0 & 0 & 1 & 0 \\
 0 & 0 & 0 & 1
 \end{matrix}
 \left|
  \hspace{5pt}
  \begin{matrix}
  \frac{5}{2} & -15 & -1 & -\frac{19}{2} \\
  1 & -11 & -1 & 7 \\
  \frac{3}{2} & -15 & 1 & -\frac{19}{2} \\
  \frac{1}{2} & -1 & 0 & -\frac{1}{2}
  \end{matrix}
 \right)
\right.
\end{equation*}

Subtraktion des vierfachen der letzten Zeile von der ersten Zeile
\begin{equation*}
\left(
 \begin{matrix}
 1 & 3 & 0 & 0 \\
 0 & 1 & 0 & 0 \\
 0 & 0 & 1 & 0 \\
 0 & 0 & 0 & 1
 \end{matrix}
 \left|
  \hspace{5pt}
  \begin{matrix}
  \frac{1}{2} & -19 & -1 & -\frac{15}{2} \\
  1 & -11 & -1 & 7 \\
  \frac{3}{2} & -15 & 1 & -\frac{19}{2} \\
  \frac{1}{2} & -1 & 0 & -\frac{1}{2}
  \end{matrix}
 \right)
\right.
\end{equation*}

Subtraktion des dreifachen der zweiten Zeile von der ersten Zeile.
\begin{equation*}
\left(
 \begin{matrix}
 1 & 0 & 0 & 0 \\
 0 & 1 & 0 & 0 \\
 0 & 0 & 1 & 0 \\
 0 & 0 & 0 & 1
 \end{matrix}
 \left|
  \hspace{5pt}
  \begin{matrix}
  -\frac{5}{2} & 14 & 8 & -\frac{57}{2} \\
  1 & -11 & -1 & 7 \\
  \frac{3}{2} & -15 & 1 & -\frac{19}{2} \\
  \frac{1}{2} & -1 & 0 & -\frac{1}{2}
  \end{matrix}
 \right)
\right.
\end{equation*}

Wir lassen uns nun das Ergebnis von Wolfram Alpha berechnen und erhalten die korrekte Lösung:
\begin{equation*}
\left(
 \begin{matrix}
 1 & 0 & 0 & 0 \\
 0 & 1 & 0 & 0 \\
 0 & 0 & 1 & 0 \\
 0 & 0 & 0 & 1
 \end{matrix}
 \left|
  \hspace{5pt}
  \begin{matrix}
    \frac{5}{2} & 22 & -29 & \frac{27}{2} \\
     1 & -11 & 15 & -7 \\
     \frac{3}{2} & -15 & -20 & -\frac{19}{2} \\
     \frac{1}{2} & -1 & 1 & -\frac{1}{2}
  \end{matrix}
 \right)
\right.
\end{equation*}

\section*{4.4 Drehmatrix}

Nicht bearbeitet.

\end{document}
