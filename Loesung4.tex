\documentclass[a4paper,german,12pt,smallheadings]{scrartcl}
\usepackage[T1]{fontenc}
\usepackage[utf8]{inputenc}
\usepackage{babel}
\usepackage{tikz}
\usepackage{geometry}
\usepackage{amsmath}
\usepackage{amssymb}
\usepackage{float}
%\usepackage{wrapfig}
\usepackage{pdflscape}
\pagenumbering{gobble}
\usepackage[thinspace,thinqspace,squaren,textstyle]{SIunits}
\restylefloat{table}
\geometry{a4paper, top=15mm, left=20mm, right=40mm, bottom=20mm, headsep=10mm, footskip=12mm}
\linespread{1.5}
\setlength\parindent{0pt}
\begin{document}
\begin{center}
\bfseries % Fettdruck einschalten
\sffamily % Serifenlose Schrift
\vspace{-40pt}
Analysis I, Sommersemester 2013, 4. Übungsblatt \\
Luis Herrmann und Markus Fenske, Tutor: Adam Schienle
\vspace{-10pt}
\end{center}

\section*{Aufgabe 4.1}
Der Schluss ist nicht korrekt.

\textbf{Gegenbeispiel}: Sei $a_n = \frac{1}{5^n}$. Dies ist eine Folge positiver Zahlen.
Der Grenzwert ist $ a = \lim_{n \to \infty} a_n = 0$ (Wie gefordert ist $a \ge
0$).

\begin{align*}
  \lim_{n \to \infty} \sqrt[n]{a_n} = \sqrt[n]{\frac{1}{5^n}} = \frac{1}{5}\qquad \neq\qquad \lim_{n \to \infty} \sqrt[n]{a} = 0
\end{align*}

\section*{Aufgabe 4.2}
Quotientenkriterium: Die Reihe ist konvergent, wenn es ein $q < 1$ gibt, so
dass für fast alle $n \in \mathbb{N}$ gilt:

\begin{align*}
  \left| \frac{a_{n+1}}{a_n} \right| \le q < 1
\end{align*}

Wir prüfen:

\begin{align*}
  \left| \frac{\frac{(n+1)!}{(n+1)^{n+1}}}{\frac{n!}{n^n}} \right| &= \left| \frac{n^n(n+1)!}{n!(n+1)^{n+1}} \right| \\ 
  &= \left| \frac{n^n n! (n+1)}{n!(n+1)^{n+1}} \right| \\
  &= \left| \frac{n^n (n+1)}{(n+1)^{n+1}} \right| \\
  &= \left| \frac{n^n}{(n+1)^{n}} \right| \\
\end{align*}

Da bereits für $n=1$ gilt, dass $n^n < (n+1)^n$, ist das Quotientenkriterium
erfüllt. Die Reihe konvergiert.

Für die andere Reihe benutzen wir das Wurzelkriterium.
\begin{align*}
  &\lim_{n \to \infty} \sqrt[n]{\left| a_n \right|} =\\
  &\lim_{n \to \infty} \sqrt[n]{\left| \frac{(n+1)^{n-1}}{(-n)^n} \right|} =\\
  &\lim_{n \to \infty} \sqrt[n]{\left| \frac{(n+1)^{n-1}}{(-n)^n} \right|} =\\
  &\lim_{n \to \infty} \sqrt[n]{\frac{(n+1)^{n-1}}{n^n}} =\\
  &\lim_{n \to \infty} \frac{1}{n} \sqrt[n]{(n+1)^{n-1}} =\\
  &\lim_{n \to \infty} \frac{1}{n} \frac{(n+1)^n}{(n+1)^n} \sqrt[n]{(n+1)^{n-1}} =\\
  &\lim_{n \to \infty} \frac{1}{n} \frac{1}{(n+1)^n} \sqrt[n]{(n+1)^n} =\\
  &\lim_{n \to \infty} \frac{1}{n} \frac{(n+1)}{(n+1)^n} =\\
  &\lim_{n \to \infty} n^{-1} (n+1)^{-n+1} =\\
  &\lim_{n \to \infty} \left(1+\frac{1}{n}\right)^{-n+1} =\\
  &\lim_{n \to \infty} \left(1+\frac{1}{n}\right)^{-n} \cdot \lim_{n \to \infty} \left(1+\frac{1}{n}\right) =\\
  &\lim_{n \to \infty} \left(1+\frac{1}{n}\right)^{-n}=\\
  &\frac{1}{\lim_{n \to \infty} \left(1+\frac{1}{n}\right)^{n}}=\\
  &\frac{1}{e} < 1
\end{align*}

Also konvergiert die Reihe.

\section*{Aufgabe 4.3}
Quotientenkriterium: Die Reihe ist konvergent, wenn es ein $q < 1$ gibt, so
dass für fast alle $n \in \mathbb{N}$ gilt:

\begin{align*}
  \left| \frac{a_{n+1}}{a_n} \right| \le q < 1
\end{align*}

Dies prüfen wir:

\begin{align*}
  \left| \frac{\frac{(n+1)^9}{\sqrt{2^{n+1}}}}{\frac{n^9}{\sqrt{2^n}}} \right| &= \left| \frac{\sqrt{2^n} (n+1)^9}{n^9 \sqrt{2^{n+1}}} \right| \\
  &= \left| \frac{(n+1)^9}{n^9 \sqrt{2}} \right| \\
  &= \left| \frac{1}{\sqrt{2}} \frac{n^9+9n^8+36n^7+84n^6+126n^5+126n^4+84n^3+36n^2+9n+1}{n^9} \right| \\
  &= \left| \frac{1}{\sqrt{2}} \left(1 + \frac{9}{n} + \frac{36}{n^2} + \frac{84}{n^3} + \frac{126}{n^4} + \frac{126}{n^5} + \frac{84}{n^6} + \frac{36}{n^7} + \frac{9}{n^8} + \frac{1}{n^9}\right) \right| \\
\end{align*}

Betrachten wir diesen Term als eine Folge $q_n$. Diese Folge konvergiert ganz
offensichtlich, denn die Brüche, die $n$-Potenzen im Nenner haben, konvergieren
für sich gegen Null. Draus folgt

\begin{align*}
  \lim_{n \to \infty} q_n = \frac{1}{\sqrt{2}} = \frac{\sqrt{2}}{2} < 1
\end{align*}

Somit ist die Reihe nach Quotientenkriterium konvergent.

Für die andere Reihe weisen wir mit Majorantenkriterium nach, dass diese divergiert. Wir wissen, dass die folgende Reihe $a$ divergiert.

\begin{align*}
  \sum_{i=1}^\infty a_n = \sum_{i=1}^\infty  \frac{1}{n} = \sum_{i=1}^\infty \frac{n}{n^2}
\end{align*}

Gemäß Majorantenkriterium (jeder Summand der Reihe $b$ ist größer als der der
entsprechende Summand in der vorherigen Reihe) divergiert dann auch die Reihe $b$:

\begin{align*}
  \sum_{i=1}^\infty b_n = \sum_{i=1}^\infty  \frac{n+4}{n^2}
\end{align*}

Denn da $n>0$ muss gelten:

\begin{align*}
  \frac{n+4}{n^2} > \frac{n}{n^2}
\end{align*}

Wir zeigen nun, dass die Terme der gebenen Reihe ab $n > 1$ alle größer sind
als die der Reihe $b$

\begin{align*}
 &\frac{n+4}{n^2-3n+1} > \frac{n+4}{n^2} \\
 \Leftrightarrow\quad& \frac{n^2-3n+1}{n+4} < \frac{n^2}{n+4} \\
 \Leftrightarrow\quad& n^2-3n+1 < n^2 \\
 \Leftrightarrow\quad& -3n+1 < 0 \\
 \Leftrightarrow\quad& -3n < -1 \\
 \Leftrightarrow\quad& 3n > 1
\end{align*}

Da $n > 1$, ist dies wahr. Damit divergiert die gegebenen Reihe nach Majorantenkriterium.

\section*{Aufgabe 4.4}
Hier bietet sich das Wurzelkriterium an. Die Reihe konvergiert dann, wenn

\begin{align*}
  \lim_{n \to \infty} \sqrt[n]{\left| a_n \right|} < 1
\end{align*}

Im umgekehrten Fall folgt die Divergenz der Reihe:

\begin{align*}
  \lim_{n \to \infty} \sqrt[n]{\left| a_n \right|} > 1
\end{align*}

Wenn der Limes jedoch $=1$ ist, sagt das Wurzelkriterium nicht aus.

Wir berechnen nun also das Wurzelkriterium:

\begin{align*}
  &\lim_{n \to \infty} \sqrt[n]{\left| a_n \right|} = \\
  &\lim_{n \to \infty} \sqrt[n]{\left| \frac{x^n}{1+x^{2n}} \right|} = \\
  &\lim_{n \to \infty} \sqrt[n]{\left| x^n \right|} \cdot \frac{1}{\lim_{n \to \infty} \sqrt[n]{\left| 1+x^{2n} \right|}} = \\
  &|x| \cdot \frac{1}{\lim_{n \to \infty} \sqrt[n]{\left| x^{2n} \right|}} = \\
  &\frac{|x|}{\left| x^2 \right|} =\\
  &\frac{1}{\left| x \right|}\\
\end{align*}

Dies bedeutet, dass die Reihe konvergiert, wenn $|x| > 1$. Für $|x| < 1$
divergiert die Reihe. Im Fall $x = 1$ ist klar, dass jedes Glied der Summe
genau $\frac{1}{2}$ ist, was bedeutet, dass die Folge divergiert.

\section*{Aufgabe 4.5}
\subsection*{Teil a}

Wir prüfen, ob gilt:
\begin{align*}
  &s_{k+1} \le \frac{1}{2}s_k \\
  \Leftrightarrow\quad&\sum_{n=2}^{\infty} \frac{1}{n^{k+1}} \le \frac{1}{2} \sum_{n=2}^{\infty} \frac{1}{n^k} \\
  \Leftrightarrow\quad&\sum_{n=2}^{\infty} \frac{1}{n^{k+1}} \le \sum_{n=2}^{\infty} \frac{1}{2n^k}
\end{align*}

Dies gilt, wenn für jeden Summanden gilt:

\begin{align*}
  &\frac{1}{n^{k+1}} \le \frac{1}{2n^k} \\
  \Leftrightarrow\quad& n^{k+1} \ge 2n^k \\
  \Leftrightarrow\quad& n \cdot n^{k} \ge 2n^k \\
  \Leftrightarrow\quad& n \ge 2
\end{align*}

Da wir erst ab $n=2$ anfangen zu summieren, ist also stets $s_{k+1} \le \frac{1}{2}s_k$.

\subsection*{Teil b}
Die Summanden der Reihe fallen monoton (das haben wir eben gezeigt). Sie sind
aber außerdem beschränkt, denn $\frac{1}{n^k}$ kann für positive $n$ und
positive $k$ niemals negativ werden. Das bedeutet, dass $s_k$ Nullfolge ist.
Nach Nullfolgenkriterium ist dann klar, dass die Reihe konvergiert.

\subsection*{Teil c}
\begin{align*}
  &\sum_{n=2}^\infty \sum_{k=2}^\infty \frac{1}{n^k} = \sum_{n=2}^\infty \frac{1}{n(n-1)} = 1
\end{align*}
\end{document}
